%%% INCLUDE FILE FOR DEFINITIONS
%%% These may require various packages.

% Shortcuts in regular text
\newcommand{\degs}{\ensuremath{^\circ}}
\newcommand{\EE}[1]{\ensuremath{\times 10^{#1}}}
\newcommand{\ttimes}{\ensuremath{{}\times{}}}
\newcommand{\cclicense}{%
  \smash{\raisebox{-0.45ex}{%
  \setlength{\unitlength}{1em}%
  \begin{picture}(1,1)%
    \put(0.5,0.5){\circle{1}}
    \put(0.5,0.5){\hbox to 0pt{\hss\raisebox{-.45ex}{\tiny\textsf{CC}}\hss}}
  \end{picture}%
  }}%
  \hskip -1em%
  \href{http://creativecommons.org/licenses/by-nc-sa/3.0/}%
  {\ \hskip 1em \textsf{BY-NC-SA}}%
}

%\newcommand{\horizsep}{{\par\noindent\centering\rule[.25ex]{.75\columnwidth}{2pt}\par}}
\newcommand{\horizsep}{\vspace{\baselineskip}\noindent\hspace{\stretch{1}}$
\ast\qquad \ast\qquad \ast\qquad
$ \hspace{\stretch{1}} \vspace{\baselineskip}}

% Research
\newcommand{\lop}[1]{\mathcal{L}\!\left[#1\right]}
\newcommand{\lopinv}[2]{\mathcal{L}_{#1}^{-1}\!\left[#2\right]}
\newcommand{\Dtens}{\mat{D}}
\newcommand{\Etens}{\mat{E}}
\newcommand{\APone}{AP$_1$}
\newcommand{\Pone}{P$_1$}
\newcommand{\SN}{S$_N$}%{S$_\text{N}$}%{$S_N$}%
\newcommand{\PN}{P$_N$}%{P$_\text{N}$}%{$P_N$}%
\newcommand{\CN}{Crank--Nicolson} %Yes, it's Nic not Nich
\newcommand{\Eddington}{\mathcal{E}} %whatever symbol I decided for Eddington
\newcommand{\RadEn}{E} %whatever symbol I decide for radiation energy
\newcommand{\Sigmatr}{\Sigma_{\mathit{tr}}}

% Program names
\newcommand{\cpp}{\textsf{C\raisebox{0.2ex}{++}}}

% General math shortcuts
\newcommand{\ud}{\mathop{}\!\mathrm{d}}
\newcommand{\pder}[2]{\frac{\partial #1}{\partial #2}}
\newcommand{\oder}[2]{\frac{\mathrm{d} #1}{\mathrm{d} #2}}
\newcommand{\tpder}[2]{{\partial #1}/{\partial #2}} %inlined
\newcommand{\toder}[2]{{\mathrm{d} #1}/{\mathrm{d} #2}} %inlined
\newcommand{\lra}{ \quad \Longrightarrow \quad }
\newcommand{\eexp}{\mathop{}\!\mathrm{e}} % upright ``e'' for exponent
\newcommand{\expp}[1]{\exp\!\left( {#1} \right)} % exp with parentheses
\newcommand{\qeq}{\stackrel{\mathrm{?}}{=}}

% Probability
\newcommand{\expectation}[1]{\mathop{}\!\mathrm{E}\!\left[ #1 \right]}
\DeclareMathOperator{\Var}{Var} % variance

% Asymptotic analysis
\DeclareMathOperator{\Ei}{Ei} % Exponential function
\newcommand{\lapl}[1]{\mathcal{L}[{#1}]} %laplace

%change the Re and Im operators from fancy curly letters
\DeclareMathOperator{\MathOpRe}{Re}
\renewcommand{\Re}{\MathOpRe}
\DeclareMathOperator{\MathOpIm}{Im}
\renewcommand{\Im}{\MathOpIm}

%imaginary ``i'' , upright 'i' or \imath
\newcommand{\iimag}{\mathrm{i}}

% Finite differences
\newcommand{\hot}{\text{h.o.t.}}
\newcommand{\inv}{^{-1}}

% Numerical Linear Algebra
\newcommand{\conj}{^{\ast}} % complex conjugate (transpose)
\newcommand{\norm}[1]{\left\| #1 \right\|} % double pipe
\newcommand{\abs}[1]{\left| #1 \right|} % single pipe
\newcommand{\eps}{\varepsilon}
\DeclareMathOperator{\fl}{fl}

\DeclareMathOperator{\acosh}{arccosh} 

% Define a command to write a nice-looking element, e.g. 4,2 He
\newcommand{\elem}[3]{\ensuremath{{}^{{#1}}_{{#2}}\mathrm{{#3}}}}

% Vector definitions
\newcommand{\mat}[1]{\mathbf{#1}} %matrix is bold upright
\renewcommand{\vec}[1]{\bm{#1}} %vector is bold italic
\newcommand{\op}[1]{\mathsf{#1}} % ``operator'' is sans serif

\newcommand{\vd}{\bm{\cdot}} % slightly bold vector dot
\newcommand{\del}{\vec{\nabla}} % gradient (Del) is bold
\newcommand{\grad}{\vec{\nabla}} % gradient

%\newcommand{\abr}[1]{\langle {#1} \rangle}
\newcommand{\abr}[1]{\left\langle {#1} \right\rangle} % angle brackets for avg.

%% topbox is useful in extended definitions of math terms inside an align
\newcommand{\topbox}[2][0.6]{\parbox[t]{#1\columnwidth}{\raggedright{}#2}}

% commands to make text in math mode appear as zero-width (better-looking
% integrals/sums, e.g.)
% from mathmode.pdf page 74, or Alexander R. Perlis ``A complement to \smash,
% \llap, and \rlap''

\def\mathllap{\mathpalette\mathllapinternal}
	\def\mathllapinternal#1#2{%
	\llap{$\mathsurround=0pt#1{#2}$}%
}
\def\clap#1{\hbox to 0pt{\hss#1\hss}}%
\def\mathclap{\mathpalette\mathclapinternal}%
\def\mathclapinternal#1#2{%
	\clap{$\mathsurround=0pt#1{#2}$}%
}
\def\mathrlap{\mathpalette\mathrlapinternal}%
\def\mathrlapinternal#1#2{%
	\rlap{$\mathsurround=0pt#1{#2}$}%
}
