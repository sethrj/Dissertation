% !TEX root = _individual/finalAbstract.tex

%%%%%%%%%%%%%%%%%%%%%%%%%%%%%%%%%%%%%%%%%%%%%%%%%%%%%%%%%%%%%%%%%%%%%%%%%%%%%%%%
In this thesis, we develop and numerically test new approximations to
time-dependent radiation transport with the goal of obtaining more accurate
answers than diffusion can generate, yet requiring less computational effort
than transport.
The first method is the nascent anisotropic diffusion (AD)
approximation, which we extend to time-dependent, finite problems; the
second is a novel anisotropic \Pone-like (\APone) approximation. These methods
are ``anisotropic'' in that, rather than operating under the assumption of
linearly anisotropic radiation, they incorporate an arbitrary amount of
anisotropy via a transport-calculated diffusion coefficient. This anisotropic
diffusion tensor is the second angular moment of a simple, purely
absorbing transport problem.

In this thesis, much of the computational testing of the new methods is performed in
``flatland'' geometry, a fictional two-dimensional universe that provides a
realistic but computationally inexpensive testbed. As work ancillary to
anisotropic
diffusion and the numerical experiments, a complete description
of flatland diffusion, including boundary conditions, is developed. Also,
implementation details for both Monte Carlo and \SN\ transport in
flatland are provided.

The two new anisotropic methods, along with a ``flux limited'' modification to
anisotropic diffusion, are tested in a variety of problems. Some aspects of the
theory, including the newly formulated boundary conditions, are tested first with
diffusive, steady-state problems. The new methods are compared against
existing ones in linear, time-dependent radiation transport problems. Finally,
the efficacy and performance of the anisotropic methods are investigated in
several thermal radiative transfer (TRT) computational experiments.

Our results demonstrate that for many multi-dimensional problems, the new
anisotropic methods perform much better than their conventional counterparts.
In every time-dependent test, flux-limited anisotropic diffusion produced the
most accurate answers of the new methods.
Based on our numerical testing, we believe this method to be a strong contender
for accurate, inexpensive simulations of time-dependent transport and
thermal radiative transfer problems.
