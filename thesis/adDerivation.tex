% !TEX root = _individual/adDerivation.tex

%%%%%%%%%%%%%%%%%%%%%%%%%%%%%%%%%%%%%%%%%%%%%%%%%%%%%%%%%%%%%%%%%%%%%%%%%%%%%%%%
\chapter{Anisotropic diffusion theory}

The previous work in anisotropic diffusion has only considered a steady-state
problem in an infinite medium \cite{Lar2009c,Mor2007}. The new understanding of
the AD method presented in this chapter provides a theoretical basis for
using the AD method in time-dependent and nonlinear contexts, and it also
addresses the heretofore unsolved problem of how to treat the boundary
conditions.

A summary of the derivation is presented up front:
\prelistpar\begin{enumerate}
  \item Define the ``anisotropic intensity'' as $\Psi = I - \frac{1}{4\pi}\phi$.
    To handle boundary conditions, split  $\Psi \equiv \tilde \Psi +
    \Psi_\mathrm{bl}$. We will approximate $\tilde \Psi$ rather than $I$, and
    use $\Psi_\mathrm{bl}$ to determine matched boundary conditions.
  \item Manipulate the radiation transport equation and conservation equation to
    get a differential transport equation for $\tilde \Psi$ and
    $\Psi_\mathrm{bl}$.  Transform the equation for $\tilde \Psi$ to an
    \emph{integral} transport equation.
  \item Assume $I=O(1)$, $\frac1c\pder{}{t}=O(\epsilon^2)$, $\grad =
    O(\epsilon)$, $\int_{4\pi} \vec{\Omega} (\cdot) \ud\Omega = O(\epsilon)$.
  \item Use Taylor series to approximate nonlocal unknowns with local
    unknowns, discarding small terms. This yields
    \begin{equation*}
      \tilde \Psi(\vec{x}, \vec{\Omega})
      \approx - f(\vec{x}, \vec{\Omega})  \vec{\Omega} \vd \grad \phi\,.
    \end{equation*}
  \item Apply the standard transport-matching procedure to $\Psi_\mathrm{bl}$
    for vacuum or incident radiation boundary conditions. Use
    the identity $\int_{4\pi} \Psi \ud\Omega=0$ to find the boundary condition
    for $f$.
  \item Take the first angular moment of $\tilde \Psi$ to get
    $\vec{F}=-\Dtens \vd \grad \phi$\,.
  \item Substitute $\vec{F}$ into the time-dependent particle
    conservation equation to get time-dependent anisotropic diffusion.
\end{enumerate}

%%%%%%%%%%%%%%%%%%%%%%%%%%%%%%%%%%%%%%%%%%%%%%%%%%%%%%%%%%%%%%%%%%%%%%%%%%%%%%%%
\section{Anisotropic transport equation}
We begin by considering the gray radiation transport equation inside a time step
$0 < t < \Delta_t$,
with opacities frozen at some value $\sigma^\ast$ (which could be
$\sigma^n$ for a semi-implicit formulation, or $\sigma^{n+1,(k)}$ if
Picard iteration \cite{Kel1995} is being used to converge the nonlinearities).
\begin{subequations} \label{eqs:fullTransport}
\begin{multline} \label{eq:fullTransportVol}
  \frac{1}{c} \pder{I}{t}(\vec{x}, \vec{\Omega}, t)
    + \vec{\Omega}\vd \grad I(\vec{x}, \vec{\Omega}, t)
    + \sigma^\ast(\vec{x}) I (\vec{x}, \vec{\Omega}, t)
    \\ = \frac{1}{4\pi} \sigma^\ast(\vec{x}) ac [T(\vec{x}, t)]^4
    + \frac{1}{4\pi} q_{r}(\vec{x}, t)
    \equiv \frac{1}{4\pi} Q(\vec{x}, t) \,,
\\
x \in V,\  0 \le t < \Delta_t, \ \vec{\Omega} \in 4\pi,
\end{multline}
with a specified incident boundary condition on some subset of the boundary,
\begin{equation} \label{eq:fullTransportBndy}
  I(\vec{x}, \vec{\Omega}, t) = I^b(\vec{x}, \vec{\Omega}, t) \,,
 \quad \vec{x} \in \partial V_b, \ \vec{\Omega} \vd \vec{n} < 0,\ 0 \le t < \Delta_t
\end{equation}
a reflecting boundary condition on the rest of the boundary,
\begin{equation} \label{eq:fullTransportRefl}
  I(\vec{x}, \vec{\Omega}, t)
  = I(\vec{x}, \vec{\Omega}_r, t)
  \,,
 \quad \vec{x} \in \partial V_r, \ \vec{\Omega} \vd \vec{n} < 0,\ 0 \le t < \Delta_t
\end{equation}
and the initial condition
\begin{equation} \label{eq:fullTransportInit}
 I(\vec{x}, \vec{\Omega}, 0) = I^i(\vec{x}, \vec{\Omega}, t) \,,
 \quad \vec{x} \in V, \ \vec{\Omega} \in 4\pi\,.
\end{equation}
The initial condition here is usually the solution from the previous time
step.
\end{subequations}
The reflected angle on the boundary with outward normal $\vec{n}$ is just
\begin{equation} \label{eq:reflection}
  \vec{\Omega}_r = \vec{\Omega} - 2(\vec{\Omega} \vd \vec{n}) \vec{n}
\end{equation}
for incident angles, $\vec{\Omega} \vd \vec{n}<0$.

%We can express the left-hand side of Eq.~\eqref{eq:fullTransportVol} as a
%transport operator $\lop{\cdot}$ that relates the unknown $I$ to the known
%sources of particles on the right-hand side:
%\begin{equation*}
%  \lop{I(\vec{x}, \vec{\Omega}, t)} = \hat Q(\vec{x}, \vec{\Omega}, t)
%  + \text{boundary conditions} + \text{initial conditions}\,.
%\end{equation*}
%
\subsection{Conservation equation}
The particle (or radiation energy) conservation equation is the zeroth moment of
the transport equation.
\begin{subequations} \label{eqs:loEquations}
Operating on Eq.~\eqref{eq:fullTransportVol} by $\int_{4\pi} (\cdot) \ud
\Omega$ gives
\begin{equation} \label{eq:loVol}
\frac{1}{c} \pder{\phi}{t} (\vec{x}, t)
  + \grad \vd\vec{F}(\vec{x}, t)
  + \sigma^\ast(\vec{x}) \phi(\vec{x}, t)
  = \sigma^\ast(\vec{x}) ac [T(\vec{x}, t)]^4 + q_{r}(\vec{x}, t)
  = Q(\vec{x}, t)\,,
  \quad \vec{x} \in V \,.
\end{equation}
Doing the same to the initial condition, Eq.~\eqref{eq:fullTransportInit}, gives 
\begin{equation} \label{eq:loInit}
\phi(\vec{x}, 0) = \int_{4\pi}  I^i(\vec{x},
\vec{\Omega}) \ud\Omega = \phi^i(\vec{x})\,, \vec{x} \in V  \,.
\end{equation}
\end{subequations}
Here we have used the scalar intensity $\phi=\int_{4\pi} I \ud\Omega$ and the
radiation flux $\vec{F} = \int_{4\pi} \vec{\Omega} I \ud\Omega$. The scalar
intensity is the energy density of the radiation scaled by the speed of light:
$\phi=cE$, so integrating $\phi$ over a volume at time $t$ and dividing by $c$
gives the amount of energy in the radiation field in that volume.


\subsection{Anisotropic intensity}
Now we define the ``anisotropic intensity,'' which is the full angular intensity
with the isotropic expansion subtracted off:
\begin{equation} \label{eq:capPsi}
  \Psi(\vec{x}, \vec{\Omega}, t) \equiv I(\vec{x}, \vec{\Omega}, t) -
  \frac{1}{4\pi} \phi(\vec{x}, t)\,.
\end{equation}
This is equivalent to taking a spherical harmonic expansion of $I$, removing the
$m=0$ term, and reconstituting the remainder to form $\Psi$.

There are two important identities that the anisotropic intensity satisfies by
its definition:
\begin{subequations} \label{eqs:capPsiIdentities}
Its zeroth moment is identically zero:
\begin{equation} \label{eq:capPsiZeroth}
  \int_{4\pi} \Psi(\vec{x}, \vec{\Omega}, t) \ud\Omega
  = \int_{4\pi} I \ud\Omega
  - \frac{1}{4\pi}\int_{4\pi} \ud\Omega \,\phi
  = \phi - \phi
  = 0\,,
\end{equation}
and its first moment is the radiation flux:
\begin{equation} \label{eq:capPsiFirst}
  \int_{4\pi} \vec{\Omega} \Psi(\vec{x}, \vec{\Omega}, t) \ud\Omega
  = \int_{4\pi} \vec{\Omega} I \ud\Omega
  - \frac{1}{4\pi} \int_{4\pi} \vec{\Omega} \ud\Omega \,\phi
  = \vec{F} - \vec{0}
  = \vec{F}(\vec{x}, t)\,.
\end{equation}
\end{subequations}

\subsubsection{Transport equation}
Multiplying the particle conservation equation~\eqref{eq:loVol} by
$\frac{1}{4\pi}$ and subtracting it from the transport
equation~\eqref{eq:fullTransportVol} cancels the isotropic source on the
right-hand side, yielding
\begin{equation*}
  \frac{1}{c} \pder{}{t}\left[ I(\vec{x}, \vec{\Omega}, t)
  - \frac{1}{4\pi} \phi(\vec{x}, t) \right]
    + \vec{\Omega}\vd \grad I(\vec{x}, \vec{\Omega}, t)
    + \sigma^\ast(\vec{x}) \left[ I(\vec{x}, \vec{\Omega}, t)
  - \frac{1}{4\pi} \phi(\vec{x}, t) \right]
  - \frac{1}{4\pi} \grad \vd\vec{F}(\vec{x}, t)
= 0 \,.
\end{equation*}
Subtracting $\vec{\Omega}\vd \grad \phi/4\pi$ from both sides,
\begin{multline*}
  \frac{1}{c} \pder{}{t}\left[ I(\vec{x}, \vec{\Omega}, t)
  - \frac{1}{4\pi} \phi(\vec{x}, t) \right]
    + \vec{\Omega}\vd \grad \left[ I(\vec{x}, \vec{\Omega}, t)
  - \frac{1}{4\pi} \phi(\vec{x}, t) \right]
    + \sigma^\ast(\vec{x}) \left[ I(\vec{x}, \vec{\Omega}, t)
  - \frac{1}{4\pi} \phi(\vec{x}, t) \right]
  \\ = \frac{1}{4\pi} \grad \vd\vec{F}(\vec{x}, t) -
  \frac{1}{4\pi} \vec{\Omega}\vd \grad \phi(\vec{x}, t)\,.
\end{multline*}
Substituting Eq.~\eqref{eq:capPsi}, we have an exact expression for $\Psi$
inside the problem:
\begin{multline} \label{eq:capPsiVol}
  \frac{1}{c} \pder{}{t}\Psi(\vec{x}, \vec{\Omega}, t)
    + \vec{\Omega}\vd \grad \Psi(\vec{x}, \vec{\Omega}, t)
    + \sigma^\ast(\vec{x}) \Psi(\vec{x}, \vec{\Omega}, t)
  \\
  = \frac{1}{4\pi} \grad \vd\vec{F}(\vec{x}, t) -
  \frac{1}{4\pi} \vec{\Omega}\vd \grad \phi(\vec{x}, t)\,,
  \qquad
x \in V,\  0 \le t < \Delta_t, \ \vec{\Omega} \in 4\pi.
\end{multline}
No approximations have been made, but now instead of an isotropic source term
on the right hand side, we have an anisotropic source term that depends on the
unknowns $\phi$ and $\vec{F}$.

\subsubsection{Incident boundary condition}
Next, for vacuum or incident radiation boundaries, we subtract $\phi/4\pi$ from 
Eq.~\eqref{eq:fullTransportBndy}:
\begin{align}\nonumber
  I(\vec{x}, \vec{\Omega}, t) - \frac{1}{4\pi} \phi(\vec{x}, t)
  &= I^b(\vec{x}, \vec{\Omega}, t) - \frac{1}{4\pi} \phi(\vec{x}, t)
  \\ 
  \intertext{Substituting Eq.~\eqref{eq:capPsi} into the left hand side and the
  boundary condition Eq.~\eqref{eq:fullTransportBndy} into the right hand side,
  we get a boundary condition for $\Psi$:
  } \label{eq:capPsiBndy}
 \Psi(\vec{x}, \vec{\Omega}, t) 
  &=I^b(\vec{x}, \vec{\Omega}, t) - \frac{1}{4\pi} \phi(\vec{x}, t)\,,
\end{align}
for $\vec{x} \in \partial V_b$, $\vec{\Omega} \vd \vec{n} < 0$,
$0 \le t < \Delta_t$.

\subsubsection{Reflecting boundary condition}
Likewise with a reflecting boundary, subtract 
Next, for vacuum or incident radiation boundaries, we subtract $\phi/4\pi$ from 
Eq.~\eqref{eq:fullTransportBndy}:
\begin{align}\nonumber
  I(\vec{x}, \vec{\Omega}, t) - \frac{1}{4\pi} \phi(\vec{x}, t)
  &= I(\vec{x}, \vec{\Omega} - 2(\vec{\Omega} \vd \vec{n}) \vec{n}, t)
   - \frac{1}{4\pi} \phi(\vec{x}, t)\,.
  \\ 
  \intertext{Substituting Eq.~\eqref{eq:capPsi} into the left hand side and the
  boundary condition Eq.~\eqref{eq:fullTransportBndy} into the right hand side,
  we get a boundary condition on $\Psi$:
  } \label{eq:capPsiRefl}
 \Psi(\vec{x}, \vec{\Omega}, t) 
  &= \Psi(\vec{x}, \vec{\Omega}_r, t)
\end{align}
for $\vec{x} \in \partial V_r$, $\vec{\Omega} \vd \vec{n} < 0$,
$0 \le t < \Delta_t$.

\subsubsection{Initial condition}
Finally, to get an initial condition for the anisotropic intensity, we
multiply Eq.~\eqref{eq:loInit} by $1/4\pi$ and subtract it from
Eq.~\eqref{eq:fullTransportInit}:
\begin{align}\nonumber
 I(\vec{x}, \vec{\Omega}, 0) - \phi(\vec{x}, 0)
 &= I^i(\vec{x}, \vec{\Omega}, t) - \frac1{4\pi} \phi^i(\vec{x})
 \\ \label{eq:capPsiInit}
 \Psi(\vec{x}, \vec{\Omega}, 0)
 &= I^i(\vec{x}, \vec{\Omega}, t) - \frac1{4\pi} \phi^i(\vec{x})
 \equiv \Psi^i(\vec{x}, \vec{\Omega}, t)
 \,,
\end{align}
for $\vec{x} \in V$, $\vec{\Omega} \in 4\pi$.

Equations \eqref{eq:capPsiVol},~\eqref{eq:capPsiBndy},%
~\eqref{eq:capPsiRefl}, and~\eqref{eq:capPsiInit}
comprise a full description of the ``anisotropic'' component of the angular
intensity. Even though they involve unknowns, they are exact.

Section \ref{sec:adDiscDiff} describes how Fick's law can be extracted from them
by using the linearly anisotropic approximation $\Psi\approx \frac{3}{4\pi}
\vec{\Omega}\vd\vec{F}$, but we shall use the anisotropic transport equations to
derive a new anisotropic approximation to $\Psi$ that yields a new expression
analogous to Fick's law.

%%%%%%%%%%%%%%%%%%%%%%%%%%%%%%%%%%%%%%%%%%%%%%%%%%%%%%%%%%%%%%%%%%%%%%%%%%%%%%%%
\section{Anisotropic diffusion}

%%%%%%%%%%%%%%%%%%%%%%%%%%%%%%%%%%%%%%%%%%%%%%%%%%%%%%%%%%%%%%%%%%%%%%%%%%%%%%%%
\subsection{Modification for boundary condition treatment}
In order to formulate transport-matched boundary conditions, we separate $\Psi$
into an internal solution $\tilde\Psi$ and a boundary layer solution
$\Psi_\mathrm{bl}$:
\begin{equation} \label{eq:boundaryLayerPsi}
  \Psi(\vec{x}, \vec{\Omega}, t)
  = \tilde\Psi(\vec{x}, \vec{\Omega}, t)
  + \Psi_\mathrm{bl}(\vec{x}, \vec{\Omega}, t)\,.
\end{equation}

\begin{subequations} \label{eqs:tCapPsi}
  The internal transport equation is just like Eq.~\eqref{eq:capPsiBndy}:
\begin{multline} \label{eq:tCapPsiVol}
  \frac{1}{c} \pder{}{t}\tilde\Psi(\vec{x}, \vec{\Omega}, t)
    + \vec{\Omega}\vd \grad \tilde\Psi(\vec{x}, \vec{\Omega}, t)
    + \sigma^\ast(\vec{x}) \tilde\Psi(\vec{x}, \vec{\Omega}, t)
  \\
  = \frac{1}{4\pi} \grad \vd\vec{F}(\vec{x}, t) -
  \frac{1}{4\pi} \vec{\Omega}\vd \grad \phi(\vec{x}, t)
  \equiv \hat Q(\vec{x}, \vec{\Omega}, t)\,,
  \qquad
x \in V,\  0 \le t < \Delta_t, \ \vec{\Omega} \in 4\pi.
\end{multline}

However, we will define incident boundary conditions for this internal solution
to be
\begin{equation} \label{eq:tCapPsiBndy}
 \tilde\Psi(\vec{x}, \vec{\Omega}, t) 
  = - \zeta(\vec{x}, \vec{\Omega}, t) \vec{\Omega}\vd \grad \phi(\vec{x}, t)
  \equiv \tilde\Psi^b(\vec{x}, \vec{\Omega}, t) 
\end{equation}
for $\vec{x} \in \partial V_b$, $\vec{\Omega} \vd \vec{n} < 0$,
$0 \le t < \Delta_t$. The function $\zeta$, which lives on the boundary, is yet
to be determined. This seemingly odd boundary condition will be justified later.

The reflecting boundary condition is just like Eq.~\eqref{eq:capPsiRefl}:
\begin{equation} \label{eq:tCapPsiRefl}
 \tilde\Psi(\vec{x}, \vec{\Omega}, t) 
  = \tilde\Psi(\vec{x}, \vec{\Omega}_r, t)
  \equiv \tilde\Psi^b(\vec{x}, \vec{\Omega}, t) 
\end{equation}
for $\vec{x} \in \partial V_r$, $\vec{\Omega} \vd \vec{n} < 0$,
$0 \le t < \Delta_t$.

Finally, the internal solution contains the same initial condition as
Eq.~\eqref{eq:capPsiInit}:
\begin{equation} \label{eq:tCapPsiInit}
 \tilde\Psi(\vec{x}, \vec{\Omega}, 0)
 = \Psi^i(\vec{x}, \vec{\Omega}, t)\,.
\end{equation}
\end{subequations}
The corresponding transport equation for $\Psi_\mathrm{bl}$ are defined to
satisfy the transport equations for $\Psi$ using the definition in
Eq.~\eqref{eq:boundaryLayerPsi}.
\begin{subequations} \label{eqs:blCapPsi}
It has the same left-hand side as Eq.~\eqref{eq:capPsiBndy} but no internal
source:
\begin{equation} \label{eq:blCapPsiVol}
  \frac{1}{c} \pder{}{t}\Psi_\mathrm{bl}(\vec{x}, \vec{\Omega}, t)
    + \vec{\Omega}\vd \grad \Psi_\mathrm{bl}(\vec{x}, \vec{\Omega}, t)
    + \sigma^\ast(\vec{x}) \Psi_\mathrm{bl}(\vec{x}, \vec{\Omega}, t)
  = 0\,,
\end{equation}
for $x \in V$, $0 \le t < \Delta_t$, $\vec{\Omega} \in 4\pi$.
The incident boundary condition accounts for the true incident boundary source
as well as the $\zeta$ term we introduced:
\begin{equation} \label{eq:blCapPsiBndy}
 \Psi_\mathrm{bl}(\vec{x}, \vec{\Omega}, t) 
  = I^b(\vec{x}, \vec{\Omega}, t) - \frac{1}{4\pi} \phi(\vec{x}, t)
  + \zeta(\vec{x}, \vec{\Omega}, t) \vec{\Omega}\vd \grad \phi(\vec{x}, t)
  \equiv \Psi_\mathrm{bl}^b(\vec{x}, \vec{\Omega}, t) \,.
\end{equation}
For $\vec{x} \in \partial V_r$, the boundary layer solution is reflecting:
\begin{equation} \label{eq:blCapPsiRefl}
 \Psi_\mathrm{bl}(\vec{x}, \vec{\Omega}, t) 
  = \Psi_\mathrm{bl}(\vec{x}, \vec{\Omega}_r, t)\,.
\end{equation}
Finally, the initial condition is zero:
Eq.~\eqref{eq:capPsiInit}:
\begin{equation} \label{eq:blCapPsiInit}
 \Psi_\mathrm{bl}(\vec{x}, \vec{\Omega}, 0)
 = 0\,.
\end{equation}
\end{subequations}

If we add Eqs.~\eqref{eqs:blCapPsi} to Eqs.~\eqref{eqs:tCapPsi}, we recover the
anisotropic transport equation. However, unlike the original transport equation,
Eqs.~\eqref{eqs:tCapPsi} and~\eqref{eqs:blCapPsi} allow us to formulate boundary
conditions for the anisotropic diffusion equation.

%%%%%%%%%%%%%%%%%%%%%%%%%%%%%%%%%%%%%%%%%%%%%%%%%%%%%%%%%%%%%%%%%%%%%%%%%%%%%%%%
\subsection{Integral transport equation}
The integral transport equation is formulated \cite{Pri2010} by taking the
right-hand side of a transport equation to be a known quantity, then integrating
along the characteristic ray $\vec{\Omega}$, accumulating particles born along
the ray and attenuating by $\sigma$. Instead of considering the integral
transport equation for $I$, we invert Eqs.~\eqref{eqs:tCapPsi}:
\begin{subequations} \label{eqs:inverseTransport}
  \begin{align} \label{eq:inverseTransportFull}
  \begin{split}
    \tilde\Psi(\vec{x}, \vec{\Omega}, t)
    &=
    \tilde\Psi^b(\vec{x} - s_b\vec{\Omega}, \vec{\Omega}, t - s_b/c)
    \eexp^{ -\tau(\vec{x}, \vec{x} - s_b \vec{\Omega})}
    U(ct - s_b)
    \\
    &\qquad + \Psi^i( \vec{x} - ct \vec{\Omega}, \vec{\Omega})
    \eexp^{ -\tau(\vec{x}, \vec{x} - ct \vec{\Omega})}
    U( s_b - ct)
    \\
    &\qquad +  \int_{0}^{s_b}
    \left[ \hat Q(\vec{x} - s \vec{\Omega}, \vec{\Omega}, t-s/c)
    \right]
    \eexp^{ -\tau(\vec{x}, \vec{x} - s \vec{\Omega})}
    \ud s
\,.
  \end{split}
  \end{align}
  Here, $U(v)$ is the Heaviside function, unity for $v \ge 0$ and zero
  otherwise. The optical thickness of the medium between points $\vec{x}$ and
  $\vec{x}'$ along direction $\vec{\Omega} = (\vec{x}'-
  \vec{x})/\norm{\vec{x}'-\vec{x}}$ is 
  \begin{equation} \label{eq:fullTauDefinition}
    \tau(\vec{x}, \vec{x}') = \int_{0}^{\norm{\vec{x} -
    \vec{x}'}} \sigma^\ast(\vec{x}-s\vec{\Omega}) \ud s \,.
  \end{equation}
  The quantity $s_b$ is the distance to the boundary along $-\vec{\Omega}$ from
  $\vec{x}$.
\end{subequations}

For brevity, we write Eq.~\eqref{eq:inverseTransportFull} as a sum of linear
operators, each of which corresponds to the local contribution of a nonlocal
particle source:
\begin{align} \nonumber
  \tilde\Psi(\vec{x}, \vec{\Omega}, t)
    &\equiv \lopinv{b}{\tilde\Psi^b}
    + \lopinv{i}{\Psi^i}
    + \lopinv{v}{\hat Q}
    \\ \label{eq:inverseTransportBrief}
  \begin{split}
    \tilde\Psi(\vec{x}, \vec{\Omega}, t)
    &\equiv
    -\lopinv{b}{\zeta \vec{\Omega}\vd \grad \phi}_{\partial V_b}
    + \lopinv{b}{\tilde\Psi(\vec{x}, \vec{\Omega}_r, t)}_{\partial V_r}
    + \lopinv{i}{\Psi^i}
  \\&\qquad
    + \lopinv{v}{\frac{1}{4\pi} \grad \vd\vec{F} }
    - \lopinv{v}{\frac{1}{4\pi} \vec{\Omega}\vd \grad \phi}
    \,.
  \end{split}
\end{align}

%A useful property of $\lopinv{b}{\cdot}$ is that, for
%$\vec{n}\vd\vec{\Omega} < 0$ on an incident boundary, $s_b=0$, so
%\begin{equation}\label{eq:bndyIdentity}
%  \lopinv{b}{\zeta} = \zeta \eexp^0 = \zeta(\vec{x}, t)\,.
%\end{equation}

No approximations or assumptions at all have been made yet. As a
result, the inverse equation~\eqref{eq:inverseTransportFull} still contains
the unknowns $\phi$, and $\vec{F}$ (as well as the exiting anisotropic flux on
any reflecting boundaries), and the local value of
$\tilde\Psi$ depends on the global value of those unknowns.

\emph{
Our goal is to make reasonable approximations to this equation to yield a
low-order approximation to $\vec{F}=\int_{4\pi}\vec{\Omega} \tilde\Psi \ud\Omega$ that
depends only on local unknowns and certain coefficients that can be calculated
without any \emph{a priori} knowledge of the exact solution.
}

%%%%%%%%%%%%%%%%%%%%%%%%%%%%%%%%%%%%%%%%
\subsection{Asymptotic ansatz and expansions}
To simplify Eq.~\eqref{eq:inverseTransportFull}, it is necessary to make some
approximations. We make an ansatz that the spatial gradients of the intensity
are weak, the intensity varies slowly in time, and the solution is mildly
(but not necessarily linearly) anisotropic:
\begin{align} \label{eq:ansatz}
  I &= O(1), &
  \grad I &= O(\epsilon), &
  \frac{1}{c}\pder{}{t} &= O(\epsilon^2), &
  \int_{4\pi} \vec{\Omega} I\ud\Omega &= O(\epsilon).
\end{align}

The contribution from $\grad \vd\vec{F}$ is $O(\epsilon^2)$, as the term
contains an $O(\epsilon)$ derivative as well as the $O(\epsilon)$ radiation
flux.  The assumption about the speed of light being very large means that the
contribution from the initial condition is $O(\epsilon^2)$.

To derive the anisotropic diffusion method, we first discard the $O(\epsilon^2)$
terms that appear in Eq.~\eqref{eq:inverseTransportBrief}:
\begin{equation} \label{eq:approxPsi1}
  \tilde\Psi \approx 
  -\lopinv{b}{\zeta \vec{\Omega}\vd \grad \phi}_{\partial V_b}
  + \lopinv{b}{\tilde\Psi(\vec{\Omega}_r)}_{\partial V_r}
  - \lopinv{v}{\frac{1}{4\pi} \vec{\Omega}\vd \grad \phi}
  + O(\epsilon^2)
\end{equation}

Another useful feature of the ansatz in Eq.~\eqref{eq:ansatz} is that the
nonlocal variables in Eq.~\eqref{eq:inverseTransportFull} can be expanded about
the local spatiotemporal point:
\begin{equation} \label{eq:taylorPhi}
  \phi(\vec{x} - s \vec{\Omega}, t-s/c)
  \sim \phi(\vec{x},t) - s \left( \frac{1}{c} \pder{}{t} + \vec{\Omega} \vd
  \grad  \right) \phi (\vec{x}, t) + O(\epsilon^2) \sim \phi(\vec{x},t) +
  O(\epsilon) \,.
\end{equation}

%%%%%%%%%%%%%%%%%%%%%%%%%%%%%%%%%%%%%%%%
\subsection{Approximating the streaming term}
The streaming term in Eq.~\eqref{eq:approxPsi1} is (see
Eq.~\eqref{eq:inverseTransportFull}):
\begin{align*}
- \lopinv{v}{\frac{1}{4\pi} \vec{\Omega}\vd \grad \phi(\vec{x}, t)}
  &= \int_{0}^{\norm{\vec{x} - \vec{x}_b}}
    \left[ -\frac1{4\pi}\vec{\Omega}\vd \grad \phi(\vec{x} - s \vec{\Omega},
    t-s/c)
    \right]
    \eexp^{ -\tau(\vec{x}, \vec{x} - s \vec{\Omega})}
    \ud s
\end{align*}
This integral describes the contribution from the volumetric source  along
$\vec{\Omega}$, evaluated at a prior
point in time ($t-s/c$, the point along $s$ at which a particle would travel
to $\vec{x}$ at time $t$), attenuated by the medium along the way (the
$\eexp^{ -\tau }$ factor).

We now make our first approximation by expanding the distant $\phi(\vec{x} - s
\vec{\Omega}, t-s/c)$ about the local $\phi(\vec{x}, t)$ using
Eq.~\eqref{eq:taylorPhi}. Thus,
\begin{equation*}
  \grad \phi(\vec{x} - s \vec{\Omega}, t-s/c)
  = \grad \phi(\vec{x}, t) + \grad O(\epsilon)
  = \grad \phi(\vec{x}, t) + O(\epsilon^2).
\end{equation*}

The expansion is a good approximation if $\phi$ is smooth, especially because the
$\eexp^{ -\tau }$ term exponentially attenuates the non-local components of the
Taylor series as $s$ increases, assuming $\sigma\ne 0$ along the ray
$\vec{\Omega}$.

We can now move the unknown $\phi$ outside the integral,
because it is no longer a function of $s$:
\begin{align}\nonumber
- \lopinv{v}{\frac{1}{4\pi} \vec{\Omega}\vd \grad \phi(\vec{x}, t)}
  &\approx \int_{0}^{\norm{\vec{x} - \vec{x}_b}}
    \left[ -\frac1{4\pi}\vec{\Omega}\vd \grad \tilde\phi(\vec{x},t) \right]
    \eexp^{ -\tau(\vec{x}, \vec{x} - s \vec{\Omega})}
    \ud s
  \\\nonumber
  &= - \int_{0}^{\norm{\vec{x} - \vec{x}_b}}
    \left[ \frac1{4\pi}\right]
    \eexp^{ -\tau(\vec{x}, \vec{x} - s \vec{\Omega})} \ud s \,
    \vec{\Omega}\vd \grad \phi(\vec{x},t)
  \\\label{eq:streamingApprox}
  &= - \lopinv{v}{ \frac1{4\pi} } \vec{\Omega}\vd \grad \phi(\vec{x},t)
  \,.
\end{align}

%%%%%%%%%%%%%%%%%%%%%%%%%%%%%%%%%%%%%%%%
\subsection{Approximating the incident boundary term}\label{sec:derBc}
The incident boundary term in Eq.~\eqref{eq:approxPsi1} is
\begin{align*}
-\lopinv{b}{\zeta \vec{\Omega}\vd \grad \phi}_{\partial V_b}
  &= -\left[\zeta(\vec{x} - s_b\vec{\Omega}, \vec{\Omega}, t - s_b/c)
   \vec{\Omega}\vd \grad \phi(\vec{x} - s_b\vec{\Omega}, t - s_b/c) \right]
  \\
   &\qquad\times
    \eexp^{ -\tau(\vec{x}, \vec{x} - s_b \vec{\Omega})}
    U(ct - s_b)
\end{align*}
It accounts for particles that start their life at a specified incident boundary
inside the current time step and stream along $\vec{\Omega}$, attenuated by
$\eexp^{-\tau}$ along their path. As the optical thickness between
$(\vec{x}, \vec{\Omega})$ and the boundary increases, this term vanishes
exponentially.

Now we apply the Taylor series expansion from Eq.~\eqref{eq:taylorPhi} to
$\phi$, but not to $\zeta$:
\begin{equation*}
  \grad \phi(\vec{x} - s_b\vec{\Omega}, t - s_b/c)
  \approx \grad \phi(\vec{x}, t) + O(\epsilon^2)\,,
\end{equation*}
and we discard the $O(\epsilon^2)$ term. Now we have
\begin{align} \nonumber
-\lopinv{b}{\zeta \vec{\Omega}\vd \grad \phi}_{\partial V_b}
&= -\left[\zeta(\vec{x} - s_b\vec{\Omega}, \vec{\Omega}, t - s_b/c) \right]
  \eexp^{ -\tau(\vec{x}, \vec{x} - s_b \vec{\Omega})} U(ct - s_b)
  \vec{\Omega}\vd \grad \phi(\vec{x}, t) 
 \\ \label{eq:bndyApprox}
&= -\lopinv{b}{\zeta}_{\partial V_b} \vec{\Omega}\vd \grad \phi \,.
\end{align}

%%%%%%%%%%%%%%%%%%%%%%%%%%%%%%%%%%%%%%%%
\subsection{Approximating the reflecting boundary term}\label{sec:derReflBc}
For a moment, let us assume that the problem has no reflecting boundaries,
$\partial V_b = \partial V$. At this point, Eq.~\eqref{eq:approxPsi1} has been
reduced to
\begin{align*}
  \tilde\Psi
  &= -\lopinv{b}{\zeta \vec{\Omega}\vd \grad \phi}
    + \lopinv{i}{\Psi^i}
    + \lopinv{v}{\frac{1}{4\pi} \grad \vd\vec{F} }
    - \lopinv{v}{\frac{1}{4\pi} \vec{\Omega}\vd \grad \phi}
\\
  &\approx
  -\lopinv{b}{\zeta} \vec{\Omega}\vd \grad \phi
  - \lopinv{v}{\frac{1}{4\pi}} \vec{\Omega}\vd \grad \phi \,.
\end{align*}

Now the decision to choose the boundary condition in Eq.~\eqref{eq:tCapPsiBndy}
is clear: under the systematic approximations made so far, the internal solution
$\tilde\Psi$ can be written
\begin{equation*}
  \tilde\Psi(\vec{x}, \vec{\Omega}, t)
  = - \left\{ \lopinv{b}{\zeta} + \lopinv{v}{\frac{1}{4\pi}}
  \right\} \vec{\Omega}\vd \grad \phi
  \equiv - f(\vec{x}, \vec{\Omega}) \vec{\Omega}\vd \grad \phi(\vec{x}, t)\,.
\end{equation*}
Let us assume that an approximation to the reflecting boundary condition can
be made that, in the general case with mixed reflecting and incident
boundaries, also allows us to write 
\begin{equation*}
  \tilde\Psi(\vec{x}, \vec{\Omega}, t)
  \approx - f(\vec{x}, \vec{\Omega}) \vec{\Omega}\vd \grad \phi(\vec{x}, t)\,.
\end{equation*}

Substituting this approximation into the internal contribution from a reflecting
boundary yields
\begin{align*}
\lopinv{b}{\tilde\Psi(\vec{\Omega}_r)}_{\partial V_r}
  &= \left[\tilde\Psi(\vec{x} - s_b\vec{\Omega}, \vec{\Omega}_r, t - s_b/c) \right]
    \eexp^{ -\tau(\vec{x}, \vec{x} - s_b \vec{\Omega})}
    U(ct - s_b)
\\
  &= \left[-f(\vec{x} - s_b\vec{\Omega}, \vec{\Omega}_r) \vec{\Omega}_r 
  \vd \grad \phi(\vec{x} - s_b\vec{\Omega}, t - s_b/c) \right]
  \eexp^{ -\tau(\vec{x}, \vec{x} - s_b \vec{\Omega})}
  U(ct - s_b) \,.
\\ 
\intertext{First, we expand $\vec{\Omega}_r$ using Eq.~\eqref{eq:reflection}.
}
\lopinv{b}{\tilde\Psi(\vec{\Omega}_r)}_{\partial V_r}
  &= - \lopinv{b}{f(\vec{x} - s_b\vec{\Omega}, \vec{\Omega}_r)
\left(  \vec{\Omega} - 2(\vec{\Omega} \vd \vec{n}) \vec{n} \right)
  \vd \grad \phi(\vec{x} - s_b\vec{\Omega}, t - s_b/c) }
  \\
  &= -\lopinv{b}{f \vec{\Omega} \vd \grad \phi}
  + \lopinv{b}{2(\vec{\Omega} \vd \vec{n}) \vec{n} \vd \grad \phi } \,.
\\ 
\intertext{On a reflecting boundary, the exact intensity satisfies $\vec{n} \vd
\grad I = 0$, which also means $\vec{n} \vd \grad \phi=0$. Thus, the second
term is zero.
}
\lopinv{b}{\tilde\Psi(\vec{\Omega}_r)}_{\partial V_r}
&= -\lopinv{b}{f \vec{\Omega} \vd \grad \phi} \,.
\\ \intertext{ Now, just like in the incident boundary situation, we apply the
Taylor
series expansion from Eq.~\eqref{eq:taylorPhi} to $\phi$ but not to $f$:
}
\lopinv{b}{\tilde\Psi(\vec{\Omega}_r)}_{\partial V_r}
&\approx
-\left[f(\vec{x} - s_b\vec{\Omega}, \vec{\Omega}_r)
  \right]
  \eexp^{ -\tau(\vec{x}, \vec{x} - s_b \vec{\Omega})}
  U(ct - s_b) \vec{\Omega} \vd \grad \phi(\vec{x}, t)\,,
\end{align*}
or, in the more simplified form,
\begin{equation} \label{eq:reflApprox}
\lopinv{b}{\tilde\Psi(\vec{\Omega}_r)}_{\partial V_r}
\approx  
- \lopinv{b}{f(\vec{\Omega}_r)}_{\partial V_r}
\vec{\Omega} \vd \grad \phi(\vec{x}, t) \,.
\end{equation}

This has the same form as the other approximations to the term. This is crucial
to forming the anisotropic diffusion approximation.

%%%%%%%%%%%%%%%%%%%%%%%%%%%%%%%%%%%%%%%%
\subsection{Completed approximation to the anisotropic intensity}
Substituting Eqs.~\eqref{eq:streamingApprox},~\eqref{eq:bndyApprox},
and~\eqref{eq:reflApprox} into Eq.~\eqref{eq:approxPsi1} gives a nearly complete
approximation to the anisotropic intensity:
\begin{align} \nonumber
  \tilde\Psi
  &\approx 
- \lopinv{b}{\zeta}_{\partial V_b} \vec{\Omega}\vd \grad \phi
- \lopinv{b}{f(\vec{\Omega}_r)}_{\partial V_r}
  \vec{\Omega}\vd \grad \phi(\vec{x}, t)
- \lopinv{v}{\frac{1}{4\pi}}  \vec{\Omega}\vd \grad \phi
\\ \label{eq:approxPsi2}
  \tilde\Psi &= 
- \left\{ \lopinv{b}{\zeta}_{\partial V_b} 
+ \lopinv{b}{f(\vec{\Omega}_r)}_{\partial V_r}
+ \lopinv{v}{\frac{1}{4\pi}} \right\} \vec{\Omega}\vd \grad \phi
\\ \label{eq:approxPsi3}
\tilde\Psi(\vec{x}, \vec{\Omega}, t) &= - f(\vec{x}, \vec{\Omega})
\vec{\Omega}\vd \grad \phi(\vec{x}, t)\,.
\end{align}

The exciting part of this representation is in the interpretation of
\begin{equation*}
  f(\vec{x}, \vec{\Omega})
  \equiv \lopinv{b}{\zeta}_{\partial V_b} 
+ \lopinv{b}{f(\vec{\Omega}_r)}_{\partial V_r}
+ \lopinv{v}{\frac{1}{4\pi}}\,.
\end{equation*}
Converting this from an integral transport representation back to a differential
transport equation, we see that $f$ is the solution of a purely absorbing
transport equation with a uniform, isotropic source:
\begin{subequations} \label{eqs:fFull}
  \begin{equation} \label{eq:fFullVol}
    \vec{\Omega}\vd \grad f(\vec{x}, \vec{\Omega})
    + \sigma^\ast f (\vec{x}, \vec{\Omega})
  = \frac{1}{4\pi} \,, \quad x \in V,\ \vec{\Omega} \in 4\pi\,,
  \end{equation}
  with to-be-determined boundary conditions where $I$ is specified for incident
  directions,
\begin{equation} \label{eq:fFullBndy}
  f(\vec{x}, \vec{\Omega}) = \zeta(\vec{x}, \vec{\Omega}) \,,
 \quad \vec{x} \in \partial V_b, \ \vec{\Omega} \vd \vec{n} < 0\,.
\end{equation}
  and with reflecting boundary conditions where $I$ is reflecting,
\begin{equation} \label{eq:fFullRefl}
  f(\vec{x}, \vec{\Omega}) = f(\vec{x}, \vec{\Omega}_r) \,,
 \quad \vec{x} \in \partial V_r, \ \vec{\Omega} \vd \vec{n} < 0\,.
\end{equation}
\end{subequations}

We have applied the approximation that $\frac{1}{c}\pder{}{t}= O(\epsilon^2)$
to turn the transport equation for $f$ into a steady-state equation, and we have
accordingly restricted $\zeta$ to a function constant within the time step.

%%%%%%%%%%%%%%%%%%%%%%%%%%%%%%%%%%%%%%%%
\subsection{Approximate radiation flux}
Now we have an equation for the local angle-dependent anisotropic intensity as a
function of this simple transport equation $f$ and the scalar intensity $\phi$.
We desire a simple low-order equation that provides a closure for the unknown
radiation flux $\vec{F}$ in the radiation conservation
equation~\eqref{eq:loVol}.

To get such a closure, we recall the property from Eq.~\eqref{eq:capPsiFirst}
that the first moment of the anisotropic intensity is the radiation flux. We
therefore apply this identity to our approximate anisotropic intensity from
Eq.~\eqref{eq:approxPsi3}:
\begin{align} \nonumber
  \vec{F}(\vec{x}, t)
  &= \int_{4\pi} \vec{\Omega} \tilde \Psi(\vec{x}, \vec{\Omega}, t) \ud\Omega
  \\ \nonumber
  &= 
  - \left[ \int_{4\pi} \vec{\Omega} \vec{\Omega} f(\vec{x}, \vec{\Omega})
  \ud\Omega \right]
  \vd \grad \phi(\vec{x},t)
  \\ \label{eq:anisotropicFicks}
  &= - \Dtens(\vec{x}) \vd \grad \phi(\vec{x},t) \,.
\end{align}
This resembles ``Fick's law,'' but instead of a scalar diffusion coefficient,
the anisotropic diffusion method has a diffusion \emph{tensor}, $\Dtens$, the
second angular moment of $f$. Just as with Fick's law for diffusion, it is
substituted into the time-dependent conservation equation to provide a simple
approximation for the scalar intensity:
\begin{equation*}
  -\grad \vd \Dtens \phi + \sigma^\ast \phi = \sigma^\ast ac T^4 + q_{r} \,.
\end{equation*}

We have yet to make use of the boundary layer equations~\eqref{eqs:blCapPsi} or
to determine the nature of $\zeta$ as a boundary condition for the calculation
of $f$.

%%%%%%%%%%%%%%%%%%%%%%%%%%%%%%%%%%%%%%%%%%%%%%%%%%%%%%%%%%%%%%%%%%%%%%%%%%%%%%%%
\section{Boundary conditions}

In this section, we address the connected issues of appropriate boundary
conditions for the low-order anisotropic diffusion equations and the boundary
condition $\zeta$ used in the transport calculation.

%%%%%%%%%%%%%%%%%%%%%%%%%%%%%%%%%%%%%%%%
\subsection{Incident boundary conditions}
In this section, a standard boundary layer analysis uses
equations~\eqref{eqs:blCapPsi} to determine a boundary condition for specified
incident boundaries for the low-order equations AD method.
{\small [citation needed]}

A boundary layer analysis 
{\small [citation needed]}
shows that the transport boundary layer, the solution of
Eqs.~\eqref{eqs:blCapPsi}, decays most rapidly if the solution of the
approximate method satisfies the boundary condition
\begin{equation} \label{eq:bcW}
  0 = \int_{\vec{\Omega} \vd \vec{n} < 0} W(\abs{\vec{\Omega} \vd \vec{n}})
  \Psi_\mathrm{bl} (\vec{x}, \vec{\Omega}, t) \ud \Omega\,,\qquad \vec{x} \in
  \partial V_b\,.
\end{equation}
Here, $W$ is related to Chandrasekhar's $H$-function \cite{Cha1960}:
\begin{equation} \label{eq:chandraW}
  W(\mu) = \frac{\sqrt{3}}{2} \mu H(\mu)
  \approx \mu + \tfrac{3}{2} \mu^2
\end{equation}
To recover the Marshak boundary condition, we could use $W(\mu) \approx 2 \mu$.

Substituting Eq.~\eqref{eq:blCapPsiBndy} into Eq.~\eqref{eq:bcW} gives the
low-order boundary condition for anisotropic diffusion:
\begin{equation} \label{eq:bcInc1}
  2\int_{\vec{\Omega}\vd \vec{n} < 0}
  W(\abs{\vec{\Omega} \vd \vec{n}}) I^b(\vec{x}, \vec{\Omega}, t) \ud\Omega
  = \phi(\vec{x}, t)
  - 2\int_{\vec{\Omega}\vd \vec{n} < 0} W(\abs{\vec{\Omega} \vd \vec{n}})
  \zeta(\vec{x}, \vec{\Omega}) \vec{\Omega} \ud\Omega
  \vd \grad \phi(\vec{x}, t) \,.
\end{equation}

\subsubsection{Determining $\zeta$}
The unknown function $\zeta(\vec{x}, \vec{\Omega})$ that lives on the boundary
is a degree of freedom introduced at the beginning of the anisotropic
diffusion analysis. It allowed us to formulate a specified boundary condition
such that the effect of $\zeta$ could be embedded in the anisotropic diffusion
tensor $\Dtens$.

To make use of this degree of freedom, we decide to enforce on the boundary the
truth from Eq.~\eqref{eq:capPsiZeroth},
\begin{equation*}
  \int_{4\pi} \Psi(\vec{x}, \vec{\Omega}, t) \ud\Omega
  = 0 \,.
\end{equation*}
Note that our approximate $\tilde\Psi$ defined in Eq.~\eqref{eq:approxPsi3}
does not generally satisfy this identity:
\begin{align*}
  0
&\qeq \int_{4\pi} \tilde\Psi(\vec{x}, \vec{\Omega}, t) \ud\Omega
\\
&\qeq \int_{4\pi} f(\vec{x}, \vec{\Omega}) \vec{\Omega}
\vd \grad \phi(\vec{x}, t)
\ud\Omega
\\
&\qeq \int_{4\pi} \vec{\Omega} f(\vec{x}, \vec{\Omega})\ud\Omega
\vd \grad \phi(\vec{x}, t) \,.
\end{align*}
This identity holds if $f$ is an even function of $\vec{\Omega}$.
%\footnote{The identity $\int_{4\pi} \vec{\Omega} f\ud\Omega \vd \grad \phi=0$
%is also true if $\grad \phi=0$, or if the first angular moment of $f$ is
%orthogonal to $\grad \phi$.} 
One such situation is many mean free paths away from internal material
boundaries, where $f$ is effectively a constant and therefore even.

On exterior source boundaries, because $\zeta$ is defined for incident
directions and $f$ is known for exiting directions, we can choose $\zeta$ such
that $f$ on the boundary is even under certain conditions.

Returning to the description of $f$ on an incident boundary in
Eq.~\eqref{eq:fFullBndy}, we can say that
\begin{align*}
  \int_{4\pi} \vec{\Omega} f(\vec{x}, \vec{\Omega})\ud\Omega
  &= \int_{\vec{\Omega} \vd \vec{n} < 0}
  \vec{\Omega} \zeta(\vec{x}, \vec{\Omega})\ud\Omega
  + \int_{\vec{\Omega} \vd \vec{n} > 0}
  \vec{\Omega} f(\vec{x}, \vec{\Omega})\ud\Omega\,.
\end{align*}
Now we set the left hand side to zero, demanding that
Eq.~\eqref{eq:capPsiZeroth} be satisfied:
\begin{align} \nonumber
  \int_{\vec{\Omega} \vd \vec{n} < 0}
  \vec{\Omega} \zeta(\vec{x}, \vec{\Omega})\ud\Omega
  &= -\int_{\vec{\Omega} \vd \vec{n} > 0}
  \vec{\Omega} f(\vec{x}, \vec{\Omega})\ud\Omega \,.
  \\ 
  \intertext{Making the substitution $\vec{\Omega}\to -\vec{\Omega}$ on the
  right hand side yields}
  \label{eq:zetaCondition}
  \int_{\vec{\Omega} \vd \vec{n} < 0}
  \vec{\Omega} \zeta(\vec{x}, \vec{\Omega})\ud\Omega
  &= \int_{\vec{\Omega} \vd \vec{n} < 0}
  \vec{\Omega} f(\vec{x}, -\vec{\Omega})\ud\Omega \,.
\end{align}

If $f(\vec{\Omega})$ is azimuthally symmetric about $\vec{n}$, then $f$ is only
a function of the cosine angle between $\vec{\Omega}$ and $\vec{n}$:
\begin{equation*}
f(\vec{\Omega}) = \hat f( \vec{\Omega} \vd \vec{n})\,.
\end{equation*}
Now recall the definition of a reflecting boundary from
Eq.~\eqref{eq:reflection},
\begin{equation*}
  \vec{\Omega}_r = \vec{\Omega} - 2(\vec{\Omega} \vd \vec{n}) \vec{n}\,.
\end{equation*}
Dotting the reflected vector with the normal vector $\vec{n}$,
\begin{equation*}
  \vec{\Omega}_r \vd \vec{n}
  = \vec{\Omega} \vd \vec{n} - 2(\vec{\Omega} \vd \vec{n}) \vec{n}\vd \vec{n}
  = - \vec{\Omega} \vd \vec{n}\,.
\end{equation*}
Thus,
\begin{equation*}
  \hat f( \vec{\Omega}_r \vd \vec{n}) = \hat f( -\vec{\Omega} \vd \vec{n})
\end{equation*}
and
\begin{equation}\label{eq:aziSymResult}
  f( \vec{\Omega}_r) = f( -\vec{\Omega} )\,.
\end{equation}

Therefore, if $f$ is azimuthally symmetric for outgoing directions on the
boundary, Eq.~\eqref{eq:zetaCondition} can be written
\begin{equation*}
  \int_{\vec{\Omega} \vd \vec{n} < 0}
  \vec{\Omega} \zeta(\vec{x}, \vec{\Omega})\ud\Omega
  = \int_{\vec{\Omega} \vd \vec{n} < 0}
  \vec{\Omega} f(\vec{x}, \vec{\Omega}_r)\ud\Omega \,,
\end{equation*}
which is satisfied by
\begin{equation} \label{eq:zeta}
  \zeta(\vec{x}, \vec{\Omega}) = f(\vec{x}, \vec{\Omega}_r) \,,
 \quad \vec{x} \in \partial V_b, \ \vec{\Omega} \vd \vec{n} < 0 \,.
\end{equation}
Now the boundary condition for $f$ in Eq.~\eqref{eq:fFullBndy} becomes
\begin{equation} \label{eq:fFullBndy2}
  f(\vec{x}, \vec{\Omega}) = f(\vec{x}, \vec{\Omega}_r) \,,
 \quad \vec{x} \in \partial V_b, \ \vec{\Omega} \vd \vec{n} < 0\,.
\end{equation}
This says that under the approximations, assumptions, and restrictions we made,
the transport equation for $f$ has reflecting boundaries everywhere, even
where the physical problem does \emph{not} have reflecting boundaries.

%%%%%%%%%%%%%%%%%%%%%%%%%%%%%%%%%%%%%%%%
\subsubsection{Low-order boundary conditions}
With a definition for $\zeta$ in hand, we return to Eq.~\eqref{eq:bcInc1} and
substitute Eq.~\eqref{eq:zeta}:
\begin{equation*}
  2\int_{\vec{\Omega}\vd \vec{n} < 0}
  W(\abs{\vec{\Omega} \vd \vec{n}}) I^b(\vec{\Omega}) \ud\Omega
  = \phi
  - 2\int_{\vec{\Omega}\vd \vec{n} < 0} W(\abs{\vec{\Omega} \vd \vec{n}})
  \vec{\Omega} f(\vec{\Omega}_r) \ud\Omega
  \vd \grad \phi \,.
\end{equation*}
(The space and time parameters have been omitted for brevity.)
We modify the right hand side slightly by making the substitution
$\vec{\Omega}\to-\vec{\Omega}$:
\begin{align*}
  - 2\int_{\vec{\Omega}\vd \vec{n} < 0} W(\abs{\vec{\Omega} \vd \vec{n}})
  \vec{\Omega} f(\vec{\Omega}_r) \ud\Omega
  &= 
  - 2\int_{\vec{\Omega}\vd \vec{n} > 0} W(\abs{-\vec{\Omega} \vd \vec{n}})
  ( - \vec{\Omega}) f(-\vec{\Omega}_r) \ud\Omega
  \\
  &= 
  2\int_{\vec{\Omega}\vd \vec{n} > 0} W(\vec{\Omega} \vd \vec{n})
  \vec{\Omega} f(-\vec{\Omega}_r) \ud\Omega
  \\ 
  \intertext{and using the boundary condition on $f$ from
  Eq.~\eqref{eq:fFullBndy2} in conjunction with Eq.~\eqref{eq:aziSymResult},
  which gives $f(\vec{\Omega}_r) = f(-\vec{\Omega})$:}
  &= 
  2\int_{\vec{\Omega}\vd \vec{n} > 0} W(\vec{\Omega} \vd \vec{n})
  \vec{\Omega} f(\vec{\Omega}) \ud\Omega \,.
\end{align*}
The low-order, transport-consistent boundary condition for an incident source
is therefore
\begin{equation}\label{eq:loBndy}
  2\int_{\vec{\Omega}\vd \vec{n} < 0}
  W(\abs{\vec{\Omega} \vd \vec{n}}) I^b(\vec{x}, \vec{\Omega}, t) \ud\Omega
  = \phi(\vec{x}, t)
  + 2\int_{\vec{\Omega}\vd \vec{n} > 0} W(\vec{\Omega} \vd \vec{n})
  \vec{\Omega} f(\vec{x}, \vec{\Omega}) \ud\Omega
  \vd \grad \phi(\vec{x}, t) \,.
\end{equation}

This form has a particular advantage if we use the Marshak-like approximation
that $W(\mu)\approx 2\mu$. Equation~\eqref{eq:loBndy} becomes
\begin{align}\nonumber
  2\int_{\vec{\Omega}\vd \vec{n} < 0}
  [2\abs{\vec{\Omega} \vd \vec{n}}] I^b(\vec{\Omega}) \ud\Omega
  &= \phi
  + 2\int_{\vec{\Omega}\vd \vec{n} > 0} [2\vec{\Omega} \vd \vec{n}]
  \vec{\Omega} f(\vec{\Omega}) \ud\Omega \vd \grad \phi
  \\ \nonumber
  4 \vec{F}^-
  &= \phi
  + 4 \vec{n} \vd \left[ \int_{\vec{\Omega}\vd \vec{n} > 0} \vec{\Omega}
  \vec{\Omega} f(\vec{\Omega}) \ud\Omega \right] \vd \grad \phi \,.
  \\ 
  \intertext{We chose the boundary condition on $f$ to ensure that it is an
  even function of $\vec{\Omega}$ on the boundary. Therefore, the integrand on
  the right hand side is also an even function of $\vec{\Omega}$, so}
  \nonumber
  4 \vec{F}^-
  &= \phi
  + 4 \vec{n} \vd  \left[ \frac{1}{2} \int_{4\pi}
  \vec{\Omega} \vec{\Omega} f(\vec{\Omega}) \ud\Omega \right] \vd \grad \phi \,.
  \\ 
  \intertext{The integral on the right hand side is the same as in
  Eq.~\eqref{eq:anisotropicFicks}, which defined the anisotropic diffusion
  tensor. Our Marshak-like boundary approximation is}
  \label{eq:marshakAd}
  4 \vec{F}^-(\vec{x}, t)
  &= \phi(\vec{x}, t)
  + 2 \vec{n} \vd \Dtens(\vec{x}) \vd \grad \phi(\vec{x}, t) \,.
\end{align}
This is entirely analogous to the standard diffusion Marshak boundary condition,
\begin{equation*}
  4 \vec{F}^- = \phi + 2 \vec{n} \vd D \grad \phi\,.
\end{equation*}

%%%%%%%%%%%%%%%%%%%%%%%%%%%%%%%%%%%%%%%%
\subsection{Reflecting boundary conditions}

%%%%%%%%%%%%%%%%%%%%%%%%%%%%%%%%%%%%%%%%%%%%%%%%%%%%%%%%%%%%%%%%%%%%%%%%%%%%%%%%
\section{Discussion}

%%%%%%%%%%%%%%%%%%%%%%%%%%%%%%%%%%%%%%%%
\subsection{Transport calculation for $f$}

%%%%%%%%%%%%%%%%%%%%%%%%%%%%%%%%%%%%%%%%
\subsection{Properties of the anisotropic diffusion method}

%%%%%%%%%%%%%%%%%%%%%%%%%%%%%%%%%%%%%%%%
\subsection{Relating the anisotropic intensity equation to
diffusion}\label{sec:adDiscDiff}
Let us return to Eq.~\eqref{eq:capPsiVol}. Instead of using the integral
transport equation and the rest, let us choose to approximate
\begin{equation*}
  \Psi(\vec{x}, \vec{\Omega}, t) \approx \frac{3}{4\pi} \vec{\Omega} \vd
  \vec{F}(\vec{x}, t)\,,
\end{equation*}
which corresponds to the \Pone\ approximation $I= \frac{1}{4\pi} (\phi +
3\vec{\Omega} \vd \vec{F})$. It also satisfies the identities given in
Eqs.~\eqref{eqs:capPsiIdentities}.

For the interior, Eq.~\eqref{eq:capPsiVol} gives
\begin{align*}
  \frac{1}{c} \pder{}{t} \left[ \frac{3}{4\pi} \vec{\Omega} \vd \vec{F} \right]
  + \vec{\Omega}\vd \grad \left[ \frac{3}{4\pi} \vec{\Omega} \vd \vec{F} \right]
  + \sigma^\ast \left[ \frac{3}{4\pi} \vec{\Omega} \vd \vec{F} \right]
  &= \frac{1}{4\pi} \grad \vd\vec{F}
  - \frac{1}{4\pi} \vec{\Omega}\vd \grad \phi\,.
\end{align*}
Taking the first moment of this equation, operating on it with $\int_{4\pi}
\vec{\Omega}(\cdot) \ud\Omega$, yields
\begin{multline*}
\frac{3}{4\pi} \left( \int_{4\pi} \vec{\Omega}\vec{\Omega}\ud\Omega \right) \vd
\frac{1}{c} \pder{}{t} \vec{F}
+ \frac{3}{4\pi} \left( \int_{4\pi}
  \vec{\Omega}\vec{\Omega}\vec{\Omega}\ud\Omega \right)
\vd \grad \vd \vec{F}
+ \frac{3}{4\pi} \left( \int_{4\pi} \vec{\Omega}\vec{\Omega}\ud\Omega \right)
\vd \sigma^\ast \vec{F}
\\
= \frac{1}{4\pi} \left( \int_{4\pi} \vec{\Omega}\ud\Omega \right)
\grad \vd\vec{F}
- \frac{1}{4\pi} \left( \int_{4\pi} \vec{\Omega}\vec{\Omega}\ud\Omega \right)\vd \grad \phi\,.
\end{multline*}
Now, basic vector identities \cite{Lar2007} reduce the parenthesized
quantities to very manageable expressions: $\int_{4\pi}
\vec{\Omega}\vec{\Omega}\ud\Omega=\frac{4\pi}{3}\Identitytens$, and the odd
multiples of $\vec{\Omega}$ integrated over the unit sphere are zero. Thus,
\begin{equation*}
  \frac{1}{c} \pder{}{t} \vec{F}
  + \sigma^\ast \vec{F}
  =
  - \frac{1}{3} \grad \phi\,.
\end{equation*}
This is the standard \Pone{} equation, although it was formulated in an
admittedly very odd way.

Now if we neglect the time derivative using the quasi-static approximation
\cite{Dud1976}, we recover Fick's law,
\begin{equation*}
\vec{F} = - \frac{1}{3\sigma^\ast} \grad \phi\,.
\end{equation*}

