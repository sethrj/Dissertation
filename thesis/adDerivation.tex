% !TEX root = _individual/adDerivation.tex

\newcommand{\epsiloncolor}[1]{#1}
%%%%%%%%%%%%%%%%%%%%%%%%%%%%%%%%%%%%%%%%%%%%%%%%%%%%%%%%%%%%%%%%%%%%%%%%%%%%%%%%
\chapter{Anisotropic diffusion theory}

The previous work in anisotropic diffusion has only considered a steady-state
problem in an infinite medium \cite{Lar2009c,Mor2007}. The new understanding of
the AD method presented in this chapter provides a theoretical basis for
using the AD method in time-dependent and nonlinear contexts, and it also
addresses the heretofore unsolved problem of how to treat the boundary
conditions.

A summary of the derivation is presented up front:
\begin{enumerate}
  \item Consider the time-dependent nonlinear radiation transport equation
    over one time step, with the opacities frozen at some particular value.
  \item Define the anisotropic intensity as $\Psi = I - \frac{1}{4\pi}\phi$.
    From the radiation transport equation and conservation equation, we get a
    transport equation for $\Psi$ with appropriate initial and boundary
    conditions.
  \item Transform this to an integral transport equation for $\Psi$, and move
    the term involving $\phi$ to the right-hand side to get an exact
    integral expression for $I$.
  \item Use Taylor series to approximate nonlocal unknowns with local
    unknowns.
  \item Require that the contribution of boundary terms throughout the problem
    space is zero: this gives boundary conditions for the new approximate
    method.
  \item Take the first angular moment of this new approximate angular
    intensity to get an approximate radiation flux.
  \item Substitute into time-dependent particle conservation equation to get
    time-dependent anisotropic diffusion.
\end{enumerate}

%%%%%%%%%%%%%%%%%%%%%%%%%%%%%%%%%%%%%%%%%%%%%%%%%%%%%%%%%%%%%%%%%%%%%%%%%%%%%%%%
\section{Derivation}
The gray radiative transfer equation inside a time step $0 < t < \Delta_t$,
with opacities frozen at some value $\sigma^\ast$ (which could be
$\sigma(\vec{x},0)$ for a semi-implicit formulation), is:
\begin{subequations} \label{eqs:fullTransport}
\begin{multline} \label{eq:fullTransportVol}
  \frac{1}{c} \pder{I}{t}(\vec{x}, \vec{\Omega}, t)
    + \vec{\Omega}\vd \grad I(\vec{x}, \vec{\Omega}, t)
    + \sigma^\ast(\vec{x}) I (\vec{x}, \vec{\Omega}, t)
    \\ = \frac{1}{4\pi} \sigma^\ast(\vec{x}) ac [T(\vec{x}, t)]^4
    + \frac{1}{4\pi} q_{r}(\vec{x}, t)
    \equiv \frac{1}{4\pi} Q(\vec{x}, t)
    \equiv \hat Q(\vec{x}, \vec{\Omega}, t) \,,
\\
x \in V,\  0 \le t < \Delta_t, \ \vec{\Omega} \in 4\pi,
\end{multline}
with the boundary condition
\begin{equation} \label{eq:fullTransportBndy}
  I(\vec{x}, \vec{\Omega}, t) = I^b(\vec{x}, \vec{\Omega}, t) \,,
 \quad \vec{x} \in \partial V, \ \vec{\Omega} \vd \vec{n} < 0,\ 0 \le t < \Delta_t
\end{equation}
and the initial condition
\begin{equation} \label{eq:fullTransportInit}
 I(\vec{x}, \vec{\Omega}, 0) = I^i(\vec{x}, \vec{\Omega}, t) \,,
 \quad \vec{x} \in V, \ \vec{\Omega} \in 4\pi\,.
\end{equation}
(The initial condition here is usually the solution from the previous time
step.)
\end{subequations}

We can express the left-hand side of Eq.~\eqref{eq:fullTransportVol} as a
transport operator $\lop{\cdot}$ that satisfies
\begin{equation*}
  \lop{I(\vec{x}, \vec{\Omega}, t)} = \hat Q(\vec{x}, \vec{\Omega}, t)
  + \text{boundary conditions} + \text{initial conditions}\,.
\end{equation*}

The particle conservation equation (a scaled form of radiation energy
conservation) is the zeroth moment of the radiation transport equation.
\begin{subequations} \label{eqs:loEquations}
Operating on Eq.~\eqref{eq:fullTransportVol} by $\int_{4\pi} (\cdot) \ud
\Omega$ gives
\begin{equation} \label{eq:loVol}
\frac{1}{c} \pder{\phi}{t} (\vec{x}, t)
  + \grad \vd\vec{F}(\vec{x}, t)
  + \sigma^\ast \phi(\vec{x}, t)
  =  Q(\vec{x}, t)\,.
\end{equation}
Doing the same to the initial condition, Eq.~\eqref{eq:fullTransportInit},
\begin{equation} \label{eq:loInit}
\phi(\vec{x}, 0) = \int_{4\pi}  I^i(\vec{x},
\vec{\Omega}) \ud \Omega = \phi^i(\vec{x}) \,.
\end{equation}
\end{subequations}
Here we have used the scalar intensity $\phi=\int_{4\pi} I \ud \Omega$ and the
radiation flux $\vec{F} = \int_{4\pi} \vec{\Omega} I \ud \Omega$. The boundary
conditions shall be addressed later.

\subsection{Integral transport equation}
Because the opacities are frozen (i.e., independent of time inside $\Delta_t$),
and assuming the arbitrary angle- and space-dependent $\hat Q(\vec{x}, t)$ is
known inside the time step,
the transport operator can be
inverted to yield an integral equation for the time-dependent angular
intensity $I$ \cite{Pri2010}. The exact result for the intensity is:
\begin{subequations} \label{eqs:inverseTransport}
  \begin{align} \label{eq:inverseTransportFull}
  \begin{split}
    I(\vec{x}, \vec{\Omega}, t)
    &=
    I^b(\vec{x}_b, \vec{\Omega}, t - \norm{\vec{x} - \vec{x}_b}/c)
    \eexp^{ -\tau(\vec{x}, \vec{x}_b)}
    U(ct - \norm{\vec{x} - \vec{x}_b})
    \\
    &\qquad + I^i( \vec{x} - ct \vec{\Omega}, \vec{\Omega})
    \eexp^{ -\tau(\vec{x}, \vec{x} - ct \vec{\Omega})}
    U( \norm{\vec{x} - \vec{x}_b} - ct)
    \\
    &\qquad +  \int_{0}^{\norm{\vec{x} - \vec{x}_b}}
    \left[ \hat Q(\vec{x} - s \vec{\Omega}, \vec{\Omega}, t-s/v)
    \right]
    \eexp^{ -\tau(\vec{x}, \vec{x} - s \vec{\Omega})}
    \ud s\,.
  \end{split}
    \\ 
    &\equiv \lopinv{b}{I^b}
    + \lopinv{i}{I^{\smash i \vphantom{\mathrm{i}}}}
    + \lopinv{v}{\hat Q} 
  \end{align}
  Here, $U(\zeta)$ is the heaviside function, unity for $\zeta \ge 0$ and zero
  otherwise. The optical thickness of the medium between points $\vec{x}$ and
  $\vec{x}'$ along direction $\vec{\Omega} = (\vec{x}'-
  \vec{x})/\norm{\vec{x}'-\vec{x}}$ is 
  \begin{equation} \label{eq:fullTauDefinition}
    \tau(\vec{x}, \vec{x}') = \int_{0}^{\norm{\vec{x} -
    \vec{x}'}} \sigma^\ast(\vec{x}-s\vec{\Omega}) \ud s \,.
  \end{equation}
  The point $\vec{x}_b$ is defined as the point on the boundary that,
  following direction $\vec{\Omega}$, intersects point $\vec{x}$. In other
  words, $\vec{x}_b = \vec{x} - s \vec{\Omega}$ where $s$ positive and fixed
  for a particular $(\vec{x}, \vec{\Omega})$.
\end{subequations}

Just as $\lop{\cdot}$ represented a differential operator, $\lopinv{}{\cdot}$
represents an integral operator.
Equation~\eqref{eq:lopinv} shows how the inverse operator is decomposed into
contributions from the incident radiation on the boundary
$\lopinv{b}{I^b}$, contributions from the initial condition
$\lopinv{i}{I^i}$, and contributions from the volumetric angle-dependent source
$\lopinv{v}{\hat Q}$. Because this ``solution'' for $I$ depends on
unknowns in $\hat Q$, it cannot be directly solved. The crux of the AD and
\APone\ methods is to systematically approximate the $\mathcal L \inv$ terms,
rather than approximating the transport equation~\eqref{eq:fullTransportVol}
itself.

Note that these operators are linear, so for some constant $c$ and any
arbitrary space- and angle-dependent functions $A$ and $B$, $\lopinv{X}{ cA +
B} = c\lopinv{X}{ A} + \lopinv{X}{ B}$. This also means $\lopinv{X}{0} = 0$.

%%%%%%%%%%%%%%%%%%%%%%%%%%%%%%%%%%%%%%%%%%%%%%%%%%%%%%%%%%%%%%%%%%%%%%%%%%%%%%%%
\subsection{Anisotropic angular intensity}
The next step is to formulate an equation for the ``anisotropic'' intensity,
i.e., the exact angular intensity with the isotropic component removed:
\begin{equation} \label{eq:capPsi}
  \Psi(\vec{x}, \vec{\Omega}, t) \equiv I(\vec{x}, \vec{\Omega}, t)
  - \frac{1}{4\pi} \phi(\vec{x}, t)\,.
\end{equation}
This definition satisfies $\int_{4\pi} \Psi(\vec{x}, \vec{\Omega}, t)=0$.

\subsubsection{Transport equation}
Multiplying the particle conservation equation~\eqref{eq:loVol} by
$\frac{1}{4\pi}$ and subtracting it from the transport
equation~\eqref{eq:fullTransportVol} cancels the isotropic source on the
right-hand side, yielding
\begin{equation*}
  \frac{1}{c} \pder{}{t}\left[ I(\vec{x}, \vec{\Omega}, t)
  - \frac{1}{4\pi} \phi(\vec{x}, t) \right]
    + \vec{\Omega}\vd \grad I(\vec{x}, \vec{\Omega}, t)
    + \sigma^\ast(\vec{x}) \left[ I(\vec{x}, \vec{\Omega}, t)
  - \frac{1}{4\pi} \phi(\vec{x}, t) \right]
  - \frac{1}{4\pi} \grad \vd\vec{F}(\vec{x}, t)
= 0 \,.
\end{equation*}
Subtracting $\vec{\Omega}\vd \grad \phi/4\pi$ from both sides,
\begin{multline*}
  \frac{1}{c} \pder{}{t}\left[ I(\vec{x}, \vec{\Omega}, t)
  - \frac{1}{4\pi} \phi(\vec{x}, t) \right]
    + \vec{\Omega}\vd \grad \left[ I(\vec{x}, \vec{\Omega}, t)
  - \frac{1}{4\pi} \phi(\vec{x}, t) \right]
    + \sigma^\ast(\vec{x}) \left[ I(\vec{x}, \vec{\Omega}, t)
  - \frac{1}{4\pi} \phi(\vec{x}, t) \right]
  \\ = \frac{1}{4\pi} \grad \vd\vec{F}(\vec{x}, t) -
  \frac{1}{4\pi} \vec{\Omega}\vd \grad \phi(\vec{x}, t)\,.
\end{multline*}
Substituting Eq.~\eqref{eq:capPsi}, we have an exact expression for $\Psi$
inside the problem:
\begin{multline} \label{eq:capPsiVol}
  \frac{1}{c} \pder{}{t}\Psi(\vec{x}, \vec{\Omega}, t)
    + \vec{\Omega}\vd \grad \Psi(\vec{x}, \vec{\Omega}, t)
    + \sigma^\ast(\vec{x}) \Psi(\vec{x}, \vec{\Omega}, t)
  \\
  = \frac{1}{4\pi} \grad \vd\vec{F}(\vec{x}, t) -
  \frac{1}{4\pi} \vec{\Omega}\vd \grad \phi(\vec{x}, t)\,,
  \qquad
x \in V,\  0 \le t < \Delta_t, \ \vec{\Omega} \in 4\pi,
\end{multline}
No approximations have been made, but now instead of an isotropic source term
on the right hand side, we have an anisotropic source term that depends on the
unknowns $\phi$ and $\vec{F}$.

\paragraph{Note} At this point, making the approximation $\Psi(\vec{x},
\vec{\Omega}, t) \approx \frac{3}{4\pi} \vec{\Omega}\vd \vec{F}$ and
integrating the equation over $\vec{\Omega} \in 4\pi$ would yield the \Pone{}
approximation. Also, if the source term from the radiation transport equation
had an anisotropic component (e.g., linearly anisotropic scattering), there
would be an extra term on the right hand side of Eq.~\eqref{eq:capPsiVol}.

\subsubsection{Boundary condition}
Next, we subtract $\phi/4\pi$ from the boundary condition,
Eq.~\eqref{eq:fullTransportBndy}:
\begin{align}\nonumber
  I(\vec{x}, \vec{\Omega}, t) - \frac{1}{4\pi} \phi(\vec{x}, t)
  &= I^b(\vec{x}, \vec{\Omega}, t) - \frac{1}{4\pi} \phi(\vec{x}, t)
  \\
  \intertext{Using the definition of $\phi$ as the integral of $I$ and
  splitting that integral into two hemispheres about the unit normal, we get 
  }\nonumber
  I(\vec{x}, \vec{\Omega}, t) - \frac{1}{4\pi} \phi(\vec{x}, t)
  &= I^b(\vec{x}, \vec{\Omega}, t) - \frac{1}{4\pi}
  \left[ 
   \int_{\vec{\Omega}' \vd \vec{n} < 0} I(\vec{x}, \vec{\Omega}', t) \ud \Omega'
 + \int_{\vec{\Omega}' \vd \vec{n} > 0} I(\vec{x}, \vec{\Omega}', t) \ud \Omega'
  \right]\,.
  \\ 
  \intertext{Substituting Eq.~\eqref{eq:capPsi} into the left hand side and the
  boundary condition Eq.~\eqref{eq:fullTransportBndy} into the right hand side,
  we get a boundary condition on $\Psi$:
  } \label{eq:capPsiBndy}
 \Psi(\vec{x}, \vec{\Omega}, t) 
  &= I^b(\vec{x}, \vec{\Omega}, t) - \frac{1}{4\pi}
 \int_{\vec{\Omega}' \vd \vec{n} < 0} I^b(\vec{x}, \vec{\Omega}', t) \ud \Omega'
 - \frac{1}{4\pi}
 \int_{\vec{\Omega}' \vd \vec{n} > 0} I(\vec{x}, \vec{\Omega}', t) \ud \Omega'
 \,,
\end{align}
for $\vec{x} \in \partial V$, $\vec{\Omega} \vd \vec{n} < 0$, $0 \le t <
\Delta_t$.

\subsubsection{Initial condition}
Finally, to get an initial condition for the anisotropic intensity, we
multiply Eq.~\eqref{eq:loInit} by $1/4\pi$ and subtract it from
Eq.~\eqref{eq:fullTransportInit}:
\begin{align}\nonumber
 I(\vec{x}, \vec{\Omega}, 0) - \phi(\vec{x}, 0)
 &= I^i(\vec{x}, \vec{\Omega}, t) - \frac1{4\pi} \phi^i(\vec{x})
 \\ \label{eq:capPsiInit}
 \Psi(\vec{x}, \vec{\Omega}, 0)
 &= I^i(\vec{x}, \vec{\Omega}, t) - \frac1{4\pi} \phi^i(\vec{x})\,,
\end{align}
for $\vec{x} \in V$, $\vec{\Omega} \in 4\pi$.

Equations \eqref{eq:capPsiVol},~\eqref{eq:capPsiBndy}, and~\eqref{eq:capPsiInit}
comprise a full description of the ``anisotropic'' component of the angular
intensity. Even though they involve unknowns, they are exact.

%%%%%%%%%%%%%%%%%%%%%%%%%%%%%%%%%%%%%%%%%%%%%%%%%%%%%%%%%%%%%%%%%%%%%%%%%%%%%%%%
\subsection{Inverse anisotropic angular intensity}

Applying the inverse transport Eq.~\eqref{eq:inverseTransportFull} to $\Psi$
instead of $I$,
\begin{align*}\nonumber
 \begin{split}
 \Psi &=
 \lopinv{b}{
  I^b
- \frac{1}{4\pi} \int_{\vec{\Omega}' \vd \vec{n} < 0} I^b \ud \Omega'
- \frac{1}{4\pi} \int_{\vec{\Omega}' \vd \vec{n} > 0} I \ud \Omega' }
\\&\qquad
+ \lopinv{i}{
  I^i(\vec{x}, \vec{\Omega}, t)
- \frac{1}{4\pi} \phi^i(\vec{x}) }
+ \lopinv{v}{
  \frac{1}{4\pi} \grad \vd\vec{F}
- \frac{1}{4\pi} \vec{\Omega}\vd \grad \phi} 
 \end{split}
 \\ 
 \intertext{Substituting the definition of Eq.~\eqref{eq:capPsi} and using the
 linear property of $\lopinv{v}{\cdot}$,}
 \begin{split}
 I - \frac{1}{4\pi} \phi &=
 \lopinv{b}{
  I^b
- \frac{1}{4\pi} \int_{\vec{\Omega}' \vd \vec{n} < 0} I^b \ud \Omega'
- \frac{1}{4\pi} \int_{\vec{\Omega}' \vd \vec{n} > 0} I \ud \Omega' }
\\&\qquad
+ \lopinv{i}{
  I^i(\vec{x}, \vec{\Omega}, t)
- \frac{1}{4\pi} \phi^i(\vec{x}) }
+ \lopinv{v}{ \frac{1}{4\pi} \grad \vd\vec{F} }
- \lopinv{v}{ \frac{1}{4\pi} \vec{\Omega}\vd \grad \phi } 
 \end{split}
\end{align*}

Moving $\phi$ to the right-hand side gives an exact expression for the angular
flux that depends on 
\begin{multline}\label{eq:inverseAnisotropicExact}
  I(\vec{x}, \vec{\Omega}, t)  =
 \frac{1}{4\pi} \phi(\vec{x}, t)
+ \lopinv{b}{
  I^b
- \frac{1}{4\pi} \int_{\vec{\Omega}' \vd \vec{n} < 0} I^b \ud \Omega'
- \frac{1}{4\pi} \int_{\vec{\Omega}' \vd \vec{n} > 0} I \ud \Omega' }
\\
+ \lopinv{i}{
  I^i(\vec{x}, \vec{\Omega}, t)
- \frac{1}{4\pi} \phi^i(\vec{x}) }
+ \lopinv{v}{ \frac{1}{4\pi} \grad \vd\vec{F} }
- \lopinv{v}{ \frac{1}{4\pi} \vec{\Omega}\vd \grad \phi } \,.
\end{multline}

Again, we emphasize that no approximations or assumptions at all have been
made. As a
result, the inverse equation~\eqref{eq:inverseAnisotropicExact} still contains
the unknowns $I$, $\phi$, and $\vec{F}$, and the local value of
$I$ depends on the global value of all the unknowns.

\emph{
Our goal is to make reasonable approximations to this equation to yield a
low-order approximation to $\vec{F}=\int_{4\pi}\vec{\Omega} I \ud \Omega$ that
depends only on local unknowns and certain coefficients that can be calculated
without any \emph{a priori} knowledge of the exact solution.
}

Now, we approximate Eq.~\eqref{eq:inverseAnisotropicExact} by
separately considering each component of the intensity:
\begin{equation}\label{eq:approxIntensity1}
  \tilde I(\vec{x}, \vec{\Omega}, t)
  = \frac1{4\pi} \tilde \phi(\vec{x}, t) 
  + \mathrm{ \tilde b }(\vec{x}, \vec{\Omega}, t)
  + \mathrm{ \tilde i }(\vec{x}, \vec{\Omega}, t)
  + \mathrm{ \tilde f }(\vec{x}, \vec{\Omega}, t)
  + \mathrm{ \tilde s }(\vec{x}, \vec{\Omega}, t) \,,
\end{equation}
where:
\begin{itemize}
  \item $\tilde I$ is our approximate representation of the intensity $I$,
  \item $\tilde \phi$ is our approximate representation of the scalar intensity
    $\phi$,
  \item $\mathrm{ \tilde b }$ is an approximation to the boundary conditions
    $\lopinv{b}{\cdots}$,
  \item $\mathrm{ \tilde i }$ is an approximation to the initial condition 
    $\lopinv{i}{\cdots}$,
  \item $\mathrm{ \tilde f }$ is an approximation to the leakage term
    $\lopinv{v}{ \frac1{4\pi} \grad \vd \vec{F} }$, and
  \item $\mathrm{ \tilde s }$ is an approximation to the streaming term
    $\lopinv{v}{ -\frac1{4\pi} \vec{\Omega}\vd \grad \phi }$.
\end{itemize}

At present, it's not clear if we can approximate the ``leakage'' term:
\begin{align*}
  \mathrm{ \tilde f }(\vec{x}, \vec{\Omega}, t) &=
   \lopinv{v}{ \frac1{4\pi}\grad \vd \vec{F} }
  \\
  &= \int_{0}^{\norm{\vec{x} - \vec{x}_b}}
    \left[ \frac1{4\pi}\grad \vd \vec{F}(\vec{x} - s \vec{\Omega}, t-s/v)
    \right]
    \eexp^{ -\tau(\vec{x}, \vec{x} - s \vec{\Omega})}
    \ud s \,.
\end{align*}
For now, we just discard it.

%%%%%%%%%%%%%%%%%%%%%%%%%%%%%%%%%%%%%%%%
\subsection{Approximating the streaming term}
The streaming term in Eq.~\eqref{eq:approxIntensity1} is (see
Eqs.~\eqref{eqs:inverseTransport}):
\begin{align*}
  \mathrm{ \tilde s }(\vec{x}, \vec{\Omega}, t) &=
    \lopinv{v}{ -\frac1{4\pi}\vec{\Omega}\vd \grad \phi }
  \\
  &= \int_{0}^{\norm{\vec{x} - \vec{x}_b}}
    \left[ -\frac1{4\pi}\vec{\Omega}\vd \grad \phi(\vec{x} - s \vec{\Omega},
    t-s/v)
    \right]
    \eexp^{ -\tau(\vec{x}, \vec{x} - s \vec{\Omega})}
    \ud s
\end{align*}
This integral describes the contribution from the volumetric source (in this
case, $-\vec{\Omega}\vd\grad\phi$) along $\vec{\Omega}$, evaluated at a prior
point in time ($t-s/v$, the point along $s$ at which a particle would travel
to $\vec{x}$ at time $t$), attenuated by the medium along the way (the
$\eexp^{ -\tau }$ factor).

We now make our first approximation by expanding the distant $\phi(\vec{x} - s
\vec{\Omega}, t-s/v)$ about the local $\phi(\vec{x}, t)$ in a Taylor series:
\begin{equation} \label{eq:taylorPhi}
  \phi(\vec{x} - s \vec{\Omega}, t-s/v)
  \sim \phi(\vec{x},t) - s \left( \frac{1}{c} \pder{}{t} + \vec{\Omega} \vd
  \grad  \right) \phi (\vec{x}, t) + \cdots \,.
\end{equation}
This is a good approximation if $\phi$ is smooth, especially because the
$\eexp^{ -\tau }$ term exponentially attenuates the non-local components of the
Taylor series as $s$ increases, assuming $\sigma\ne 0$ along the ray
$\vec{\Omega}$.

Applying Eq.~\eqref{eq:taylorPhi} to the nonlocal term in $\mathrm{ \tilde s
}$, we can move the unknown approximate $\tilde \phi$ outside the integral,
because it is no longer a function of $s$:
\begin{align}\nonumber
  \mathrm{ \tilde s }(\vec{x}, \vec{\Omega}, t)
  &\approx \int_{0}^{\norm{\vec{x} - \vec{x}_b}}
    \left[ -\frac1{4\pi}\vec{\Omega}\vd \grad \tilde\phi(\vec{x},t) \right]
    \eexp^{ -\tau(\vec{x}, \vec{x} - s \vec{\Omega})}
    \ud s
  \\\nonumber
  &= - \int_{0}^{\norm{\vec{x} - \vec{x}_b}}
    \left[ \frac1{4\pi}\right]
    \eexp^{ -\tau(\vec{x}, \vec{x} - s \vec{\Omega})} \ud s
    \vec{\Omega}\vd \grad \tilde \phi(\vec{x},t)
  \\\nonumber
  &= - \lopinv{v}{ \frac1{4\pi} } \vec{\Omega}\vd \grad \tilde \phi(\vec{x},t)
  \\\label{eq:tildeS}
  &= -f(\vec{x}, \vec{\Omega}, t) \vec{\Omega}\vd \grad \tilde \phi(\vec{x},t)
  \,.
\end{align}

Here,
\begin{align*}
  f(\vec{x}, \vec{\Omega}, t) &= \lopinv{v}{ \frac1{4\pi} }
%\\
  = \lopinv{b}{0} + \lopinv{i}{0} + \lopinv{v}{ \frac1{4\pi} },
\end{align*}
which (see Eq.~\eqref{eq:inverseTransportFull}) is the description of a
time-dependent transport problem with a homogeneous isotropic source, zero
initial condition, and zero boundary condition:
\begin{subequations} \label{eqs:fFull}
  \begin{equation} \label{eq:fFullVol}
  \frac{1}{c} \pder{f}{t}(\vec{x}, \vec{\Omega}, t)
    + \vec{\Omega}\vd \grad f(\vec{x}, \vec{\Omega}, t)
    + \sigma^\ast f (\vec{x}, \vec{\Omega}, t)
  =  \frac{1}{4\pi} \,, \quad x \in V,\  0 \le t < \Delta_t, \ \vec{\Omega}
  \in 4\pi\,,
  \end{equation}
  with boundary conditions
\begin{equation} \label{eq:fFullBndy}
  f(\vec{x}, \vec{\Omega}, t) = 0 \,,
 \quad \vec{x} \in \partial V, \ \vec{\Omega} \vd \vec{n} < 0,\ 0 \le t <
 \Delta_t \,,
\end{equation}
and the initial condition
\begin{equation} \label{eq:fFullInit}
 f(\vec{x}, \vec{\Omega}, 0) = 0 \,,
 \quad \vec{x} \in V, \ \vec{\Omega} \in 4\pi\,.
\end{equation}
\end{subequations}
We will return to this transport problem later to discuss some of its
advantageous properties.

\paragraph{Note} If we take the limit as $t \to \infty$, the transport
equation for $f$ gives the same transport equation as in anisotropic diffusion,
\begin{equation*}
    \vec{\Omega}\vd \grad f(\vec{x}, \vec{\Omega}, t)
    + \sigma^\ast f (\vec{x}, \vec{\Omega}, t)
  =  \frac{1}{4\pi} \,.
\end{equation*}
Then, taking the first moment of $\tilde I$ in Eq.~\eqref{eq:approxIntensity1}
and discarding the contributions from the boundary, the initial condition, and
the leakage terms,
\begin{equation*}
  \tilde F(\vec{x}, t)
  = \approx\frac{1}{4\pi} \int_{4\pi} \vec{\Omega} \tilde I \ud \Omega
  = 0 - \int_{4\pi} \vec{\Omega} f \vec{\Omega} \ud \Omega
    \vd \grad \tilde \phi(\vec{x},t) 
  = - \Dtens \vd \grad \tilde \phi(\vec{x},t) \,.
\end{equation*}
This is the standard AD result for the interior of a problem, ignoring initial
conditions. However, now that we have expressions for the other terms $\mathrm{
\tilde b}$, $\mathrm{ \tilde i}$, and $\mathrm{ \tilde f}$, we can do better.

%%%%%%%%%%%%%%%%%%%%%%%%%%%%%%%%%%%%%%%%
\subsection{Approximating the initial condition term}\label{sec:derIc}
The term accounting for the initial condition in
Eq.~\eqref{eq:approxIntensity1} is (see Eqs.~\eqref{eqs:inverseTransport}):
\begin{align*}
  \mathrm{ \tilde i }(\vec{x}, \vec{\Omega}, t) &=
    \lopinv{i}{I^i - \frac1{4\pi} \phi^i}
  \\
  &= \left[ I^i - \frac1{4\pi} \phi^i \right]_{( \vec{x} - ct
  \vec{\Omega}, \vec{\Omega})}
    \eexp^{ -\tau(\vec{x}, \vec{x} - ct \vec{\Omega})}
    U( \norm{\vec{x} - \vec{x}_b} - ct)\,.
\end{align*}
This describes the contribution at $(\vec{x}, \vec{\Omega}, t)$ from particles
that started at $t=0$ along the direction $\vec{\Omega}$, which have
traveled a distance $ct$ to get to the current point. The Heaviside function
$U$ is so that only particles that started in the interior are considered. The
particles are attenuated by a factor $\eexp^{ -\tau}$ along their travel, so
the contribution from this term in a non-vacuum is exponentially small at large
times.

Now we approximate the terms in the bracket, which is the evaluated nonlocal
initial condition. Let's expand the initial condition in spherical harmonics:
\begin{equation*}
  I^i(\vec{x}, \vec{\Omega}) = \frac{1}{4\pi} \phi^i(\vec{x})
  + \frac{3}{4\pi} \vec{\Omega} \vd \vec{F}^i(\vec{x}) + \varphi(\vec{x}, \vec{\Omega})\,,
\end{equation*}
where $\varphi$ contains all the spherical moments of $I^i$ that are more than
linearly anisotropic. Substituting the spherical harmonic expansion into
$\mathrm{ \tilde i }$ cancels the isotropic component of the initial condition,
leaving
\begin{align} \nonumber
  \mathrm{ \tilde i }(\vec{x}, \vec{\Omega}, t)
  &= \left[ \frac{3}{4\pi} \vec{\Omega} \vd \vec{F}^i(\vec{x})
  + \varphi(\vec{x}, \vec{\Omega}) \right]_{( \vec{x} - ct \vec{\Omega},
  \vec{\Omega})}
    \eexp^{ -\tau(\vec{x}, \vec{x} - ct \vec{\Omega})}
    U( \norm{\vec{x} - \vec{x}_b} - ct)\,.
  \\
  \intertext{Now we make the dubious but (perhaps for now) necessary
  approximation of discarding the higher-order angular moments:} \nonumber
  \mathrm{ \tilde i }(\vec{x}, \vec{\Omega}, t)
  &= \left[ \frac{3}{4\pi} \vec{\Omega} \vd \vec{F}^i(\vec{x}) \right]_{( \vec{x} - ct \vec{\Omega}, \vec{\Omega})}
    \eexp^{ -\tau(\vec{x}, \vec{x} - ct \vec{\Omega})}
    U( \norm{\vec{x} - \vec{x}_b} - ct)\,.
  \\
  \intertext{At this point, we apply the same Taylor series expansion as in
  Eq.~\eqref{eq:taylorPhi}, but this time to $\vec{F}^i$, allowing us to move
  the term outside the bracket:} \nonumber
  \mathrm{ \tilde i }(\vec{x}, \vec{\Omega}, t)
  &= \left[ \frac{3}{4\pi} \right]_{( \vec{x} - ct \vec{\Omega}, \vec{\Omega})}
    \eexp^{ -\tau(\vec{x}, \vec{x} - ct \vec{\Omega})}
    U( \norm{\vec{x} - \vec{x}_b} - ct)  \vec{\Omega} \vd \vec{F}^i(\vec{x})
  \\ \nonumber
  &= \lopinv{i}{\frac{3}{4\pi}} \vec{\Omega} \vd \vec{F}^i(\vec{x})\,.
  \\\label{eq:tildeI}
  &= g(\vec{x}, \vec{\Omega}, t) \vec{\Omega} \vd \vec{F}^i(\vec{x})\,.
\end{align}

Now we have another simple transport problem:
\begin{equation*}
  g(\vec{x}, \vec{\Omega}, t) = 
  \lopinv{b}{0} + \lopinv{i}{\frac{3}{4\pi}} + \lopinv{v}{ 0 },
\end{equation*}
which describes a time-dependent transport problem with an isotropic initial
condition but no source:
\begin{subequations} \label{eqs:gFull}
  \begin{equation} \label{eq:gFullVol}
  \frac{1}{c} \pder{g}{t}(\vec{x}, \vec{\Omega}, t)
    + \vec{\Omega}\vd \grad g(\vec{x}, \vec{\Omega}, t)
    + \sigma^\ast g (\vec{x}, \vec{\Omega}, t)
  =  0 \,, \quad x \in V,\  0 \le t < \Delta_t, \ \vec{\Omega}
  \in 4\pi\,,
  \end{equation}
  with boundary conditions
\begin{equation} \label{eq:gFullInit}
 g(\vec{x}, \vec{\Omega}, 0) = \frac{3}{4\pi} \,,
 \quad \vec{x} \in V, \ \vec{\Omega} \in 4\pi\,,
\end{equation}
and boundary conditions to be determined.
\end{subequations}

%%%%%%%%%%%%%%%%%%%%%%%%%%%%%%%%%%%%%%%%
\subsection{Approximating the boundary term}\label{sec:derBc}
With Eqs.~\eqref{eq:tildeS} and~\eqref{eq:tildeI}, and if the $\mathrm{ \tilde f }$ term is discarded, our approximate
equation~\eqref{eq:approxIntensity1} for the angular intensity becomes
\begin{equation} \label{eq:approxIntensity2}
  \tilde I(\vec{x}, \vec{\Omega}, t)
  =
  \frac{1}{4\pi} \tilde \phi(\vec{x}, t)
+ \mathrm{ \tilde b }(\vec{x}, \vec{\Omega}, t) 
+ g(\vec{x}, \vec{\Omega}, t) \vec{\Omega} \vd \vec{F}^i(\vec{x})
- f(\vec{x}, \vec{\Omega}, t) \vec{\Omega}\vd \grad \tilde \phi(\vec{x},t) \,,
\end{equation}
where $f$ and $g$ are the solutions to the simple transport equations derived
earlier, Eqs.~\eqref{eqs:fFull} and~\eqref{eqs:gFull}. Even though the boundary
contribution $ \mathrm{ \tilde b }$ is mathematically small away from the
boundaries, we seek to eliminate it globally, including at the boundary
itself.

%%%%%%%%%%%%%%%%%%%%%%%%%%%%%%%%%%%%%%%%
\subsubsection{Low-order ``Marshak'' boundary}
Now let us consider the boundary condition from the true transport problem,
Eq.~\eqref{eq:fullTransportBndy}:
\begin{equation*}
  I(\vec{x}, \vec{\Omega}, t)
  = I^b(\vec{x}, \vec{\Omega}, t) \,,
\end{equation*}
for $\vec{x} \in \partial V, \ \vec{\Omega} \vd \vec{n} < 0,
 \ 0 \le t < \Delta_t$.
 Multiplying both sides by $\abs{\vec{\Omega} \vd \vec{n}}$ and
 integrating over incident directions gives an exact expression for the
 incident radiation flux:
\begin{equation*}
 \int_{\vec{\Omega} \vd \vec{n} < 0} \abs{\vec{\Omega} \vd \vec{n}}
 I(\vec{x}, \vec{\Omega}, t) \ud\Omega
 = F^\mathrm{in}(\vec{x}, t)\,.
\end{equation*}
Substituting our approximate expression for $I$,
Eq.~\eqref{eq:approxIntensity1}, and using the approximations to
$\mathrm{ \tilde f }$, $\mathrm{ \tilde s }$, and $\mathrm{ \tilde i }$ derived
in previous sections, operating under the assumption that we have cancelled the
$\mathrm{ \tilde b }$ term,
\begin{align}\nonumber
  F^\mathrm{in}(\vec{x}, t)
  &= \int_{\vec{\Omega} \vd \vec{n} < 0} \abs{\vec{\Omega} \vd \vec{n}}
 \left[
   \frac1{4\pi} \tilde \phi(\vec{x}, t) 
  + \mathrm{ \tilde b }
  + \mathrm{ \tilde i }
  + \mathrm{ \tilde s }
  + \mathrm{ \tilde f }\right]\ud\Omega
 \\\nonumber
  F^\mathrm{in}(\vec{x}, t)
 &=
 \frac1{4} \tilde \phi(\vec{x}, t) 
 + \int_{\vec{\Omega} \vd \vec{n} < 0} \abs{\vec{\Omega} \vd \vec{n}}
\left[
 0
  + g(\vec{x}, \vec{\Omega}, t) \vec{\Omega} \vd \vec{F}^i(\vec{x})
  - f(\vec{x}, \vec{\Omega}, t) \vec{\Omega} \vd \grad \tilde\phi(\vec{x}, t)
  + 0 \right]\ud\Omega\,.
\\ \nonumber
 F^\mathrm{in}(\vec{x}, t)
 &=
 \frac1{4} \tilde \phi(\vec{x}, t) 
 - \int_{\vec{\Omega} \vd \vec{n} < 0} (\vec{\Omega} \vd \vec{n})
 \vec{\Omega}\:
 g(\vec{x}, \vec{\Omega}, t) \ud\Omega \vd \vec{F}^i(\vec{x})
 + 4\int_{\vec{\Omega} \vd \vec{n} < 0} (\vec{\Omega} \vd \vec{n})
 \vec{\Omega}\:
 f(\vec{x}, \vec{\Omega}, t) \ud\Omega \vd \grad \tilde\phi(\vec{x}, t)
\,.
\\ \label{eq:bndyAnisotropic}
 4F^\mathrm{in}(\vec{x}, t)
 &=
 \tilde \phi(\vec{x}, t) 
 - 4 \vec{n} \vd \int_{\vec{\Omega} \vd \vec{n} < 0}
 \vec{\Omega} \vec{\Omega}\:
 g(\vec{x}, \vec{\Omega}, t) \ud\Omega \vd \vec{F}^i(\vec{x})
 + 4 \vec{n} \vd \int_{\vec{\Omega} \vd \vec{n} < 0}
 \vec{\Omega} \vec{\Omega}\:
 f(\vec{x}, \vec{\Omega}, t) \ud\Omega \vd \grad \tilde\phi(\vec{x}, t)
\,.
\end{align}

%%%%%%%%%%%%%%%%%%%%%%%%%%%%%%%%%%%%%%%%
\subsubsection{High-order boundary term}

The boundary contribution to the approximate intensity,
Eq.~\eqref{eq:approxIntensity1}, is:
\begin{align*}
  \mathrm{ \tilde b }(\vec{x}, \vec{\Omega}, t)
  &= \left[
  I^b
- \frac{1}{4\pi} \int_{\vec{\Omega}' \vd \vec{n} < 0} I^b \ud \Omega'
- \frac{1}{4\pi} \int_{\vec{\Omega}' \vd \vec{n} > 0} I \ud \Omega'
 \right]_{(\vec{x}_b, \vec{\Omega}, t - \norm{\vec{x} - \vec{x}_b}/c)}
    \eexp^{ -\tau(\vec{x}, \vec{x}_b)} U(ct - \norm{\vec{x} - \vec{x}_b})
  \\
  &=
  \lopinv{b}{ I^b
- \frac{1}{4\pi} \int_{\vec{\Omega}' \vd \vec{n} < 0} I^b \ud \Omega'
- \frac{1}{4\pi} \int_{\vec{\Omega}' \vd \vec{n} > 0} I \ud \Omega' }
 \,.
\end{align*}
It accounts for particles that start their life at the boundary inside the
current time step, streaming along $\vec{\Omega}$, attenuated by $\eexp^{
-\tau}$ along their path.

If we can somehow cancel the term in brackets at all points on the boundary,
then there will be no contribution to the internal solution from the boundary
term. Demanding this cancellation gives the equation
\begin{equation*}
I^b(\vec{x}, \vec{\Omega}, t)
- \frac{1}{4\pi} \int_{\vec{\Omega}' \vd \vec{n} < 0} I^b(\vec{x},
\vec{\Omega}', t) \ud \Omega'
= \frac{1}{4\pi} \int_{\vec{\Omega}' \vd \vec{n} > 0} \tilde I(\vec{x},
\vec{\Omega}', t) \ud \Omega'
 \quad \vec{x} \in \partial V, \ \vec{\Omega} \vd \vec{n} < 0,
 \ 0 \le t < \Delta_t\,,
\end{equation*}
where $\tilde I$ is the approximation of the intensity, given in
Eq.~\eqref{eq:approxIntensity2}.
Substituting that approximate expression under the assumption that
$\mathrm{ \tilde b }=0$,
\begin{align*}
\lefteqn{I^b(\vec{x}, \vec{\Omega}, t)
- \frac{1}{4\pi} \int_{\vec{\Omega}' \vd \vec{n} < 0} I^b(\vec{x},
\vec{\Omega}', t) \ud \Omega'}\qquad
\\
&= \frac{1}{4\pi} \int_{\vec{\Omega}' \vd \vec{n} > 0} 
\Bigg[
 \frac{1}{4\pi} \tilde \phi(\vec{x}, t)
 + g(\vec{x}, \vec{\Omega}', t) \vec{\Omega}' \vd \vec{F}^i(\vec{x})
 - f(\vec{x}, \vec{\Omega}', t) \vec{\Omega}'\vd \grad \tilde\phi(\vec{x},t)
 \Bigg] \ud \Omega'
 \\
&= \frac{1}{8\pi} \tilde \phi(\vec{x}, t)
+ \frac{1}{4\pi} \int_{\vec{\Omega}' \vd \vec{n} > 0} 
 g(\vec{x}, \vec{\Omega}', t) \vec{\Omega}' \vd \vec{F}^i(\vec{x})
 \ud \Omega'
- \frac{1}{4\pi} \int_{\vec{\Omega}' \vd \vec{n} > 0} 
 f(\vec{x}, \vec{\Omega}', t) \vec{\Omega}'\vd \grad \tilde\phi(\vec{x},t)
 \ud \Omega'\,.
\end{align*}

To proceed, we assume that $I^b( \vec{x}, \vec{\Omega}, t)
\equiv \frac{1}{\pi} F^\mathrm{in}( \vec{x}, t)$, an isotropic incident
radiation condition with radiation flux $F^\mathrm{in}$.\footnote{Rather than
requiring that the flux be isotropic, we could separate $I_b$ into an
angle-independent component of magnitude $F^\mathrm{in}/\pi$ and a remainder
term. The remainder term would introduce some unaccounted-for boundary terms
into the problem.} Under this assumption, the left-hand side is
\begin{equation*}
I^b( \vec{x},\vec{\Omega},  t)
- \frac{1}{4\pi} \int_{\vec{\Omega}' \vd \vec{n} < 0}
I^b( \vec{x}, \vec{\Omega}', t) \ud \Omega'
= \frac{1}{\pi} F^\mathrm{in}( \vec{x}, t)
- \frac{1}{\pi} F^\mathrm{in}( \vec{x}, t) \left[ \frac{1}{4\pi}
\int_{\vec{\Omega}' \vd \vec{n} < 0} \ud \Omega'\right]
= \frac{1}{2\pi} F^\mathrm{in}( \vec{x}, t)\,.
\end{equation*}
Thus, to cancel the boundary term,
\begin{align}\nonumber
  \frac{1}{2\pi} F^\mathrm{in}( \vec{x}, t)
  &=
  \frac{1}{8\pi} \tilde \phi(\vec{x}, t)
+ \frac{1}{4\pi} \int_{\vec{\Omega}' \vd \vec{n} > 0} 
 g(\vec{x}, \vec{\Omega}', t) \vec{\Omega}' \ud \Omega'
 \vd \vec{F}^i(\vec{x})
- \frac{1}{4\pi} \int_{\vec{\Omega}' \vd \vec{n} > 0} 
 f(\vec{x}, \vec{\Omega}', t) \vec{\Omega}' \ud \Omega'
 \vd \grad \tilde\phi(\vec{x},t)
 \\ \label{eq:bndyElimCond}
 4 F^\mathrm{in}( \vec{x}, t)
  &=
  \tilde \phi(\vec{x}, t)
+ 2\int_{\vec{\Omega}' \vd \vec{n} > 0} 
 g(\vec{x}, \vec{\Omega}', t) \vec{\Omega}' \ud \Omega'
 \vd \vec{F}^i(\vec{x})
- 2\int_{\vec{\Omega}' \vd \vec{n} > 0} 
 f(\vec{x}, \vec{\Omega}', t) \vec{\Omega}' \ud \Omega'
 \vd \grad \tilde\phi(\vec{x},t)
 \ud \Omega'\,.
\end{align}

%%%%%%%%%%%%%%%%%%%%%%%%%%%%%%%%%%%%%%%%
\subsubsection{Resulting boundary conditions}

In summary, the low-order equation~\eqref{eq:bndyAnisotropic} demands that
\begin{align*}
 4F^\mathrm{in}(\vec{x}, t)
 &=
 \tilde \phi(\vec{x}, t) 
 - 4 \vec{n} \vd \left[ \int_{\vec{\Omega} \vd \vec{n} < 0}
 \vec{\Omega} \vec{\Omega}\:
 g(\vec{x}, \vec{\Omega}, t) \ud\Omega \right] \vd \vec{F}^i(\vec{x})
 + 4 \vec{n} \vd \left[ \int_{\vec{\Omega} \vd \vec{n} < 0}
 \vec{\Omega} \vec{\Omega}\:
 f(\vec{x}, \vec{\Omega}, t) \ud\Omega \right] \vd \grad \tilde\phi(\vec{x}, t)
\\ 
\intertext{and we want to satisfy Eq.~\eqref{eq:bndyElimCond},}
 4 F^\mathrm{in}( \vec{x}, t)
  &=
  \tilde \phi(\vec{x}, t)
+ 2 \left[ \int_{\vec{\Omega} \vd \vec{n} > 0} 
 g(\vec{x}, \vec{\Omega}, t) \vec{\Omega} \ud \Omega \right]
 \vd \vec{F}^i(\vec{x})
- 2\left[\int_{\vec{\Omega} \vd \vec{n} > 0} 
 f(\vec{x}, \vec{\Omega}, t) \vec{\Omega} \ud \Omega \right]
\vd \grad \tilde\phi(\vec{x},t)
 \,.
\end{align*}
WHAT JUST HAPPENED TO MY SIGNS?

If we use the old way I did it,
\begin{align*}
  \mathrm{\tilde b} 
  &= \lopinv{b}{I^b
  - \frac{1}{2\pi} \int_{\vec{\Omega}' \vd \vec{n} < 0} I^b \ud \Omega'}
  \\
  &= \lopinv{b}{I^b
  - \frac{1}{4\pi} \int_{\vec{\Omega}' \vd \vec{n} < 0} I^b \ud \Omega'
  - \frac{1}{4\pi} \int_{\vec{\Omega}' \vd \vec{n} < 0} I^b \ud \Omega'}
  \\
  &= \lopinv{b}{I^b
  - \frac{1}{4\pi} \int_{\vec{\Omega}' \vd \vec{n} < 0} I^b \ud \Omega'
  - \frac{1}{4\pi}\left[ \phi - \int_{\vec{\Omega}' \vd \vec{n} > 0} I \ud \Omega' \right]
  }
\end{align*}
i.e.,
\begin{equation*}
  I^b
  - \frac{1}{4\pi} \int_{\vec{\Omega}' \vd \vec{n} < 0} I^b \ud \Omega'
  = \frac{1}{4\pi} \phi
  - \frac{1}{4\pi}\int_{\vec{\Omega}' \vd \vec{n} > 0} I \ud \Omega'
\end{equation*}
then we get
\begin{equation*}
 4 F^\mathrm{in}( \vec{x}, t)
  =
  \tilde \phi(\vec{x}, t)
- 2\int_{\vec{\Omega} \vd \vec{n} > 0} 
 g(\vec{x}, \vec{\Omega}, t) \vec{\Omega} \vd \vec{F}^i(\vec{x})
 \ud \Omega
+ 2\int_{\vec{\Omega} \vd \vec{n} > 0} 
 f(\vec{x}, \vec{\Omega}, t) \vec{\Omega}\vd \grad \tilde\phi(\vec{x},t)
 \ud \Omega\,.
\end{equation*}
and the signs match like we'd expect.

We are ??? in luck. By inspection, both equations holds true for all
values of
$\tilde\phi$, $\vec{F}^i$, and $F^\mathrm{in}$ if:
\begin{subequations} \label{eqs:bndyTransport}
\begin{align} \label{eq:bndyF}
  \int_{\vec{\Omega} \vd \vec{n} > 0} \vec{\Omega}
 \underbrace{f(\vec{x}, \vec{\Omega}, t)}_{\text{exiting}} \ud\Omega
 &= 2 \vec{n} \vd \int_{\vec{\Omega} \vd \vec{n} < 0}
 \vec{\Omega} \vec{\Omega}
 \underbrace{f(\vec{x}, \vec{\Omega}, t)}_{\text{incident}} \ud\Omega
 \\ \intertext{and} \label{eq:bndyG}
  \int_{\vec{\Omega} \vd \vec{n} > 0} \vec{\Omega}
 \underbrace{g(\vec{x}, \vec{\Omega}, t)}_{\text{exiting}} \ud\Omega
 &= 2 \vec{n} \vd \int_{\vec{\Omega} \vd \vec{n} < 0}
 \vec{\Omega} \vec{\Omega}
 \underbrace{g(\vec{x}, \vec{\Omega}, t)}_{\text{incident}} \ud\Omega
\end{align}
\end{subequations}
for all $\vec{x} \in \partial V$, $0\le t < \Delta_t$.

Thus, if $f$ and $g$ have boundary conditions that satisfy
Eqs.~\eqref{eqs:bndyTransport}, and the true incident radiation is isotropic,
and Eq.~\eqref{eq:bndyAnisotropic} is used as the low-order boundary condition,
then we have developed a set of consistent conditions that eliminate
$\mathrm{\tilde b}$ throughout the approximate solution $\tilde\phi$ and
produced a low-order boundary condition for the \APone\ problem.

%%%%%%%%%%%%%%%%%%%%%%%%%%%%%%%%%%%%%%%%
\subsubsection{A further approximation}

If $f$ and $g$ have no azimuthal dependence at the boundary, then
\begin{equation*}
  \int_{\vec{\Omega} \vd \vec{n} > 0} \vec{\Omega}
  f(\vec{x}, \vec{\Omega}, t) \ud\Omega
  = \vec{n} \int_{\vec{\Omega} \vd \vec{n} > 0}
  \left( \vec{\Omega} \vd \vec{n} \right)
  f(\vec{x}, \vec{\Omega}, t) \ud\Omega
  = \vec{n} j^\textrm{out}
\end{equation*}
where $j^\textrm{out}$ is the ``exiting radiation flux'' in the transport
calculation that calculates $f$.

and we get an albedo white boundary condition. (Note that in a 1-D problem,
there is no azimuthal dependence at all, so the albedo white boundary is not an
approximation.)

The assumption of negligible azimuthal dependence on the exiting $f$ has an
additional benefit. Without azimuthal dependence, there is no transverse
leakage on the boundary, so formulating a discretized version of the boundary
conditions is much easier.

%%%%%%%%%%%%%%%%%%%%%%%%%%%%%%%%%%%%%%%%%%%%%%%%%%%%%%%%%%%%%%%%%%%%%%%%%%%%%%%%
\subsection{Approximate radiation flux}
Integrating the approximate expression for the intensity in
Eq.~\eqref{eq:approxIntensity2} yields an approximation to the radiation flux:
\begin{align} \nonumber
  \tilde{\vec{F}} (\vec{x}, t)
  &= \int_{4\pi} \vec{\Omega} \tilde I(\vec{x}, \vec{\Omega}, t) \ud\Omega
  \\ \nonumber
  &= 
  \int_{4\pi} \vec{\Omega} g(\vec{x}, \vec{\Omega}, t) \vec{\Omega} \ud\Omega
  \vd \vec{F}^i(\vec{x})
  - \int_{4\pi} \vec{\Omega} f(\vec{x}, \vec{\Omega}, t) \vec{\Omega} \ud\Omega
  \vd \grad \tilde \phi(\vec{x},t)
  \\ \label{eq:anisotropicP1}
  &= \mat{E}(\vec{x}, t) \vd \vec{F}^i(\vec{x})
  - \Dtens(\vec{x}, t) \vd \grad \tilde \phi(\vec{x},t) \,.
\end{align}
Here, $\mat{E}$ is the second angular moment of $g$, and $\Dtens$ is the second
angular moment of $f$.

Combining this with the zeroth moment of the transport equation, and with the
two transport calculations that yield $\mat{E}$ and $\Dtens$, we have
derived the anisotropic \Pone\ equations.

From this point onward, we discard the tildes because we are aware that our
solutions $\phi$ and $\vec{F}$ are only approximations to the true scalar
intensity and radiation flux.

%%%%%%%%%%%%%%%%%%%%%%%%%%%%%%%%%%%%%%%%%%%%%%%%%%%%%%%%%%%%%%%%%%%%%%%%%%%%%%%%
\subsection{Implicit time-dependent formulation}
To take the implicit Euler approximation of Eq.~\eqref{eq:anisotropicP1}, we
average over the time step $0 < t < \Delta_t$, and approximate the
time-average unknowns by their values at $t^{n+1} = t^i + \Delta_t$. This
procedure yields an approximate but numerically convenient equation,
\begin{equation}\label{eq:implicitFlux1}
  \vec{F}^{n+1}(\vec{x}) = \mat{E}^{n+1} (\vec{x}) \vd \vec{F}^i(\vec{x})
  - \Dtens^{n+1}(\vec{x}) \vd \grad \phi^{n+1} (\vec{x}) 
\end{equation}

\begin{subequations} \label{eqs:implicitD}
The implicit approximation to $\Dtens$ is
\begin{equation}\label{eq:implicitDtens}
  \Dtens^{n+1} = \int_{4\pi} \vec{\Omega} \vec{\Omega} \:
  f^{n+1}(\vec{x}, \vec{\Omega}) \ud\Omega\,,
\end{equation}
and the transport equation~\eqref{eqs:fFull} for $f$ discretized implicitly
over time is:
\begin{equation*}
  \frac{f^{n+1} - f^{0}}{c \Delta_t}
  + \vec{\Omega} \vd \grad f^{n+1}
  + \sigma^\ast f^{n+1}
  =  \frac{1}{4\pi} \,,
\end{equation*}
with appropriate boundary conditions and the initial condition $f^{0}=0$. This
can be rearranged to the form of a steady-state transport solve, with modified
opacities:
\begin{equation} \label{eq:implicitF}
  \vec{\Omega} \vd \grad f^{n+1}
  + \left[ \sigma^\ast + \frac{1}{c \Delta_t} \right]f^{n+1}
  =  \frac{1}{4\pi} \,.
\end{equation}
This transport equation takes only one ``sweep'' to solve!
\end{subequations}

\begin{subequations} \label{eqs:implicitE}
Likewise, the implicit approximation to $\mat{E}$ is
\begin{equation}\label{eq:implicitEtens}
  \mat{E}^{n+1} = \int_{4\pi} \vec{\Omega} \vec{\Omega} \:
  g^{n+1}(\vec{x}, \vec{\Omega}) \ud\Omega\,,
\end{equation}
and the transport equation~\eqref{eqs:gFull} for $g$ discretized implicitly
over time is:
\begin{equation*}
  \frac{g^{n+1} - g^{0}}{c \Delta_t}
  + \vec{\Omega} \vd \grad g^{n+1}
  + \sigma^\ast g^{n+1}
=  0 \,,
\end{equation*}
with appropriate boundary conditions and the initial condition
$g^{0}=\frac{3}{4\pi}$. This too can be rearranged! Substituting the
isotropic initial condition $g^i$,
\begin{equation} \label{eq:implicitG}
  \vec{\Omega} \vd \grad g^{n+1}
  + \left[ \sigma^\ast + \frac{1}{c \Delta_t} \right]g^{n+1}
  = \frac{1}{c \Delta_t} \frac{3}{4\pi} \,.
\end{equation}
This transport equation \emph{also} takes only one ``sweep'' to solve, but
there is another glaringly obvious fact that will make our lives even easier:
with the implicit approximation, $g^{n+1} = \frac{3}{c \Delta t} f^{n+1}$!
\end{subequations}

Therefore, $\mat{E}^{n+1} =  \frac{3}{c \Delta t} \mat{D}^{n+1}$, so
Eq.~\eqref{eq:implicitFlux1} becomes
\begin{equation}\label{eq:implicitFlux}
  \vec{F}^{n+1}(\vec{x}) =  \frac{3}{c \Delta t} \Dtens^{n+1} (\vec{x}) \vd
  \vec{F}^i(\vec{x})
  - \Dtens^{n+1}(\vec{x}) \vd \grad \phi^{n+1} (\vec{x}) \,.
\end{equation}
Our implicit approximation comprises this approximation to the first moment,
Eq.~\eqref{eq:implicitFlux}, and the transport equation used to calculate
$\Dtens$, Eqs.~\eqref{eqs:implicitD}.

\paragraph{Note} This is essentially the same result I got last August when I
first implicitly time-discretized the transport equation and then applied the
old (standard) AD derivation. I guess I should have paid more attention then!
At least this time, I have derived it fairly rigorously, I've got a
time-dependent form that we can improve, and I show how the opacities over a
time step don't need to be approximated if they're lagged to $\sigma^\ast$.

%%%%%%%%%%%%%%%%%%%%%%%%%%%%%%%%%%%%%%%%%%%%%%%%%%%%%%%%%%%%%%%%%%%%%%%%%%%%%%%%
\section{Analysis}


\subsection{Time dependence}
Starting here, we only consider the implicit Euler discretization of the
\APone\ equations. \emph{However}, the fact that we can calculate
$\mat{E}^{n+1}$ and $\Dtens^{n+1}$ however we want means that we could do crazy
stuff like, for example, using a Monte Carlo transport solve to yield
coefficients that take into account the finite speed of the particles, yielding
true time-averaged values for the two tensors, and possibly obviating the error
inherent in implicit time discretization.

%%%%%%%%%%%%%%%%%%%%%%%%%%%%%%%%%%%%%%%%%%%%%%%%%%%%%%%%%%%%%%%%%%%%%%%%%%%%%%%%
\subsubsection{Limit for large time steps}
As $\Delta_t\to \infty$, the modified opacity in Eq.~\eqref{eq:implicitF}
approaches the true space-dependent opacity $\sigma^\ast$. Thus, $\Dtens$
approaches the same $\Dtens$ as the standard anisotropic diffusion method. In
an infinite homogeneous medium, $f\to 1/4\pi \sigma$ so $\Dtens \to
\mat{I}/3\sigma$, the standard diffusion result.

For these large time steps, the contribution of the initial condition
$\vec{F}^i$ should approach zero, so $\mat{E}^{n+1}$ should approach $
\mat{0}$. We find that it does, because the effect of the initial condition
in Eq.~\eqref{eq:implicitG} diminishes exponentially for large time steps. Thus,
for $c \Delta_t \gg \sigma^\ast$, $g^{n+1}=0$ and therefore
$\mat{E}^{n+1}=\mat{0}$.

Therefore, in the limit of large time steps, the \APone\ method limits to the
standard AD method. This is good!

\paragraph{Note} What makes a time step ``large'' is now a non-local factor,
not a local factor as in \Pone.

%%%%%%%%%%%%%%%%%%%%%%%%%%%%%%%%%%%%%%%%%%%%%%%%%%%%%%%%%%%%%%%%%%%%%%%%%%%%%%%%
\subsubsection{Limit for small time steps}
TODO the same analysis makes it clear that as $\Delta_t \to 0$,
$\Dtens^{n+1}$ approaches $\mat{0}$ and $\mat{E}^{n+1}$ approaches $\mat{I}$,
yielding
\begin{equation*}
  \vec{F}^{n+1} \approx \vec{F}^{n}
\end{equation*}
as $\Delta_t \to 0$.

%%%%%%%%%%%%%%%%%%%%%%%%%%%%%%%%%%%%%%%%%%%%%%%%%%%%%%%%%%%%%%%%%%%%%%%%%%%%%%%%
\subsubsection{Infinite homogeneous medium case}
In an infinite homogeneous medium, Eq.~\eqref{eq:implicitFlux} limits to a
form, obviating the error inherent in implicit discretizations.
of the implicitly time-discretized \Pone\ equation.

The solution to Eq.~\eqref{eq:implicitF} is
\begin{equation*}
  f^{n+1}
  = \frac{1}{4\pi} \frac{1}{\sigma^\ast + 1 /c \Delta_t}
  = \frac{1}{4\pi} \frac{\sigma^\ast}{\sigma^\ast} \frac{c \Delta t}{1 + \sigma^\ast c \Delta_t}
  = \frac{1}{4\pi \sigma} ( 1 - \alpha^i) \,,
\end{equation*}
where $\alpha^i = 1/ \sigma^\ast c \Delta_t$. (This has a physical
interpretation, because $\sigma^\ast c \Delta_t$ is the number of mean free
paths a photon travels inside a time step.) Equation~\eqref{eq:implicitDtens}
then gives
\begin{equation*}
  \Dtens = \frac{1}{3 \sigma} ( 1 - \alpha^i) \mat{I}\,.
\end{equation*}

The solution to Eq.~\eqref{eq:implicitG} is
\begin{equation*}
  g^{n+1}
  = \frac{1}{c \Delta_t} \frac{3}{4\pi} \frac{1}{\sigma^\ast + 1 /c \Delta_t}
  = \frac{3}{4\pi} \frac{1}{1 + \sigma^\ast c \Delta_t}
  = \frac{3}{4\pi} \alpha^i \,.
\end{equation*}
Equation~\eqref{eq:implicitEtens} then gives
\begin{equation*}
  \mat{E} = \alpha^i \mat{I}\,.
\end{equation*}

Substituting $\Dtens$ and $\mat{E}$ into Eq.~\eqref{eq:implicitFlux}, we get the
infinite-medium solution
\begin{align*}
  \vec{F}^{n+1}(\vec{x})
  &= \alpha^i \mat{I} \vd \vec{F}^i(\vec{x})
  - \frac{1}{3 \sigma} ( 1 - \alpha^i) \mat{I} \vd \grad \phi^{n+1} (\vec{x}) 
  \\
  &= \alpha^i \vec{F}^i(\vec{x})
  - \frac{1}{3 \sigma} ( 1 - \alpha^i) \grad \phi^{n+1} (\vec{x})\,.
\end{align*}

We can reshape the \Pone\ equations into this exact form:
\begin{align*}
  \frac{\vec{F}^{n+1} - \vec{F}^i}{c \Delta_t} + \frac{1}{3} \grad \phi^{n+1}
  + \sigma^\ast \vec{F}^{n+1} &= 0
  \\
  \vec{F}^{n+1} - \vec{F}^i 
  + \sigma^\ast c \Delta_t \vec{F}^{n+1}
  &= - c \Delta_t \frac{1}{3} \grad \phi^{n+1}
  \\
  \left[ 1 + \sigma^\ast c \Delta_t \right] \vec{F}^{n+1}
  &= \vec{F}^i - \sigma^\ast c \Delta_t \frac{1}{3\sigma^\ast} \grad \phi^{n+1}
  \\
  \vec{F}^{n+1}
  &= \frac{1}{1 + \sigma^\ast c \Delta_t} \vec{F}^i - \frac{\sigma^\ast c
  \Delta_t}{1 + \sigma^\ast c \Delta_t} \frac{1}{3\sigma^\ast} \grad \phi^{n+1}
  \\
  \vec{F}^{n+1}
  &= \alpha^i \vec{F}^i - (1 - \alpha^i) \frac{1}{3\sigma^\ast} \grad \phi^{n+1}
\end{align*}

Ta-da! Our approximation of the previous time step in \S\ref{sec:derIc} as
being linear in angle is really what gives the undesirable $\frac13\grad\phi$
(which has an incorrect wave speed). It may be possible to either improve that
approximation by supposing $f$ for the previous time step is like $f$ from the
current time step (and same with $g$), which would definitely not be the case
if $\Delta_t$ is changing from one time step to the next.

%%%%%%%%%%%%%%%%%%%%%%%%%%%%%%%%%%%%%%%%%%%%%%%%%%%%%%%%%%%%%%%%%%%%%%%%%%%%%%%%
\subsection{Boundaries}
In the general 1-D case during a time step, the transport equation is:
\begin{subequations} \label{eqs:oneDFullTransport}
\begin{multline} \label{eq:oneDFullTransportVol}
  \frac{1}{c} \pder{I}{t}(x, \mu, t)
  + \mu \pder{I}{x}(x, \mu, t)
    + \sigma^\ast(x) I (x, \mu, t)
  = \frac{1}{2} \sigma^\ast(x) ac [T(x, t)]^4
    + \frac{1}{2} q_{r}(x, t)
    \equiv \frac{1}{2} Q(x, t)\,,
\\
0 < x < X,\  0 \le t < \Delta_t, \ -1 < \mu < 1,
\end{multline}
with a specified incident radiation boundary condition on the left:
\begin{equation} \label{eq:oneDFullTransportBndySrc}
  I(0, \mu, t) =I^{\ell,b}(\mu, t) \,,
 \quad 0 < \mu \le 1,\ 0 \le t < \Delta_t
\end{equation}
and a reflecting boundary on the right
\begin{equation} \label{eq:oneDFullTransportBndyRefl}
  I(X, \mu, t) =I(X, -\mu, t) \,,
 \quad -1 \le \mu < 0,\ 0 \le t < \Delta_t
\end{equation}
and the initial condition
\begin{equation} \label{eq:oneDFullTransportInit}
 I(x, \mu, 0) =I^i(x, \mu, t) \,,
\quad 0 < x < X,\  -1 < \mu < 1.
\end{equation}
\end{subequations}

Calculating the incident radiation flux $F^{\ell,\mathrm{in}}$ for both
boundary conditions gives us a starting point for the low-order boundary
conditions. On the left, applying $\int_{0}^{1} \abs{\mu} (\cdot) \ud \mu$,
\begin{subequations} \label{eqs:oneDFin}
\begin{equation} \label{eq:oneDFinLeft}
  \int_{0}^{1} \mu I(0, \mu, t) \ud \mu
  = \int_{0}^{1} \abs{\mu} I^{\ell,b}(\mu, t) \ud \mu
  \equiv F^{\ell,\mathrm{in}}(t) \,.
\end{equation}
On the right, applying $\int_{-1}^{0} \abs{\mu} (\cdot) \ud \mu$,
\begin{align} \nonumber
  -\int_{-1}^{0} \mu I(X, \mu, t) \ud \mu
  &= -\int_{-1}^{0}\mu I(X, -\mu, t) \ud \mu
  \\ \nonumber
  -\int_{-1}^{0} \mu I(X, \mu, t) \ud \mu
  &= -\int_{0}^{1} (-\mu) I(X, \mu, t) \ud \mu
  \\ \nonumber
  0
  &= \int_{-1}^{0} \mu I(X, \mu, t) \ud \mu
  +\int_{0}^{1} \mu I(X, \mu, t) \ud \mu
  \\ \nonumber
  0 &= \int_{-1}^{0} \mu I(X, \mu, t) \ud \mu
  +\int_{0}^{1} \mu I(X, \mu, t) \ud \mu
  \\ \label{eq:oneDFinRight}
  0 &= F(X, t) \,.
\end{align}
Thus, a reflecting boundary condition implies a zero-flux boundary condition.
\end{subequations}

Now, in 1-D, the \APone{} expression for the intensity given by
Eq.~\eqref{eq:approxIntensity2} simplifies to
\begin{equation} \label{eq:approxIntensityOned}
  I(x, \mu, t) \approx \frac{1}{2} \phi(x, t) + \mu g(x, \mu, t) F^i(x)
  -  \mu f(x,\mu,t) \pder{\phi}{x}(x, t)\,.
\end{equation}
where $g$ is determined by the transport equation
\begin{equation*}
  \frac{1}{c} \pder{g}{t}(x, \mu, t)
  + \mu \pder{g}{x}(x, \mu, t) + \sigma^\ast(x) g (x, \mu, t)
  = 0\,, \quad
0 < x < X,\  0 \le t < \Delta_t, \ -1 < \mu < 1,
\end{equation*}
with the initial condition
\begin{equation*}
  g(x, \mu, 0) = \frac{3}{2}\,,
\end{equation*}
and boundary conditions to be determined. The function $f$ is likewise
determined by a simple transport problem
\begin{equation*}
  \frac{1}{c} \pder{f}{t}(x, \mu, t)
  + \mu \pder{f}{x}(x, \mu, t) + \sigma^\ast(x) f (x, \mu, t)
  = \frac{1}{2} \,, \quad
0 < x < X,\  0 \le t < \Delta_t, \ -1 < \mu < 1,
\end{equation*}
with the initial condition
\begin{equation*}
  f(x, \mu, 0) = 0\,,
\end{equation*}
and some boundary conditions which we will calculate.

Substituting the approximate Eq.~\eqref{eq:approxIntensityOned} into the
low-order incident boundary condition Eq.~\eqref{eq:oneDFinLeft} gives the 1-D
equivalent of Eq.~\eqref{eq:bndyAnisotropic}:
\begin{align} \nonumber
  F^{\ell,\mathrm{in}}(t)
  &= \int_{0}^{1} \mu I(0, \mu, t) \ud \mu
  \\ \nonumber
  &= \int_{0}^{1} \mu \left[ \frac{1}{2} \phi(0, t) + \mu g(0, \mu, t) F^i(0)
  -  \mu f(0,\mu,t) \pder{\phi}{x}(0, t) \right] \ud \mu
  \\ \label{eq:bndyAnisotropicOned}
  &= \frac{1}{4} \phi(0, t)
  + \int_{0}^{1} \mu^2 g(0, \mu, t) \ud \mu F^i(0)
  - \int_{0}^{1} \mu^2 f(0, \mu, t) \ud \mu \pder{\phi}{x}(0, t) \,.
\end{align}
(If this were standard diffusion, we would substitute $I \approx (\phi - 3 D
\pder{\phi}{x})/2$ and end up with Marshak boundary conditions.)

Demanding that the boundary term $\mathrm{ \tilde b}$ is zero gives the 1-D
equivalent of Eqs.~\eqref{eqs:bndyTransport}, where $\vec{n} = -1$:
\begin{subequations} \label{eqs:bndyTransportOned}
\begin{align} \label{eq:bndyFOned}
  \frac{1}{2} \int_{-1}^{0} \mu f(0, \mu, t) \ud\mu
  &= - \int_{0}^{1} \mu^2 f(x, \mu, t) \ud\mu
 \\ \intertext{and} \label{eq:bndyGOned}
  \frac{1}{2} \int_{-1}^{0} \mu g(0, \mu, t) \ud\mu
  &= - \int_{0}^{1} \mu^2 g(x, \mu, t) \ud\mu \,.
\end{align}
\end{subequations}

On the right-hand side, TODO

\subsubsection{Steady-state case, incident boundary condition}
In the steady-state problem, $g(x, \mu, t)=0$, and the argument $t$
disappears.
%So we have
%\begin{equation*}
%  I(x, \mu)
%  \approx \frac{1}{2} \phi(0) - f(0,\mu) \mu \pder{\phi}{x}(0)\,,
%\end{equation*}
%and the boundary condition on $f$,
We only have one condition to satisfy, the boundary on $f$:
\begin{equation*}
  \int_{0}^{1} \mu^2 \underbrace{f(0, \mu)}_{\text{incident}} \ud\mu
  = \frac{1}{2} \int_{-1}^{0} \abs{\mu} 
  \underbrace{f(0, \mu) }_{\text{exiting}} \ud\mu
  \equiv \frac{1}{2} j^{\ell,\mathrm{out}}\,.
\end{equation*}
The term $j^{\ell,\mathrm{out}}$ is just the exiting partial current of the
transport problem that calculates $f$.

We can satisfy this condition by letting the incoming values of $f$ be
isotropic, $f(0, \mu)$ = $f^\mathrm{in}$ for $0 < \mu < 1$:
\begin{align*}
  \int_{0}^{1} \mu^2 f^\mathrm{in}\ud \mu
  = \frac{1}{2} j^{\ell,\mathrm{out}}
  \lra
  f^\mathrm{in} \frac{1}{3}
  = \frac{1}{2} j^{\ell,\mathrm{out}}
  \lra
  f^\mathrm{in} = \frac{3}{2} j^{\ell,\mathrm{out}}
\end{align*}
or
\begin{equation} \label{eq:bndyFOnedIncident}
  f(0, \mu) = \frac{3}{2} \int_{-1}^{0} \abs{\mu'} f(0, \mu') \ud\mu' \,.
\end{equation}
This is like a ``white'' boundary but with an albedo factor of 0.75: I would
call it a ``gray'' boundary except that would be ambiguous with the
radiation frequency treatment. I will just call it albedo-white from here
forward.

The low-order boundary condition on the left side of the problem, assuming we
enforce the boundary condition on $f$, simplifies to
\begin{align} \nonumber
  F^{\ell,\mathrm{in}}
  &= \frac{1}{4} \phi(0)
  - \int_{0}^{1} \mu^2 f(0, \mu) \ud \mu \pder{\phi}{x}(0) \,.
  \\ \nonumber
  F^{\ell,\mathrm{in}}
  &= \frac{1}{4} \phi(0)
  - \frac{1}{2} \int_{-1}^{0} \abs{\mu} f(0, \mu) \ud\mu \pder{\phi}{x}(0) \,.
  \\ \label{eq:bndyAnisotropicOnedSs}
  F^{\ell,\mathrm{in}}
  &= \frac{1}{4} \phi(0)
  - \frac{1}{2} j^{\ell,\mathrm{out}} \pder{\phi}{x}(0) \,.
\end{align}
where $j^\mathrm{\ell,out}(0) = \int_{-1}^{0} \mu f(0, \mu) \ud\mu$, which
is almost always calculated with transport solvers to determine leakage rates
from the problem.

Compare this to the standard Marshak boundary condition for diffusion:
\begin{equation*}
  F^{\ell,\mathrm{in}}
  = \frac{1}{4} \phi(0)
  - \frac{1}{2} \frac{1}{3\sigma(0)} \pder{\phi}{x}(0) \,.
\end{equation*}

So there are a couple of things to notice here:
\begin{itemize}
  \item The boundary condition actually requires the value of $f$ on the
    boundary; using $D=\int_{-1}^{1} \mu^2 f \ud \mu$ near the boundary is
    inappropriate.
  \item In an infinite homogeneous medium, where $\sigma$ is constant
    throughout the problem, this does \emph{not} reduce to the standard
    diffusion boundary conditions! We will get 
    Our AD method is more advanced than
    diffusion, because it accounts for some transport effects using $f$.
    Numerical results will show whether this difference (which leads to
    changing values of $D$ near a boundary, even in a semi-infinite
    homogeneous medium) is an improvement or detraction.
\end{itemize}
