% !TEX root = _individual/adDerivation.tex

%%%%%%%%%%%%%%%%%%%%%%%%%%%%%%%%%%%%%%%%%%%%%%%%%%%%%%%%%%%%%%%%%%%%%%%%%%%%%%%%
\chapter{Anisotropic diffusion}\label{chap:adDerivation}

Existing radiation transport methods, the most common of which were presented in
the last chapter, approximate the angular dependence of the intensity each in an
utterly different way. The crudest of these, diffusion, has only one unknown
$\phi$ at each point in space and time; the more complex such as \SN\ have many
unknowns, leading to greater accuracy but greater computer run time and memory
usage. In this chapter, we derive the anisotropic diffusion method, which
approximates the intensity in such a way as to retain an arbitrary amount of
anisotropy (like a true transport method) while solving for a single unknown
(like a diffusion method).

The previous work in anisotropic diffusion has only considered a steady-state
problem in an infinite medium \cite{Lar2009c,Mor2007}. The new derivation of
the anisotropic diffusion approximation presented in this chapter provides a
theoretical basis for using AD in time-dependent and nonlinear contexts, and it
furthermore addresses boundary conditions for the AD method.

%%%%%%%%%%%%%%%%%%%%%%%%%%%%%%%%%%%%%%%%%%%%%%%%%%%%%%%%%%%%%%%%%%%%%%%%%%%%%%%%
\section{A modified transport equation}\label{sec:capPsi}

We begin by considering the 3-D gray radiation transport equation inside a time
step
$0 \le t \le \Delta_t$, with opacities frozen at some value $\sigmast$ that
satisfies $\sigma(0) \le \sigmast \le \sigma(\Delta_t)$. The transport equation
for the radiation intensity $I$ is:
\begin{subequations} \label{eqs:fullTransport}
\begin{multline} \label{eq:fullTransportVol}
  \frac{1}{c} \pder{I}{t}(\vec{x}, \vec{\Omega}, t)
    + \vec{\Omega}\vd \grad I(\vec{x}, \vec{\Omega}, t)
    + \sigmast(\vec{x}) I (\vec{x}, \vec{\Omega}, t)
    \\ = \frac{1}{4\pi} \sigmast(\vec{x}) ac [T(\vec{x}, t)]^4
    + \frac{1}{4\pi} q_{r}(\vec{x}, t)
    \equiv \frac{1}{4\pi} Q(\vec{x}, t) \,,
\\
x \in V,\  0 \le t \le \Delta_t, \ \vec{\Omega} \in 4\pi\,.
\end{multline}
Some subset of the boundary has an incident radiation source,
represented by the following Dirichlet boundary condition:
\begin{equation} \label{eq:fullTransportBndy}
  I(\vec{x}, \vec{\Omega}, t) = I^b(\vec{x}, \vec{\Omega}, t) \,,
 \quad \vec{x} \in \partial V_b, \ \vec{\Omega} \vd \vec{n} < 0,
 \ 0 \le t \le \Delta_t\,.
\end{equation}
The remainder of the boundary is specularly reflecting:
\begin{equation} \label{eq:fullTransportRefl}
  I(\vec{x}, \vec{\Omega}, t)
  = I(\vec{x}, \vec{\Omega}_r, t)
  \,,
 \quad \vec{x} \in \partial V_r, \ \vec{\Omega} \vd \vec{n} < 0,
 \ 0 \le t \le \Delta_t\,.
\end{equation}
Here, $\vec{\Omega}_r$ is the reflected angle on a boundary surface with outward
normal $\vec{n}$:
\begin{equation} \label{eq:reflection}
  \vec{\Omega}_r = \vec{\Omega} - 2(\vec{\Omega} \vd \vec{n}) \vec{n}\,.
\end{equation}
The intensity has the initial condition
\begin{equation} \label{eq:fullTransportInit}
 I(\vec{x}, \vec{\Omega}, 0) = I^i(\vec{x}, \vec{\Omega}) \,,
 \quad \vec{x} \in V, \ \vec{\Omega} \in 4\pi\,,
\end{equation}
which is the solution from the previous time step or a user-specified initial
condition.
\end{subequations}

%%%%%%%%%%%%%%%%%%%%%%%%%%%%%%%%%%%%%%%%
The particle (or radiation energy) conservation equation is the zeroth moment of
the transport equation, obtained by integrating Eq.~\eqref{eq:fullTransportVol}
over all angles, $\int_{4\pi} (\cdot) \ud \Omega$:
\begin{subequations} \label{eqs:loEquations}
\begin{equation} \label{eq:loVol}
\frac{1}{c} \pder{\phi}{t} (\vec{x}, t)
  + \grad \vd\vec{F}(\vec{x}, t)
  + \sigmast(\vec{x}) \phi(\vec{x}, t)
  = \sigmast(\vec{x}) ac [T(\vec{x}, t)]^4 + q_{r}(\vec{x}, t)
  = Q(\vec{x}, t)\,.
\end{equation}
%for $\vec{x} \in V$ and $0 \le t \le \Delta_t$.
Doing the same to the initial condition, Eq.~\eqref{eq:fullTransportInit}, gives 
\begin{equation} \label{eq:loInit}
\phi(\vec{x}, 0) = \int_{4\pi}  I^i(\vec{x},
\vec{\Omega}) \ud\Omega = \phi^i(\vec{x}) \,.
\end{equation}
\end{subequations}
Here, we have used the first two moments of the
intensity: the scalar intensity $\phi=\int_{4\pi} I \ud\Omega$, and the
radiation flux $\vec{F} = \int_{4\pi} \vec{\Omega} I \ud\Omega$ (see \S\ref{sec:bgTrtEquations}).
Equations~\eqref{eqs:loEquations} form the first piece of the ``low-order'' set
of equations for the AD method. As in other approximate methods, we seek to
``close'' the low-order equation with an approximation to the radiation
flux $\vec{F}$.

%%%%%%%%%%%%%%%%%%%%%%%%%%%%%%%%%%%%%%%%
Adding $\vec{\Omega}\vd \grad \phi$ to both sides of Eq.~\eqref{eq:loVol},
multiplying by $\frac{1}{4\pi}$, and subtracting from
Eq.~\eqref{eq:fullTransportVol}, we cancel the isotropic radiation source $Q$
to obtain the following equation:
\begin{equation*}
  \frac{1}{c} \pder{}{t}\left[ I
  - \frac{1}{4\pi} \phi \right]
    + \vec{\Omega}\vd \grad \left[ I
  - \frac{1}{4\pi} \phi \right]
   + \sigmast \left[ I
  - \frac{1}{4\pi} \phi \right]
  = 0 + \frac{1}{4\pi} \grad \vd\vec{F} -
  \frac{1}{4\pi} \vec{\Omega}\vd \grad \phi\,.
\end{equation*}
Defining
\begin{equation} \label{eq:capPsi}
  \Psi(\vec{x}, \vec{\Omega}, t) \equiv I(\vec{x}, \vec{\Omega}, t) -
  \frac{1}{4\pi} \phi(\vec{x}, t)\,,
\end{equation}
we have a transport equation for $\Psi$:
\begin{equation} \label{eq:capPsiVol}
  \frac{1}{c} \pder{}{t}\Psi(\vec{x}, \vec{\Omega}, t)
    + \vec{\Omega}\vd \grad \Psi(\vec{x}, \vec{\Omega}, t)
    + \sigmast(\vec{x}) \Psi(\vec{x}, \vec{\Omega}, t)
  = \frac{1}{4\pi} \grad \vd\vec{F}(\vec{x}, t) -
  \frac{1}{4\pi} \vec{\Omega}\vd \grad \phi(\vec{x}, t)\,,
\end{equation}
which has the same domain as the original transport equation: $\vec{x} \in
V$, $\vec{\Omega} \in 4\pi$, $0 \le t \le \Delta_t$. The
transport equation for the exact intensity $I$ satisfies
Eq.~\eqref{eq:capPsiVol} exactly:
no approximation has yet been made. However, instead of having an isotropic source term
on the right hand side, the equation for $\Psi$ has an anisotropic source term
that depends on the
unknowns $\phi$ and $\vec{F}$.

The quantity $\Psi$ can be thought of as a spherical harmonic expansion
of $I$ with the first (isotropic) term removed, then reconstituted: it is a full
representation of the anisotropic components of $\Psi$. By definition, $\Psi$
satisfies two important identities.
\begin{subequations} \label{eqs:capPsiIdentities}
Its zeroth angular moment is identically zero:
\begin{equation} \label{eq:capPsiZeroth}
  \int_{4\pi} \Psi(\vec{x}, \vec{\Omega}, t) \ud\Omega
  = \int_{4\pi} I \ud\Omega
  - \frac{1}{4\pi}\int_{4\pi} \ud\Omega \,\phi
  = \phi - \phi
  = 0\,,
\end{equation}
and its first moment is the radiation flux:
\begin{equation} \label{eq:capPsiFirst}
  \int_{4\pi} \vec{\Omega} \Psi(\vec{x}, \vec{\Omega}, t) \ud\Omega
  = \int_{4\pi} \vec{\Omega} I \ud\Omega
  - \frac{1}{4\pi} \int_{4\pi} \vec{\Omega} \ud\Omega \,\phi
  = \vec{F} - \vec{0}
  = \vec{F}(\vec{x}, t)\,.
\end{equation}
\end{subequations}
All higher moments of $\Psi$ are also identical to those of $I$.

Now we formulate boundary conditions for $\Psi$ from the transport equation.
For specified incident radiation boundaries, we subtract $\phi/4\pi$ from 
Eq.~\eqref{eq:fullTransportBndy}:
\begin{align}\nonumber
  I(\vec{x}, \vec{\Omega}, t) - \frac{1}{4\pi} \phi(\vec{x}, t)
  &= I^b(\vec{x}, \vec{\Omega}, t) - \frac{1}{4\pi} \phi(\vec{x}, t)
  \\ 
  \intertext{Substituting Eq.~\eqref{eq:capPsi} into the left hand side gives a
  boundary condition for $\Psi$:
  } \label{eq:capPsiBndy}
 \Psi(\vec{x}, \vec{\Omega}, t) 
  &=I^b(\vec{x}, \vec{\Omega}, t) - \frac{1}{4\pi} \phi(\vec{x}, t)\,.
\end{align}
Likewise, $\phi/4\pi$ is subtracted from the reflecting boundary condition, 
Eq.~\eqref{eq:fullTransportBndy}:
\begin{align}\nonumber
  I(\vec{x}, \vec{\Omega}, t) - \frac{1}{4\pi} \phi(\vec{x}, t)
  &= I(\vec{x}, \vec{\Omega} - 2(\vec{\Omega} \vd \vec{n}) \vec{n}, t)
   - \frac{1}{4\pi} \phi(\vec{x}, t)\,.
  \\ \label{eq:capPsiRefl}
 \Psi(\vec{x}, \vec{\Omega}, t) 
  &= \Psi(\vec{x}, \vec{\Omega}_r, t)\,.
\end{align}

Finally, to get an initial condition for $\Psi$, we
multiply the low-order initial condition \eqref{eq:loInit} by $1/4\pi$ and
subtract it from the initial condition for the intensity,
Eq.~\eqref{eq:fullTransportInit}:
\begin{align}\nonumber
  I(\vec{x}, \vec{\Omega}, 0) - \frac{1}{4\pi}\phi(\vec{x}, 0)
 &= I^i(\vec{x}, \vec{\Omega}) - \frac1{4\pi} \phi^i(\vec{x})
 \\ \label{eq:capPsiInit}
 \Psi(\vec{x}, \vec{\Omega}, 0)
 &\equiv \Psi^i(\vec{x}, \vec{\Omega}, t)
 \,.
\end{align}

Equations \eqref{eq:capPsiVol},~\eqref{eq:capPsiBndy},%
~\eqref{eq:capPsiRefl}, and~\eqref{eq:capPsiInit} constitute the transport
equation for $\Psi$; they are satisfied by the exact transport solution. They
were derived without approximation but at this stage they are not particularly
helpful: instead of a ``known'' isotropic source $Q$ on the right-hand side of
the equation, the transport equation for $\Psi$ has an anisotropic source that
depends on the zeroth and first moments of the solution.

Cryptic as it may appear, this form is the starting point of the AD derivation;
it serves as a replacement for the first angular moment of the transport
equation.  In fact, as described later in \S\ref{sec:adDiscDiff}, Fick's law can
be derived from the equations for $\Psi$
by using the linearly anisotropic approximation \mbox{$\Psi\approx \frac{3}{4\pi}
\vec{\Omega}\vd\vec{F}$}. Just as Fick's law can be derived from the first moment
equation using the linear-in-angle approximation, a new ``anisotropic'' analogue
to Fick's law can be derived from the transport equation for $\Psi$ using a
higher fidelity approximation.

%%%%%%%%%%%%%%%%%%%%%%%%%%%%%%%%%%%%%%%%%%%%%%%%%%%%%%%%%%%%%%%%%%%%%%%%%%%%%%%%
\section{Anisotropic diffusion derivation}\label{sec:adDerivation}
The anisotropic diffusion equations are obtained from the differential transport
equation for $\Psi$ by using an integral transport equation and making
assumptions about the weakness of gradients in the solution. Some care must be
taken in deriving
a suitable boundary condition, because we desire a simple expression for
$\vec{F}$ that only depends on \textsl{(i)} the low-order unknown $\phi$ and
\textsl{(ii)} some quantity that depends only
on problem data such as $\sigma$.

%%%%%%%%%%%%%%%%%%%%%%%%%%%%%%%%%%%%%%%%%%%%%%%%%%%%%%%%%%%%%%%%%%%%%%%%%%%%%%%%
\subsection{Modification for boundary condition treatment}
In order to formulate transport-matched boundary conditions---%
i.e.~boundary conditions that yield the transport solution in the interior when
the anisotropic diffusion approximation is accurate---%
we separate $\Psi$ into an ``interior'' solution $\tilde\Psi$ and a ``boundary
layer'' solution $\Psi_\mathrm{bl}$:
\begin{equation} \label{eq:boundaryLayerPsi}
  \Psi(\vec{x}, \vec{\Omega}, t)
  = \tilde\Psi(\vec{x}, \vec{\Omega}, t)
  + \Psi_\mathrm{bl}(\vec{x}, \vec{\Omega}, t)\,.
\end{equation}
The boundary layer problem is formulated to tend to zero rapidly with distance
from the outer boundary.

\begin{subequations} \label{eqs:tCapPsi}
  The interior transport equation resembles Eq.~\eqref{eq:capPsiBndy} and
  accounts for the right hand side of the equation:
\begin{multline} \label{eq:tCapPsiVol}
  \frac{1}{c} \pder{}{t}\tilde\Psi(\vec{x}, \vec{\Omega}, t)
    + \vec{\Omega}\vd \grad \tilde\Psi(\vec{x}, \vec{\Omega}, t)
    + \sigmast(\vec{x}) \tilde\Psi(\vec{x}, \vec{\Omega}, t)
  \\
  = \frac{1}{4\pi} \grad \vd\vec{F}(\vec{x}, t) -
  \frac{1}{4\pi} \vec{\Omega}\vd \grad \phi(\vec{x}, t)
  \equiv \hat Q(\vec{x}, \vec{\Omega}, t)\,,
  \qquad
x \in V,\  0 \le t \le \Delta_t, \ \vec{\Omega} \in 4\pi.
\end{multline}
However, it does not need to use a physical boundary condition. We will
define the incident boundary condition for the interior solution
to be
\begin{equation} \label{eq:tCapPsiBndy}
 \tilde\Psi(\vec{x}, \vec{\Omega}, t) 
  = - \zeta(\vec{x}, \vec{\Omega}, t) \vec{\Omega}\vd \grad \phi(\vec{x}, t)
  \equiv \tilde\Psi^b(\vec{x}, \vec{\Omega}, t) \,,
\end{equation}
for $\vec{x} \in \partial V_b$, $\vec{\Omega} \vd \vec{n} < 0$,
$0 \le t \le \Delta_t$. The function $\zeta$, which lives on the boundary for
incident directions, is yet
to be determined.
The reflecting boundary condition is effectively unchanged from
Eq.~\eqref{eq:capPsiRefl}:
\begin{equation} \label{eq:tCapPsiRefl}
 \tilde\Psi(\vec{x}, \vec{\Omega}, t) 
  = \tilde\Psi(\vec{x}, \vec{\Omega}_r, t)
  \equiv \tilde\Psi^b(\vec{x}, \vec{\Omega}, t) \,,
\end{equation}
for $\vec{x} \in \partial V_r$, $\vec{\Omega} \vd \vec{n} < 0$,
$0 \le t \le \Delta_t$.
Finally, the interior solution has the same initial condition as
Eq.~\eqref{eq:capPsiInit}:
\begin{equation} \label{eq:tCapPsiInit}
 \tilde\Psi(\vec{x}, \vec{\Omega}, 0)
 = \Psi^i(\vec{x}, \vec{\Omega}, t)\,.
\end{equation}
\end{subequations}

The corresponding transport problem for $\Psi_\mathrm{bl}$ 
complements the interior problem for $\tilde\Psi$, satisfying the exact
transport equations for
$\Psi$ using the definition in Eq.~\eqref{eq:boundaryLayerPsi}.
\begin{subequations} \label{eqs:blCapPsi}
It has the same left-hand side as Eq.~\eqref{eq:capPsiBndy} but no internal
source:
\begin{equation} \label{eq:blCapPsiVol}
  \frac{1}{c} \pder{}{t}\Psi_\mathrm{bl}(\vec{x}, \vec{\Omega}, t)
    + \vec{\Omega}\vd \grad \Psi_\mathrm{bl}(\vec{x}, \vec{\Omega}, t)
    + \sigmast(\vec{x}) \Psi_\mathrm{bl}(\vec{x}, \vec{\Omega}, t)
  = 0\,, \quad
x \in V, \ \vec{\Omega} \in 4\pi,\  0 \le t \le \Delta_t.
\end{equation}
The incident boundary condition accounts for the true incident boundary source
as well as the $\zeta$ term we introduced:
\begin{equation} \label{eq:blCapPsiBndy}
 \Psi_\mathrm{bl}(\vec{x}, \vec{\Omega}, t) 
  = I^b(\vec{x}, \vec{\Omega}, t) - \frac{1}{4\pi} \phi(\vec{x}, t)
  + \zeta(\vec{x}, \vec{\Omega}, t) \vec{\Omega}\vd \grad \phi(\vec{x}, t)
  \equiv \Psi_\mathrm{bl}^b(\vec{x}, \vec{\Omega}, t) \,.
\end{equation}
For $\vec{x} \in \partial V_r$, the boundary layer solution is reflecting:
\begin{equation} \label{eq:blCapPsiRefl}
 \Psi_\mathrm{bl}(\vec{x}, \vec{\Omega}, t) 
  = \Psi_\mathrm{bl}(\vec{x}, \vec{\Omega}_r, t)\,.
\end{equation}
Finally, because $\tilde\Psi$ accounts for the initial condition, the initial
condition for $\Psi_\mathrm{bl}$ is zero:
Eq.~\eqref{eq:capPsiInit}:
\begin{equation} \label{eq:blCapPsiInit}
 \Psi_\mathrm{bl}(\vec{x}, \vec{\Omega}, 0)
 = 0\,.
\end{equation}
\end{subequations}

If we add Eqs.~\eqref{eqs:blCapPsi} to Eqs.~\eqref{eqs:tCapPsi}, we recover the
original transport equation for $\Psi$. However, unlike the original transport
equation,
the separation of the interior and boundary layer transport equations allows us
to formulate boundary
conditions for the anisotropic diffusion method.

%%%%%%%%%%%%%%%%%%%%%%%%%%%%%%%%%%%%%%%%%%%%%%%%%%%%%%%%%%%%%%%%%%%%%%%%%%%%%%%%
\subsection{Integral transport equation}\label{sec:integralTransport}
The integral transport equation is obtained \cite{Pri2010} by taking the
right-hand side of a transport equation to be a known quantity, then integrating
along the characteristic ray $\vec{\Omega}$, accumulating particles born along
the ray and attenuating by collisions during their flight. Instead of taking
the integral transport equation for $I$, we use the integral form of the
``anisotropic'' interior solution $\tilde\Psi$ from Eqs.~\eqref{eqs:tCapPsi}:
\begin{subequations} \label{eqs:inverseTransport}
  \begin{align} \label{eq:inverseTransportFull}
  \begin{split}
    \tilde\Psi(\vec{x}, \vec{\Omega}, t)
    &=
    \tilde\Psi^b(\vec{x} - s_b\vec{\Omega}, \vec{\Omega}, t - s_b/c)
    \eexp^{ -\tau(\vec{x}, \vec{x} - s_b \vec{\Omega})}
    U(ct - s_b)
    \\
    &\qquad + \Psi^i( \vec{x} - ct \vec{\Omega}, \vec{\Omega})
    \eexp^{ -\tau(\vec{x}, \vec{x} - ct \vec{\Omega})}
    U( s_b - ct)
    \\
    &\qquad +  \int_{0}^{s_b}
    \left[ \hat Q(\vec{x} - s \vec{\Omega}, \vec{\Omega}, t-s/c)
    \right]
    \eexp^{ -\tau(\vec{x}, \vec{x} - s \vec{\Omega})}
    \ud s
\,.
  \end{split}
  \end{align}
  Here, $U(x)$ is the Heaviside function, unity for $x \ge 0$ and zero
  otherwise. The optical thickness of the medium between points $\vec{x}$ and
  $\vec{x}'$ along the direction $\vec{\Omega} = (\vec{x}'-
  \vec{x})/\norm{\vec{x}'-\vec{x}}$ is 
  \begin{equation} \label{eq:fullTauDefinition}
    \tau(\vec{x}, \vec{x}') = \int_{0}^{\norm{\vec{x} -
    \vec{x}'}} \sigmast(\vec{x}-s\vec{\Omega}) \ud s \,.
  \end{equation}
  The quantity $s_b$ is the distance to the boundary along $-\vec{\Omega}$ from
  $\vec{x}$.
\end{subequations}

For brevity, we write Eq.~\eqref{eq:inverseTransportFull} as a sum of linear
operators, each of which corresponds to the contribution of a nonlocal
particle source to the local point $(\vec{x},t)$:
\begin{align} \nonumber
  \tilde\Psi(\vec{x}, \vec{\Omega}, t)
    &\equiv \lopinv{b}{\tilde\Psi^b}
    + \lopinv{i}{\Psi^i}
    + \lopinv{v}{\hat Q}
    \\ \label{eq:inverseTransportBrief}
  \begin{split}
    \tilde\Psi(\vec{x}, \vec{\Omega}, t)
    &\equiv
    -\lopinv{b}{\zeta \vec{\Omega}\vd \grad \phi}_{\partial V_b}
    + \lopinv{b}{\tilde\Psi(\vec{x}, \vec{\Omega}_r, t)}_{\partial V_r}
    + \lopinv{i}{\Psi^i}
  \\&\qquad
    + \lopinv{v}{\frac{1}{4\pi} \grad \vd\vec{F} }
    - \lopinv{v}{\frac{1}{4\pi} \vec{\Omega}\vd \grad \phi}
    \,.
  \end{split}
\end{align}

Because this equation is exact approximation,
the integral equation~\eqref{eq:inverseTransportFull} still contains
the unknowns $\phi$ and $\vec{F}$ (as well as the exiting $\tilde\Psi$ on
any reflecting boundaries). Furthermore, the local value of
$\tilde\Psi$ depends on the value of those unknowns along the entire
characteristic ray.

Our goal is to make reasonable assumptions that reduce this equation to 
a form that depends only on local, low-order unknowns and nonlocal
knowns. Taking the first moment of the resulting approximate $\tilde\Psi$ will
yield a simple relationship between $\vec{F}$ and $\phi$.

%%%%%%%%%%%%%%%%%%%%%%%%%%%%%%%%%%%%%%%%
\subsection{Asymptotic ansatz and expansions}
To simplify the integral transport
equation~\eqref{eq:inverseTransportBrief}, it is necessary to make some
approximations. We make an ansatz that the spatial gradients of the intensity
are weak, the intensity varies slowly in time, and the solution is mildly
(but not necessarily linearly) anisotropic:
\begin{align} \label{eq:ansatz}
  I &= O(1), &
  \grad I &= O(\epsilon), &
  \frac{1}{c}\pder{I}{t} &= O(\epsilon^2), &
  \vec{F} = \int_{4\pi} \vec{\Omega} I\ud\Omega &= O(\epsilon).
\end{align}
These scalings are similar to those used in an asymptotic derivation of the
standard diffusion equation. However, rather than taking an asymptotic expansion
of the transport equations for $I$, we apply the scaling to the integral
equation for $\tilde\Psi$.

The contribution from $\grad \vd\vec{F}$ is $O(\epsilon^2)$, as the term
contains an $O(\epsilon)$ derivative as well as the $O(\epsilon)$ radiation
flux (first moment). Additionally, the contribution from the initial condition is
$O(\epsilon^2)$. Neglecting those terms in Eq.~\eqref{eq:inverseTransportBrief},
the integral transport equation becomes:
\begin{equation} \label{eq:approxPsi1}
  \tilde\Psi \approx 
  -\lopinv{b}{\zeta \vec{\Omega}\vd \grad \phi}_{\partial V_b}
  + \lopinv{b}{\tilde\Psi(\vec{\Omega}_r)}_{\partial V_r}
  - \lopinv{v}{\frac{1}{4\pi} \vec{\Omega}\vd \grad \phi}
  + O(\epsilon^2)\,.
\end{equation}
The terms that we have retained are all $O(\epsilon)$, so that $\tilde\Psi =
O(\epsilon)$ and accordingly $\vec{F} = \int_{4\pi} \vec{\Omega} \tilde\Psi
\ud\Omega = O(\epsilon)$.

We can also expand the nonlocal variables in Eq.~\eqref{eq:inverseTransportFull}
about the local spatiotemporal point in powers of $\epsilon$:
\begin{equation} \label{eq:taylorPhi}
  \phi(\vec{x} - s \vec{\Omega}, t-s/c)
  \sim \phi(\vec{x},t) - s \left( \frac{1}{c} \pder{}{t} + \vec{\Omega} \vd
  \grad  \right) \phi (\vec{x}, t) + O(\epsilon^2) \sim \phi(\vec{x},t) +
  O(\epsilon) \,.
\end{equation}
This Taylor series will enable, to leading order, the approximation of nonlocal
quantities inside the operators $\lopinv{}{\cdot}$ by local quantities.

%%%%%%%%%%%%%%%%%%%%%%%%%%%%%%%%%%%%%%%%
\subsection{Approximating the streaming integral term}\label{sec:adStreaming}
The streaming term in Eq.~\eqref{eq:approxPsi1} is (see
Eq.~\eqref{eq:inverseTransportFull}):
\begin{align*}
- \lopinv{v}{\frac{1}{4\pi} \vec{\Omega}\vd \grad \phi(\vec{x}, t)}
  &= \int_{0}^{\norm{\vec{x} - \vec{x}_b}}
    \left[ -\frac1{4\pi}\vec{\Omega}\vd \grad \phi(\vec{x} - s \vec{\Omega},
    t-s/c)
    \right]
    \eexp^{ -\tau(\vec{x}, \vec{x} - s \vec{\Omega})}
    \ud s\,.
\end{align*}
This integral describes the contribution from the volumetric source  along
$\vec{\Omega}$, evaluated at a prior
point in time ($t-s/c$, the point along $s$ at which a particle would travel
to $\vec{x}$ at time $t$), attenuated by the medium along the way by collisions.

We now make our first approximation by expanding the distant $\phi(\vec{x} - s
\vec{\Omega}, t-s/c)$ about the local $\phi(\vec{x}, t)$ using
Eq.~\eqref{eq:taylorPhi}. Thus,
\begin{equation*}
  \grad \phi(\vec{x} - s \vec{\Omega}, t-s/c)
  = \grad \left[ \phi(\vec{x}, t) + O(\epsilon) \right]
  = \grad \phi(\vec{x}, t) + O(\epsilon^2).
\end{equation*}
The expansion is a good approximation if $\phi$ is smooth.

We can now move the unknown $\phi$ outside the integral,
because it is no longer a function of $s$:
\begin{align}\nonumber
- \lopinv{v}{\frac{1}{4\pi} \vec{\Omega}\vd \grad \phi(\vec{x}, t)}
  &\approx \int_{0}^{\norm{\vec{x} - \vec{x}_b}}
    \left[ -\frac1{4\pi}\vec{\Omega}\vd \grad \phi(\vec{x},t) \right]
    \eexp^{ -\tau(\vec{x}, \vec{x} - s \vec{\Omega})}
    \ud s
  \\\nonumber
  &= - \int_{0}^{\norm{\vec{x} - \vec{x}_b}}
    \left[ \frac1{4\pi}\right]
    \eexp^{ -\tau(\vec{x}, \vec{x} - s \vec{\Omega})} \ud s \,
    \vec{\Omega}\vd \grad \phi(\vec{x},t)
  \\\label{eq:streamingApprox}
  &= - \lopinv{v}{ \frac1{4\pi} } \vec{\Omega}\vd \grad \phi(\vec{x},t)
  \,.
\end{align}
Thus the function of nonlocal unknowns has been approximated by a nonlocal known, the
$\lopinv{v}{\cdot}$ term, and a local unknown, $\phi(\vec{x},t)$.

%%%%%%%%%%%%%%%%%%%%%%%%%%%%%%%%%%%%%%%%
\subsection{Approximating the incident boundary integral term}
\label{sec:adIncident}
The incident boundary term in Eq.~\eqref{eq:approxPsi1} is
\begin{multline*}
-\lopinv{b}{\zeta \vec{\Omega}\vd \grad \phi}_{\partial V_b}
  = -\left[\zeta(\vec{x} - s_b\vec{\Omega}, \vec{\Omega}, t - s_b/c)
   \vec{\Omega}\vd \grad \phi(\vec{x} - s_b\vec{\Omega}, t - s_b/c) \right]
   \\
\times
    \eexp^{ -\tau(\vec{x}, \vec{x} - s_b \vec{\Omega})}
    U(ct - s_b) \,.
\end{multline*}
It accounts for particles that start their life at an incident radiation
boundary
inside the current time step and stream along $\vec{\Omega}$, attenuated by
collisions along their path. As the optical thickness between
$\vec{x}$ and the boundary increases, this term vanishes
exponentially quickly.

Now we apply the Taylor series expansion from Eq.~\eqref{eq:taylorPhi} to
$\phi$, but not to $\zeta$:
\begin{equation*}
  \grad \phi(\vec{x} - s_b\vec{\Omega}, t - s_b/c)
  \approx \grad \phi(\vec{x}, t) + O(\epsilon^2)\,,
\end{equation*}
and we discard the $O(\epsilon^2)$ term. This gives
\begin{align} \nonumber
-\lopinv{b}{\zeta \vec{\Omega}\vd \grad \phi}_{\partial V_b}
&\approx -\left[\zeta(\vec{x} - s_b\vec{\Omega}, \vec{\Omega}, t - s_b/c) \right]
  \eexp^{ -\tau(\vec{x}, \vec{x} - s_b \vec{\Omega})} U(ct - s_b)
  \vec{\Omega}\vd \grad \phi(\vec{x}, t)
 \\ \label{eq:bndyApprox}
&= -\lopinv{b}{\zeta}_{\partial V_b} \vec{\Omega}\vd \grad \phi \,.
\end{align}

%%%%%%%%%%%%%%%%%%%%%%%%%%%%%%%%%%%%%%%%
\subsection{Approximating the reflecting boundary integral term}\label{sec:derReflBc}
For a moment, let us consider a problem without reflecting boundaries, so
$\partial V_b = \partial V$. At this point, Eq.~\eqref{eq:approxPsi1} has been
reduced to
\begin{align*}
  \tilde\Psi
  &= -\lopinv{b}{\zeta \vec{\Omega}\vd \grad \phi}
    + \lopinv{i}{\Psi^i}
    + \lopinv{v}{\frac{1}{4\pi} \grad \vd\vec{F} }
    - \lopinv{v}{\frac{1}{4\pi} \vec{\Omega}\vd \grad \phi}
\\
  &\approx
  -\lopinv{b}{\zeta} \vec{\Omega}\vd \grad \phi
  - \lopinv{v}{\frac{1}{4\pi}} \vec{\Omega}\vd \grad \phi \,.
\end{align*}
Now the decision to choose the particular form for the boundary condition in
Eq.~\eqref{eq:tCapPsiBndy}
is clear: under the systematic approximations made so far, the interior solution
$\tilde\Psi$ can be written
\begin{equation*}
  \tilde\Psi(\vec{x}, \vec{\Omega}, t)
  = - \left\{ \lopinv{b}{\zeta} + \lopinv{v}{\frac{1}{4\pi}}
  \right\} \vec{\Omega}\vd \grad \phi
  \equiv - f(\vec{x}, \vec{\Omega}) \vec{\Omega}\vd \grad \phi(\vec{x}, t)\,.
\end{equation*}
Let us assume that an approximation to the reflecting boundary condition can
be made that, in the general case with mixed reflecting and incident
boundaries, also allows us to write 
\begin{equation*}
  \tilde\Psi(\vec{x}, \vec{\Omega}, t)
  \approx - f(\vec{x}, \vec{\Omega}) \vec{\Omega}\vd \grad \phi(\vec{x}, t)\,.
\end{equation*}

Substituting this approximate $\tilde\Psi$ into the reflecting boundary term in
Eq.~\eqref{eq:inverseTransportBrief} yields
\begin{align*}
\lopinv{b}{\tilde\Psi(\vec{\Omega}_r)}_{\partial V_r}
  &= \left[\tilde\Psi(\vec{x} - s_b\vec{\Omega}, \vec{\Omega}_r, t - s_b/c)
  \right]
    \eexp^{ -\tau(\vec{x}, \vec{x} - s_b \vec{\Omega})}
    U(ct - s_b)
\\
  &= \left[-f(\vec{x} - s_b\vec{\Omega}, \vec{\Omega}_r) \vec{\Omega}_r 
  \vd \grad \phi(\vec{x} - s_b\vec{\Omega}, t - s_b/c) \right]
  \eexp^{ -\tau(\vec{x}, \vec{x} - s_b \vec{\Omega})}
  U(ct - s_b) \,.
\\ 
\intertext{Now we write $\vec{\Omega}_r$ using its definition in
Eq.~\eqref{eq:reflection}.
}
\lopinv{b}{\tilde\Psi(\vec{\Omega}_r)}_{\partial V_r}
  &= - \lopinv{b}{f(\vec{x} - s_b\vec{\Omega}, \vec{\Omega}_r)
\left(  \vec{\Omega} - 2(\vec{\Omega} \vd \vec{n}) \vec{n} \right)
  \vd \grad \phi(\vec{x} - s_b\vec{\Omega}, t - s_b/c) }
  \\
  &= -\lopinv{b}{f(\vec{x}_b,\vec{\Omega}_r) \vec{\Omega} \vd \grad
  \phi(\vec{x}_b,t_b)}
  + \lopinv{b}{f(\vec{x}_b,\vec{\Omega}_r) 2(\vec{\Omega} \vd \vec{n}) \vec{n} \vd
  \grad \phi(\vec{x}_b,t_b) } \,.
\\ 
\intertext{On a reflecting boundary at any point $\vec{x}_b$, the exact
intensity satisfies $\vec{n} \vd
\grad I = 0$, which also means $\vec{n} \vd \grad \phi=0$. Thus, the second
term is zero, and the reflecting boundary term is
}
\lopinv{b}{\tilde\Psi(\vec{\Omega}_r)}_{\partial V_r}
&= -\lopinv{b}{f \vec{\Omega} \vd \grad \phi} \,.
\\ \intertext{Now, as with in the incident boundary condition, we apply the
Taylor
series expansion from Eq.~\eqref{eq:taylorPhi} to $\phi$ but not to $f$:
}
\lopinv{b}{\tilde\Psi(\vec{\Omega}_r)}_{\partial V_r}
&\approx
-\left[f(\vec{x} - s_b\vec{\Omega}, \vec{\Omega}_r)
  \right]
  \eexp^{ -\tau(\vec{x}, \vec{x} - s_b \vec{\Omega})}
  U(ct - s_b) \vec{\Omega} \vd \grad \phi(\vec{x}, t)\,,
\end{align*}
or, in the more simplified form,
\begin{equation} \label{eq:reflApprox}
\lopinv{b}{\tilde\Psi(\vec{\Omega}_r)}_{\partial V_r}
\approx  
- \lopinv{b}{f(\vec{\Omega}_r)}_{\partial V_r}
\vec{\Omega} \vd \grad \phi(\vec{x}, t) \,.
\end{equation}

This has the same form as the other approximations to the term, a property
crucial
to forming the anisotropic diffusion approximation.

%%%%%%%%%%%%%%%%%%%%%%%%%%%%%%%%%%%%%%%%
\subsection{Anisotropic diffusion approximation to \texorpdfstring{$\Psi$}{Psi}}
Substituting Eqs.~\eqref{eq:streamingApprox},~\eqref{eq:bndyApprox},
and~\eqref{eq:reflApprox} into Eq.~\eqref{eq:approxPsi1} gives a nearly complete
approximation to the anisotropic component of the angular intensity,
$\Psi=I-\frac{1}{4\pi}\phi$:
\begin{align} \nonumber
  \tilde\Psi
  &\approx 
- \lopinv{b}{\zeta}_{\partial V_b} \vec{\Omega}\vd \grad \phi
- \lopinv{b}{f(\vec{\Omega}_r)}_{\partial V_r}
  \vec{\Omega}\vd \grad \phi
- \lopinv{v}{\frac{1}{4\pi}}  \vec{\Omega}\vd \grad \phi
\\ \label{eq:approxPsi2}
  \tilde\Psi &= 
- \left\{ \lopinv{b}{\zeta}_{\partial V_b} 
+ \lopinv{b}{f(\vec{\Omega}_r)}_{\partial V_r}
+ \lopinv{v}{\frac{1}{4\pi}} \right\} \vec{\Omega}\vd \grad \phi
\\ \label{eq:approxPsi3}
\tilde\Psi(\vec{x}, \vec{\Omega}, t) &= - f(\vec{x}, \vec{\Omega})
\vec{\Omega}\vd \grad \phi(\vec{x}, t)\,.
\end{align}

Here, we have defined
\begin{equation*}
  f(\vec{x}, \vec{\Omega})
  \equiv \lopinv{b}{\zeta}_{\partial V_b} 
+ \lopinv{b}{f(\vec{\Omega}_r)}_{\partial V_r}
+ \lopinv{v}{\frac{1}{4\pi}}\,,
\end{equation*}
which is an integral transport equation [see Eqs.~\eqref{eqs:inverseTransport}].
Converting the integral transport equation to a differential
transport equation shows $f$ to be the solution to a purely absorbing transport
problem with a uniform, isotropic unit source:
\begin{subequations} \label{eqs:fFull}
  \begin{equation} \label{eq:fFullVol}
    \vec{\Omega}\vd \grad f(\vec{x}, \vec{\Omega})
    + \sigmast f (\vec{x}, \vec{\Omega})
  = \frac{1}{4\pi} \,, \quad x \in V,\ \vec{\Omega} \in 4\pi\,,
  \end{equation}
with to-be-determined boundary conditions where the incident radiation is specified,
\begin{equation} \label{eq:fFullBndy}
  f(\vec{x}, \vec{\Omega}) = \zeta(\vec{x}, \vec{\Omega}) \,,
 \quad \vec{x} \in \partial V_b, \ \vec{\Omega} \vd \vec{n} < 0\,.
\end{equation}
  and with reflecting boundary conditions where the physical problem is
  reflecting,
\begin{equation} \label{eq:fFullRefl}
  f(\vec{x}, \vec{\Omega}) = f(\vec{x}, \vec{\Omega}_r) \,,
 \quad \vec{x} \in \partial V_r, \ \vec{\Omega} \vd \vec{n} < 0\,.
\end{equation}
\end{subequations}

Note that the transport equation is steady-state because $\frac{1}{c}\pder{}{t}=
O(\epsilon^2)$. Accordingly, $\zeta$ is restricted to a function constant within
the time step.

%%%%%%%%%%%%%%%%%%%%%%%%%%%%%%%%%%%%%%%%
\subsection{Approximate radiation flux}
Now we have an equation for $\tilde\Psi(\vec{x}, \vec{\Omega}, t)$ as a
separable
function of this simple transport equation $f(\vec{x}, \vec{\Omega})$ and the
scalar intensity $\phi(\vec{x},t)$.
We desire a simple low-order equation that provides a closure for the unknown
radiation flux $\vec{F}(\vec{x},t)$ in the radiation conservation
equation~\eqref{eq:loVol}.

We use the property from Eq.~\eqref{eq:capPsiFirst}
that the first moment of $\Psi$ is the flux $\vec{F}$. Operating on 
Eq.~\eqref{eq:approxPsi3} by $\int_{4\pi} \vec{\Omega} (\cdot) \ud\Omega$, we
obtain the following simple expression:
\begin{align} \nonumber
  \vec{F}(\vec{x}, t)
  &= \int_{4\pi} \vec{\Omega} \tilde \Psi(\vec{x}, \vec{\Omega}, t) \ud\Omega
  \\ \nonumber
  &= 
  - \left[ \int_{4\pi} \vec{\Omega} \vec{\Omega} f(\vec{x}, \vec{\Omega})
  \ud\Omega \right]
  \vd \grad \phi(\vec{x},t)
  \\ \label{eq:anisotropicFicks}
  &= - \Dtens(\vec{x}) \vd \grad \phi(\vec{x},t) \,.
\end{align}
This resembles ``Fick's law,'' but instead of a scalar diffusion
\emph{coefficient},
the anisotropic diffusion method has a diffusion \emph{tensor}, $\Dtens$, the
second angular moment of $f$:
\begin{equation}\label{eq:dDefinition}
  \Dtens(\vec{x}) \equiv \int_{4\pi} \vec{\Omega} \vec{\Omega}
  f(\vec{x}, \vec{\Omega}) \ud\Omega \,.
\end{equation}

Like Fick's law, Eq.~\eqref{eq:anisotropicFicks} is substituted into
the low-order conservation equation~\eqref{eq:loVol}.
However, this ``Fick's law'' does not directly describe the behavior of $f$ or
$\phi$ at the boundary. For boundary conditions that relate $\phi$, $f$, and
$\zeta$, we turn to the boundary layer equations~\eqref{eqs:blCapPsi}.

%%%%%%%%%%%%%%%%%%%%%%%%%%%%%%%%%%%%%%%%
\subsection{Incident boundary condition}\label{sec:zeta}

The unknown function $\zeta(\vec{x}, \vec{\Omega})$ that lives on the boundary
is a degree of freedom introduced at the beginning of the anisotropic
diffusion derivation. The replacement of $\Psi_b$ with $-\zeta \vec{\Omega} \vd
\grad \phi$ allowed us to formulate a specified boundary condition
such that the effect of incident radiation boundaries could be embedded in the
anisotropic diffusion tensor $\Dtens$.

To make use of this degree of freedom, we decide to enforce on the boundary the
truth from Eq.~\eqref{eq:capPsiZeroth},
\begin{equation*}
  \int_{4\pi} \Psi(\vec{x}, \vec{\Omega}, t) \ud\Omega
  = 0 \,.
\end{equation*}
Note that our approximate $\tilde\Psi$ defined in Eq.~\eqref{eq:approxPsi3}
does not generally satisfy this identity:
\begin{align*}
  0
&\qeq \int_{4\pi} \tilde\Psi(\vec{x}, \vec{\Omega}, t) \ud\Omega
\\
0 &\qeq \int_{4\pi} \left[ - f(\vec{x}, \vec{\Omega}) \vec{\Omega}
\vd \grad \phi(\vec{x}, t)\right]
\ud\Omega
\\
\vec{0} &\qeq \int_{4\pi} \vec{\Omega} f(\vec{x}, \vec{\Omega})\ud\Omega \,.
\end{align*}
However, on exterior source boundaries, because $\zeta$ is defined for incident
directions and $f$ is known for exiting directions, we can choose $\zeta$ such
that this identity holds true under certain conditions.

Returning to the description of $f$ on an incident boundary in
Eq.~\eqref{eq:fFullBndy}, we write
\begin{align*}
  \int_{4\pi} \vec{\Omega} f(\vec{x}, \vec{\Omega})\ud\Omega
  &= \int_{\vec{\Omega} \vd \vec{n} < 0}
  \vec{\Omega} \zeta(\vec{x}, \vec{\Omega})\ud\Omega
  + \int_{\vec{\Omega} \vd \vec{n} > 0}
  \vec{\Omega} f(\vec{x}, \vec{\Omega})\ud\Omega\,.
\end{align*}
Now we set the left hand side to zero to satisfy $\int_{4\pi} \tilde\Psi \ud\Omega = 0$:
\begin{align}
  \label{eq:zetaCondition1}
  \int_{\vec{\Omega} \vd \vec{n} < 0}
  \vec{\Omega} \zeta(\vec{x}, \vec{\Omega})\ud\Omega
  &= -\int_{\vec{\Omega} \vd \vec{n} > 0}
  \vec{\Omega} f(\vec{x}, \vec{\Omega})\ud\Omega \,.
  \\ 
  \intertext{Making the substitution $\vec{\Omega}\to -\vec{\Omega}$ on the
  right hand side yields}
  \label{eq:zetaCondition2}
  \int_{\vec{\Omega} \vd \vec{n} < 0}
  \vec{\Omega} \zeta(\vec{x}, \vec{\Omega})\ud\Omega
  &= \int_{\vec{\Omega} \vd \vec{n} < 0}
  \vec{\Omega} f(\vec{x}, -\vec{\Omega})\ud\Omega \,.
\end{align}

If $f(\vec{\Omega})$ is azimuthally symmetric about $\vec{n}$, then $f$ is only
a function of the cosine angle between $\vec{\Omega}$ and $\vec{n}$:
\begin{equation*}
f(\vec{\Omega}) = \hat f( \vec{\Omega} \vd \vec{n})\,.
\end{equation*}
Now recall the definition of a reflecting boundary from
Eq.~\eqref{eq:reflection},
\begin{equation*}
  \vec{\Omega}_r = \vec{\Omega} - 2(\vec{\Omega} \vd \vec{n}) \vec{n}\,.
\end{equation*}
Dotting the reflected vector with the normal vector $\vec{n}$,
\begin{equation*}
  \vec{\Omega}_r \vd \vec{n}
  = \vec{\Omega} \vd \vec{n} - 2(\vec{\Omega} \vd \vec{n}) \vec{n}\vd \vec{n}
  = - \vec{\Omega} \vd \vec{n}\,.
\end{equation*}
Thus,
\begin{equation*}
  \hat f( \vec{\Omega}_r \vd \vec{n}) = \hat f( -\vec{\Omega} \vd \vec{n})
\end{equation*}
and
\begin{equation}\label{eq:aziSymResult}
  f( \vec{\Omega}_r) = f( -\vec{\Omega} )\,.
\end{equation}

Therefore, Eq.~\eqref{eq:zetaCondition2} can be written
\begin{equation}\label{eq:zetaCondition3}
  \int_{\vec{\Omega} \vd \vec{n} < 0}
  \vec{\Omega} \zeta(\vec{x}, \vec{\Omega})\ud\Omega
  = \int_{\vec{\Omega} \vd \vec{n} < 0}
  \vec{\Omega} f(\vec{x}, \vec{\Omega}_r)\ud\Omega \,,
\end{equation}
which is satisfied by
\begin{equation} \label{eq:zetaReflecting}
  \zeta(\vec{x}, \vec{\Omega}) = f(\vec{x}, \vec{\Omega}_r) \,,
 \quad \vec{x} \in \partial V_b, \ \vec{\Omega} \vd \vec{n} < 0 \,.
\end{equation}
This is not the only definition that satisfies Eq.~\eqref{eq:zetaCondition1},
but it straightforward and has the advantage that all half-space angular moments
of $\zeta$ are equal to $f$, an identity that is used to derive Marshak-like
boundary conditions later.

A more general condition than Eq.~\eqref{eq:zetaReflecting}, also valid if $f$ is
azimuthally symmetric for outgoing directions, is to substitute the identity
\begin{equation*}
  \int_{\vec{\Omega} \vd \vec{n} > 0}
  \vec{\Omega} f(\vec{x}, \vec{\Omega})\ud\Omega
  = \vec{n} 
  \int_{\vec{\Omega} \vd \vec{n} > 0}
  (\vec{\Omega} \vd \vec{n} ) f(\vec{x}, \vec{\Omega}) \ud\Omega
\end{equation*}
into Eq.~\eqref{eq:zetaCondition1} to give
\begin{align} \nonumber
  -\vec{n} \int_{\vec{\Omega} \vd \vec{n} < 0}
  (\vec{n} \vd \vec{\Omega}) \zeta(\vec{x}, \vec{\Omega})\ud\Omega
  &= \vec{n} \int_{\vec{\Omega} \vd \vec{n} > 0}
  (\vec{\Omega} \vd \vec{n} ) f(\vec{x}, \vec{\Omega}) \ud\Omega
\\ \label{eq:zetaGeneral}
  \int_{\vec{\Omega} \vd \vec{n} < 0}
  \abs{\vec{n} \vd \vec{\Omega}} \zeta(\vec{x}, \vec{\Omega}) \ud\Omega
  &= \int_{\vec{\Omega} \vd \vec{n} > 0}
  (\vec{\Omega} \vd \vec{n} ) f(\vec{x}, \vec{\Omega}) \ud\Omega\,,
\end{align}
which says that only the exiting component of the first moment of $f$ on the
boundary need be preserved. This allows, for example, a ``white'' boundary
condition on $f$:
\begin{equation}\label{eq:zetaWhite}
  \zeta(\vec{x}, \vec{\Omega})
  = \frac{1}{\pi} \int_{\vec{\Omega}' \vd \vec{n} > 0}
  (\vec{\Omega}' \vd \vec{n} ) f(\vec{x}, \vec{\Omega}') \ud\Omega'
\end{equation}
which is an isotropic distribution with the same ``strength'' as the exiting
component of $f$. It does not have any relation to the actual strength or
distribution of the intensity $I$ on the boundary.

%This says that under the approximations, assumptions, and restrictions we made,
%the transport equation for $f$ has reflecting boundaries everywhere, even
%where the physical problem does \emph{not} have reflecting boundaries.

%%%%%%%%%%%%%%%%%%%%%%%%%%%%%%%%%%%%%%%%
\subsubsection{Low-order boundary condition}
A boundary layer analysis \cite{Mal1991}
shows that the transport boundary layer, the solution of
Eqs.~\eqref{eqs:blCapPsi}, decays most rapidly under the condition
\begin{equation} \label{eq:bcW}
  0 = \int_{\vec{\Omega} \vd \vec{n} < 0} W(\abs{\vec{\Omega} \vd \vec{n}})
  \Psi_\mathrm{bl} (\vec{x}, \vec{\Omega}, t) \ud \Omega\,,\qquad \vec{x} \in
  \partial V_b\,,\ 0 \le t \le \Delta_t\,.
\end{equation}
$W$ is related to Chandrasekhar's $H$-function \cite{Cha1960} and is
well-approximated by a simple polynomial \cite{Mal1991}:
\begin{equation} \label{eq:chandraW}
  W(\mu) = \frac{\sqrt{3}}{2} \mu H(\mu)
  \approx \mu + \tfrac{3}{2} \mu^2 \,.
\end{equation}
To recover the Marshak boundary condition, we could use $W(\mu) \approx 2 \mu$.

Substituting Eq.~\eqref{eq:blCapPsiBndy} into Eq.~\eqref{eq:bcW} gives the
low-order boundary condition for anisotropic diffusion:
\begin{equation*}
  0 = \int_{\vec{\Omega} \vd \vec{n} < 0} W(\abs{\vec{\Omega} \vd \vec{n}})
  \left[  I^b(\vec{x}, \vec{\Omega}, t) - \frac{1}{4\pi} \phi(\vec{x}, t)
  + \zeta(\vec{x}, \vec{\Omega}, t) \vec{\Omega}\vd \grad \phi(\vec{x}, t)
\right] \ud \Omega \,,
\end{equation*}
or
\begin{equation} \label{eq:bcInc1}
  2\int_{\vec{\Omega}\vd \vec{n} < 0}
  W(\abs{\vec{\Omega} \vd \vec{n}}) I^b(\vec{x}, \vec{\Omega}, t) \ud\Omega
  = \phi(\vec{x}, t)
  - 2\int_{\vec{\Omega}\vd \vec{n} < 0} W(\abs{\vec{\Omega} \vd \vec{n}})
  \zeta(\vec{x}, \vec{\Omega}) \vec{\Omega} \ud\Omega
  \vd \grad \phi(\vec{x}, t) \,.
\end{equation}

If we use the reflecting boundary condition given by
Eq.~\eqref{eq:zetaReflecting}, further simplifications are possible. We
substitute $\zeta = f(\vec{\Omega}_r)$ into Eq.~\eqref{eq:bcInc1}:
\begin{equation*}
  2\int_{\vec{\Omega}\vd \vec{n} < 0}
  W(\abs{\vec{\Omega} \vd \vec{n}}) I^b(\vec{\Omega}) \ud\Omega
  = \phi
  - 2\int_{\vec{\Omega}\vd \vec{n} < 0} W(\abs{\vec{\Omega} \vd \vec{n}})
  \vec{\Omega} f(\vec{\Omega}_r) \ud\Omega
  \vd \grad \phi \,.
\end{equation*}
We can make the right-hand side clearer by expressing the integral over exiting
values of $f$. Making the substitution $\vec{\Omega}\to-\vec{\Omega}$ in the
integral, we get
\begin{align*}
  - 2\int_{\vec{\Omega}\vd \vec{n} < 0} W(\abs{\vec{\Omega} \vd \vec{n}})
  \vec{\Omega} f(\vec{\Omega}_r) \ud\Omega
  &= 
  - 2\int_{\vec{\Omega}\vd \vec{n} > 0} W(\abs{-\vec{\Omega} \vd \vec{n}})
  ( - \vec{\Omega}) f(-\vec{\Omega}_r) \ud\Omega
  \\
  &= 
  2\int_{\vec{\Omega}\vd \vec{n} > 0} W(\vec{\Omega} \vd \vec{n})
  \vec{\Omega} f(-\vec{\Omega}_r) \ud\Omega
  \\ 
  &= 
  2\int_{\vec{\Omega}\vd \vec{n} > 0} W(\vec{\Omega} \vd \vec{n})
  \vec{\Omega} f(\vec{\Omega}) \ud\Omega \,.
\end{align*}
The boundary condition on $f$ from Eq.~\eqref{eq:zetaReflecting} in conjunction
with Eq.~\eqref{eq:aziSymResult} give the equality $f(-\vec{\Omega}_r) =
f(\vec{\Omega})$, still under the condition that $f$ is azimuthally symmetric.

The low-order, transport-consistent boundary condition for an incident source
is therefore
\begin{equation}\label{eq:loBndy}
  2\int_{\vec{\Omega}\vd \vec{n} < 0}
  W(\abs{\vec{\Omega} \vd \vec{n}}) I^b(\vec{x}, \vec{\Omega}, t) \ud\Omega
  = \phi(\vec{x}, t)
  + 2\int_{\vec{\Omega}\vd \vec{n} > 0} W(\vec{\Omega} \vd \vec{n})
  \vec{\Omega} f(\vec{x}, \vec{\Omega}) \ud\Omega
  \vd \grad \phi(\vec{x}, t) \,.
\end{equation}

%%%%%%%%%%%%%%%%%%%%%%%%%%%%%%%%%%%%%%%%
\subsubsection{Marshak-like boundary condition}
This form has a particular advantage if we use the Marshak-like approximation
that $W(\mu)\approx 2\mu$. Equation~\eqref{eq:loBndy} becomes
\begin{align*}
  2\int_{\vec{\Omega}\vd \vec{n} < 0}
  ( 2\abs{\vec{\Omega} \vd \vec{n}} ) I^b(\vec{\Omega}) \ud\Omega
  &= \phi
  + 2\int_{\vec{\Omega}\vd \vec{n} > 0} ( 2\vec{\Omega} \vd \vec{n} )
  \vec{\Omega} f(\vec{\Omega}) \ud\Omega \vd \grad \phi
  \\
  4 F^-
  &= \phi
  + 4 \vec{n} \vd \left[ \int_{\vec{\Omega}\vd \vec{n} > 0} \vec{\Omega}
  \vec{\Omega} f(\vec{\Omega}) \ud\Omega \right] \vd \grad \phi \,.
  \\ 
  \intertext{Because our chosen boundary condition of $f$ gives
  $f(\vec{\Omega})=f(-\vec{\Omega})$, the integrand in
  brackets is an even function of $\vec{\Omega}$, so}
  4 F^-
  &= \phi
  + 4 \vec{n} \vd  \left[ \frac{1}{2} \int_{4\pi}
  \vec{\Omega} \vec{\Omega} f(\vec{\Omega}) \ud\Omega \right] \vd \grad \phi \,.
  \\
  4 F^-
  &= \phi
  + 2 \vec{n} \vd \int_{4\pi}
  \vec{\Omega} \vec{\Omega} f(\vec{\Omega}) \ud\Omega \vd \grad \phi \,.
\end{align*}
The integral on the right hand side is the same as in
  Eq.~\eqref{eq:anisotropicFicks}, which defined the anisotropic diffusion
  tensor. Our Marshak-like boundary approximation is
\begin{equation}\label{eq:marshakAd}
  4 F^-(\vec{x}, t)
  = \phi(\vec{x}, t)
  + 2 \vec{n} \vd \Dtens(\vec{x}) \vd \grad \phi(\vec{x}, t) \,.
\end{equation}
This is entirely analogous to the standard diffusion Marshak boundary condition,
\begin{equation*}
  4 F^-(\vec{x}, t) = \phi(\vec{x}, t)
  + 2  D(\vec{x}) \vec{n} \vd \grad \phi(\vec{x}, t)\,.
\end{equation*}

%%%%%%%%%%%%%%%%%%%%%%%%%%%%%%%%%%%%%%%%
\subsection{Reflecting boundary condition}
The final piece of the anisotropic diffusion equations is the low-order
condition on a reflecting boundary, which is described by
Eqs.~\eqref{eq:fullTransportRefl} and~\eqref{eq:reflection}:
\begin{equation*}
  I(\vec{x}, \vec{\Omega}, t)
  = I(\vec{x}, \vec{\Omega} - 2(\vec{\Omega} \vd \vec{n}) \vec{n}, t) \,.
\end{equation*}
This implies that there is no net flux of radiation energy normal to the
boundary:
\begin{equation}\label{eq:reflFirst}
  \vec{n} \vd \vec{F}(\vec{x}, t) = 0\,.
\end{equation}

Substituting Eq.~\eqref{eq:anisotropicFicks}, the first moment of the AD
approximation to $\tilde\Psi$, we find
\begin{equation}\label{eq:reflLo1}
  \vec{n} \vd \Dtens(\vec{x}) \vd \grad \phi(\vec{x}, t) = 0\,.
\end{equation}
As noted before, the exact intensity on a reflecting boundary also satisfies
\begin{equation}\label{eq:reflZeroth}
  \vec{n}\vd\grad \phi(\vec{x}, t) = 0 \,.
\end{equation}
This is only compatible with Eq.~\eqref{eq:reflLo1} when $\vec{n}$ is an
eigenvector of $\Dtens$:
\begin{equation*}
  \vec{n} \vd \Dtens(\vec{x}) = \lambda \vec{n}\,.
\end{equation*}
Interestingly, as shown in the discussion section \ref{sec:eigenvectors}, this is the case when $f$ is
azimuthally symmetric about
$\vec{n}$, which is exactly what we demanded of $f$ on non-reflecting boundaries.

Therefore, in the interest of consistency with the other boundaries and with the
property of the exact intensity, we use the low-order AD reflecting boundary
condition of
\begin{equation}\label{eq:reflLo2}
  \vec{n} \vd \grad \phi(\vec{x}, t) = 0\,.
\end{equation}

%\begin{align*}
%  D n &= \lambda n \\
%  n\conj D\conj &= \lambda n\conj \\
%  \\ 
%  \intertext{Since the diffusion tensor is symmetric, $D\conj=D$, so}
%  n\conj D &= \lambda n\conj \,.
%\end{align*}

%%%%%%%%%%%%%%%%%%%%%%%%%%%%%%%%%%%%%%%%
\subsection{Summary}
We have now derived a full description of the time-dependent anisotropic
diffusion equations in a finite medium. 

The anisotropic diffusion method approximates the transport
equations~\eqref{eqs:fullTransport} with a set of low-order equations for the
scalar intensity $\phi$ that use a diffusion coefficient calculated from a
simple high-order transport equation.

The low order equation is the result of substituting the approximate Fick's law,
Eq.~\eqref{eq:anisotropicFicks}, into the conservation equation~\eqref{eq:loVol}:
\begin{equation*}
\frac{1}{c} \pder{\phi}{t} (\vec{x}, t)
  - \grad \vd \Dtens(\vec{x}) \vd \grad \phi(\vec{x},t)
  + \sigmast(\vec{x}) \phi(\vec{x}, t)
  = Q(\vec{x}, t) \,,
  \quad \vec{x} \in V,\ 0 \le t \le \Delta_t \,.
\end{equation*}
From Eq.~\eqref{eq:loInit}, it has an initial condition
\begin{equation*}
\phi(\vec{x}, 0) = \phi^i(\vec{x})\,, \vec{x} \in V  \,.
\end{equation*}
The incident source boundary condition with the simpler Marshak-like
approximation from Eq.~\eqref{eq:marshakAd} is
\begin{equation*}
  4 F^-(\vec{x}, t)
  = \phi(\vec{x}, t)
  + 2 \vec{n} \vd \Dtens(\vec{x}) \vd \grad \phi(\vec{x}, t) \,.
 \quad \vec{x} \in \partial V_b,\ 0 \le t \le \Delta_t \,.
\end{equation*}
The reflecting boundary condition from Eq.~\eqref{eq:reflLo2} is
\begin{equation*}
  \vec{n}\vd\grad \phi(\vec{x}, t) = 0 \,,
 \quad \vec{x} \in \partial V_r,\ 0 \le t \le \Delta_t \,.
\end{equation*}

The anisotropic diffusion tensor is defined in Eq.~\eqref{eq:dDefinition},
\begin{equation*}
  \Dtens(\vec{x}) \equiv \int_{4\pi} \vec{\Omega} \vec{\Omega}
  f(\vec{x}, \vec{\Omega}) \ud\Omega \,.
\end{equation*}
Here, $f$ is the solution of the purely absorbing transport equation described by
Eqs.~\eqref{eqs:fFull}:
\begin{subequations} \label{eqs:fFull}
  \begin{equation} \label{eq:fFullVol}
    \vec{\Omega}\vd \grad f(\vec{x}, \vec{\Omega})
    + \sigmast f (\vec{x}, \vec{\Omega})
  = \frac{1}{4\pi} \,, \quad x \in V,\ \vec{\Omega} \in 4\pi\,,
  \end{equation}
\begin{equation} \label{eq:fFullBndy}
  f(\vec{x}, \vec{\Omega}) = \zeta(\vec{x}, \vec{\Omega}) \,,
 \quad \vec{x} \in \partial V_b, \ \vec{\Omega} \vd \vec{n} < 0\,.
\end{equation}
\begin{equation} \label{eq:fFullRefl}
  f(\vec{x}, \vec{\Omega}) = f(\vec{x}, \vec{\Omega}_r) \,,
 \quad \vec{x} \in \partial V_r, \ \vec{\Omega} \vd \vec{n} < 0\,.
\end{equation}
On $\partial V_b$ where $I$ is specified, $f$ can take
several suitable forms as discussed in \S\ref{sec:zeta}, but we will usually
take the condition from Eq.~\eqref{eq:zetaReflecting}:
\begin{equation} \label{eq:fFullBndy2}
  f(\vec{x}, \vec{\Omega}) = \zeta(\vec{x}, \vec{\Omega}) = f(\vec{x},
  \vec{\Omega}_r) \,,
 \quad \vec{x} \in \partial V_b, \ \vec{\Omega} \vd \vec{n} < 0\,.
\end{equation}
\end{subequations}

%%%%%%%%%%%%%%%%%%%%%%%%%%%%%%%%%%%%%%%%%%%%%%%%%%%%%%%%%%%%%%%%%%%%%%%%%%%%%%%%
\section{Anisotropic diffusion discussion}
Even without numerical results for the anisotropic diffusion equations, a
number of interesting and beneficial properties can be deduced from the
low-order AD equations and the transport equations for $f$.

%%%%%%%%%%%%%%%%%%%%%%%%%%%%%%%%%%%%%%%%
\subsection{Specific application to nonlinear radiation transport}

The low-order anisotropic diffusion equation in TRT applications, from
Eq.~\eqref{eq:loVol}, is
\begin{equation*}
\frac{1}{c} \pder{\phi}{t} (\vec{x}, t)
  - \grad \vd \Dtens(\vec{x}) \vd \grad \phi(\vec{x},t)
  + \sigmast(\vec{x}) \phi(\vec{x}, t)
  = \sigmast(\vec{x}) ac [T(\vec{x}, t)]^4 + q_{r}(\vec{x}, t)\,.
\end{equation*}
The only limitations of this form are that $\Dtens$ and $\sigmast$ are constant
during a time step.  Yet, by leaving the specific value of $\sigmast$
open-ended, we allow for multiple nonlinear treatments. For example, the
semi-implicit treatment would be to lag the opacity, $\sigmast \approx
\sigma(t=0)$. If nonlinear convergence were desired, perhaps via JFNK
\cite{Kno2004}, $\sigmast$ would assume the value of the opacity at the current
estimated temperature. Because $\Dtens$ depends on $\sigmast$ through the
calculation for $f$, true nonlinear convergence would also require updating
$\Dtens$ at every nonlinear iteration. The generality of the low-order AD
equation also allows for any arbitrary time discretization.

As formulated, the time-dependent anisotropic diffusion approximation does not
depend on any physics particular to thermal radiative transfer. For example, we
only demanded that the source term is isotropic (as is the case with black body
emission), implicitly allowing for isotropic scattering and sources. (Linearly
anisotropic sources may even be accounted for: see
\cite{Kel2010} for an application of steady-state infinite medium anisotropic
diffusion with linearly anisotropic scattering in reactors.)

%%%%%%%%%%%%%%%%%%%%%%%%%%%%%%%%%%%%%%%%
\subsection{Transport calculation for \texorpdfstring{$f$}{f}}

The anisotropic diffusion tensors are calculated from $f$, the solution to the
purely absorbing transport problem in Eqs.~\eqref{eqs:fFull}  with a unit
isotropic source,
reflecting boundary conditions, and the same opacities as the physical problem
being simulated. It is steady-state, although it needs to be recalculated at
every time step as $\sigmast$ changes.

Were it not for the reflecting boundary condition, a discrete ordinates (\SN)
solution for $f$ would take only one transport sweep to complete, because there
is no scattering source to converge. However, because the boundary conditions
rely on exiting values of $f$, it will in practice take more than one. The
larger the optical thickness between boundaries, the faster the convergence will
be: this poses a problem if
two boundaries on opposite sides of the problem are separated by only a
fraction of a mean free path (e.g., a voided channel), especially if a very fine
angular
quadrature set is used (one with ordinates that are nearly perpendicular to the
boundary).

The convergence problem can be obviated somewhat by taking advantage of a
loophole left in the boundary condition function $\zeta$. Because the condition
$\int_{4\pi} \vec{\Omega} f \ud\Omega$ allows for other angular shapes on the
boundary, as shown by Eq.~\eqref{eq:zetaGeneral}, we could replace the
offending reflecting boundary with a white boundary, which should converge more
quickly.

In the case of a time-dependent problem where $\sigma$ remains constant from one
time step to the next, the calculation for $f$ needs only to be run once. In
nonlinear problems where $\sigma$ is a function of time, storing $f$ on the
outer boundaries of the problem (which is already necessary for reflecting
boundary conditions) will greatly improve the recalculation of $f$. After the
first time step, a good guess for $f$ on the boundary means that only large and
far-reaching changes to $\sigma$ near the boundary will require more than one
transport sweep to converge.

Another desirable property of $f$ is that, because it is a steady-state
solution, and because only the second angular moment
is required, the
full angle-dependent solution does not need to be stored: the angular moment can
just be accumulated during the transport sweep as is done with steady-state \SN\
transport. This is a tremendous
advantage over traditional time-dependent transport methods, which require the
memory-intensive storage of $I(\vec{x},\vec{\Omega},t)$.

If the problem is homogeneous ($\sigmast=\sigma$ is constant), then the solution is
$f=1/(4\pi\sigma)$. Taking the second moment of $f$ then yields
\begin{equation*}
  \Dtens = \frac{1}{4\pi\sigma} \int_{4\pi} \vec{\Omega} \vec{\Omega} \ud \Omega
  = \frac{1}{3\sigma} \Identitytens\,.
\end{equation*}
Substituting this into the anisotropic Fick's law,
Eq.~\eqref{eq:anisotropicFicks}, we recover the standard Fick's law:
\begin{equation*}
  \vec{F} = - \frac{1}{3\sigma} \grad \phi\,.
\end{equation*}
In other words, for a homogeneous medium, the anisotropic diffusion method
reduces to the standard diffusion method.

Indeed, even in an inhomogeneous medium, if the transport equation for $f$ is
rewritten as
\begin{equation*}
  \epsilon \vec{\Omega}\vd f(\vec{x},\vec{\Omega})
  + \sigmast(\vec{x})  f(\vec{x},\vec{\Omega}) = \frac{1}{4\pi}\,,
\end{equation*}
then as $\epsilon\to 0$,
\begin{equation*}
  f(\vec{x}, \vec{\Omega}) \to \frac{1}{4\pi \sigmast(\vec{x})} \lra
  \Dtens(\vec{x}) \to \frac{1}{3 \sigmast(\vec{x})} \Identitytens\,,
\end{equation*}
which is the standard diffusion coefficient. The value $\epsilon=1$ yields the
anisotropic diffusion tensor.

%%%%%%%%%%%%%%%%%%%%%%%%%%%%%%%%%%%%%%%%
\subsection{Properties of the anisotropic diffusion tensor}
The diffusion tensor is defined in Eq.~\eqref{eq:dDefinition} to be
\begin{equation*}
  \Dtens(\vec{x}) \equiv \int_{4\pi} \vec{\Omega} \vec{\Omega}
  f(\vec{x}, \vec{\Omega}) \ud\Omega \,.
\end{equation*}
Equivalently, the component in row $i$, column $j$ of $\Dtens$ is
\begin{equation}\label{eq:dij}
  D^{ij} = \int_{4\pi} \Omega^i \Omega^j
  f(\vec{x}, \vec{\Omega}) \ud\Omega \,,
\end{equation}
where $\Omega^i$ is the $i$th component of the angular vector $\vec{\Omega}$
(e.g., $i=1$ corresponds to the polar cosine angle $\mu$).

\subsubsection{Fick's law}
Fick's law for diffusion states that a gradient in $\phi$ will induce particles
to flow from the area of higher density to lower density along the gradient:
\begin{equation*}
  \vec{F}(\vec{x},t) = - D(\vec{x}) \grad \phi(\vec{x},t)\,.
\end{equation*}
The anisotropic diffusion tensor has an interesting twist: it implies that a
gradient in $\phi$ can induce
particle flow in a direction \emph{other than the direction of the gradient}. In
2-D, the anisotropic Fick's law has the form
\begin{align*}
  \vec{F} &= - \Dtens \vd \grad \phi
  \\
  \begin{bmatrix}
    F^x \\
    F^y
  \end{bmatrix}
  &=
  -
  \begin{bmatrix}
    D^{xx} & D^{yx} \\
    D^{xy} & D^{yy}
  \end{bmatrix}
  \begin{bmatrix}
    \tpder \phi x \\
    \tpder \phi y
  \end{bmatrix} \,.
\end{align*}

If we calculate the ``leakage'' $\vec{n} \vd \vec{F}$ averaged over a planar
surface normal to $\vec{n}$, standard diffusion will only consider the gradient
\emph{normal} to the surface. For example, on a surface normal to the $x$ axis,
diffusion gives the leakage term
\begin{equation*}
  F^x = - D \pder{\phi}{x}\,.
\end{equation*}

However, anisotropic diffusion has the following leakage term:
\begin{equation*}
  F^x = - D^{xx} \pder{\phi}{x} - D^{yx} \pder{\phi}{y}\,.
\end{equation*}
We refer to the leakage normal to the face, $- D^{xx} \pder{\phi}{x}$, as
``normal'' leakage; the leakage along the face, $- D^{yx}
\pder{\phi}{y}$, we term the ``transverse'' leakage. As we will see in
chapter~\ref{chap:implementation}, the transverse leakage adds a layer of
complexity to anisotropic diffusion discretization schemes that is not present
in standard diffusion discretizations.

\subsubsection{Limited magnitude}
The standard diffusion coefficient is defined as
\begin{equation*}
  D(\vec{x}) = \frac{1}{3\sigma(\vec{x})} \,.
\end{equation*}
As $\sigma\to0$ locally, $D\to \infty$. An infinite diffusion coefficient in a
region gives a spatially constant $\phi$, which is almost always an
unphysical result. Small values of $\sigma$ can also cause numerical
difficulties in computer implementations, as well.

In contrast to standard diffusion, the anisotropic diffusion approximation has
a \emph{nonlocal} dependence on $\sigma$ through $f(\vec{x}, \vec{\Omega})$. Since $f$ remains
bounded, even when $\sigma=0$ at some point, $\Dtens$ will also remain bounded.

\subsubsection{Symmetric positive definiteness}
From Eq.~\eqref{eq:dij}, $\Dtens$ is clearly symmetric: $D^{ij}=D^{ji}$. Yet
$\Dtens$ is also symmetric positive definite (SPD), satisfying
\begin{equation*}
  \vec{a} \vd \Dtens \vd \vec{a} > 0
\end{equation*}
for all non-zero, real vectors $\vec{a}$ \cite{Tre1997}. To show this, we write 
\begin{align*}
  \vec{a} \vd \Dtens \vd \vec{a} &=
  \vec{a} \vd \left[\int_{4\pi} \vec{\Omega} \vec{\Omega}
  f(\vec{x}, \vec{\Omega}) \ud\Omega\right] \vd \vec{a}
  \\
  &=
  \int_{4\pi} (\vec{\Omega} \vd
  \vec{a}) (\vec{\Omega} \vd \vec{a})
  f(\vec{x}, \vec{\Omega}) \ud\Omega\,.
  \\
  &=
  \int_{4\pi} (\vec{\Omega} \vd \vec{a})^2
  f(\vec{x}, \vec{\Omega}) \ud\Omega\,.
\end{align*}
Because the solution for $f$ is strictly positive for all $\vec{\Omega}\in
4\pi$, and $(\vec{\Omega} \vd \vec{a})^2$ is positive%
\footnote{The quantity $(\vec{\Omega} \vd \vec{a})^2$ will technically be zero where
$\vec{\Omega}$ is orthogonal to $\vec{\alpha}$, but it will be nonzero most
everywhere in the domain of integration, yielding a positive result.
}%
for non-zero
$\vec{a}$, this integral will always be positive. Therefore, $\Dtens$ is SPD.

Because $\Dtens$ is symmetric positive definite, the ``anisotropic Fick's law''
in Eq.~\eqref{eq:anisotropicFicks} is self-adjoint.  Consequently, many
reasonable discretizations of the anisotropic diffusion equations will be SPD as
well, allowing solution by the conjugate gradient (CG) method \cite{Tre1997}.
This is in contrast to quasidiffusion methods (see \S\ref{sec:bgPn}), which
require a more computationally expensive solver such as GMRES \cite{War2003}.

\subsubsection{Eigenvectors}\label{sec:eigenvectors}
As stated in \S\ref{sec:zeta}, if $f(\vec{\Omega})$ is azimuthally
symmetric about some unit vector $\vec{a}$, then $f$ is only a function of the
cosine angle
between $\vec{\Omega}$ and $\vec{a}$:
\begin{equation*}
f(\vec{\Omega}) = \hat f( \vec{\Omega} \vd \vec{a})\,.
\end{equation*}

If $\vec{a}$ is an eigenvector of $\Dtens$, then
\begin{equation*}
  \Dtens \vd \vec{a} = \lambda \vec{a}\,,
\end{equation*}
where $\lambda$ is a constant. Taking the dot product of the diffusion tensor's
definition in Eq.~\eqref{eq:dDefinition} and
$\vec{a}$, omitting the $\vec{x}$ parameter for brevity, we find
\begin{align*}
  \Dtens \vd \vec{a}
  &= \int_{4\pi} \vec{\Omega} \vec{\Omega} f(\vec{\Omega}) \ud\Omega \vd \vec{a}
  \\
  &= \int_{4\pi} (\vec{\Omega} \vd \vec{a}) \vec{\Omega} f(\vec{\Omega}) \ud\Omega
  \,.
\end{align*}
If $f$ is azimuthal about $\vec{a}$, then
\begin{align*}
  \Dtens \vd \vec{a}
  &= \int_{4\pi} (\vec{\Omega} \vd \vec{a}) \vec{\Omega}
    \hat f(\vec{\Omega} \vd \vec{a}) \ud\Omega \,.
\end{align*}
We choose an angular coordinate system where $\vec{i}=\vec{a}$, giving
$\vec{\Omega} \vd \vec{a}=\mu$:
\begin{align*}
  \Dtens \vd \vec{a}
  &= \int_{0}^{2\pi} \int_{-1}^{1} (\mu)
  \left( \mu \vec{a} + \sqrt{1-\mu^2} \cos \theta \vec{u} + \sqrt{1-\mu^2} \sin
  \theta \vec{v} \right) \hat f(\mu) \ud\mu \ud\theta \,.
\end{align*}
Because $\hat f$ is \emph{not} a function of $\theta$, the integrand is an
odd function of $\theta$, so the $\vec{u}$ and $\vec{v}$ components are zero.
\begin{align*}
  \Dtens \vd \vec{a}
  &= 2\pi \int_{-1}^{1} \mu^2 \hat f(\mu) \ud\mu \vec{a}
  \\
  &= \lambda \vec{a} \,.
\end{align*}
Thus, if $f$ is azimuthally symmetric about $\vec{a}$, $\vec{a}$ is an
eigenvector of $\Dtens$.

Thus, if $\sigma$ in a 3-D problem varies only along
the $x$ axis, then the vectors along \emph{each} primary axis are
eigenvectors of $\Dtens$: i.e., the diffusion tensor has only
components along the diagonal, and there is no transverse leakage on any
Cartesian face in the problem. For example, the 2-D VHTR problem in
\cite{Lar2009c} featured a total cross section that varied only along the $x$
axis. Therefore, even though the physical problem was a function of $(x,y)$
because of the problem's source term, the diffusion tensor had
$D^{xy}=D^{yx}=0$, allowing the use of a very simple spatial discretization.

%%%%%%%%%%%%%%%%%%%%%%%%%%%%%%%%%%%%%%%%
\subsection{Revisiting the asymptotic assumptions}

The first assumptions we made in deriving the anisotropic diffusion
approximation were, from Eq.~\eqref{eq:ansatz}, that
\begin{align*}
  I &= O(1), &
  \grad I &= O(\epsilon), &
  \frac{1}{c}\pder{I}{t} &= O(\epsilon^2), &
  \vec{F} = \int_{4\pi} \vec{\Omega} I\ud\Omega &= O(\epsilon).
\end{align*}
It is justifiable to test how consistently these assumptions match the resulting
approximation to $I$,
\begin{equation*}
I(\vec{x}, \vec{\Omega}, t)
= \frac{1}{4\pi}\phi(\vec{x}, t) + \tilde\Psi(\vec{x}, \vec{\Omega}, t)
= - f(\vec{x}, \vec{\Omega}) \vec{\Omega}\vd \grad \phi(\vec{x}, t)\,.
\end{equation*}

The transport solution for $f$ is $O(1)$; it is roughly the
same magnitude as the opacity in the problem.
Furthermore, because $f$ is continuous in space, the diffusion tensor is also
continuous. The solution for $\phi$ therefore has a smooth first
derivative,%
\footnote{
The smooth result differs from standard diffusion, which has
``kinks'' in $\phi$ at discontinuous material interfaces, where $D$ is
discontinuous. The exact solution to $I$ has both a ``kink'' and a boundary
layer.
}
a fact that is compatible with the supposition that $\grad \phi$
is $O(\epsilon)$. Because $\phi$ is $O(1)$, $I$ is also $O(1)$.
Multiplying the equation for $I$ by $\vec{\Omega}$ and integrating yields an
$O(1)$
diffusion tensor and the $O(\epsilon)$ gradient, so $\vec{F}$ is also
$O(\epsilon)$.

%%%%%%%%%%%%%%%%%%%%%%%%%%%%%%%%%%%%%%%%
\subsection{Relating the \texorpdfstring{$\Psi$}{Psi} equations to
diffusion}\label{sec:adDiscDiff}
Let us return to Eq.~\eqref{eq:capPsiVol}, where we formulated an equation for
the ``anisotropic'' components of $I$ by defining
\begin{equation*}
  \Psi(\vec{x}, \vec{\Omega}, t) \equiv I(\vec{x}, \vec{\Omega}, t) -
  \frac{1}{4\pi} \phi(\vec{x}, t)\,,
\end{equation*}
and manipulating the transport equation and its zeroth moment.

Instead of
using the integral transport equation to derive the AD method, we could
approximate $\Psi$
as a linear function in angle,
\begin{equation*}
  \Psi(\vec{x}, \vec{\Omega}, t) \approx \frac{3}{4\pi} \vec{\Omega} \vd
  \vec{F}(\vec{x}, t)\,,
\end{equation*}
which corresponds to the \Pone\ approximation $I= \frac{1}{4\pi} (\phi +
3\vec{\Omega} \vd \vec{F})$. It also satisfies the identities given in
Eqs.~\eqref{eqs:capPsiIdentities}: $\int_{4\pi} \Psi \ud\Omega = 0$ and
$\int_{4\pi} \vec{\Omega} \Psi \ud\Omega = \vec{F}$.

Substituting this approximation into the transport equation~\eqref{eq:capPsiVol}
for $\Psi$, we get
\begin{align*}
  \frac{1}{c} \pder{}{t} \left[ \frac{3}{4\pi} \vec{\Omega} \vd \vec{F} \right]
  + \vec{\Omega}\vd \grad \left[ \frac{3}{4\pi} \vec{\Omega} \vd \vec{F} \right]
  + \sigmast \left[ \frac{3}{4\pi} \vec{\Omega} \vd \vec{F} \right]
  &= \frac{1}{4\pi} \grad \vd\vec{F}
  - \frac{1}{4\pi} \vec{\Omega}\vd \grad \phi\,.
\end{align*}
The function parameters have been omitted for brevity.
Taking the first angular moment of this equation with $\int_{4\pi}
\vec{\Omega}(\cdot) \ud\Omega$ yields
\begin{multline*}
\frac{3}{4\pi} \left( \int_{4\pi} \vec{\Omega}\vec{\Omega}\ud\Omega \right) \vd
\frac{1}{c} \pder{}{t} \vec{F}
+ \frac{3}{4\pi} \left( \int_{4\pi}
  \vec{\Omega}\vec{\Omega}\vec{\Omega}\ud\Omega \right)
\vd \grad \vd \vec{F}
+ \sigmast \frac{3}{4\pi} \left( \int_{4\pi}
\vec{\Omega}\vec{\Omega}\ud\Omega \right) \vd\vec{F}
\\
= \frac{1}{4\pi} \left( \int_{4\pi} \vec{\Omega}\ud\Omega \right)
\grad \vd\vec{F}
- \frac{1}{4\pi} \left( \int_{4\pi} \vec{\Omega}\vec{\Omega}\ud\Omega \right)
\vd \grad \phi\,.
\end{multline*}
Now, basic vector identities \cite{Lar2007} reduce the parenthesized
quantities to very manageable expressions: $\int_{4\pi}
\vec{\Omega}\vec{\Omega}\ud\Omega=\frac{4\pi}{3}\Identitytens$, and the odd
multiples of $\vec{\Omega}$ integrated over the unit sphere are zero. Simplified,
the equation becomes the standard \Pone\ equation:
\begin{equation*}
  \frac{1}{c} \pder{}{t} \vec{F}
  + \sigmast \vec{F}
  = - \frac{1}{3} \grad \phi\,.
\end{equation*}

Now if we neglect the time derivative of the radiation flux, making the
quasi-static approximation, we recover Fick's law:
\begin{equation*}
\vec{F}(\vec{x}, t) = -\frac{1}{3\sigmast(\vec{x})} \grad\phi(\vec{x}, t) \,.
\end{equation*}

%%%%%%%%%%%%%%%%%%%%%%%%%%%%%%%%%%%%%%%%%%%%%%%%%%%%%%%%%%%%%%%%%%%%%%%%%%%%%%%%
\section{Flux-limited anisotropic diffusion}

Like standard diffusion, the time-dependent anisotropic diffusion approximation
is parabolic \cite{Pom1982,Ols2000}, allowing radiation energy to unphysically
propagate faster than the speed of light.  This undesirable property results
from the assumption that the time derivative of the intensity varied slowly in
time, $\frac{1}{c}\pder{I}{t} = O(\epsilon^2)$, effectively the same
quasi-static approximation as standard diffusion.

From \S\ref{sec:bgFld}, the exact
intensity always satisfies the relation
\begin{equation*}
  \norm{\vec{F}} \le \phi\,.
\end{equation*}
Yet the new anisotropic Fick's law, Eq.~\eqref{eq:anisotropicFicks}, can violate
this relation:
\begin{equation*}
  \norm{\Dtens \vd \grad \phi} \stackrel{?}{\le} \phi \,.
\end{equation*}
The left hand side can exceed the right if either $\Dtens$ or $\grad \phi$ is
``large'':  yet the nature of this tensor product makes the interpretation of
``large'' much less clear than the flux limiting of standard diffusion, where
the scalar $D$ can be moved outside the norm.

However, the straightforward ``max'' limiter can be easily written. A
semi-implicit implementation of this limiter is
\begin{equation*}
  \vec{F}^{n+1} = -\Dtens^{n}\vd \del \phi^{n+1} \times 
  \max\left( 1, \norm{\Dtens^{n}\vd \frac{\del \phi^{n}}{\phi^{n}}}
  \right)\inv \,.
\end{equation*}
Effectively, the limiter tests whether an estimate of $\vec{F}^{n+1}$ (using the
previous time step's solution) exceeds the flux limit; if it does, then it
uniformly scales the diffusion tensor to satisfy the estimated limit.

It is an admittedly \emph{ad hoc} correction, but it can restore a qualitative
behavior of the true intensity when the assumptions that led to anisotropic
diffusion break down.

%%%%%%%%%%%%%%%%%%%%%%%%%%%%%%%%%%%%%%%%%%%%%%%%%%%%%%%%%%%%%%%%%%%%%%%%%%%%%%%%
\section{Summary}
To derive the anisotropic diffusion method, we:
\begin{enumerate}
  \item manipulated the radiation transport equation and conservation equation
    to get a differential transport equation for $\Psi$;
  \item separated $\Psi$ into an interior solution $\tilde \Psi$ and boundary
    layer solution $\Psi_\mathrm{bl}$;
  \item transformed the equation for $\tilde \Psi$ to an \emph{integral}
    transport equation;
  \item made assumptions about the strength of the solution's gradients and
    anisotropy;
  \item used Taylor series to approximate nonlocal unknowns with local
    unknowns and discarded small terms;
  \item applied a transport-matching procedure to $\Psi_\mathrm{bl}$ 
    and used an identity to find boundary conditions for $f$ and $\phi$;
  \item took the first angular moment of $\tilde \Psi$ to obtain an anisotropic
    ``Fick's law''; and
  \item substituted that result into the time-dependent particle conservation
    equation.
\end{enumerate}
This procedure results in two simple sets of equations. The first set
is a simple transport equation for $f$, whose second angular moment is
the anisotropic diffusion tensor $\Dtens$. The second set of equations uses
$\Dtens$ to approximate the flow of radiation inside a time step. 

These equations limit to the standard diffusion approximation in a homogeneous
medium, but they do not make the diffusion approximation that the radiation
intensity is linear in
angle. We therefore expect the AD method to give much more accurate answers
where the intensity is a complex function of angle.
Chapters~\ref{chap:simpleNumericalResults}
and~\ref{chap:trtNumericalResults} will put this expectation to the test.

