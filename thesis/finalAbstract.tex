\documentclass{umthesis}
\usepackage[inner=1.5in,outer=1in,top=1in,bottom=1in,headheight=0pt]{geometry}

\usepackage[T1]{fontenc}
\usepackage{fourier}
\usepackage{microtype}

%%%%%%%%%%%%%%%%%%%%%%%%%%%%%%%%%%%%%%%%%%%%%%%%%%%%%%%%%%%%%%%%%%%%%%%%%%%%%%%%
\author{Seth R.~Johnson}
\title{Anisotropic Diffusion Approximations for Time-dependent Particle
Transport}

\program{Nuclear Engineering and Radiological Sciences}
\degree{Doctor of Philosophy}
\chaircommitteemember{Edward W.~Larsen}{Professor}
\committeemember{Thomas J.~Downar}{Professor}
\committeemember{James P.~Holloway}{Professor}
\committeemember{William R.~Martin}{Professor}
\committeemember{Katsuyo S.~Thornton}{Professor}

%%%%%%%%%%%%%%%%%%%%%%%%%%%%%%%%%%%%%%%%%%%%%%%%%%%%%%%%%%%%%%%%%%%%%%%%%%%%%%%%
\begin{document}

% ABSTRACT: up to 350 words ; vim: [select] g CTRL+g
\begin{finalabstract}
Motivated by the Center for Radiative Shock Hydrodynamics (CRASH) uncertainty
quantification project, which needs to simulate hundreds of instances of a
complex multiphysics problem, we desire a computational method more accurate
than diffusion
but less costly than time-dependent transport. We believe we have found this
middle ground: an anisotropic diffusion (AD) method that uses inexpensive
transport-calculated AD coefficients to solve the thermal radiative transfer
equations.

Using the integral transport equation for the angular intensity, we derive an
expression for the radiation flux that depends on a non-local function of the
opacity. This nonlocal function is the solution of a purely absorbing transport
equation with albedo boundary conditions, and the function's second angular
moment is the anisotropic diffusion tensor. A system of low-order equations
solves for the scalar intensity using the AD coefficients. To demonstrate the
AD method's speed and efficacy, we simulate a mock-up of the CRASH project's
problem of interest, radiation flow down a channel in ``flatland'' geometry.
\end{finalabstract}

%%%%%%%%%%%%%%%%%%%%%%%%%%%%%%%%%%%%%%%%%%%%%%%%%%%%%%%%%%%%%%%%%%%%%%%%%%%%%%%%
\end{document}
