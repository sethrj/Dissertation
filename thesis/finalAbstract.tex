\documentclass{umthesis}
\usepackage[inner=1.5in,outer=1in,top=1in,bottom=1in,headheight=0pt]{geometry}

\usepackage[T1]{fontenc}
\usepackage{fourier}
\usepackage{microtype}
\usepackage{amsmath}
/Users/seth/Documents/Compositions/SRJinclude.tex

%%%%%%%%%%%%%%%%%%%%%%%%%%%%%%%%%%%%%%%%%%%%%%%%%%%%%%%%%%%%%%%%%%%%%%%%%%%%%%%%
\author{Seth R.~Johnson}
\title{Anisotropic Diffusion Approximations for Time-dependent Particle
Transport}

\program{Nuclear Engineering and Radiological Sciences}
\degree{Doctor of Philosophy}
\chaircommitteemember{Edward W.~Larsen}{Professor}
\committeemember{Thomas J.~Downar}{Professor}
\committeemember{James P.~Holloway}{Professor}
\committeemember{William R.~Martin}{Professor}
\committeemember{Katsuyo S.~Thornton}{Professor}

%%%%%%%%%%%%%%%%%%%%%%%%%%%%%%%%%%%%%%%%%%%%%%%%%%%%%%%%%%%%%%%%%%%%%%%%%%%%%%%%
\begin{document}

% ABSTRACT: up to 350 words ; vim: [select] g CTRL+g
\begin{finalabstract}

In this thesis, we develop and numerically test two approximations to
time-dependent radiation transport with the goal of better accuracy than
diffusion and significantly lower computational cost compared to
transport. The first method is the nascent anisotropic diffusion (AD)
approximation, which we extend to time-dependent, finite problems; the
second is a novel anisotropic \Pone-like (\APone) approximation. These methods
are ``anisotropic'' in that, rather than operating under the assumption of
linearly anisotropic radiation, they incorporate an arbitrary amount of
anisotropy via a transport-calculated diffusion tensor. This anisotropic
diffusion coefficient is the second angular moment of a simple, purely
absorbing transport problem.

Much of the computational testing of these new methods is performed in
``flatland'' geometry, a fictional two-dimensional universe that provides a
realistic but computationally simpler testbed. As work ancillary to anisotropic
diffusion and the numerical experiments,a complete description
of flatland diffusion, including boundary conditions, is developed. We also
provide methodologies for implementing both Monte Carlo and \SN\ transport in
flatland.

The two anisotropic methods, along with a ``flux limited'' modification to
anisotropic diffusion, are tested in a variety of problems. Some aspects of the
theory, including the newly formulated boundary conditions, are tested with
diffusive steady-state problems. Next, we compare the new methods against
existing approximations to linear, time-dependent radiation transport. Finally,
we
investigate the efficacy and performance of the anisotropic methods in a variety
of thermal radiative transfer (TRT) computational experiments.

Our results demonstrate that for many problems, the new anisotropic
methods perform far better than their conventional counterparts.
\end{finalabstract}

%%%%%%%%%%%%%%%%%%%%%%%%%%%%%%%%%%%%%%%%%%%%%%%%%%%%%%%%%%%%%%%%%%%%%%%%%%%%%%%%
\end{document}

