% !TEX root = _individual/trtNumericalResults.tex

%%%%%%%%%%%%%%%%%%%%%%%%%%%%%%%%%%%%%%%%%%%%%%%%%%%%%%%%%%%%%%%%%%%%%%%%%%%%%%%%
\chapter{Numerical Results: Thermal Radiative Transfer}
\label{chap:trtNumericalResults}

The large phase space of 2-D thermal radiative transfer makes visualization---%
especially when comparing several methods---%
a formidable challenge. The radiation unknown is a function of time, \xy\
space, and angle; the material unknown is also time- and space-dependent.
Analysis is further complicated by how to express the results; for example,
should we compare the radiation energy density $\phi$ or
the radiation temperature $T_\text{rad}=\phi^{1/4}$? Does the entire solution
space matter, or do we only care about one particular output such as the
position of the radiation wavefront?

%These questions are unanswerable in the general sense. To guide us, we have
%prior, and hopefully more well-informed, authors.

One standard TRT benchmark, the Su--Olson problem \cite{Su1997}, is absent from
our comparisons. The justification for its omission is obvious: it has an
opacity constant in space and time; the AD method then gives a solution identical
to diffusion.

For these numerical simulations, we compare the results of several methods.

%%%%%%%%%%%%%%%%%%%%%%%%%%%%%%%%%%%%%%%%
\paragraph{Implicit Monte Carlo}

%%%%%%%%%%%%%%%%%%%%%%%%%%%%%%%%%%%%%%%%
\paragraph{Discrete ordinates}

%%%%%%%%%%%%%%%%%%%%%%%%%%%%%%%%%%%%%%%%
\paragraph{Flux-limited diffusion}

%%%%%%%%%%%%%%%%%%%%%%%%%%%%%%%%%%%%%%%%
\paragraph{Anisotropic diffusion}

%%%%%%%%%%%%%%%%%%%%%%%%%%%%%%%%%%%%%%%%
\paragraph{Flux-limited anisotropic diffusion}

%%%%%%%%%%%%%%%%%%%%%%%%%%%%%%%%%%%%%%%%%%%%%%%%%%%%%%%%%%%%%%%%%%%%%%%%%%%%%%%%
\section{1-D blast wave}
This test problem is from \cite{Rau2005,Ols2007}.

%%%%%%%%%%%%%%%%%%%%%%%%%%%%%%%%%%%%%%%%%%%%%%%%%%%%%%%%%%%%%%%%%%%%%%%%%%%%%%%%
\section{1-D boundary source}

%%%%%%%%%%%%%%%%%%%%%%%%%%%%%%%%%%%%%%%%%%%%%%%%%%%%%%%%%%%%%%%%%%%%%%%%%%%%%%%%
\section{Flatland pipe with blast wave}

%%%%%%%%%%%%%%%%%%%%%%%%%%%%%%%%%%%%%%%%%%%%%%%%%%%%%%%%%%%%%%%%%%%%%%%%%%%%%%%%
\section{Flatland pipe with boundary source}

%%%%%%%%%%%%%%%%%%%%%%%%%%%%%%%%%%%%%%%%%%%%%%%%%%%%%%%%%%%%%%%%%%%%%%%%%%%%%%%%
\section{CRASH problem}
The profile of energy deposition in the wall is an important diagnostic in the
radiation hydrodynamics problem relevant to the CRASH program \cite{HolCom2011}.

%%%%%%%%%%%%%%%%%%%%%%%%%%%%%%%%%%%%%%%%%%%%%%%%%%%%%%%%%%%%%%%%%%%%%%%%%%%%%%%%
\section{2-D obstacle problems}
\cite{Mou2006}

%%%%%%%%%%%%%%%%%%%%%%%%%%%%%%%%%%%%%%%%%%%%%%%%%%%%%%%%%%%%%%%%%%%%%%%%%%%%%%%%
\section{Summary}

