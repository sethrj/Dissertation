% !TEX root = _individual/simpleNumericalResults.tex

%%%%%%%%%%%%%%%%%%%%%%%%%%%%%%%%%%%%%%%%%%%%%%%%%%%%%%%%%%%%%%%%%%%%%%%%%%%%%%%%
\chapter{Numerical Results: Test Problems}


%%%%%%%%%%%%%%%%%%%%%%%%%%%%%%%%%%%%%%%%%%%%%%%%%%%%%%%%%%%%%%%%%%%%%%%%%%%%%%%%
\section{Low-order spatial discretization}

%%%%%%%%%%%%%%%%%%%%%%%%%%%%%%%%%%%%%%%%%%%%%%%%%%%%%%%%%%%%%%%%%%%%%%%%%%%%%%%%
\section{Boundary conditions}

Angular distributions

\begin{table}[htb]
  \centering
  \begin{tabular}{ccc}
\toprule
    Distribution & 1-D & Flatland
\\ \midrule
Isotropic & $I(\mu) = \frac{1}{2}$ & $I(\theta) = \frac{\pi}{2}$
\\
Normal & $I(\mu) = \delta(\mu-1)$ & $I(\theta) = \delta(\theta-\pi/2)$
\\
Grazing & $I(\mu) = \delta(\mu-0.1)$ & $I(\theta) = \delta(\theta-0.1)$
\\ \bottomrule
  \end{tabular}
  \caption{Angular distributions used in boundary condition tests.}
  \label{tab:angularDistributions}
\end{table}

These are the same angular distributions used in a boundary matching analysis
in \cite{Dav2006}.

\horizsep

As a test problem, let's look at an optically thick one-dimensional problem
with a specified incident boundary source on the left, and a reflecting
boundary many mean free paths to the right. It has a total opacity $\sigma$, a
scattering opacity $c\sigma$, and a uniform source $q$. Because the problem is
optically thick (approximately semi-infinite), $f$ will have the infinite
homogeneous medium value of $1/2\sigma$ for $-1 < \mu < 0$.

For the ``exact'' solution, we use S$_{32}$ with the step characteristic
discretization. A cell-centered diffusion approximation is used for
the low-order solution. The ``anisotropic diffusion'' coefficients are
calculated using step characteristic transport.

\horizsep

Instead of applying the true albedo-white boundary condition to the
left hand side, we could use an incident isotropic source matched to the
analytic expected albedo-white value. For the semi-consistent (not transport
corrected) value,
\begin{align*}
  j^{\ell,\mathrm{in}} &= \frac{3}{4} \int_{-1}^{0} \abs{\mu'} f(0, \mu') \ud\mu' \\
  &= \frac{3}{4} \int_{-1}^{0} \abs{\mu'} \frac{1}{2\sigma} \ud\mu' \\
  &= \frac{1}{16\sigma} \,,
  \\ 
  \intertext{which corresponds to}
  f(0, \mu) &=  \frac{1}{16\sigma} \,,\quad 0 < \mu \le 1\,.
\end{align*}
So the high order transport problem for $f$ effectively has an incident
isotropic boundary on the left with incident current $1/16\sigma$.

The semi-consistent boundary condition for $f$ yields the blue ``anisotropic''
diffusion coefficient in
Fig.~\ref{fig:bndycondDcoeff}.  The black line uses the transport-corrected
boundary albedo.  Compare these to the red line, which is the
standard diffusion coefficient of $1/3\sigma$, and to the red line, which
is the na\"ive way of calculating the AD coefficients with an incident
vacuum boundary of $j^{\ell,\mathrm{in}}=0$.
\begin{figure}[htb]
  \centering
  \input{bndyconditions/dcoeff-bl-homog/include}
  \caption{Standard and ``anisotropic'' (transport-calculated) diffusion
  coefficients for the case with $\sigma=1$, $F^{\ell,\mathrm{in}}=0$.}
  \label{fig:bndycondDcoeff}
\end{figure}

The boundary conditions on $f$ must be paired with consistent boundary
conditions on the low-order problem for $\phi$,
Eq.~\eqref{eq:bndyAnisotropicOnedSs}. In this simple semi-infinite problem,
that evaluates to
\begin{equation*}
  F^{\ell,\mathrm{in}}
  = \frac{1}{4} \phi(0)
  + \frac{1}{2} j^{\ell,\mathrm{out}} \pder{\phi}{x}(0)
  = \frac{1}{4} \phi(0)
  + \frac{1}{2} \frac{1}{4\sigma} \pder{\phi}{x}(0) \,.
\end{equation*}
Compare this to the standard Marshak boundary condition used by the na\"ive AD
method (where a vacuum incident boundary on $f$ gives $D(0)=1/6\sigma$):
\begin{equation*}
  F^{\ell,\mathrm{in}}
  = \frac{1}{4} \phi(0)
  + \frac{1}{2} D(0) \pder{\phi}{x}(0)
  = \frac{1}{4} \phi(0)
  + \frac{1}{2} \frac{1}{6\sigma} \pder{\phi}{x}(0) \,.
\end{equation*}

\begin{figure}[htb]
  \centering
  \hspace{-.6in}
  \subfigure[$q=0.001$, $F^{\ell,\mathrm{in}}=0$]{
  \input{bndyconditions/phi-bl-homog/include}
  }
  \hspace{-.2in}
  \subfigure[$q=0$, $F^{\ell,\mathrm{in}}=1$]{
  \input{bndyconditions/phi-bl-homog-nosource/include}
  }
  \hspace{-.6in}
  \caption{Solutions with a semi-infinite homogeneous medium, $\sigma=1$,
  $c=0.999$.}
  \label{fig:bndycondSolutions}
\end{figure}

\begin{figure}[htb]
  \centering
  \hspace{-.6in}
  \subfigure[$q=0.001$, $F^{\ell,\mathrm{in}}=0$]{
  \input{bndyconditions/err-bl-homog/include}
  }
  \hspace{-.2in}
  \subfigure[$q=0$, $F^{\ell,\mathrm{in}}=1$]{
  \input{bndyconditions/err-bl-homog-nosource/include}
  }
  \hspace{-.6in}
  \caption{Absolute errors with a semi-infinite homogeneous medium, $\sigma=1$,
  $c=0.999$.}
  \label{fig:bndycondErrors}
\end{figure}

\begin{figure}[htb]
  \centering
  \input{bndyconditions/plot-bl-homog-c/include}
  \caption{Absolute error in $\phi$ at three points with a semi-infinite
  homogeneous medium, vacuum boundary on the left, $\sigma=1$,
  $q=\sigma(1-c)$, as a function of the scattering ratio $c$ for the three
  methods.}
  \label{fig:bndycondHomogC}
\end{figure}

\begin{figure}[htb]
  \centering
  \input{bndyconditions/plot-bl-homog-c-nosource/include}
  \caption{Absolute error in $\phi$ at three points with a semi-infinite
  homogeneous medium, incident source
  boundary of $F^{\ell,\mathrm{in}}=1$ on the left, $\sigma=1$, $q=0$, as a
  function of the scattering ratio $c$.}
  \label{fig:bndycondHomogC}
\end{figure}
%
%\begin{figure}[htb]
%  \centering
%  \input{bndyconditions/plot-bl-homog-left-inc-nosource/include}
%  \caption{Error norms with a semi-infinite homogeneous medium, incident
%  boundary on the left, $\sigma=1$, with a varying incident intensity on the
%  left.}
%  \label{fig:bndycondHomogC}
%\end{figure}

Standard diffusion approximates the intensity as
\begin{equation*}
  I_\mathrm{D}(x, \mu) \approx \frac{1}{2} \phi(x) - \frac{3}{2} \mu D(x) \pder{\phi}{x}(x)\,,
\end{equation*}
and AD approximates it as
\begin{equation*}
  I_\mathrm{AD}(x, \mu) \approx \frac{1}{2} \phi(x) - \mu f(x, \mu)
  \pder{\phi}{x}(x)\,
\end{equation*}
where $f$ is determined by a transport problem. Normally, only the second
moment of $f$ is calculated by accumulation during a transport sweep, but if we
store the full angular form, we can plot $I_\mathrm{AD}(x, \mu)$.

Fig.~\ref{fig:bndycondAdfOut} plots $f$ at the boundary and at one mean free
path into the problem. The coefficient that gives $\int \mu^2 f(x,\mu) \ud \mu
=D $ is $f=1/2\sigma$.

\begin{figure}[htb]
  \centering
  \hspace{-.6in}
  \subfigure[$x=0.00125$]{
  \input{bndyconditions/adf-bl-homog-x0/include}
  }
  \hspace{-.2in}
  \subfigure[$x=1$]{
  \input{bndyconditions/adf-bl-homog-x1/include}
  }
  \hspace{-.6in}
  \caption{Calculated values for $f$.}
  \label{fig:bndycondAdfOut}
\end{figure}

Figure~\ref{fig:bndycondAngularIntensity} plots the angular intensity at the
boundary (a), half a mean free path away (b), and one mean free path away (c),
for the four compared methods. 
\begin{figure}[htb]
  \centering
  \hspace{-.6in}
  \subfigure[$x=0.00125$]{
  \input{bndyconditions/angle-bl-homog-x0/include}
  }
  \hspace{-.2in}
  \subfigure[$x=0.5$]{
  \input{bndyconditions/angle-bl-homog-x0-5/include}
  }
  \hspace{-.6in}

  \subfigure[$x=1$]{
  \input{bndyconditions/angle-bl-homog-x1/include}
  }
  \caption{Angular intensity near a vacuum boundary in a semi-infinite
  homogeneous medium.}
  \label{fig:bndycondAngularIntensity}
\end{figure}


\clearpage
\begin{figure}[htb]
  \centering
  \hspace{-.6in}
  \subfigure[Scalar intensity]{
  \input{bndyconditions/phi-bl-deltasource/include}
  }
  \hspace{-.2in}
  \subfigure[Relative error]{
  \input{bndyconditions/relerr-bl-deltasource/include}
  }
  \hspace{-.6in}
  \caption{Incident delta source on the left, $\sigma=1$,
  $c=0.999$, no internal source, vacuum on the right.}
  \label{fig:bndycondSolutions}
\end{figure}

