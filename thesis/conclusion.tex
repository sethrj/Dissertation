% !TEX root = _individual/conclusion.tex

%%%%%%%%%%%%%%%%%%%%%%%%%%%%%%%%%%%%%%%%%%%%%%%%%%%%%%%%%%%%%%%%%%%%%%%%%%%%%%%%
\chapter{Conclusion}\label{chap:conclusion}
%%%%%%%%%%%%%%%%%%%%%%%%%%%%%%%%%%%%%%%%%%%%%%%%%%%%%%%%%%%%%%%%%%%%%%%%%%%%%%%%

Our new method is simply the best ever! Please let me graduate.

\section{Future work}
%%%%%%%%%%%%%%%%%%%%%%%%%%%%%%%%%%%%%%%%%%%%%%%%%%%%%%%%%%%%%%%%%%%%%%%%%%%%%%%%
\section{Improved discontinuity treatment}
In the derivation of the AD method, we made the ansatz that
$\vec{F}=O(\epsilon)$. If we instead supposed that $\vec{F}=O(1)$, the same
Taylor expansion would yield an extra term:
\begin{equation*}
  \tilde\Psi(\vec{x}, \vec{\Omega}, t)
  \approx \cdots + \lopinv{v}{ \frac1{4\pi} } \grad \vd \vec{F}(\vec{x},t)\,,
\end{equation*}
resulting in
\begin{equation*}
  \vec{F}(\vec{x}, t)
  \approx - \Dtens \vd \grad \phi(\vec{x},t)
  + \left\{ \int_{4\pi} \vec{\Omega} \lopinv{v}{ \frac1{4\pi} } \ud\Omega
  \right\} \grad \vd \vec{F}(\vec{x},t)
\end{equation*}

Away from material discontinuities, the term in brackets would be
exponentially small (since $f$ would be an even function of $\vec{\Omega}$).
Near a change in material, however, this might yield a better solution that has
a ``kink'' (discontinuous first derivative) like the transport solution.

%%%%%%%%%%%%%%%%%%%%%%%%%%%%%%%%%%%%%%%%%%%%%%%%%%%%%%%%%%%%%%%%%%%%%%%%%%%%%%%%
\section{Higher-order anisotropic P$_N$ methods}
The conceptualization of an anisotropic transport equation opens the door to
any number of potential approximations to radiation transport.

For example, instead of defining the anisotropic intensity as $I -
\frac{1}{4\pi}\phi$, we could also subtract off the linearly anisotropic
component:
\begin{equation*}
  \Psi' \equiv I - \frac{1}{4\pi}\phi - \frac{3}{4\pi} \vec{\Omega} \vd
  \vec{F}\,.
\end{equation*}
We could then perform a similar analysis that in chapter~\ref{chap:adDerivation}
and get an approximate form for the second moment of the intensity,
\begin{equation*}
  \int_{4\pi} \vec{\Omega}\vec{\Omega} I \ud \Omega
  = \int_{4\pi} \vec{\Omega}\vec{\Omega} \Psi' \ud \Omega \,.
\end{equation*}
This would be substituted into the first moment of the transport equation,
\begin{equation*}
  \frac{1}{c}\pder{\vec{F}}{t}
  + \vec{\Omega} \vd \int_{4\pi} \vec{\Omega}\vec{\Omega} I \ud \Omega
  + \sigma \vec{F}
  = 0 \,.
\end{equation*}
It would almost certainly involve higher-order gradients of $\phi$ and $\vec{F}$
as well as more complex tensors.

