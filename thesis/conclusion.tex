% !TEX root = _individual/conclusion.tex

%%%%%%%%%%%%%%%%%%%%%%%%%%%%%%%%%%%%%%%%%%%%%%%%%%%%%%%%%%%%%%%%%%%%%%%%%%%%%%%%
\chapter{Conclusions}\label{chap:conclusion}

In this thesis, we have developed and numerically tested three anisotropic
diffusion approximations to time-dependent radiation transport and
nonlinear radiative transfer. Ancillary to the numerical testing, we
investigated flatland geometry and new discretization schemes suitable for the
anisotropic diffusion methods. We conclude by reviewing the new developments and
results, then discussing future refinements and improvements to the methods.

%%%%%%%%%%%%%%%%%%%%%%%%%%%%%%%%%%%%%%%%%%%%%%%%%%%%%%%%%%%%%%%%%%%%%%%%%%%%%%%%
\section{Anisotropic diffusion}

The anisotropic diffusion (AD) approximation, initially formulated in prior work, has been extended to finite and time-dependent problems
(Chapter~\ref{chap:adDerivation}). Using an asymptotic
derivation that appeals to the physical scaling of the problem, we
obtained a new, arbitrarily anisotropic (as opposed to linearly anisotropic in
the case of standard diffusion) approximation to the angular intensity.
Boundary and initial layer analyses gave transport-matched boundary and initial
conditions for the new method.

We then discussed properties of the new AD approximation solely on the basis of
the underlying equations. We postulated that the method should be only slightly
more expensive than diffusion: in an infinite-medium problem with
time-independent opacities, only one transport sweep is needed to calculate
the anisotropic diffusion coefficients for all time steps. (In finite medium
problems, an additional few sweeps may be needed to converge the opposing
reflecting/white boundary conditions.) Furthermore, unlike time-dependent
transport methods, the time-dependent AD methods do not require the storage of
the full angle-dependent intensity. Anisotropic diffusion is faster and less
memory-intensive than transport methods.

One complication present in the anisotropic diffusion methods is
that, in most problems, the diffusion tensor has nonzero off-diagonal terms.
These terms cause a gradient in one direction to induce particle flow in an
orthogonal direction. To account for this ``transverse'' leakage, we generalized
an existing cell-centered difference scheme (Chapter~\ref{chap:implementation}).
It reduces to a conventional 5-point finite difference method when the diffusion
tensors are isotropic, and it does not increase the number of unknowns compared
to cell-centered finite difference diffusion. With the penalty of a slightly
less sparse matrix structure, these new difference schemes account for the
transverse leakage in anisotropic diffusion

Our numerical results (Chapter~\ref{chap:simpleNumericalResults}) began by
testing the fidelity of \SN-calculated AD coefficients. We obtained accurate
answers with a coarse quadrature set (16 ordinates per quadrant), few
transport sweeps (about two), and a coarse grid (so long as material boundaries
were preserved).

Comparing plots of the intensity as a function of angle, we confirmed the
hypothesis that the arbitrary anisotropy inherent to the AD approximation
allows it to better represent the shape of the true intensity. We found
the linear-in-angle representation of conventional diffusion to be inadequate
in problems with strong anisotropy in the solution.

We also tested the newly derived boundary conditions. In most cases, they
accurately preserved the transport solution away from the boundaries, even in
the presence of voids. The exceptions that performed poorly were in cases of
strongly anisotropic specified boundaries incident on a void. (Because the AD
boundary
conditions reduce to the diffusion boundary conditions in a homogeneous medium,
these same boundary conditions in a problem without a void perform well.)

In time-dependent problems, the anisotropic diffusion approximation produced
more accurate answers than standard diffusion, but in voids, it allowed energy
to travel too quickly. The same behavior was seen in thermal radiative transfer
problems but was less pronounced: even the streaming areas are
relatively opaque at their initial cold temperatures.

The anisotropic diffusion method shows great promise in steady-state problems.
Yet because the anisotropic diffusion equations are parabolic, they have the
shortcoming that in time-dependent problems, radiation may travel faster than
$c$,~the speed of light. We addressed this issue by proposing a modification to
the AD approximation (flux limiting) and by developing a second ``anisotropic''
approximation (anisotropic \Pone).

%%%%%%%%%%%%%%%%%%%%%%%%%%%%%%%%%%%%%%%%%%%%%%%%%%%%%%%%%%%%%%%%%%%%%%%%%%%%%%%%
\section{Flux-limited anisotropic diffusion}

Adding a flux limiter to the anisotropic diffusion approximation
(\S\ref{sec:flad}) artificially reduced the diffusion tensor's magnitude in
the presence of strong gradients, inhibiting the spread of energy at a wavefront
to restore the qualitative feature of transport that transfer is limited to the
speed of light. Flux-limited anisotropic diffusion (FLAD) yields results
similar to AD in problems with weak time derivatives and large opacities, but
our results show that it improves the accuracy of the solution during the
initial transients of a Marshak wave.
Testing showed the ``max'' limiter with AD is used less frequently than with
standard diffusion: as expected, the \emph{ad hoc}
fix-up is less necessary for the anisotropic diffusion approximation.

We found that FLAD produced very accurate solutions in a wide
variety of problems with voided channel-like regions. The anisotropic diffusion
approximation, when modified by the flux limiter, was especially efficacious in
pipe-like problems with strong multi-dimensional behavior, which conventional
flux-limited diffusion could not approximate well.

%%%%%%%%%%%%%%%%%%%%%%%%%%%%%%%%%%%%%%%%%%%%%%%%%%%%%%%%%%%%%%%%%%%%%%%%%%%%%%%%
\section{Anisotropic \texorpdfstring{\Pone}{P1}}

In the derivation of the anisotropic diffusion approximation, we assumed a very
slowly changing solution. Revisiting that assumption and allowing a stronger
$O(\epsilon)$ time derivative, we
derived a set of \Pone-like equations that contained the anisotropic diffusion
tensor $\Dtens$ (Chapter~\ref{chap:aponeDerivation}). 

These equations
had a shortcoming which was corrected by the \emph{ad hoc} replacement of the
opacity $\sigma$ with a nonlocal $\varsigma$ that could be
calculated in the same transport sweep as $\Dtens$. The resulting Anisotropic
\Pone\ (\APone) approximation behaves well in voids, reduces to the AD
approximation at steady state, and becomes the standard \Pone\ approximation in
a homogeneous medium.
Using a boundary layer analysis, we also obtained boundary conditions for this
new method.

The low-order unknowns for \APone\ comprise both $\phi$ and $\vec{F}$, compared
to just $\phi$ needed with the AD method.  As a result, a different spatial
discretization is also needed. We formulated a staggered-mesh
scheme for \APone\ based on an extant \Pone\ discretization.

We found that \APone-calculated solutions behaved like standard \Pone\ in some
problems (with large voided regions) and like anisotropic diffusion in others
(with smaller voided regions). In the case of large voids, \APone\ showed less
unphysically oscillatory behavior than \Pone.

There were no problems in which \APone\ was superior to FLAD,
despite that the equations of \APone\ are more complex than those of FLAD. (They
are not
self-adjoint, and they have a larger unknown space and therefore require more
computer memory.)
As a result, we cannot recommend the new anisotropic \Pone\ approximation over
flux-limited anisotropic diffusion.

%%%%%%%%%%%%%%%%%%%%%%%%%%%%%%%%%%%%%%%%%%%%%%%%%%%%%%%%%%%%%%%%%%%%%%%%%%%%%%%%
\section{Flatland geometry}

Many of our test problems used the ``flatland'' geometry
(Chapter~\ref{chap:flatland}), which has a smaller angular phase space than 2-D
geometry and qualitatively simulates channels more realistically. (In 2-D
geometry, what looks like a channel is the projection of two infinite walls.)
Flatland geometry is not as straightforward as it seemed at first glance,
however. The diffusion coefficients, Marshak boundary conditions,
transport-matched boundary conditions, and black-body radiation boundary
conditions all are different from their standard 2-D formulations.

In addition to prescribing sampling routines for flatland Monte Carlo
calculations and presenting \SN\ quadrature sets that preserve the flatland
angular moments (the equivalent of spherical harmonic functions), we more
thoroughly investigated the diffusion approximation in flatland. We derived
flatland Marshak boundary conditions and variational boundary conditions for
diffusion, and generalized them so that they could apply to AD and \APone\ as
well. We verified these boundary conditions numerically in
Chapter~\ref{chap:simpleNumericalResults}.

%%%%%%%%%%%%%%%%%%%%%%%%%%%%%%%%%%%%%%%%%%%%%%%%%%%%%%%%%%%%%%%%%%%%%%%%%%%%%%%%
\section{Future work}

The anisotropic diffusion approximations are still in their infancy. Their
accuracy may be improved in a number of ways, which we now discuss.

%%%%%%%%%%%%%%%%%%%%%%%%%%%%%%%%%%%%%%%%
\subsection{Improved discontinuity treatment}

In the derivation of the anisotropic diffusion method, we discarded
terms of $O(\epsilon^2)$ using a particular asymptotic scaling. Retaining more
terms---making fewer approximations---should result in a more accurate
approximation to the radiative intensity.

By retaining the $\frac{1}{4\pi} \grad \vd \vec{F}$ term in
Eq.~\eqref{eq:capPsiVol}, we can derive at a new approximation that takes the place
of Fick's law:
\begin{equation*}
  \vec{F} = - \Dtens \vd \grad \phi + \vec{E} (\grad \vd\vec{F})\,,
\end{equation*}
where
\begin{equation*}
  \vec{E} = \int_{4\pi} \vec{\Omega} f \ud\Omega\,.
\end{equation*}

Away from material discontinuities, $\vec{E}$ is exponentially small; but
near a discontinuity, $\vec{E}$ is non-negligible. This additional advection
term may
improve the accuracy of the anisotropic diffusion approximation in an
asymptotic order-of-accuracy sense, and it will re-introduce a qualitative
property that our anisotropic
diffusion approximation lacks: a ``kink'' in the solution $\phi$. (Because
$\vec{E}$ is discontinuous at a material boundary, it will induce a
discontinuity in $\grad \phi$.) This new method will then have at a material
interface both a boundary layer (which transport and AD have but standard
diffusion does not) and a discontinuous first derivative (which transport and
standard diffusion have but AD does not).

The difficulty with this extra term is that advection-diffusion equations
require troublesome discretization schemes. Also, advection-diffusion is
not self-adjoint,
so the fast and simple conjugate gradient (CG) method could no longer be used to
solve the resulting matrix: more costly methods (such as GMRES) would be
required.

%%%%%%%%%%%%%%%%%%%%%%%%%%%%%%%%%%%%%%%%
\subsection{Alternative flux limiters}

In \S\ref{sec:flad}, we proposed a flux-limited modification to anisotropic
diffusion theory. For ease of understanding and implementation, we chose a
``max'' limiter that multiplies the tensor by a scalar that preserves the
physical property $\vec{F} \le \phi$. However, other flux limiting
methods---some of which may be more well behaved numerically---are possible.
These flux limiters would be more complicated than the standard scalar diffusion
coefficients: the extra degrees of freedom in being anisotropic
\emph{tensors} rather than isotropic \emph{scalars} would allow more options in
formulating the tensor.
They may also be more computationally expensive. For a ``square root''-type
limiter, as an
example, a ``matrix square root'' would have to be calculated in each cell at
each time step.

%The
%two properties a flux-limited diffusion tensor $\tilde\Dtens$ must retain are
%that:
%\begin{enumerate}
%  \item it must reduce to the true, transport-calculated $\Dtens$ in the absence
%    of strong gradients; and
%  \item in the presence of strong gradients it must satisfy $ \norm{
%    \tilde\Dtens \vd \grad \phi} = \phi$.
%\end{enumerate}
%Because the anisotropic diffusion coefficient is, in multiple spatial
%dimensions, a rank-2 tensor, there is a degree of flexibility not present in
%standard diffusion theory.

%%%%%%%%%%%%%%%%%%%%%%%%%%%%%%%%%%%%%%%%
\subsection{Multigroup}

Flux-limited diffusion is often used with multiple energy groups (frequency
groups) to improve the
fidelity of the simulation. Because of the strong variation of opacity with
frequency, parts of the problem that are optically thick in one group may be optically
thin in another. The AD approximation will likely provide a better treatment of
these groups than standard FLD. However, a disadvantage of multigroup solutions is that
each group will require its own transport-calculated $\Dtens$, This might be
too costly a requirement for it to remain competitive against IMC and other
transport methods.

%%%%%%%%%%%%%%%%%%%%%%%%%%%%%%%%%%%%%%%%
\subsection{Three-dimensional problems}

Time-dependent anisotropic diffusion has been formulated for general
3-D geometry (see Chapter~\ref{chap:adDerivation}), but it has not
been implemented or tested in 3-D. The major differences between our work
and 3-D are that the
diffusion tensor is a $3\times 3$ matrix, the diffusion matrix is more
complex, and the cost of the transport sweep increases.

%%%%%%%%%%%%%%%%%%%%%%%%%%%%%%%%%%%%%%%%
\subsection{Discretization schemes}

The cell-centered discretization schemes of Chapter~\ref{chap:implementation}
are only a first attempt at developing an accurate, cheap discretization for
anisotropic diffusion. The Support Operators Method
\cite{Mor1998,Run2006}, even though it uses more unknowns than
cell-centered diffusion, may compensate for this added expense by being more accurate. Finite element
methods might also be applied effectively to the anisotropic diffusion methods.
Because the cell-centered scheme we derived is only applicable to ``brick''
geometry (regular Cartesian mesh), other discretization schemes will undoubtedly
be needed (and equally undoubtedly already exist in other fields) for more
complex geometries.

%%%%%%%%%%%%%%%%%%%%%%%%%%%%%%%%%%%%%%%%
\subsection{Preconditioners}

As with standard diffusion methods, the performance of the anisotropic diffusion method could be improved
by using Krylov solvers with well-chosen preconditioners. Multigrid
preconditioning has already been applied to nonlinear diffusion \cite{Rid1999};
the extension to anisotropic diffusion should be straightforward.

%%%%%%%%%%%%%%%%%%%%%%%%%%%%%%%%%%%%%%%%%%%%%%%%%%%%%%%%%%%%%%%%%%%%%%%%%%%%%%%%
\section{Final remarks}

The anisotropic diffusion methods that we derived and tested incorporate an
arbitrary amount of anisotropy into their approximation of the angular
intensity, producing results superior to diffusion in a number of problems. Of
these new methods, flux-limited anisotropic diffusion is the most accurate.
Even though the calculation of the anisotropic diffusion tensor
requires the moderate cost of a transport sweep, the FLAD solution in many
problems is commensurately more accurate compared to conventional FLD. Our
goal of developing a newer method more accurate than FLD yet less expensive
than transport has been attained. Based on
our numerical testing, we believe FLAD to be a strong contender for
accurate, inexpensive multi-dimensional simulations for time-dependent
transport and thermal radiative transfer.

