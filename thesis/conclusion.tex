% !TEX root = _individual/conclusion.tex

%%%%%%%%%%%%%%%%%%%%%%%%%%%%%%%%%%%%%%%%%%%%%%%%%%%%%%%%%%%%%%%%%%%%%%%%%%%%%%%%
\chapter{Conclusion}
%%%%%%%%%%%%%%%%%%%%%%%%%%%%%%%%%%%%%%%%%%%%%%%%%%%%%%%%%%%%%%%%%%%%%%%%%%%%%%%%

Our new method is simply the best ever! Please let me graduate.

\section{Future work}
%%%%%%%%%%%%%%%%%%%%%%%%%%%%%%%%%%%%%%%%%%%%%%%%%%%%%%%%%%%%%%%%%%%%%%%%%%%%%%%%
In the derivation of the AD method, we made the ansatz that
$\vec{F}=O(\epsilon)$. If we instead supposed that $\vec{F}=O(1)$, the same
Taylor expansion would yield an extra term:
\begin{equation*}
  \tilde\Psi(\vec{x}, \vec{\Omega}, t)
  \approx \cdots + \lopinv{v}{ \frac1{4\pi} } \grad \vd \vec{F}(\vec{x},t)\,,
\end{equation*}
resulting in
\begin{equation*}
  \vec{F}(\vec{x}, t)
  \approx - \Dtens \vd \grad \phi(\vec{x},t)
  + \left\{ \int_{4\pi} \vec{\Omega} \lopinv{v}{ \frac1{4\pi} } \ud\Omega
  \right\} \grad \vd \vec{F}(\vec{x},t)
\end{equation*}

Away from material discontinuities, the term in brackets would be
exponentially small (since $f$ would be an even function of $\vec{\Omega}$).
Near a change in material, however, this might yield a better solution that has
a ``kink'' (discontinuous first derivative) like the transport solution.

