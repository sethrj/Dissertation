% !TEX root = _individual/conclusion.tex

%%%%%%%%%%%%%%%%%%%%%%%%%%%%%%%%%%%%%%%%%%%%%%%%%%%%%%%%%%%%%%%%%%%%%%%%%%%%%%%%
\chapter{Conclusions}\label{chap:conclusion}



%%%%%%%%%%%%%%%%%%%%%%%%%%%%%%%%%%%%%%%%%%%%%%%%%%%%%%%%%%%%%%%%%%%%%%%%%%%%%%%%
\section{Anisotropic diffusion}

%%%%%%%%%%%%%%%%%%%%%%%%%%%%%%%%%%%%%%%%%%%%%%%%%%%%%%%%%%%%%%%%%%%%%%%%%%%%%%%%
\section{Flux-limited anisotropic diffusion}

%%%%%%%%%%%%%%%%%%%%%%%%%%%%%%%%%%%%%%%%%%%%%%%%%%%%%%%%%%%%%%%%%%%%%%%%%%%%%%%%
\section{Anisotropic \texorpdfstring{\Pone}{P1}}

%%%%%%%%%%%%%%%%%%%%%%%%%%%%%%%%%%%%%%%%%%%%%%%%%%%%%%%%%%%%%%%%%%%%%%%%%%%%%%%%
\section{Future work}

The anisotropic diffusion approximation is still in its infancy. Its novel
formulation, which uses an integral transport operation and asymptotic
expansions, can be extended to generate entirely new methods. The application
of anisotropic diffusion to time-dependent transport also has room to grow,
both in the use of flux limiters and in 

%%%%%%%%%%%%%%%%%%%%%%%%%%%%%%%%%%%%%%%%
\subsection{Improved discontinuity treatment}
In the derivation of the anisotropic diffusion method, we discarded
terms of $O(\epsilon^2)$ using the following scaling of Eq.~\eqref{eq:capPsiVol}:
\begin{equation*}
  \Big[ \underbrace{\frac{1}{c}\pder{}{t}}_{O(\epsilon^2)}
  + \underbrace{\vec{\Omega}\vd \grad}_{O(\epsilon^2)}
  + \underbrace{\sigma}_{O(1)} \Big]
   \underbrace{\left( \Iv
  - \frac{1}{4\pi} \phi \right)}_{O(1)}
  = \frac{1}{4\pi}  \underbrace{\grad \vd\vec{F}}_{O(\epsilon^2)} -
  \frac{1}{4\pi} \vec{\Omega}\vd  \underbrace{\grad \phi}_{O(\epsilon)}\,.
\end{equation*}
If we instead take the scaling that $I - \frac{1}{4\pi}\phi=
O(\epsilon)$, which is true for the traditional diffusion scaling, then:
\begin{equation*}
  \Big[ \underbrace{\frac{1}{c}\pder{}{t}}_{O(\epsilon^2)}
  + \underbrace{\vec{\Omega}\vd \grad}_{O(\epsilon^2)}
  + \underbrace{\sigma}_{O(1)} \Big]
   \underbrace{\left( \Iv
  - \frac{1}{4\pi} \phi \right)}_{O(\epsilon)}
  = \frac{1}{4\pi}  \underbrace{\grad \vd\vec{F}}_{O(\epsilon^2)} -
  \frac{1}{4\pi} \vec{\Omega}\vd  \underbrace{\grad \phi}_{O(\epsilon)}\,,
\end{equation*}
where the $O(\epsilon^2)$ terms must be retained:
\begin{equation*}
  \Big[ \vec{\Omega}\vd \grad
  + \underbrace{\sigma} \Big]
   \left( \Iv - \frac{1}{4\pi} \phi \right)
  = \frac{1}{4\pi}  \grad \vd\vec{F}
 - \frac{1}{4\pi} \vec{\Omega}\vd  \grad \phi + O(\epsilon^3)\,.
\end{equation*}
Inverting the left-hand side, expanding the resulting nonlocal unknowns about
the local point and discarding small terms, we arrive at a different
approximation to the radiative intensity than given in
Chapter~\ref{chap:adDerivation}:
\begin{equation*}
  \Iv = \frac{1}{4\pi} \phi - f(\vec{x},\vec{\Omega}) \vec{\Omega}\vd  \grad \phi 
  + f(\vec{x},\vec{\Omega})\grad \vd\vec{F} \,.
\end{equation*}
The first angular moment is a new approximation:
\begin{equation*}
  \vec{F} = - \Dtens \vd \grad \phi + \vec{E} (\grad \vd\vec{F})\,,
\end{equation*}
where
\begin{equation*}
  \vec{E} = \int_{4\pi} \vec{\Omega} f \ud\Omega\,.
\end{equation*}
Away from material discontinuities, $\vec{E}$ is exponentially small; but
near a discontinuity, $\vec{E}$ is non-negligible. This additional term may
improve the accuracy of the anisotropic diffusion approximation in asymptotic
sense, but it will re-introduce a qualitative property that our anisotropic
diffusion approximation lacks: a ``kink'' in the solution $\phi$. Because
$\vec{E}$ is discontinuous at a material boundary, it will induce a
discontinuity in $\grad \phi$.

%%%%%%%%%%%%%%%%%%%%%%%%%%%%%%%%%%%%%%%%
\subsection{Alternative flux limiters}

In \S\ref{sec:flad}, we proposed a flux-limited modification to anisotropic
diffusion theory. For ease of understanding and implementation, we chose a
``max'' limiter that multiplied the tensor by a scalar that preserved the
physical property $\vec{F} \le \phi$. However, other flux limiting
methods---some of which may be more well behaved numerically---are possible.
These flux limiters would be more complicated than the standard scalar diffusion
coefficients: the extra degrees of freedom in being anisotropic tensors rather
than isotropic scalars would allow a more options in formulating the tensor.
An additional complication is that for a ``square root''-type limiter, as an
example, a ``matrix square root'' would have to be calculated.
%The
%two properties a flux-limited diffusion tensor $\tilde\Dtens$ must retain are
%that:
%\begin{enumerate}
%  \item it must reduce to the true, transport-calculated $\Dtens$ in the absence
%    of strong gradients; and
%  \item in the presence of strong gradients it must satisfy $ \norm{
%    \tilde\Dtens \vd \grad \phi} = \phi$.
%\end{enumerate}
%Because the anisotropic diffusion coefficient is, in multiple spatial
%dimensions, a rank-2 tensor, there is a degree of flexibility not present in
%standard diffusion theory.

%%%%%%%%%%%%%%%%%%%%%%%%%%%%%%%%%%%%%%%%
\subsection{Higher-order anisotropic \texorpdfstring{\PN}{PN} methods}
The new derivation of the AD method opens the door to other potential
approximations to radiation transport.  $I -
\frac{1}{4\pi}\phi$, we could also subtract off the linearly anisotropic
component:
\begin{equation*}
  \Psi' \equiv I - \frac{1}{4\pi}\phi - \frac{3}{4\pi} \vec{\Omega} \vd
  \vec{F}\,.
\end{equation*}
We could then perform a similar analysis that in chapter~\ref{chap:adDerivation}
and get an approximate form for the second moment of the intensity,
\begin{equation*}
  \int_{4\pi} \vec{\Omega}\vec{\Omega} I \ud \Omega
  = \int_{4\pi} \vec{\Omega}\vec{\Omega} \Psi' \ud \Omega \,.
\end{equation*}
This would be substituted into the first moment of the transport equation,
\begin{equation*}
  \frac{1}{c}\pder{\vec{F}}{t}
  + \vec{\Omega} \vd \int_{4\pi} \vec{\Omega}\vec{\Omega} I \ud \Omega
  + \sigma \vec{F}
  = 0 \,.
\end{equation*}
It would almost certainly involve higher-order gradients of $\phi$ and $\vec{F}$
as well as more complex tensors.

%%%%%%%%%%%%%%%%%%%%%%%%%%%%%%%%%%%%%%%%
\subsection{Multigroup}

We observed that 

%%%%%%%%%%%%%%%%%%%%%%%%%%%%%%%%%%%%%%%%
\subsection{Three-dimensional problems}

%%%%%%%%%%%%%%%%%%%%%%%%%%%%%%%%%%%%%%%%
\subsection{Preconditioners}

