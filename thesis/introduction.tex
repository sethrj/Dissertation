% !TEX root = _individual/introduction.tex

%%%%%%%%%%%%%%%%%%%%%%%%%%%%%%%%%%%%%%%%%%%%%%%%%%%%%%%%%%%%%%%%%%%%%%%%%%%%%%%%
\chapter{Introduction}\label{chap:introduction}
%%%%%%%%%%%%%%%%%%%%%%%%%%%%%%%%%%%%%%%%%%%%%%%%%%%%%%%%%%%%%%%%%%%%%%%%%%%%%%%%
This is the introduction.

Thermal radiative transfer \cite{Urb2006,McC2008a}, nonlinear radiative transfer
\cite{Den2009}, radiative transfer and material temperature
\cite{Gen2001,Mor2000}, thermal radiation transport \cite{McC2007,Dav2010},

Morel's tensor diffusion work \cite{Mor2007} says to use regular diffusion near
problem boundaries, etc. It also talks about how only one transport sweep is
needed.

Material energy density \cite{Su2001}

\section{Synopsis}
The remainder of this thesis is organized into the following chapters:

\chaptersynopsis{chap:trtBackground}
The assertions about the difficulty of computational modelling of thermal
radiative transfer are bolstered by presenting the equations themselves. We give
a brief overview of existing approximations to the TRT regime and discuss how
those approximations are used in our work. Particular emphasis is given to the
semi-implicit treatment, which allows the nonlinear problem to be approximated
by a system of linear equations.

\chaptersynopsis{chap:adDerivation}
With the transport equation in hand, we derive a new approximation to radiation
transport, anisotropic diffusion. The derivation accounts for both the time
dependence and boundary conditions. We then discuss some of the properties of
the AD method and make predictions for its range of applicability.

\chaptersynopsis{chap:aponeDerivation}
The derivation for the time-dependent AD equations assumed that the solution
changes very slowly in time, which can be a poor approximation when applied to
TRT that leads to the nonphysical transfer of energy faster than the speed of
light. This chapter addresses that shortcoming in two very different ways. The
first is to apply the physically motivated but \emph{ad hoc} method of flux
limiting to the AD formulation. The second is to modify the ansatz used in
deriving the anisotropic diffusion equations, leading to the \APone\ method.

\chaptersynopsis{chap:implementation}
The leakage terms for anisotropic diffusion are more complex than standard
diffusion: rather than a scalar diffusion coefficient, AD has a diffusion
tensor. This necessitates unusual discretization schemes in all but the simplest
of problems. We derive new discretization schemes for Cartesian \xy\ geometry
that can account for the transverse leakage induced by the anisotropic diffusion
tensor.

\chaptersynopsis{chap:flatland}
As mentioned earlier, the ``flatland'' geometry has recently proven to be a
valuable test bed for new transport methods because of its smaller phase space
and correspondingly easier solution. This chapter gives a thorough overview of
the differences between flatland and true 3D geometries, with a focus on
implementing flatland solvers. We also explore diffusion in flatland, not only
deriving the prior result that the diffusion coefficient is different but also
formulating correct diffusion boundary conditions. Finally, we present the AD
equations in flatland geometry.

\chaptersynopsis{chap:simpleNumericalResults}
Before applying the anisotropic approximations to full nonlinear transport in
multi-dimensional geometries, it is important that we test individual components
of the derivation. We detail several steady-state problems that test the
discretization schemes, flatland diffusion boundary conditions, and anisotropic
diffusion boundary conditions.

\chaptersynopsis{chap:trtNumericalResults}

\chaptersynopsis{chap:conclusion}

