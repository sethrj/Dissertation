% !TEX root = _individual/flatland.tex

%%%%%%%%%%%%%%%%%%%%%%%%%%%%%%%%%%%%%%%%%%%%%%%%%%%%%%%%%%%%%%%%%%%%%%%%%%%%%%%%
\chapter{Flatland Geometry}

Flatland geometry is a fictional two-dimensional space where particles are
constrained to the page \cite{Asa2008}. This differs from standard 2-D
geometry, which represents a 3-D problem that is invariant in the $z$ axis,
where particles can travel at different polar angles out of the page. The
constraint of living in the page reduces the phase space of the transport
equation, as the flatland solution is only a function of the azimuthal angle
rather than both azimuthal and polar angles. This reduction in phase space makes 
flatland geometry a computationally less burdensome testing ground for new
methods.

Despite being easier computationally to solve, flatland has a few subtle quirks
that need to be taken into account before implementing a solver in that
geometry. Previous work in flatland \cite{Asa2008,Lar2009c} has shown that the
diffusion coefficient for flatland geometry $\frac{1}{2\sigma}$ is different from
the physical diffusion coefficients $\frac{1}{3\sigma}$, but flatland boundary
conditions have remained an unanswered and indeed unasked question. This
chapter answers that question in addition to providing other insights into this
strange geometry.

%%%%%%%%%%%%%%%%%%%%%%%%%%%%%%%%%%%%%%%%%%%%%%%%%%%%%%%%%%%%%%%%%%%%%%%%%%%%%%%%
\section{Transport in flatland}

\subsection{An insightful comparison problem}

\begin{table}[htb]
  \centering
  \begin{tabular}{lll}
\toprule
 $\tau$ & Flatland & 2-D
\\ \midrule
\phantom{1}0.01 & 0.985 & 0.950 \\
\phantom{1}0.1 & 0.863 & 0.726 \\
\phantom{1}1 & 0.274 & 0.148 \\
10 & $1.63\EE{-5}$ & $3.83\EE{-6}$
 \\
\bottomrule
  \end{tabular}
  \caption{Comparison of the probability of crossing a channel $\tau$ mfp thick
  without colliding.}
  \label{tab:collision}
\end{table}
%%%%%%%%%%%%%%%%%%%%%%%%%%%%%%%%%%%%%%%%%%%%%%%%%%%%%%%%%%%%%%%%%%%%%%%%%%%%%%%%
\section{Diffusion in flatland}

\subsection{Interior diffusion approximation}

\subsection{Boundary conditions}

%%%%%%%%%%%%%%%%%%%%%%%%%%%%%%%%%%%%%%%%%%%%%%%%%%%%%%%%%%%%%%%%%%%%%%%%%%%%%%%%

