% !TEX root = _individual/trtAnisotropicDerivation.tex

\newcommand{\epsiloncolor}[1]{#1}
%%%%%%%%%%%%%%%%%%%%%%%%%%%%%%%%%%%%%%%%%%%%%%%%%%%%%%%%%%%%%%%%%%%%%%%%%%%%%%%%
\chapter{Anisotropic diffusion for TRT}
%%%%%%%%%%%%%%%%%%%%%%%%%%%%%%%%%%%%%%%%%%%%%%%%%%%%%%%%%%%%%%%%%%%%%%%%%%%%%%%%
The radiation field in TRT is described by a Boltzmann transport equation:
\begin{equation}\label{eq:fullGrayTransport2}
  \frac{1}{c} \pder{I}{t}(\vec{x}, \vec{\Omega}, t)
  + \vec{\Omega} \vd \del I(\vec{x}, \vec{\Omega}, t) +
 \sigma(\vec{x}, T) I(\vec{x}, \vec{\Omega}, t)
  = \frac{\sigma(\vec{x}, T) a c [T(\vec{x},t)]^4}{4\pi} 
  + \frac{c Q(\vec{x},t)}{4\pi} \,, \qquad \vec{x} \in V, \ \vec{\Omega} \in
  4\pi, \ t >= 0.
\end{equation}
It has boundary conditions
\begin{alignat*}{3}
  I(\vec{x}, \vec{\Omega}, t) &= I^b(\vec{x}, \vec{\Omega}, t) \,, &\qquad&
  \vec{x} \in \partial V, \  \vec{\Omega}\vd \vec{n} < 0, \ t > 0,
  \\
  \intertext{and an initial condition}
  I(\vec{x}, \vec{\Omega}, 0) &= I^i(\vec{x}, \vec{\Omega}) \,, &&
  \vec{x} \in V,\ \vec{\Omega} \in 4\pi.
\end{alignat*}
The time-dependence of the opacity results from its explicit dependence on
the material temperature, $\sigma = \sigma(\vec{x}, T[\vec{x},t])$.

Taking the zeroth angular moment of Eq.~\eqref{eq:fullGrayTransport2} gives a
conservation statement for the radiation:
\begin{equation} \label{eq:grayTransportZeroth}
  \frac{1}{c} \pder{\phi}{t}(\vec{x}, t)
  +\del \vd \vec{F}(\vec{x}, t) +
 \sigma(\vec{x}, t) \phi(\vec{x}, t)
  = \sigma(\vec{x}, t) a c [T(\vec{x},t)]^4
  + c Q(\vec{x},t)
\end{equation}
where $\phi$ is the zeroth moment of the radiative intensity (so $\phi/c$ is
the energy density of the radiation field), and the radiation flux $\vec{F}$ is
the first moment of $I$.

If the right hand side of Eq.~\eqref{eq:fullGrayTransport2} is treated as a known
time-dependent quantity, the integrodifferential Boltzmann equation can be
transformed to an integral transport equation by considering sources and
attenuation along the characteristic ray that passes through the point
$\vec{x}$ along $\vec{\Omega}$ \cite{Pri2010}. Including the initial and
boundary conditions, the integral transport equation is
\begin{subequations} \label{eqs:integralTrtAngularFlux}
  \begin{equation} \label{eq:integralTrtAngularFluxFull}
  \begin{split}
    I(\vec{x}, \vec{\Omega}, t)
    &=
    I^b(\vec{x}_b, \vec{\Omega}, t - \norm{\vec{x} - \vec{x}_b}/c)
    \eexp^{ -\tau(\vec{x}, \vec{x}_b, \vec{\Omega}, t)}
    U(c t - \norm{\vec{x} - \vec{x}_b})
    \\
    &\qquad + I^i( \vec{x} - c t \vec{\Omega}, \vec{\Omega})
    \eexp^{ -\tau(\vec{x}, \vec{x}- c t \vec{\Omega}, \vec{\Omega}, t)}
    U( \norm{\vec{x} - \vec{x}_b} - c t)
    \\
    &\qquad +  \int_{0}^{\norm{\vec{x} - \vec{x}_b}}
    \eexp^{ -\tau(\vec{x}, \vec{x} - s \vec{\Omega}, \vec{\Omega}, t)}
    \hat Q(\vec{x} - s \vec{\Omega}, \vec{\Omega}, t-s/c) \ud s\,.
  \end{split}
  \end{equation}
  Here, $U(\zeta)$ is the heaviside function, unity for $\zeta \ge 0$ and zero
  otherwise. It effectively selects whether the initial condition or the
  boundary condition is used along a characteristic ray, depending on the
  travel time to the boundary.

  The ``known'' source $\hat Q$ is evaluated along the characteristic ray and
  is defined by the right hand side in Eq.~\eqref{eq:fullGrayTransport2},
  \begin{equation} \label{eq:integralTrtSource}
    \hat Q(\vec{x}, \vec{\Omega}, t) =  \frac{\sigma(\vec{x}, t) a c
    [T(\vec{x},t)]^4}{4\pi} + \frac{c Q(\vec{x},t)}{4\pi}\,.
  \end{equation}
  The optical thickness of the medium between points $\vec{x}$ and
  $\vec{x}'$ along direction $\vec{\Omega}$ is 
  \begin{equation} \label{eq:integralTrtTauDefinition}
    \tau(\vec{x}, \vec{x}', \vec{\Omega}, t) = \int_{0}^{\norm{\vec{x} -
    \vec{x}'}} \sigma(\vec{x}-s'\vec{\Omega}, t-s'/c) \ud s' \,.
  \end{equation}
  The point $\vec{x}_b$ is defined as the point on the boundary that,
  following direction $\vec{\Omega}$, intersects point $\vec{x}$. In other
  words, some geometry-dependent positive value of $d$ satisfies $\vec{x}_b =
  \vec{x} - d \vec{\Omega}$.
\end{subequations}

%%%%%%%%%%%%%%%%%%%%%%%%%%%%%%%%%%%%%%%%%%%%%%%%%%%%%%%%%%%%%%%%%%%%%%%%%%%%%%%%
\section{Anisotropic diffusion derivation in interior}
%%%%%%%%%%%%%%%%%%%%%%%%%%%%%%%%%%%%%%%%%%%%%%%%%%%%%%%%%%%%%%%%%%%%%%%%%%%%%%%%
Far from the problem's boundaries, $\norm{\vec{x} - \vec{x}_b} \sim \infty$,
and the optical thickness $\tau$ is very large, so only an exponentially small
approximation is made by discarding the $I^b$ term and writing the line
integral's upper limit as $\infty$. If $t\gg 0$, temporally distant from the
initial condition, then the term with $I^i$ is also small because of the large
number of mean free paths between $\vec{x}$ and $\vec{x} - c t$. It too may
therefore be neglected in the spatiotemporal interior.

Discarding those boundary terms and substituting $\hat Q$, Eq.~\eqref{eq:integralTrtAngularFluxFull} is, with only a small
approximation in the interior,
\begin{equation*}
    I(\vec{x}, \vec{\Omega}, t)
    = \frac{1}{4\pi}\int_{0}^{\infty}
    \eexp^{ -\tau(\vec{x}, \vec{x} - s \vec{\Omega}, \vec{\Omega}, t)}
    \left[ \sigma a c T^4 + c Q \right]_{(\vec{x} - s
    \vec{\Omega}, t-s/c)} \ud s\,.
\end{equation*}
The quantity in brackets is just the right-hand side of the conservation
equation~\eqref{eq:grayTransportZeroth}. Replacing it with the left-hand
side,
\begin{equation}\label{eq:integralTrtLhs}
    I(\vec{x}, \vec{\Omega}, t)
    = \frac{1}{4\pi}\int_{0}^{\infty}
    \eexp^{ -\tau(\vec{x}, \vec{x} - s \vec{\Omega}, \vec{\Omega}, t)}
    \left[ \frac{1}{c} \pder{\phi}{t} +\del \vd \vec{F} + \sigma \phi
    \right]_{(\vec{x} - s \vec{\Omega}, t-s/c)} \ud s\,.
\end{equation}

To expand and simplify the integral, it is necessary to make an ansatz about
the strength of the derivatives and moments are in
Eq.~\eqref{eq:fullGrayTransport2}:
\begin{align*}
  I &= O(\epsiloncolor{1}), &
  \sigma &= O(\epsiloncolor{1}), &
  \del I &= O(\epsiloncolor{\epsilon}), &
  \frac1c\pder{I}{t} &= O(\epsiloncolor{\epsilon}), &
  \int_{4\pi} \vec{\Omega} I\ud \Omega &= O(\epsiloncolor{\epsilon}).
\end{align*}
This set of assumptions differs from the standard asymptotic diffusion
derivation, where only the physics ($c$, $\sigma$, $c_v$) are scaled, and the
strength of the derivatives result naturally from the asymptotic matching.

We consider each term in Eq.~\eqref{eq:integralTrtLhs},
\begin{align*}
    \begin{split}
    I(\vec{x}, \vec{\Omega}, t)
    &=
    \frac{1}{4\pi}\int_{0}^{\infty}
    \eexp^{ -\tau(\vec{x}, \vec{x} - s \vec{\Omega}, \vec{\Omega}, t)}
    \sigma(\vec{x} - s \vec{\Omega}, t-s/c)
    \phi(\vec{x} - s \vec{\Omega}, t-s/c)
    \ud s
    \\
    &\quad+ \frac{1}{4\pi}\frac{1}{c} \pder{}{t} \int_{0}^{\infty}
    \eexp^{ -\tau(\vec{x}, \vec{x} - s \vec{\Omega}, \vec{\Omega}, t)}
    \phi(\vec{x} - s \vec{\Omega}, t-s/c)
    \ud s
    \\
    &\quad+ \frac{1}{4\pi}\del \vd \int_{0}^{\infty}
    \eexp^{ -\tau(\vec{x}, \vec{x} - s \vec{\Omega}, \vec{\Omega}, t)}
    \vec{F}(\vec{x} - s \vec{\Omega}, t-s/c)
    \ud s
    \end{split}
    \\
    &= A + B + C
    \\
    &= O(\epsiloncolor{1}) + O(\epsiloncolor{\epsilon})+
    O(\epsiloncolor{\epsilon^2})\,.
\end{align*}
To begin, we discard $C$, which is $O(\epsilon^2)$.

Next, we consider $A$ by recognizing that, by the fundamental theorem of
calculus,
\begin{align*}
  \lefteqn{\eexp^{ -\tau(\vec{x}, \vec{x} - s \vec{\Omega}, \vec{\Omega}, t)}
  \sigma(\vec{x} - s \vec{\Omega}, t-s/c)}\qquad&
  \\
  &= \sigma(\vec{x} - s \vec{\Omega}, t-s/c) \expp{ -\int_{0}^{s}
  \sigma(\vec{x}-s'\vec{\Omega}, t-s'/c) \ud s'}
  \\
  &= -\oder{}{s}
    \expp{ -\int_{0}^{s} \sigma(\vec{x}-s'\vec{\Omega}, t-s'/c) \ud s'}\,.
\end{align*}
Integration by parts lets $A$ be evaluated:
\begin{align*}
  A &= 
  -\frac{1}{4\pi}\int_{0}^{\infty}
  \oder{}{s}
    \expp{- \int_{0}^{s} \sigma(\vec{x}-s'\vec{\Omega}, t-s'/c) \ud s'}
  \phi(\vec{x} - s \vec{\Omega}, t-s/c)
  \ud s
  \\
    \begin{split}
  &=-\frac{1}{4\pi} \Bigg[ 
\expp{- \int_{0}^{s} \sigma(\vec{x}-s'\vec{\Omega}, t-s'/c) \ud s'} 
\phi(\vec{x} - s \vec{\Omega}, t-s/c) \Bigg|_{0}^{\infty}
\\
&\qquad\qquad- \int_{0}^{\infty} \expp{- \int_{0}^{s} \sigma(\vec{x}-s'\vec{\Omega}, t-s'/c) \ud s'}
\oder{}{s} \phi(\vec{x} - s \vec{\Omega}, t-s/c)
\ud s
  \Bigg]
    \end{split}
  \\
  &=-\frac{1}{4\pi} \bigg[ 
0 -  
\eexp^0 \phi(\vec{x}, t)
- \int_{0}^{\infty} \eexp^{-\tau(\vec{x}, \vec{x} - s \vec{\Omega}, \vec{\Omega}, t)}
\oder{}{s} \phi(\vec{x} - s \vec{\Omega}, t-s/c)
\ud s
  \bigg]
  \\
 &=\frac{1}{4\pi}\phi(\vec{x}, t)
+ \frac{1}{4\pi}\ \int_{0}^{\infty} \eexp^{-\tau(\vec{x}, \vec{x} - s \vec{\Omega}, \vec{\Omega}, t)}
\oder{}{s} \phi(\vec{x} - s \vec{\Omega}, t-s/c)
\ud s
\end{align*}
The derivative along the streaming direction works out to be the streaming plus
time derivative operator:
\begin{align*}
  \oder{}{s} f(\vec{x} - s \vec{\Omega}, t-s/c)
  &=  \oder{}{s} f(x_i - s \Omega_i, t-s/c)
  = \oder{}{s} f(x_1',x_2',x_3',t')
  \\
  &= \sum_{i=1}^{3}
  \pder{x_i'}{s}\pder{x_i}{x_i'} \pder{f}{x_i}
  + \pder{t'}{s}\pder{t}{t'} \pder{f}{t}
  \\
  &= \sum_{i=1}^{3} [- \Omega_i][1]\pder{f}{x_i} 
  + \left[ -\frac{1}{c} \right]  \pder{f}{t}
  \\
  &= \left[ -\vec{\Omega} \vd \del - \pder{}{t} \right]
    f(\vec{x} - s \vec{\Omega}, t-s/c)
\end{align*}
Applying this to $\oder{}{s}\phi$ in the $A$ term,
\begin{equation*}
A = \frac{1}{4\pi}\phi(\vec{x}, t)
+ \frac{1}{4\pi}\ \int_{0}^{\infty} \eexp^{-\tau(\vec{x}, \vec{x} - s \vec{\Omega}, \vec{\Omega}, t)}
\left[ -\vec{\Omega} \vd \del - \pder{}{t} \right] \phi(\vec{x} - s \vec{\Omega}, t-s/c)
\ud s
\end{equation*}
Next, a Taylor series about the local point $(\vec{x}, t)$ expands the
nonlocal unknown $\phi$. Since the derivatives were assumed to be small, each
term is an increasing power in $\epsilon$:
\begin{align*}
  f(\vec{x} - s \vec{\Omega}, t-s/c)
  &\sim
  f(\vec{x}, t) - s \vec{\Omega} \vd \del f(\vec{x}, t)
  - s\frac{1}{c} \pder{f}{t}(\vec{x}, t) + \cdots
  \\
  &= f(\vec{x}, t) - s \left[ \vec{\Omega} \vd \del
  + \frac{1}{c} \pder{}{t} \right] f(\vec{x}, t) + \cdots
  \\
  &= O(\epsiloncolor{1}) +
  O(\epsiloncolor{\epsilon}) + \cdots
\end{align*}
With the Taylor series approximation to $\phi$, keeping only the leading order
term,
\begin{align*}
A &= \frac{1}{4\pi}\phi(\vec{x}, t)
+ \frac{1}{4\pi}\ \int_{0}^{\infty} \eexp^{-\tau(\vec{x}, \vec{x} - s \vec{\Omega}, \vec{\Omega}, t)}
\left[ -\vec{\Omega} \vd \del - \pder{}{t} \right][\phi(\vec{x}, t)]
\ud s
\\
&= \frac{1}{4\pi}\phi(\vec{x}, t)
- \frac{1}{4\pi}\ \int_{0}^{\infty} \eexp^{-\tau(\vec{x}, \vec{x} - s \vec{\Omega}, \vec{\Omega}, t)}
\ud s
\left[ \vec{\Omega} \vd \del + \pder{}{t} \right]\phi(\vec{x}, t)
\end{align*}

A slightly different Taylor series allows the expansion of the $\sigma$ term
embedded inside $\tau$:
\begin{align*}
  f(\vec{x} - s \vec{\Omega}, t-s/c)
  &\sim
  f(\vec{x} - s \vec{\Omega}, t)
  - s\frac{1}{c} \pder{f}{t}(\vec{x} - s \vec{\Omega}, t) + \cdots
  \\
  &= O(\epsiloncolor{1}) + O(\epsiloncolor{\epsilon}) + \cdots
\end{align*}
so
\begin{align*}
  \tau(\vec{x}, \vec{x} - s \vec{\Omega}, \vec{\Omega}, t)
  &=  \int_{0}^{s} \sigma(\vec{x}-s'\vec{\Omega}, t-s'/c) \ud s'
  \\
  &\sim \int_{0}^{s} [\sigma(\vec{x}-s'\vec{\Omega}, t)-s\frac{1}{c}
  \pder{\sigma}{t}(\vec{x}, t)] \ud s'
  \\
  &\sim \int_{0}^{s} \sigma(\vec{x}-s'\vec{\Omega}, t) \ud s'\,.
\end{align*}
This approximation is good if the speed of light $c$ is large or the opacities
vary slowly. It obviates the expense of storing the opacities at all
prior points in time.

With these approximations, the $\sigma I$ term in the integral expression for
$I$ in Eq.~\eqref{eq:integralTrtLhs} is
\begin{equation*}
  A \sim \frac{1}{4\pi}\phi(\vec{x}, t)
- \frac{1}{4\pi}\ \int_{0}^{\infty} \eexp^{-\int_{0}^{s} \sigma(\vec{x}-s'\vec{\Omega}, t) \ud s'}
\ud s
\left[ \vec{\Omega} \vd \del + \pder{}{t} \right]\phi(\vec{x}, t)
=  O(\epsiloncolor{1}) + O(\epsiloncolor{\epsilon})
\end{equation*}
Applying the same two Taylor series approximations to the $\frac1c\pder It$
term gives:
\begin{align*}
  B &=
  \frac{1}{4\pi}\frac{1}{c} \pder{}{t} \int_{0}^{\infty}
  \eexp^{ -\int_{0}^{s} \sigma(\vec{x}-s'\vec{\Omega}, t-s'/c) \ud s'}
  \phi(\vec{x} - s \vec{\Omega}, t-s/c)
  \ud s
\\
  &\sim
  \frac{1}{4\pi}\frac{1}{c} \pder{}{t} \int_{0}^{\infty}
  \expp{ -\int_{0}^{s} \left[ \sigma(\vec{x}-s'\vec{\Omega}, t) +
  O(\epsiloncolor{\epsilon}) \right]}
  \left[ \phi(\vec{x} - s \vec{\Omega}, t-s/c) + O(\epsiloncolor{\epsilon}) \right]
  \ud s
\\
&= \frac{1}{4\pi} \int_{0}^{\infty}
  \eexp^{ -\int_{0}^{s} \sigma(\vec{x}-s'\vec{\Omega},t) \ud s'}
 \frac{1}{c} \pder{}{t} \phi(\vec{x} - s \vec{\Omega}, t)
  \ud s
  + O(\epsiloncolor{\epsilon})\,.
\end{align*}

With the Taylor series approximations that allow the $O(\epsilon^2)$ terms to
be neglected, Eq.~\eqref{eq:integralTrtLhs} is
\begin{align*}
  I(\vec{x}, \vec{\Omega}, t) 
  &\sim A + B + O(\epsiloncolor{\epsilon^2}) 
  \\
  \begin{split}
  &= \frac{1}{4\pi}\phi(\vec{x}, t)
- \frac{1}{4\pi}\int_{0}^{\infty} \eexp^{-\int_{0}^{s} \sigma(\vec{x}-s'\vec{\Omega}, t) \ud s'}
\ud s
\left[ \vec{\Omega} \vd \del\phi(\vec{x}, t) + \pder{}{t}\phi(\vec{x}, t)
\right]
\\
&\qquad+ \frac{1}{4\pi} \int_{0}^{\infty}
  \eexp^{ -\int_{0}^{s} \sigma(\vec{x}-s'\vec{\Omega},t) \ud s'}
 \frac{1}{c} \pder{}{t} \phi(\vec{x} - s \vec{\Omega}, t)
  \ud s
  \end{split}
  \\
  &= \frac{1}{4\pi}\phi(\vec{x}, t)
  + \frac{1}{4\pi}\int_{0}^{\infty} \eexp^{-\int_{0}^{s}
  \sigma(\vec{x}-s'\vec{\Omega}, t) \ud s'} \ud s \left[ - \vec{\Omega} \vd
  \del\phi(\vec{x}, t) - \pder{}{t}\phi(\vec{x}, t)
  + \pder{}{t}\phi(\vec{x}, t)
  \right]
  \\
  &= \frac{1}{4\pi}\phi(\vec{x}, t)
  - \frac{1}{4\pi}\int_{0}^{\infty} \eexp^{-\int_{0}^{s}
  \sigma(\vec{x}-s'\vec{\Omega}, t) \ud s'} \ud s \,
  \vec{\Omega} \vd \del\phi(\vec{x}, t)
  \\
  &= O(1) + O(\epsiloncolor{\epsilon})
\end{align*}
The time derivatives have canceled out, leaving an isotropic leading order
term and an angle-dependent 

%\begin{figure}[htb]
%  \centering
%  \small\input{crashpipe2/dcoeff_channel_t10/include.tex}
%  \caption{Diffusion coefficient example at $t=10$}
%  \label{fig:test}
%\end{figure}
