% !TEX root = _individual/aponeDerivation.tex

%%%%%%%%%%%%%%%%%%%%%%%%%%%%%%%%%%%%%%%%%%%%%%%%%%%%%%%%%%%%%%%%%%%%%%%%%%%%%%%%
% [Anisotropic P1]
\chapter{Anisotropic \Pone}\label{chap:aponeDerivation}

The anisotropic diffusion method is derived under the assumptions that the
intensity has weak spatial gradients, mild anisotropy, and very slow time
dependence. By reducing the asymptotic smallness of the time derivative, we
derive here a new ``anisotropic'' method, anisotropic \Pone\ (\APone), that
should produce a more accurate answer in strongly time-dependent problems. In
the steady-state case, it reduces to the anisotropic diffusion approximation; in
the homogeneous case, it reduces to the standard \Pone\ (spherical harmonics)
approximation. 

%%%%%%%%%%%%%%%%%%%%%%%%%%%%%%%%%%%%%%%%%%%%%%%%%%%%%%%%%%%%%%%%%%%%%%%%%%%%%%%%
\section{Derivation}

We begin with the linear, time-dependent transport equation with isotropic
scattering. By omitting the complication of material--radiation coupling and
nonlinearities, we may derive the \APone\ equations with a straightforward
linear asymptotic analysis. We shall then show how the results may be applied to
the diffusive regime of nonlinear thermal radiative transfer.

The linear transport equation, given in Chapter~\ref{chap:adDerivation}
%Eqs.~\eqref{eqs:tdTransport}
and repeated here, comprise the Boltzmann equation,
\begin{multline} \tagref{eq:tdTransportVol}
  \frac{1}{c}\pder{I}{t}(\vec{x},\vec{\Omega},t)
  + \vec{\Omega}\vd \grad I(\vec{x},\vec{\Omega},t)
  + \sigma(\vec{x}) I(\vec{x},\vec{\Omega},t)
  \\ = \frac{\sigma_s(\vec{x})}{4\pi}
  \int_{4\pi} I(\vec{x},\vec{\Omega}',t) \ud \Omega'
  + \frac{q(\vec{x},t)}{4\pi}
  \,, \quad \vec{x}\in V,\ \vec{\Omega}\in4\pi,\ t \ge 0;
\end{multline}
the boundary condition,
\begin{equation} \tagref{eq:tdTransportBndy}
  I(\vec{x},\vec{\Omega},t) = I^b(\vec{x},\vec{\Omega},t) \,,
  \quad \vec{x}\in \partial V ,\ \vec{\Omega}\vd \vec{n} < 0,\ t > 0;
\end{equation}
and the initial condition,
\begin{equation} \tagref{eq:tdTransportInit}
  I(\vec{x},\vec{\Omega},0) = I^i(\vec{x},\vec{\Omega}) \,,
  \quad \vec{x}\in V ,\ \vec{\Omega} \in 4\pi.
\end{equation}

To perform the asymptotic analysis, we again consider the intensity as the
linear superposition of three distinct transport solutions:
\begin{equation} \tagref{eq:tdSuperposition}
  I(\vec{x},\vec{\Omega},t)
  \equiv \Iv(\vec{x},\vec{\Omega},t)
  + \Ibl(\vec{x},\vec{\Omega},t)
  + \Iil(\vec{x},\vec{\Omega},t)\,.
\end{equation}
Here, $\Iv$ is an ``interior'' solution, $\Ibl$ is a ``boundary layer'' solution,
and $\Iil$ is an ``initial layer'' solution. For a more detailed description of
these, see \S\ref{sec:adDerivation} and Fig.~\ref{fig:layers}.

We shall approximate the interior solution $\Iv$ using a slightly different set
of assumptions than the anisotropic diffusion method uses. The boundary and
initial layer solutions are used to match the resulting interior approximation
to the transport solution at the boundary and initial time.

%%%%%%%%%%%%%%%%%%%%%%%%%%%%%%%%%%%%%%%%%%%%%%%%%%%%%%%%%%%%%%%%%%%%%%%%%%%%%%%%
\subsection{Interior solution}

The interior transport equation is valid away from the problem and initial
layer. It is identical to the interior equation in \S\ref{sec:adInterior}, and
the first steps in developing the \APone\ approximation are identical to those
for the AD approximation.

The interior transport equation is
\begin{equation} \tagref{eq:tdVol}
  \frac{1}{c}\pder{\Iv}{t}(\vec{x},\vec{\Omega},t)
  + \vec{\Omega}\vd \grad \Iv(\vec{x},\vec{\Omega},t)
  + \sigma(\vec{x}) \Iv(\vec{x},\vec{\Omega},t)
  = \frac{\sigma_s(\vec{x})}{4\pi}
  \phi(\vec{x},t) + \frac{q(\vec{x},t)}{4\pi} \,,
\end{equation}
with the \emph{interior} scalar intensity defined as
\begin{equation} \tagref{eq:tdPhi}
  \phi(\vec{x}) \equiv \int_{4\pi} \Iv(\vec{x}, \vec{\Omega},t) \ud \Omega\,.
\end{equation}

The zeroth moment of the interior transport equation is the following
conservation equation:
\begin{equation} \tagref{eq:loVol}
\frac{1}{c} \pder{\phi}{t} (\vec{x}, t)
  + \grad \vd\vec{F}(\vec{x}, t)
  + \sigma(\vec{x}) \phi(\vec{x}, t)
 = \sigma_s(\vec{x}) \phi(\vec{x},t) + q(\vec{x},t)\,,
\end{equation}
with the \emph{interior} radiation flux defined as
\begin{equation}\tagref{eq:tdF}
  \vec{F} \equiv \int_{4\pi} \vec{\Omega} \Iv(\vec{x}, \vec{\Omega},t) \ud
  \Omega\,.
\end{equation}

By combining Eqs.~\eqref{eq:tdVol} and~\eqref{eq:loVol} (see
\S\ref{sec:adInterior}), we eliminate the scattering and extraneous sources on
the right-hand side of the transport equation to obtain
\begin{equation}\tagref{eq:capPsiVol}
  \left[ \frac{1}{c}\pder{}{t}
  + \vec{\Omega}\vd \grad
  + \sigma \right]
   \left( \Iv
  - \frac{1}{4\pi} \phi \right)
  = \frac{1}{4\pi} \grad \vd\vec{F} -
  \frac{1}{4\pi} \vec{\Omega}\vd \grad \phi\,.
\end{equation}

Now we shall approximate Eq.~\eqref{eq:capPsiVol} by making an asymptotic ansatz
about the behavior of $I$ and discarding ``small'' terms. Here the derivation
diverges from the anisotropic diffusion approximation: rather than assuming
$\tpder{I}{t} = O(\epsilon^2)$, we take the scaling $\tpder{I}{t} =
O(\epsilon)$. Thus the assumed magnitude of the intensity, its derivatives, and
its anisotropy are:
\begin{align} \label{eq:ap1ansatz}
  \sigma &= O(1), &
  I &= O(1), &
  \int_{4\pi} \vec{\Omega} I \ud\Omega &= O(\epsilon), &
  \grad I &= O(\epsilon), &
  \pder{I}{t} &= O(\epsilon) \,.
\end{align}

The stronger magnitude of the time derivative means that the time derivative in
Eq.~\eqref{eq:capPsiVol} may not be discarded while preserving $O(epsilon^2)$
accuracy as in the anisotropic diffusion approximation. The only asymptotically
small term is $\grad \vd\vec{F} = O(\epsilon^2)$, which we discard.
Equation~\eqref{eq:capPsiVol} is then
\begin{equation*}
  \left[ \frac{1}{c}\pder{}{t}
  + \vec{\Omega}\vd \grad
  + \sigma \right]
   \left( \Iv - \frac{1}{4\pi} \phi \right)
  = - \frac{1}{4\pi} \vec{\Omega}\vd \grad \phi\,.
\end{equation*}

Now we decompose the ``time${}+{}$leakage${}+{}$collision'' operator on the
left-hand side:
\begin{equation*}
  \frac{1}{c}\pder{}{t}
  + \vec{\Omega}\vd \grad
  + \sigma
  =
  \left( \vec{\Omega}\vd \grad + \sigma \right)
  \left( \frac{1}{\sigma} \frac{1}{c}\pder{}{t} + 1 \right)
  -  \vec{\Omega}\vd \grad \frac{1}{\sigma} \frac{1}{c}\pder{}{t} \,.
\end{equation*}
Formally, the rightmost term is $O(\epsilon^2)$: $\sigma$ is $O(1)$,
$\vec{\Omega}\vd\grad$ is $O(\epsilon)$, and $\frac{1}{c}\pder{}{t}$ is
$O(\epsilon)$. Thus the previous equation can be written with $O(\epsilon^2)$
error as:
\begin{equation*}
  \left( \vec{\Omega}\vd \grad + \sigma \right)
  \left( \frac{1}{\sigma} \frac{1}{c}\pder{}{t} + 1 \right)
   \left( \Iv - \frac{1}{4\pi} \phi \right)
  = - \frac{1}{4\pi} \vec{\Omega}\vd \grad \phi\,.
\end{equation*}
By making the definition
\begin{equation}\label{eq:ap1capPsi}
  \Psi(\vec{x},\vec{\Omega},t)
  \equiv
  \left( \frac{1}{\sigma(\vec{x})} \frac{1}{c}\pder{}{t} + 1 \right)
  \left( \Iv(\vec{x},\vec{\Omega},t) - \frac{1}{4\pi} \phi(\vec{x},t) \right)\,,
\end{equation}
we obtain the following equation:
\begin{equation}\label{eq:ap1Ss}
  \left[ \vec{\Omega}\vd \grad + \sigma(\vec{x}) \right] \Psi(\vec{x},\vec{\Omega},t)
  = - \frac{1}{4\pi} \vec{\Omega}\vd \grad \phi (\vec{x},\vec{\Omega},t)\,.
\end{equation}
Here, time is a parameter, not a variable: if $\phi(\vec{x},t)$ is known,
$\Psi(\vec{x},\vec{\Omega},t)$ can be calculated without needing any prior
$\phi$ or $\Psi$.

Equation~\eqref{eq:ap1Ss} is very similar to Eq.~\eqref{eq:tdVolApprox1} of the
AD derivation. As we did there, we formally invert the streaming${}+{}$collision
operator on the left, interpreting the inverse as an integral transport
operator:
\begin{align*}
  \Psi
  &= \left[ \vec{\Omega}\vd \grad + \sigma \right]\inv \left(
  - \frac{1}{4\pi} \vec{\Omega}\vd \grad \phi \right)
  \\
  &= - \int_{0}^{\infty}
  \left( \frac{1}{4\pi} \vec{\Omega}\vd \grad \phi(\vec{x} - s \vec{\Omega},
  t)\right)
  \eexp^{ -\tau(\vec{x}, \vec{x} - s \vec{\Omega})}
  \ud s \,.
\end{align*}
Taylor-expanding the non-local $\phi$ about the local point $\vec{x}$, and
recalling the assumption that $\grad\phi = O(\epsilon)$, the above becomes
\begin{align} \nonumber
\Psi
&= - \left[ \int_{0}^{\infty} \frac{1}{4\pi} 
    \eexp^{ -\tau(\vec{x}, \vec{x} - s \vec{\Omega})}
      \ud s \right] \vec{\Omega}\vd \grad\phi(\vec{x}, t) + O(\epsilon^2)
\\ \nonumber
&= - 
  \left[\left(\vec{\Omega}\vd \grad  + \sigma \right)\inv
  \frac{1}{4\pi}\right] \vec{\Omega}\vd \grad \phi(\vec{x}, t)
\\ \label{eq:ap1capPsiApprox}
&= - \left[ f(\vec{x},\vec{\Omega}) \right] \vec{\Omega}\vd \grad \phi(\vec{x}, t)\,.
\end{align}
Here, $f$ is the solution to the same purely absorbing, steady-state transport
problem as in \S\ref{sec:adInterior}:
\begin{equation} \tagref{eq:fVol}
  \vec{\Omega}\vd \grad f(\vec{x}, \vec{\Omega})
  + \sigma(\vec{x}) f (\vec{x}, \vec{\Omega})
= \frac{1}{4\pi} \,.
\end{equation}

With an $O(\epsilon^2)$-accurate expression for $\Psi$ in hand, we return to
Eq.~\eqref{eq:ap1capPsi} and substitute Eq.~\eqref{eq:ap1capPsiApprox}:
\begin{equation}\label{eq:ap1Interior1}
  \left( \frac{1}{\sigma(\vec{x})} \frac{1}{c}\pder{}{t} + 1 \right)
  \left( \Iv(\vec{x},\vec{\Omega},t) - \frac{1}{4\pi} \phi(\vec{x},t) \right)
  = - f(\vec{x},\vec{\Omega}) \vec{\Omega}\vd \grad \phi(\vec{x}, t) \,.
\end{equation}
Taking the first angular moment of this equation eliminates $\phi$ on the
left-hand side to yield the following equation:
\begin{equation}\label{eq:ap1ficksLaw1}
  \left( \frac{1}{\sigma(\vec{x})} \frac{1}{c}\pder{}{t} + 1 \right)
  \vec{F}(\vec{x},t) 
  = - \int_{4\pi} f(\vec{x},\vec{\Omega}) \vec{\Omega} \ud\Omega
  \vd \grad \phi(\vec{x}, t) \,,
\end{equation}
or, multiplying through by $\sigma$,
\begin{equation}\label{eq:ap1ficksLaw1a}
  \frac{1}{c}\pder{\vec{F}}{t}(\vec{x},t) + \sigma(\vec{x})\vec{F}(\vec{x},t) 
  = - \sigma(\vec{x}) \Dtens(\vec{x}) \vd \grad \phi(\vec{x}, t) \,,
\end{equation}
where we have used the definition of the anisotropic diffusion tensor from
Chapter~\ref{chap:adDerivation},
\begin{equation}\tagref{eq:dDefinition}
  \Dtens(\vec{x}) \equiv \int_{4\pi} \vec{\Omega} \vec{\Omega}
  f(\vec{x}, \vec{\Omega}) \ud\Omega \,.
\end{equation}

Equation~\eqref{eq:ap1ficksLaw1a} is our first attempt at an ``anisotropic
\Pone'' equation. It approximates the radiation using the scalar unknown $\phi$
and the vector unknown $\vec{F}$ in tandem with the calculated anisotropic
diffusion tensor $\Dtens$.

In a homogeneous medium, $\Dtens$ limits to $\Identitytens/(3\sigma)$, so
Eq.~\eqref{eq:ap1ficksLaw1a} becomes:
\begin{equation*}
  \frac{1}{c}\pder{\vec{F}}{t}(\vec{x},t) - \frac{1}{3} \grad\phi(\vec{x}, t)
  + \sigma\vec{F}(\vec{x},t) 
  = 0\,,
\end{equation*}
which is the standard \Pone\ equation.

%Formally inverting the left-hand side, we obtain an approximate expression for
%the interior solution $\Iv$ 
%\begin{equation*}
%   \Iv(\vec{x},\vec{\Omega},t) 
%   = \frac{1}{4\pi} \phi(\vec{x},t) 
%  - \left( \frac{1}{\sigma(\vec{x})} \frac{1}{c}\pder{}{t} + 1 \right)\inv
%  f(\vec{x},\vec{\Omega}) \vec{\Omega}\vd \grad \phi(\vec{x}, t) \,.
%\end{equation*}

%%%%%%%%%%%%%%%%%%%%%%%%%%%%%%%%%%%%%%%%%%%%%%%%%%%%%%%%%%%%%%%%%%%%%%%%%%%%%%%%
\subsubsection{An important substitution}
In problem with voided regions, $\sigma\approx 0$, Eq.~\eqref{eq:ap1ficksLaw1a}
has a serious deficiency. The nonlocal dependence of $f$ on $\sigma$ (see
\S\ref{sec:adVoids}) means that $\Dtens \ne 0$. Thus in a void,
Eq.~\eqref{eq:ap1ficksLaw1a} becomes the unphysical
\begin{equation*}
  \frac{1}{c}\pder{\vec{F}}{t}(\vec{x},t) = 0\,.
\end{equation*}

To understand this behavior, we show how a similar procedure yields the standard
\Pone\ equation,
\begin{equation*}
  \frac{1}{c}\pder{\vec{F}}{t}(\vec{x},t) - \frac{1}{3}\grad \phi(\vec{x},t)
  + \sigma \vec{F} = 0\,.
\end{equation*}
Equation~\eqref{eq:ap1Ss} is an $O(\epsilon^2)$-accurate approximation:
\begin{equation*}
  \Psi(\vec{x},\vec{\Omega},t)
  = - \left[ \vec{\Omega}\vd \grad + \sigma(\vec{x}) \right]\inv \frac{1}{4\pi} \vec{\Omega}\vd \grad \phi (\vec{x},t)\,.
\end{equation*}
The term in brackets can be asymptotically expanded as
\begin{equation*}
  \left[ \epsilon \vec{\Omega}\vd \grad + \sigma(\vec{x}) \right]\inv
  \sim \frac{1}{\sigma(\vec{x})} \left( 1
  - \epsilon \frac{1}{\sigma} \vec{\Omega}\vd \grad + O(\epsilon^2) \right)\,.
\end{equation*}
Substituting into Eq.~\eqref{eq:ap1capPsiApprox} gives
\begin{equation*}
  \left( \frac{1}{\sigma(\vec{x})} \frac{1}{c}\pder{}{t} + 1 \right)
  \left( \Iv(\vec{x},\vec{\Omega},t) - \frac{1}{4\pi} \phi(\vec{x},t) \right)
  = - \frac{1}{\sigma(\vec{x})} \left( 1 - \frac{1}{\sigma} \vec{\Omega}\vd
  \grad \right) \frac{1}{4\pi} \vec{\Omega}\vd \grad \phi
  (\vec{x},,t) \,.
\end{equation*}
The first angular moment of this equation is:
\begin{equation*}
  \frac{1}{\sigma} \frac{1}{c}\pder{\vec{F}}{t} 
  + \vec{F}
  = - \frac{1}{3}\frac{1}{\sigma} \grad \phi \,.
\end{equation*}
Multiplying by $\sigma$ gives the \Pone\ equation:
\begin{equation*}
  \frac{1}{c}\pder{\vec{F}}{t} + \frac{1}{3}\frac{1}{\sigma} \grad \phi
  + \sigma\vec{F}
  = 0 \,.
\end{equation*}

The asymptotic expansion used in the derivation of the \Pone\ equation used a
local approximation to $\sigma$ that corresponded to the other local $\sigma$.
In contrast, the derivation of the ``anisotropic''
equation~\eqref{eq:ap1ficksLaw1a} uses both a local $\sigma$ and a nonlocal $f$.
To rectify this imbalance, we wish to replace the $1/\sigma$ with a function
that does not blow up as $\sigma\to 0$ but which becomes $1/\sigma$ in the
diffusive limit.

We make the following substitution, which is accurate to $O(\epsilon^2)$ in the
diffusive limit:
\begin{align*}
  \frac{1}{\sigma(\vec{x})}
  &\approx \int_{4\pi} f(\vec{x},\vec{\Omega}) \ud \Omega
  \\
  &= \int_{4\pi} (\vec{\Omega} \vd \grad + \sigma(\vec{x}) )\inv
  \frac{1}{4\pi} \ud \Omega
  \\
  &= \int_{4\pi} \frac{1}{\sigma} \left( \frac{1}{\sigma} \vec{\Omega} \vd
  \grad + 1 \right)\inv \frac{1}{4\pi} \ud \Omega
  \\
  &\sim \int_{4\pi} \frac{1}{\sigma} \left(1 - \frac{1}{\sigma} \vec{\Omega} \vd
  \grad + O(\epsilon^2) \right) \frac{1}{4\pi} \ud \Omega
  \\
  &\sim \frac{1}{\sigma} + O(\epsilon^2) \,.
\end{align*}

The resulting \APone\ approximation is:
\begin{equation*}
  \frac{1}{\varsigma(\vec{x})} \pder{\vec{F}}{t}(\vec{x},t)
  + \vec{F}(\vec{x},t) = - \Dtens(\vec{x}) \vd \grad \phi \,,
\end{equation*}
where we have defined
\begin{equation}\label{eq:varSigma}
  \frac{1}{\varsigma(\vec{x})}
  \equiv \int_{4\pi} f(\vec{x},\vec{\Omega}) \ud \Omega
  \sim \frac{1}{\sigma(\vec{x})} + O(\epsilon^2)\,.
\end{equation}

Now Eq.~\eqref{eq:ap1ficksLaw1a} is replaced by
\begin{equation}\label{eq:ap1ficksLawFinal}
  \frac{1}{c}\pder{\vec{F}}{t}(\vec{x},t) + \varsigma(\vec{x})\vec{F}(\vec{x},t) 
  = - \varsigma(\vec{x}) \Dtens(\vec{x}) \vd \grad \phi(\vec{x}, t) \,.
\end{equation}

%%%%%%%%%%%%%%%%%%%%%%%%%%%%%%%%%%%%%%%%%%%%%%%%%%%%%%%%%%%%%%%%%%%%%%%%%%%%%%%%
\subsubsection{The interior approximate intensity}

We return to Eq.~\eqref{eq:ap1Interior1}, the expression for the approximate
intensity in the interior, and replace $\sigma$ with $\varsigma$:
\begin{equation*}
  \left( \frac{1}{\varsigma(\vec{x})} \frac{1}{c}\pder{}{t} + 1 \right)
  \left( \Iv(\vec{x},\vec{\Omega},t) - \frac{1}{4\pi} \phi(\vec{x},t) \right)
  = - f(\vec{x},\vec{\Omega}) \vec{\Omega}\vd \grad \phi(\vec{x}, t) \,.
\end{equation*}
Formally inverting the operator on the left-hand side, we obtain an expression
for the \APone\ approximation to the radiation intensity:
\begin{equation}\label{eq:ap1Interior2}
 \Iv(\vec{x},\vec{\Omega},t)
 =
 \frac{1}{4\pi} \phi(\vec{x},t)
 - f(\vec{x},\vec{\Omega}) \vec{\Omega} \vd
  \left( \frac{1}{\varsigma(\vec{x})} \frac{1}{c}\pder{}{t} + 1 \right)\inv 
  \grad \phi(\vec{x}, t) \,.
\end{equation}
This expression is unwieldy both in presentation and interpretation. We
therefore desire to rewrite it as a function of the known, concrete $f$ and its
second moment $\Dtens$, as and of the unknown $\phi$ and $\vec{F}$ that are
solved using the conservation equation~\eqref{eq:loVol} and
the \APone\ equation~\eqref{eq:ap1ficksLawFinal}.

The first angular moment of Eq.~\eqref{eq:ap1Interior2} is a restatement of
Eq.~\eqref{eq:ap1ficksLawFinal}:
\begin{equation*}
  \vec{F}(\vec{x},t)
  =
 - \Dtens(\vec{x}) \vd
 \left( \frac{1}{\varsigma(\vec{x})} \frac{1}{c}\pder{}{t} + 1 \right)\inv
 \grad \phi(\vec{x}, t) \,.
\end{equation*}
Left-multiplying by the matrix inverse of $\Dtens$, we obtain:
\begin{equation*}
  \Dtens\inv(\vec{x}) \vd \vec{F}(\vec{x},t)
  =
 - \left( \frac{1}{\varsigma(\vec{x})} \frac{1}{c}\pder{}{t} + 1 \right)\inv
 \grad \phi(\vec{x}, t) \,.
\end{equation*}
The right-hand side of this equation is present in Eq.~\eqref{eq:ap1Interior2}.
We substitute it for the left-hand side:
\begin{equation}\label{eq:ap1InteriorFinal}
 \Iv(\vec{x},\vec{\Omega},t)
 =
 \frac{1}{4\pi} \phi(\vec{x},t)
 - f(\vec{x},\vec{\Omega}) \vec{\Omega} \vd
  \Dtens\inv(\vec{x}) \vd \vec{F}(\vec{x},t) \,.
\end{equation}
This is the anisotropic \Pone\ approximation to the radiation intensity, where
$\phi$ and $\vec{F}$ are the solutions of Eqs.~\eqref{eq:loVol} and
Eq.~\eqref{eq:ap1ficksLawFinal}.

%%%%%%%%%%%%%%%%%%%%%%%%%%%%%%%%%%%%%%%%%%%%%%%%%%%%%%%%%%%%%%%%%%%%%%%%%%%%%%%%
\subsection{Initial layer}

Formally, as applied in Chapter~\ref{chap:adDerivation}, the initial layer
solution matches the transport initial condition,
Eq.~\eqref{eq:tdTransportInit}, to the interior approximation of the radiation
intensity, which in the case of \APone\ is Eq.~\eqref{eq:ap1InteriorFinal}. From
Eq.~\eqref{eq:tdSuperposition}, 
\begin{equation*}
  I^i(\vec{x},\vec{\Omega})
  = \Iv(\vec{x},\vec{\Omega},0) + \Iil(\vec{x},\vec{\Omega},0)\,,
\end{equation*}
and $\Iil$ must rapidly diminish as $t\to\infty$.

In the case of anisotropic \Pone, which uses two unknowns $\phi$ and
$\vec{F}$, the asymptotic matching procedure for the initial condition is not as
clear as with anisotropic diffusion, which has the single unknown $\phi$. (We
are also not aware of any asymptotic procedure to derive the standard \Pone\
equations or match them to initial conditions.)

We therefore take the sensible approach of approximating $\Iil\approx0$, and
setting the zeroth and first moments of the \APone\ initial condition to the
zeroth and first moments of the transport initial condition:
\begin{equation}\label{eq:ap1init}
  \phi(\vec{x},0) = \phi^i(\vec{x}) \,,\qquad\text{and}\qquad
  \vec{F}(\vec{x},0) = \vec{F}^i(\vec{x}) \,.
\end{equation}

%%%%%%%%%%%%%%%%%%%%%%%%%%%%%%%%%%%%%%%%%%%%%%%%%%%%%%%%%%%%%%%%%%%%%%%%%%%%%%%%
\subsection{Boundary layer}

The boundary layer describes the transition from the transport boundary
condition to the interior solution. The boundary layer solution decays to zero
rapidly in the spatial interior, and it satisfies the superposition
equation~\eqref{eq:tdSuperposition}:
\begin{equation*}
  I^b(\vec{x},\vec{\Omega},t)
  = \Iv(\vec{x},\vec{\Omega},t) + \Ibl(\vec{x},\vec{\Omega},t)\,,
  \quad \vec{x}\in \partial V, \vec{\Omega}\vd \vec{n} < 0\,.
\end{equation*}
As described in \S\ref{sec:adBoundary}, the condition that causes the boundary
layer equation to vanish in the interior is 
\begin{equation}\tagref{eq:tdKillBndy}
  \int_{\vec{\Omega}\vd\vec{n} < 0}
  W(\abs{\vec{\Omega}\vd\vec{n}}) \Ibl(\vec{x},\vec{\Omega},t) \ud\Omega
  = 0\,,
  \quad \vec{x}\in \partial V ,\ \vec{\Omega}\vd \vec{n} < 0.
\end{equation}

Operating on the superposition equation at the boundary, and substituting the
\APone\ approximation in the interior from Eq.~\eqref{eq:ap1InteriorFinal}, we
obtain the following relation:
\begin{align*}
  \int_{\vec{\Omega}\vd\vec{n} < 0}
  W(\abs{\vec{\Omega}\vd\vec{n}}) I^b(\vec{x},\vec{\Omega},t \ud\Omega
 &= 
 \int_{\vec{\Omega}\vd\vec{n} < 0}
  W(\abs{\vec{\Omega}\vd\vec{n}}) \left[ \frac{1}{4\pi} \phi(\vec{x},t)
  - f(\vec{x},\vec{\Omega}) \vec{\Omega} \vd
  \Dtens\inv(\vec{x}) \vd \vec{F}(\vec{x},t)\right] \ud\Omega
\\
&= \frac{1}{2} \phi(\vec{x},t)
-  \left[ \int_{\vec{\Omega}\vd\vec{n} < 0} W(\abs{\vec{\Omega}\vd\vec{n}})
  \vec{\Omega} f(\vec{x},\vec{\Omega}) \ud\Omega\right]
 \Dtens\inv(\vec{x}) \vd \vec{F}(\vec{x},t) \,.
\end{align*}

%%%%%%%%%%%%%%%%%%%%%%%%%%%%%%%%%%%%%%%%
\subsection{Summary}

%%%%%%%%%%%%%%%%%%%%%%%%%%%%%%%%%%%%%%%%%%%%%%%%%%%%%%%%%%%%%%%%%%%%%%%%%%%%%%%%
\section{Discussion}



%%%%%%%%%%%%%%%%%%%%%%%%%%%%%%%%%%%%%%%%%%%%%%%%%%%%%%%%%%%%%%%%%%%%%%%%%%%%%%%%
\subsection{Steady-state limit}
Just as the \Pone\ method limits to standard diffusion as $t\to\infty$, the
\APone\ method limits to the anisotropic diffusion method.
\begin{equation*}
  \frac{1}{c}\pder{\vec{F}}{t}(\vec{x},t) + \varsigma(\vec{x})\vec{F}(\vec{x},t) 
  = - \varsigma(\vec{x}) \Dtens(\vec{x}) \vd \grad \phi(\vec{x}, t) \,.
\end{equation*}

%%%%%%%%%%%%%%%%%%%%%%%%%%%%%%%%%%%%%%%%
\subsubsection{Homogeneous limit}
In a homogeneous medium, the \APone\ equation~\eqref{eq:ap1ficksLawFinal} limits
to the \Pone\ equation.

The time-dependent \Pone\ equation is
\begin{equation*}
  \frac{1}{c}\pder{\vec{F}}{t}(\vec{x},t)
  + \frac{1}{3} \grad \phi(\vec{x}, t)
  + \sigma(\vec{x})\vec{F}(\vec{x},t) 
  = 0 \,.
\end{equation*}

In a homogeneous problem, the purely absorbing transport problem for $f$ has the
constant solution $f=1/(4\pi\sigma)$. The nonlocal opacity, $\varsigma$ from
Eq.~\eqref{eq:varSigma}, becomes
\begin{equation*}
  \varsigma
  = \left[ \int_{4\pi} f \ud \Omega \right]\inv
  = \left[ \int_{4\pi} \frac{1}{4\pi\sigma} \ud \Omega \right]\inv
  = \sigma \,.
\end{equation*}
The anisotropic diffusion tensor likewise simplifies to
\begin{equation*}
  \Dtens = \int_{4\pi} \vec{\Omega}\vec{\Omega} f \ud \Omega =
  \frac{1}{3\sigma}\Identitytens \,.
\end{equation*}
Substituting these into the \APone\ equation~\eqref{eq:ap1ficksLawFinal}, we obtain
\begin{equation*}
  \frac{1}{c}\pder{\vec{F}}{t}(\vec{x},t) + \sigma\vec{F}(\vec{x},t) 
  = - \sigma \left[ \frac{1}{3\sigma}\Identitytens \right]
  \vd \grad \phi(\vec{x}, t) \,,
\end{equation*}
which is the \Pone\ approximation:
\begin{equation*}
  \frac{1}{c}\pder{\vec{F}}{t}(\vec{x},t) + \frac{1}{3} \grad \phi
  + \sigma\vec{F}(\vec{x},t) 
  = 0\,.
\end{equation*}

%%%%%%%%%%%%%%%%%%%%%%%%%%%%%%%%%%%%%%%%%%%%%%%%%%%%%%%%%%%%%%%%%%%%%%%%%%%%%%%%
\subsection{Application to nonlinear radiation transport}

As with the anisotropic diffusion approximation in
Chapter~\ref{chap:adDerivation}, the \APone\ method was derived for a linear,
time-dependent transport problem with isotropic scattering. The same arguments
apply. Furthermore, in the diffusive limit, $\Dtens \to \frac{1}{3\sigma} +
O(\epsilon^2)$, and the \APone\ equation reduces to the \Pone\ equation. From
\cite{Mor2000}, the \Pone\ approximation has the correct asymptotic behavior in
the diffusive limit; we therefore assert that the \APone\ approximation does
as well. 


%%%%%%%%%%%%%%%%%%%%%%%%%%%%%%%%%%%%%%%%%%%%%%%%%%%%%%%%%%%%%%%%%%%%%%%%%%%%%%%%
\section{Summary}
The anisotropic \Pone\ approximation was derived using the same procedure as 
time-dependent anisotropic diffusion. The assumption of a stronger time
dependence of $I$ led to \dots

