% !TEX root = _individual/aponeDerivation.tex

%%%%%%%%%%%%%%%%%%%%%%%%%%%%%%%%%%%%%%%%%%%%%%%%%%%%%%%%%%%%%%%%%%%%%%%%%%%%%%%%
% [Anisotropic P1]
\chapter{Anisotropic \texorpdfstring{\Pone}{P1}}\label{chap:aponeDerivation}

The anisotropic diffusion method was derived under the assumptions that the
intensity has weak spatial gradients, mild anisotropy, and very slow time
dependence. By reducing the asymptotic smallness of the time derivative, we
derive a new ``anisotropic'' method, which we name anisotropic \Pone\ (\APone).
In the steady-state case, it reduces to the anisotropic diffusion approximation;
in
the homogeneous-medium case, it reduces to the standard \Pone\ (spherical
harmonic)
approximation. 

%%%%%%%%%%%%%%%%%%%%%%%%%%%%%%%%%%%%%%%%%%%%%%%%%%%%%%%%%%%%%%%%%%%%%%%%%%%%%%%%
\section{Derivation}

We begin with the linear, time-dependent transport equation with isotropic
scattering. By omitting the complication of material--radiation coupling and
nonlinearities, we derive the \APone\ equations with a straightforward
linear asymptotic analysis. We then show how the results may be applied to
the thermal radiative transfer equations.

The linear transport equation, given in Chapter~\ref{chap:adDerivation}
%Eqs.~\eqref{eqs:tdTransport}
and repeated here, comprise the Boltzmann equation,
\begin{multline} \label{eq:tdTransportVol_4}
  \frac{1}{c}\pder{I}{t}(\vec{x},\vec{\Omega},t)
  + \vec{\Omega}\vd \grad I(\vec{x},\vec{\Omega},t)
  + \sigma(\vec{x}) I(\vec{x},\vec{\Omega},t)
  \\ = \frac{\sigma_s(\vec{x})}{4\pi}
  \int_{4\pi} I(\vec{x},\vec{\Omega}',t) \ud \Omega'
  + \frac{q(\vec{x},t)}{4\pi}
  \,, \quad \vec{x}\in V,\ \vec{\Omega}\in4\pi,\ t \ge 0;
\end{multline}
the boundary condition,
\begin{equation} \label{eq:tdTransportBndy_4}
  I(\vec{x},\vec{\Omega},t) = I^b(\vec{x},\vec{\Omega},t) \,,
  \quad \vec{x}\in \partial V ,\ \vec{\Omega}\vd \vec{n} < 0,\ t > 0;
\end{equation}
and the initial condition,
\begin{equation} \label{eq:tdTransportInit_4}
  I(\vec{x},\vec{\Omega},0) = I^i(\vec{x},\vec{\Omega}) \,,
  \quad \vec{x}\in V ,\ \vec{\Omega} \in 4\pi.
\end{equation}

To perform the asymptotic analysis, we again consider the intensity as the
linear superposition of three distinct transport solutions:
\begin{equation} \label{eq:tdSuperposition_4}
  I(\vec{x},\vec{\Omega},t)
  \equiv \Iv(\vec{x},\vec{\Omega},t)
  + \Ibl(\vec{x},\vec{\Omega},t)
  + \Iil(\vec{x},\vec{\Omega},t)\,.
\end{equation}
Here, $\Iv$ is an ``interior'' solution, $\Ibl$ is a ``boundary layer'' solution,
and $\Iil$ is an ``initial layer'' solution. For a more detailed description of
these, see \S\ref{sec:adDerivation} and Fig.~\ref{fig:layers}.

We shall approximate the interior solution $\Iv$ using a slightly different set
of assumptions than were used to arrive at the anisotropic diffusion
approximation. As before, the boundary and
initial layer solutions are used to match the resulting interior result
to the transport solution at the boundary and initial time.

%%%%%%%%%%%%%%%%%%%%%%%%%%%%%%%%%%%%%%%%%%%%%%%%%%%%%%%%%%%%%%%%%%%%%%%%%%%%%%%%
\subsection{Interior solution}

The interior transport equation is valid several mean free paths away from the
outer boundary, and several mean free times after the initial time.
It is identical to the interior equation in \S\ref{sec:adInterior}, and
the first steps in developing the \APone\ approximation strongly resemble those
for the AD approximation.
The derivation differs because the \APone\
assumptions do not allow the time derivative to be discarded: we instead
``split'' the transport operator into a component that depends on angle and a
component that depends on time. The result is an interior approximation $\Iv$
that requires the storage of both $\phi$ and $\vec{F}$.

The interior transport equation is
\begin{equation} \label{eq:tdVol_4}
  \frac{1}{c}\pder{\Iv}{t}(\vec{x},\vec{\Omega},t)
  + \vec{\Omega}\vd \grad \Iv(\vec{x},\vec{\Omega},t)
  + \sigma(\vec{x}) \Iv(\vec{x},\vec{\Omega},t)
  = \frac{\sigma_s(\vec{x})}{4\pi}
  \phi(\vec{x},t) + \frac{q(\vec{x},t)}{4\pi} \,,
\end{equation}
with the \emph{interior} scalar intensity defined as
\begin{equation} \label{eq:tdPhi_4}
  \phi(\vec{x}) \equiv \int_{4\pi} \Iv(\vec{x}, \vec{\Omega},t) \ud \Omega\,.
\end{equation}

The zeroth angular moment of the interior transport equation is the following
conservation equation:
\begin{equation} \label{eq:loVol_4}
\frac{1}{c} \pder{\phi}{t} (\vec{x}, t)
  + \grad \vd\vec{F}(\vec{x}, t)
  + \sigma(\vec{x}) \phi(\vec{x}, t)
 = \sigma_s(\vec{x}) \phi(\vec{x},t) + q(\vec{x},t)\,,
\end{equation}
with the \emph{interior} radiation flux defined as
\begin{equation}\label{eq:tdF_4}
  \vec{F} \equiv \int_{4\pi} \vec{\Omega} \Iv(\vec{x}, \vec{\Omega},t) \ud
  \Omega\,.
\end{equation}

By combining Eqs.~\eqref{eq:tdVol_4} and~\eqref{eq:loVol_4} (see
\S\ref{sec:adInterior}), we eliminate the isotropic scattering and extraneous
sources on
the right-hand side of the transport equation to obtain
\begin{equation}\label{eq:capPsiVol_4}
  \left[ \frac{1}{c}\pder{}{t}
  + \vec{\Omega}\vd \grad
  + \sigma \right]
   \left( \Iv
  - \frac{1}{4\pi} \phi \right)
  = \frac{1}{4\pi} \grad \vd\vec{F} -
  \frac{1}{4\pi} \vec{\Omega}\vd \grad \phi\,.
\end{equation}

Next, we approximate Eq.~\eqref{eq:capPsiVol_4} by making an asymptotic ansatz
about the behavior of $I$ and then discarding ``small'' terms. Here the
derivation diverges from anisotropic diffusion: rather than assuming
$\tpder{I}{t} = O(\epsilon^2)$, we take the scaling $\tpder{I}{t} =
O(\epsilon)$. Thus the assumed magnitude of the intensity, its derivatives, and
its anisotropy are:
\begin{align} \label{eq:ap1ansatz}
  \sigma &= O(1), &
  I &= O(1), &
  \int_{4\pi} \vec{\Omega} I \ud\Omega &= O(\epsilon), &
  \grad I &= O(\epsilon), &
  \pder{I}{t} &= O(\epsilon) \,.
\end{align}

The stronger magnitude of the time derivative requires that the time derivative
in Eq.~\eqref{eq:capPsiVol_4} must be retained to preserving $O(\epsilon^2)$
accuracy. (This differs from anisotropic diffusion, in which the term was
discarded.) The only asymptotically
small term is $\grad \vd\vec{F} = O(\epsilon^2)$, which we discard.
Equation~\eqref{eq:capPsiVol_4} is then
\begin{equation*}
  \left[ \frac{1}{c}\pder{}{t}
  + \vec{\Omega}\vd \grad
  + \sigma \right]
   \left( \Iv - \frac{1}{4\pi} \phi \right)
  = - \frac{1}{4\pi} \vec{\Omega}\vd \grad \phi\,.
\end{equation*}

Now we decompose the ``time${}+{}$leakage${}+{}$collision'' operator on the
left-hand side into an $(\vec{x},t)$-dependent operator, an
$(\vec{x},\vec{\Omega})$-dependent operator, and an asymptotically small
remainder:
\begin{equation*}
  \frac{1}{c}\pder{}{t}
  + \vec{\Omega}\vd \grad
  + \sigma
  =
  \left( \vec{\Omega}\vd \grad + \sigma \right)
  \left( \frac{1}{\sigma} \frac{1}{c}\pder{}{t} + 1 \right)
  -  \vec{\Omega}\vd \grad \frac{1}{\sigma} \frac{1}{c}\pder{}{t} \,.
\end{equation*}
Formally, the rightmost term is $O(\epsilon^2)$: $\sigma$ is $O(1)$,
$\vec{\Omega}\vd\grad$ is $O(\epsilon)$, and $\frac{1}{c}\pder{}{t}$ is
$O(\epsilon)$. Thus the previous equation can be written with $O(\epsilon^2)$
error as:
\begin{equation*}
  \left( \vec{\Omega}\vd \grad + \sigma \right)
  \left( \frac{1}{\sigma} \frac{1}{c}\pder{}{t} + 1 \right)
   \left( \Iv - \frac{1}{4\pi} \phi \right)
  = - \frac{1}{4\pi} \vec{\Omega}\vd \grad \phi\,.
\end{equation*}
By making the definition
\begin{equation}\label{eq:ap1capPsi}
  \Psi(\vec{x},\vec{\Omega},t)
  \equiv
  \left( \frac{1}{\sigma(\vec{x})} \frac{1}{c}\pder{}{t} + 1 \right)
  \left( \Iv(\vec{x},\vec{\Omega},t) - \frac{1}{4\pi} \phi(\vec{x},t) \right)\,,
\end{equation}
we obtain the following ``steady-state'' equation:
\begin{equation}\label{eq:ap1Ss}
  \left[ \vec{\Omega}\vd \grad + \sigma(\vec{x}) \right] \Psi(\vec{x},\vec{\Omega},t)
  = - \frac{1}{4\pi} \vec{\Omega}\vd \grad \phi (\vec{x},t)\,.
\end{equation}
Here, time is a parameter, not a variable: if $\phi(\vec{x},t)$ is known,
$\Psi(\vec{x},\vec{\Omega},t)$ can be calculated without knowledge of any prior
$\phi$ or $\Psi$.

Equation~\eqref{eq:ap1Ss} is very similar to Eq.~\eqref{eq:tdVolApprox1} of the
AD derivation. As we did there, we formally invert the streaming${}+{}$collision
operator on the left, interpreting the inverse as an integral transport
operator:
\begin{align*}
  \Psi
  &= \left[ \vec{\Omega}\vd \grad + \sigma \right]\inv \left(
  - \frac{1}{4\pi} \vec{\Omega}\vd \grad \phi \right)
  \\
  &= - \int_{0}^{\infty}
  \left( \frac{1}{4\pi} \vec{\Omega}\vd \grad \phi(\vec{x} - s \vec{\Omega},
  t)\right)
  \eexp^{ -\tau(\vec{x}, \vec{x} - s \vec{\Omega})}
  \ud s \,.
\end{align*}
Taylor-expanding the non-local $\phi$ about the local point $\vec{x}$, and
recalling the assumption that $\grad\phi = O(\epsilon)$, we move $\phi$ outside
the operator to obtain the following:
\begin{align} \nonumber
\Psi(\vec{x},\vec{\Omega},t)
&= - \left[ \int_{0}^{\infty} \frac{1}{4\pi} 
    \eexp^{ -\tau(\vec{x}, \vec{x} - s \vec{\Omega})}
      \ud s \right] \vec{\Omega}\vd \grad\phi(\vec{x}, t) + O(\epsilon^2)
\\ \nonumber
&= - 
  \left[\left(\vec{\Omega}\vd \grad  + \sigma \right)\inv
  \frac{1}{4\pi}\right] \vec{\Omega}\vd \grad \phi(\vec{x}, t)
\\ \label{eq:ap1capPsiApprox}
&= - \left[ f(\vec{x},\vec{\Omega}) \right] \vec{\Omega}\vd \grad \phi(\vec{x}, t)\,.
\end{align}
Here, $f$ is the solution to the same purely absorbing, steady-state transport
problem as in \S\ref{sec:adInterior}:
\begin{equation} \label{eq:fFullVol_4}
  \vec{\Omega}\vd \grad f(\vec{x}, \vec{\Omega})
  + \sigma(\vec{x}) f (\vec{x}, \vec{\Omega})
= \frac{1}{4\pi} \,.
\end{equation}

We substitute the $O(\epsilon^2)$-accurate expression for $\Psi$,
Eq.~\eqref{eq:ap1capPsiApprox}, into Eq.~\eqref{eq:ap1capPsi}:
\begin{equation}\label{eq:ap1Interior1}
  \left( \frac{1}{\sigma(\vec{x})} \frac{1}{c}\pder{}{t} + 1 \right)
  \left( \Iv(\vec{x},\vec{\Omega},t) - \frac{1}{4\pi} \phi(\vec{x},t) \right)
  = - f(\vec{x},\vec{\Omega}) \vec{\Omega}\vd \grad \phi(\vec{x}, t) \,.
\end{equation}
Taking the first angular moment of this equation eliminates $\phi$ on the
left-hand side to yield the following equation:
\begin{equation}\label{eq:ap1FicksLaw1}
  \left( \frac{1}{\sigma(\vec{x})} \frac{1}{c}\pder{}{t} + 1 \right)
  \vec{F}(\vec{x},t) 
  = - \int_{4\pi} f(\vec{x},\vec{\Omega}) \vec{\Omega} \ud\Omega
  \vd \grad \phi(\vec{x}, t) \,.
\end{equation}
Multiplying through by $\sigma$ and rearranging, we get:
\begin{equation}\label{eq:ap1FicksLaw1a}
  \frac{1}{c}\pder{\vec{F}}{t}(\vec{x},t)
  + \sigma(\vec{x}) \Dtens(\vec{x}) \vd \grad \phi(\vec{x}, t) 
  + \sigma(\vec{x})\vec{F}(\vec{x},t) 
  = 0 \,,
\end{equation}
where we have used the definition of the anisotropic diffusion tensor from
Chapter~\ref{chap:adDerivation},
\begin{equation}\label{eq:dDefinition_4}
  \Dtens(\vec{x}) \equiv \int_{4\pi} \vec{\Omega} \vec{\Omega}
  f(\vec{x}, \vec{\Omega}) \ud\Omega \,.
\end{equation}

Equation~\eqref{eq:ap1FicksLaw1a} is our first attempt at an ``anisotropic
\Pone'' equation. It approximates the radiation using the scalar unknown $\phi$
and the vector unknown $\vec{F}$ in tandem with the calculated anisotropic
diffusion tensor $\Dtens$.

In a homogeneous medium, $\Dtens$ limits to $\Identitytens/(3\sigma)$, so
Eq.~\eqref{eq:ap1FicksLaw1a} becomes:
\begin{equation*}
  \frac{1}{c}\pder{\vec{F}}{t}(\vec{x},t) - \frac{1}{3} \grad\phi(\vec{x}, t)
  + \sigma\vec{F}(\vec{x},t) 
  = 0\,,
\end{equation*}
which is the standard \Pone\ equation.

%Formally inverting the left-hand side, we obtain an approximate expression for
%the interior solution $\Iv$ 
%\begin{equation*}
%   \Iv(\vec{x},\vec{\Omega},t) 
%   = \frac{1}{4\pi} \phi(\vec{x},t) 
%  - \left( \frac{1}{\sigma(\vec{x})} \frac{1}{c}\pder{}{t} + 1 \right)\inv
%  f(\vec{x},\vec{\Omega}) \vec{\Omega}\vd \grad \phi(\vec{x}, t) \,.
%\end{equation*}

%%%%%%%%%%%%%%%%%%%%%%%%%%%%%%%%%%%%%%%%%%%%%%%%%%%%%%%%%%%%%%%%%%%%%%%%%%%%%%%%
\subsubsection{An important substitution}
In a voided region, $\sigma\approx 0$, Eq.~\eqref{eq:ap1FicksLaw1a}
has a serious deficiency. The nonlocal dependence of $f$ on $\sigma$ (see
\S\ref{sec:adVoids}) prevents $\Dtens$ from ``blowing up,'' but in the \APone\
equation, the local $\sigma=0$ eliminates two of the terms. In a void,
Eq.~\eqref{eq:ap1FicksLaw1a} becomes the unphysical
\begin{equation*}
  \frac{1}{c}\pder{\vec{F}}{t}(\vec{x},t) = 0\,.
\end{equation*}
(Preliminary numerical tests rightly demonstrated this to be a fatal
shortcoming in any problem with optically thin regions.)

To understand this behavior, we show how a similar procedure yields the standard
\Pone\ equation,
\begin{equation*}
  \frac{1}{c}\pder{\vec{F}}{t}(\vec{x},t) - \frac{1}{3}\grad \phi(\vec{x},t)
  + \sigma \vec{F} = 0\,.
\end{equation*}
We return to Equation~\eqref{eq:ap1Ss}, just after we split the time and angle
operators with an asymptotic error of $O(\epsilon^2)$:
\begin{equation*}
  \Psi(\vec{x},\vec{\Omega},t)
  = - \left[ \vec{\Omega}\vd \grad + \sigma(\vec{x}) \right]\inv \frac{1}{4\pi} \vec{\Omega}\vd \grad \phi (\vec{x},t)\,.
\end{equation*}
The term in brackets can be asymptotically expanded as
\begin{equation*}
  \left[ \epsilon \vec{\Omega}\vd \grad + \sigma \right]\inv
  \sim \frac{1}{\sigma} \left( 1
  - \epsilon \frac{1}{\sigma} \vec{\Omega}\vd \grad + O(\epsilon^2) \right)\,.
\end{equation*}
Substituting into Eq.~\eqref{eq:ap1capPsiApprox} gives
\begin{equation*}
  \left( \frac{1}{\sigma(\vec{x})} \frac{1}{c}\pder{}{t} + 1 \right)
  \left( \Iv(\vec{x},\vec{\Omega},t) - \frac{1}{4\pi} \phi(\vec{x},t) \right)
  = - \frac{1}{\sigma(\vec{x})} \left( 1 - \frac{1}{\sigma} \vec{\Omega}\vd
  \grad \right) \frac{1}{4\pi} \vec{\Omega}\vd \grad \phi(\vec{x},t) \,.
\end{equation*}
The first angular moment of this equation is:
\begin{equation*}
  \frac{1}{\sigma} \frac{1}{c}\pder{\vec{F}}{t} 
  + \vec{F}
  = - \frac{1}{3}\frac{1}{\sigma} \grad \phi \,.
\end{equation*}
Multiplying by $\sigma$ gives the \Pone\ equation:
\begin{equation*}
  \frac{1}{c}\pder{\vec{F}}{t} + \frac{1}{3}\frac{1}{\sigma} \grad \phi
  + \sigma\vec{F}
  = 0 \,.
\end{equation*}

The asymptotic expansion used in the derivation of the \Pone\ equation used a
local approximation to $\sigma$ that corresponded to the local $\sigma$ in the
split operator.
In contrast, the derivation of the ``anisotropic''
equation~\eqref{eq:ap1FicksLaw1a} uses both a local $\sigma$ and a nonlocal $f$.
To rectify this imbalance, we wish to replace the $1/\sigma$ in
Eq.~\eqref{eq:ap1FicksLaw1} with a quantity 
that does not blow up as $\sigma\to 0$ but which approaches $1/\sigma$ in the
diffusive limit.

We make the following substitution, which is accurate to $O(\epsilon^2)$ in the
diffusive limit (in which the opacity has weak spatial derivatives):
\begin{align*}
  \frac{1}{\sigma(\vec{x})}
  &\approx \int_{4\pi} f(\vec{x},\vec{\Omega}) \ud \Omega
  \\
  &= \int_{4\pi} (\vec{\Omega} \vd \grad + \sigma(\vec{x}) )\inv
  \frac{1}{4\pi} \ud \Omega
  \\
  &= \int_{4\pi} \left( \frac{1}{\sigma} \vec{\Omega} \vd
  \grad + 1 \right)\inv \frac{1}{4\pi\sigma} \ud \Omega
  \\
  &\sim \int_{4\pi} \left(1 - \frac{1}{\sigma} \vec{\Omega} \vd
  \grad + O(\epsilon^2) \right) \frac{1}{4\pi\sigma} \ud \Omega
  \\
  &\sim \frac{1}{\sigma} + O(\epsilon^2) \,.
\end{align*}

The resulting \APone\ approximation is:
\begin{equation*}
  \frac{1}{\varsigma(\vec{x})} \pder{\vec{F}}{t}(\vec{x},t)
  + \vec{F}(\vec{x},t) = - \Dtens(\vec{x}) \vd \grad \phi \,,
\end{equation*}
where we have defined
\begin{equation}\label{eq:varSigma}
  \frac{1}{\varsigma(\vec{x})}
  \equiv \int_{4\pi} f(\vec{x},\vec{\Omega}) \ud \Omega
  \sim \frac{1}{\sigma(\vec{x})} + O(\epsilon^2)\,.
\end{equation}

Now Eq.~\eqref{eq:ap1FicksLaw1a} is replaced by the new anisotropic \Pone
equation:
\begin{equation}\label{eq:ap1FicksLawFinal}
  \frac{1}{c}\pder{\vec{F}}{t}(\vec{x},t)
  + \varsigma(\vec{x}) \Dtens(\vec{x}) \vd \grad \phi(\vec{x}, t)
  + \varsigma(\vec{x})\vec{F}(\vec{x},t) 
  = 0 \,.
\end{equation}
This is a closure for the unknown $\vec{F}$ in the conservation
equation~\eqref{eq:loVol_4}; $\Iv$ in the interior is thus approximated by
the unknowns $\phi$ and $\vec{F}$ using the coefficients $\varsigma$ and
$\Dtens$ calculated from the purely absorbing transport equation for $f$.

%%%%%%%%%%%%%%%%%%%%%%%%%%%%%%%%%%%%%%%%%%%%%%%%%%%%%%%%%%%%%%%%%%%%%%%%%%%%%%%%
\subsubsection{The interior approximate intensity}

We return to Eq.~\eqref{eq:ap1Interior1} and replace $\sigma$ with $\varsigma$
as discussed in the previous section:
\begin{equation*}
  \left( \frac{1}{\varsigma(\vec{x})} \frac{1}{c}\pder{}{t} + 1 \right)
  \left( \Iv(\vec{x},\vec{\Omega},t) - \frac{1}{4\pi} \phi(\vec{x},t) \right)
  = - f(\vec{x},\vec{\Omega}) \vec{\Omega}\vd \grad \phi(\vec{x}, t) \,.
\end{equation*}
Formally inverting the operator on the left-hand side, we obtain an expression
for the \APone\ approximation to the radiation intensity:
\begin{equation}\label{eq:ap1Interior2}
 \Iv(\vec{x},\vec{\Omega},t)
 =
 \frac{1}{4\pi} \phi(\vec{x},t)
 - f(\vec{x},\vec{\Omega}) \vec{\Omega} \vd
  \left( \frac{1}{\varsigma(\vec{x})} \frac{1}{c}\pder{}{t} + 1 \right)\inv 
  \grad \phi(\vec{x}, t) \,.
\end{equation}
This expression is unwieldy. We
therefore rewrite it as a function of the known $f$ and its
second moment $\Dtens$, and of the unknown $\phi$ and $\vec{F}$ that are
solved using the conservation equation~\eqref{eq:loVol_4} and
the \APone\ equation~\eqref{eq:ap1FicksLawFinal}.

The first angular moment of Eq.~\eqref{eq:ap1Interior2} is a restatement of
Eq.~\eqref{eq:ap1FicksLawFinal}:
\begin{equation*}
  \vec{F}(\vec{x},t)
  =
 - \Dtens(\vec{x}) \vd
 \left( \frac{1}{\varsigma(\vec{x})} \frac{1}{c}\pder{}{t} + 1 \right)\inv
 \grad \phi(\vec{x}, t) \,.
\end{equation*}
Left-multiplying by the matrix inverse of $\Dtens$, we obtain:
\begin{equation*}
  \Dtens\inv(\vec{x}) \vd \vec{F}(\vec{x},t)
  =
 - \left( \frac{1}{\varsigma(\vec{x})} \frac{1}{c}\pder{}{t} + 1 \right)\inv
 \grad \phi(\vec{x}, t) \,.
\end{equation*}
The right-hand side of this equation is present in Eq.~\eqref{eq:ap1Interior2}.
We substitute it for the left-hand side to obtain the anisotropic \Pone\
approximation to the radiation intensity:
\begin{equation}\label{eq:ap1InteriorFinal}
 \Iv(\vec{x},\vec{\Omega},t)
 =
 \frac{1}{4\pi} \phi(\vec{x},t)
 + f(\vec{x},\vec{\Omega}) \vec{\Omega} \vd
  \Dtens\inv(\vec{x}) \vd \vec{F}(\vec{x},t) \,.
\end{equation}
Here, $\phi$ and $\vec{F}$ are the solutions of Eqs.~\eqref{eq:loVol_4} and
Eq.~\eqref{eq:ap1FicksLawFinal}.

In a steady-state problem, $\vec{F} = -\Dtens \vd \grad \phi$, and the above
equation reduces to the anisotropic diffusion approximation to $\Iv$.

%%%%%%%%%%%%%%%%%%%%%%%%%%%%%%%%%%%%%%%%%%%%%%%%%%%%%%%%%%%%%%%%%%%%%%%%%%%%%%%%
\subsection{Initial layer}

Formally, as shown in Chapter~\ref{chap:adDerivation}, the initial layer
solution matches the transport initial condition,
Eq.~\eqref{eq:tdTransportInit_4}, to the interior approximation of the radiation
intensity, which in the case of \APone\ is Eq.~\eqref{eq:ap1InteriorFinal}. From
Eq.~\eqref{eq:tdSuperposition_4}, the initial conditions of the original transport
equation, the interior approximation, and the initial layer satisfy:
\begin{equation*}
  I^i(\vec{x},\vec{\Omega})
  = \Iv(\vec{x},\vec{\Omega},0) + \Iil(\vec{x},\vec{\Omega},0)\,,
\end{equation*}
and $\Iil$ must rapidly diminish as $t\to\infty$.

In the case of anisotropic \Pone, which uses two unknowns $\phi$ and
$\vec{F}$, the asymptotic matching procedure for the initial condition is not as
clear as with anisotropic diffusion, which has the single unknown $\phi$. (We
are also not aware of any asymptotic procedure to derive the standard \Pone\
equations or match them to initial conditions.)

We therefore take the sensible approach of approximating $\Iil\approx0$, and
setting the zeroth and first moments of the \APone\ initial condition to the
zeroth and first moments of the transport initial condition:
\begin{equation}\label{eq:ap1init}
  \phi(\vec{x},0) = \phi^i(\vec{x}) \,,\qquad\text{and}\qquad
  \vec{F}(\vec{x},0) = \vec{F}^i(\vec{x}) \,.
\end{equation}

%%%%%%%%%%%%%%%%%%%%%%%%%%%%%%%%%%%%%%%%%%%%%%%%%%%%%%%%%%%%%%%%%%%%%%%%%%%%%%%%
\subsection{Boundary layer}

The boundary layer describes the transition from the transport boundary
condition to the interior solution. The boundary layer solution decays to zero
rapidly in the spatial interior, and it satisfies the superposition
equation~\eqref{eq:tdSuperposition_4}:
\begin{equation*}
  I^b(\vec{x},\vec{\Omega},t)
  = \Iv(\vec{x},\vec{\Omega},t) + \Ibl(\vec{x},\vec{\Omega},t)\,,
  \quad \vec{x}\in \partial V, \vec{\Omega}\vd \vec{n} < 0\,.
\end{equation*}
As described in \S\ref{sec:adBoundary}, the condition that causes the boundary
layer solution to vanish in the interior is 
\begin{equation}\label{eq:tdKillBndy_4}
  \int_{\vec{\Omega}\vd\vec{n} < 0}
  W(\abs{\vec{\Omega}\vd\vec{n}}) \Ibl(\vec{x},\vec{\Omega},t) \ud\Omega
  = 0\,,
  \quad \vec{x}\in \partial V ,\ \vec{\Omega}\vd \vec{n} < 0.
\end{equation}
(The function $W$ is related to the Chandrasekhar function, $W(\mu)\approx \mu +
\frac32 \mu^2$.)

Operating by $ \int_{\vec{\Omega}\vd\vec{n} < 0}
W(\abs{\vec{\Omega}\vd\vec{n}}) (\cdot) \ud\Omega$ on the superposition
equation at the boundary, and substituting the
\APone\ approximation in the interior from Eq.~\eqref{eq:ap1InteriorFinal}, we
obtain the following relation:
\begin{align*}
  \lefteqn{\int_{\vec{\Omega}\vd\vec{n} < 0}
  W(\abs{\vec{\Omega}\vd\vec{n}}) I^b(\vec{x},\vec{\Omega},t) \ud\Omega}
  \qquad&
  \\
 &= 
 \int_{\vec{\Omega}\vd\vec{n} < 0}
  W(\abs{\vec{\Omega}\vd\vec{n}}) \left[ \frac{1}{4\pi} \phi(\vec{x},t)
  + f(\vec{x},\vec{\Omega}) \vec{\Omega} \vd
  \Dtens\inv(\vec{x}) \vd \vec{F}(\vec{x},t)\right] \ud\Omega
\\
&= \frac{1}{2} \phi(\vec{x},t)
-  \left[ - \int_{\vec{\Omega}\vd\vec{n} < 0} W(\abs{\vec{\Omega}\vd\vec{n}})
  \vec{\Omega} f(\vec{x},\vec{\Omega}) \ud\Omega\right]
 \Dtens\inv(\vec{x}) \vd \vec{F}(\vec{x},t) \,.
\end{align*}
The quantity in brackets is the same vector $\vec{d}$ as in the anisotropic
diffusion approximation,
\begin{equation} \label{eq:adBoundaryIntegral2_4}
  \vec{d}(\vec{x}) = -\int_{\vec{\Omega}\vd\vec{n} < 0} W(\abs{\vec{\Omega}\vd\vec{n}})
\vec{\Omega} f(\vec{x}, \vec{\Omega}) \ud\Omega\,.
\end{equation}
Substituting this into the previous equation and multiplying by 2, we obtain
the anisotropic \Pone\ boundary condition:
\begin{equation} \label{eq:ap1BoundaryCondition}
  2 \int_{\vec{\Omega}\vd\vec{n} < 0}
  W(\abs{\vec{\Omega}\vd\vec{n}}) I^b(\vec{x},\vec{\Omega},t) \ud\Omega
  = \phi(\vec{x},t)
  - 2 \vec{d}(\vec{x}) \vd \Dtens\inv(\vec{x}) \vd \vec{F}(\vec{x},t) \,.
\end{equation}
If the problem is steady-state, $\vec{F} = - \Dtens \vd \grad \phi$, and the
boundary condition reduces to the anisotropic diffusion boundary condition from
Chapter~\ref{chap:adDerivation}.

The boundary condition for $f$ can, as with anisotropic diffusion, be derived by
demanding that Eq.~\eqref{eq:tdPhi_4} hold on the boundary:
\begin{equation*}
 \phi(\vec{x},t)
 = \int_{4\pi} \Iv(\vec{x},\vec{\Omega},t) \ud \Omega
 = \int_{4\pi}  \left( 
\frac{1}{4\pi} \phi(\vec{x},t)
 + f(\vec{x},\vec{\Omega}) \vec{\Omega} \vd
  \Dtens\inv(\vec{x}) \vd \vec{F}(\vec{x},t)
 \right) \,.
\end{equation*}
The result is the same as Eq.~\eqref{eq:hoBc}:
\begin{equation}\label{eq:hoBc_4}
  \int_{\vec{\Omega}\vd\vec{n} > 0} (\vec{\Omega}\vd \vec{n})
  f(\vec{x}, \vec{\Omega}) \ud\Omega
  =
  \int_{\vec{\Omega}\vd\vec{n} < 0} \abs{\vec{\Omega}\vd \vec{n}}
  f(\vec{x}, \vec{\Omega}) \ud\Omega \,,
\end{equation}
which can be satisfied by a white or a reflecting boundary on $f$.

Recall that under the assumption that $f$ is rotationally invariant about
$\vec{n}$ at the boundary, and if the Marshak approximation $W(\mu) \approx
2\mu$ is used, then $\vec{d} = \vec{n} \vd \Dtens$, canceling $\Dtens\inv$ in
the above expression to yield the standard Marshak boundary condition:
\begin{equation*}
  4 F^-(\vec{x},t) =  \phi(\vec{x},t) - 2 \vec{n} \vd \vec{F}(\vec{x},t)\,,
\end{equation*}
where $\vec{F}$ satisfies the anisotropic \Pone\
equation~\eqref{eq:ap1FicksLawFinal} rather than Fick's law.


%%%%%%%%%%%%%%%%%%%%%%%%%%%%%%%%%%%%%%%%
\subsection{Summary}

By making several asymptotically valid approximations and substitutions, we have
derived a new ``anisotropic \Pone'' approximation to time-dependent radiation
transport. Like the anisotropic diffusion approximation, it uses
a simple, transport-calculated diffusion tensor in conjunction with a low-order
conservation equation. Yet, like the \Pone\ method, the low-order unknowns now
comprise not only $\phi$ but also $\vec{F}$, the zeroth and first moments of the
radiation field, rather than merely the zeroth.

The low-order conservation equation~\eqref{eq:loVol_4} is:
\begin{equation*}
\frac{1}{c} \pder{\phi}{t} (\vec{x}, t)
+ \grad \vd \vec{F}(\vec{x},t)
+ \sigma(\vec{x}) \phi(\vec{x}, t)
  = Q(\vec{x}, t) \,,
  \quad \vec{x} \in V,\ t > 0 \,.
\end{equation*}
This is coupled with Eq.~\eqref{eq:ap1FicksLawFinal}, which acts as a
replacement to Fick's law by relating the radiation flux $\vec{F}$ to the
scalar intensity $\phi$:
\begin{equation*}
  \frac{1}{c}\pder{\vec{F}}{t}(\vec{x},t)
  + \varsigma(\vec{x}) \Dtens(\vec{x}) \vd \grad \phi(\vec{x}, t)
  + \varsigma(\vec{x})\vec{F}(\vec{x},t) 
  = 0 \,.
\end{equation*}
In this equation are embedded two moments of the simple transport solution $f$:
the zeroth moment from Eq.~\eqref{eq:varSigma},
\begin{equation*}
  \varsigma(\vec{x})
  = \left[ \int_{4\pi} f(\vec{x},\vec{\Omega}) \ud \Omega \right]\inv \,,
\end{equation*}
and the second moment from Eq.~\eqref{eq:dDefinition_4},
\begin{equation*}
  \Dtens(\vec{x}) \equiv \int_{4\pi} \vec{\Omega} \vec{\Omega}
  f(\vec{x}, \vec{\Omega}) \ud\Omega \,.
\end{equation*}
The general boundary condition for the low-order equation is given in
Eq.~\eqref{eq:ap1BoundaryCondition}:
\begin{equation*}
  2 \int_{\vec{\Omega}\vd\vec{n} < 0}
  W(\abs{\vec{\Omega}\vd\vec{n}}) I^b(\vec{x},\vec{\Omega},t) \ud\Omega
  = \phi(\vec{x},t)
  - 2 \vec{d}(\vec{x}) \vd \Dtens\inv(\vec{x}) \vd \vec{F}(\vec{x},t) \,,
\end{equation*}
where $\vec{d}$, from Eq.~\eqref{eq:adBoundaryIntegral}, is a particular
angular moment of the transport solution $f$, evaluated at the boundary for
incident angles:
\begin{equation*}
  \vec{d}(\vec{x})
  = \left[ - \int_{\vec{\Omega}\vd\vec{n} < 0}
  \abs{\vec{\Omega}\vd\vec{n}} W(\abs{\vec{\Omega}\vd\vec{n}})
  f(\vec{x}, \vec{\Omega}) \ud\Omega\right] \vec{n}\,.
\end{equation*}
(This equation is written under the assumption that $f$ is rotationally
invariant about $\vec{n}$ at the boundary.)

The transport problem for $f$ is the same as in standard anisotropic diffusion:
a steady-state, purely absorbing transport
equation described by Eq.~\eqref{eq:fFullVol_4}:
\begin{equation*}
  \vec{\Omega}\vd \grad f(\vec{x}, \vec{\Omega})
  + \sigma(\vec{x}) f (\vec{x}, \vec{\Omega}) 
  = \frac{1}{4\pi} \,, \quad x \in V,\ \vec{\Omega} \in 4\pi\,,
\end{equation*}
with boundary conditions, usually taken to be reflecting, that satisfy
Eq.~\eqref{eq:hoBc_4}:
\begin{equation*}
  \int_{\vec{\Omega}\vd\vec{n} > 0} (\vec{\Omega}\vd \vec{n})
  f(\vec{x}, \vec{\Omega}) \ud\Omega
  =
  \int_{\vec{\Omega}\vd\vec{n} < 0} \abs{\vec{\Omega}\vd \vec{n}}
  f(\vec{x}, \vec{\Omega}) \ud\Omega \,.
\end{equation*}

%%%%%%%%%%%%%%%%%%%%%%%%%%%%%%%%%%%%%%%%%%%%%%%%%%%%%%%%%%%%%%%%%%%%%%%%%%%%%%%%
\section{Discussion}

Because the transport equation for $f$ in anisotropic \Pone\ is the same as in
anisotropic diffusion, the diffusion tensor $\Dtens$ is also the same, and much
of the discussion there also applies here.
The transport calculation for $f$ has no scattering source to converge; the
anisotropic diffusion tensor $\Dtens$ is therefore computationally inexpensive.

%%%%%%%%%%%%%%%%%%%%%%%%%%%%%%%%%%%%%%%%%%%%%%%%%%%%%%%%%%%%%%%%%%%%%%%%%%%%%%%%
\subsection{Steady-state limit}
Just as the \Pone\ method limits to standard diffusion as $\tpder I t \to 0$,
the \APone\ method limits to the anisotropic diffusion method. We return to
Eq.~\eqref{eq:ap1FicksLawFinal}:
\begin{equation*}
  \frac{1}{c}\pder{\vec{F}}{t}(\vec{x},t)
  + \varsigma(\vec{x}) \Dtens(\vec{x}) \vd \grad \phi(\vec{x}, t)
  + \varsigma(\vec{x})\vec{F}(\vec{x},t)
  = 0 \,.
\end{equation*}
Letting $\frac{1}{c}\pder{\vec{F}}{t}\to 0$,
\begin{equation*}
  \varsigma(\vec{x})\vec{F}(\vec{x},t)
  = -\varsigma(\vec{x}) \Dtens(\vec{x}) \vd \grad \phi(\vec{x}, t) \,.
\end{equation*}
Dividing through by $\varsigma$, which is strictly positive, we obtain the
anisotropic Fick's law of Eq.~\eqref{eq:adFicks}:
\begin{equation*}
 \vec{F}(\vec{x},t)
  = -\varsigma(\vec{x}) \Dtens(\vec{x}) \vd \grad \phi(\vec{x}, t) \,. 
\end{equation*}

%%%%%%%%%%%%%%%%%%%%%%%%%%%%%%%%%%%%%%%%
\subsection{Homogeneous medium limit}
In a homogeneous medium, the \APone\ equation~\eqref{eq:ap1FicksLawFinal} limits
to the \Pone\ equation.

The time-dependent \Pone\ equation is
\begin{equation*}
  \frac{1}{c}\pder{\vec{F}}{t}(\vec{x},t)
  + \frac{1}{3} \grad \phi(\vec{x}, t)
  + \sigma(\vec{x})\vec{F}(\vec{x},t) 
  = 0 \,.
\end{equation*}

When $\sigma$ is constant, the purely absorbing transport problem for $f$ has the
constant solution $f=1/(4\pi\sigma)$. The nonlocal opacity, $\varsigma$ from
Eq.~\eqref{eq:varSigma}, becomes
\begin{equation*}
  \varsigma
  = \left[ \int_{4\pi} f \ud \Omega \right]\inv
  = \left[ \int_{4\pi} \frac{1}{4\pi\sigma} \ud \Omega \right]\inv
  = \sigma \,.
\end{equation*}
The anisotropic diffusion tensor likewise simplifies to
\begin{equation*}
  \Dtens = \int_{4\pi} \vec{\Omega}\vec{\Omega} f \ud \Omega =
  \frac{1}{3\sigma}\Identitytens \,.
\end{equation*}
Substituting these into the \APone\ equation~\eqref{eq:ap1FicksLawFinal}, we obtain
\begin{equation*}
  \frac{1}{c}\pder{\vec{F}}{t}(\vec{x},t)
  + \sigma \left[ \frac{1}{3\sigma}\Identitytens \right] \vd \grad \phi(\vec{x}, t)
  + \sigma\vec{F}(\vec{x},t) 
  = 0 \,,
\end{equation*}
which is the conventional \Pone\ approximation:
\begin{equation*}
  \frac{1}{c}\pder{\vec{F}}{t}(\vec{x},t) + \frac{1}{3} \grad \phi
  + \sigma\vec{F}(\vec{x},t) 
  = 0\,.
\end{equation*}

%%%%%%%%%%%%%%%%%%%%%%%%%%%%%%%%%%%%%%%%%%%%%%%%%%%%%%%%%%%%%%%%%%%%%%%%%%%%%%%%
\subsection{Application to nonlinear radiation transport}

As with the anisotropic diffusion approximation in
Chapter~\ref{chap:adDerivation}, the \APone\ method was derived for a linear,
time-dependent transport problem with isotropic scattering. We again argue that
it has the proper behavior in the diffusive limit of nonlinear radiation
transport. We note that the semi-implicit linearization of the TRT
equations result in a linear transport equation with isotropic scattering, the
subject of the asymptotic analysis in this section.
Furthermore, in the diffusive limit, $\Dtens \to \frac{1}{3\sigma} +
O(\epsilon^2)$, and the \APone\ equation reduces to the \Pone\ equation. From
Ref.~\cite{Mor2000}, the \Pone\ approximation has the correct asymptotic behavior in
the diffusive limit; we therefore assert that the \APone\ approximation does
as well. 


%%%%%%%%%%%%%%%%%%%%%%%%%%%%%%%%%%%%%%%%%%%%%%%%%%%%%%%%%%%%%%%%%%%%%%%%%%%%%%%%
\section{Summary}
The anisotropic \Pone\ approximation was derived using the same procedure as 
time-dependent anisotropic diffusion. The assumption of a stronger time
dependence of the radiative intensity led to a larger unknown space---now the
zeroth \emph{and} first moments of $I$---and we therefore expect this method to
be more accurate in problems with rapid transient behavior.

