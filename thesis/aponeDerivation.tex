% !TEX root = _individual/aponeDerivation.tex

%%%%%%%%%%%%%%%%%%%%%%%%%%%%%%%%%%%%%%%%%%%%%%%%%%%%%%%%%%%%%%%%%%%%%%%%%%%%%%%%
\chapter{Advanced time-dependent anisotropic theory}

%%%%%%%%%%%%%%%%%%%%%%%%%%%%%%%%%%%%%%%%%%%%%%%%%%%%%%%%%%%%%%%%%%%%%%%%%%%%%%%%
\section{Flux-limited anisotropic diffusion}

reference to FLD chapter

\begin{equation*}
  \vec{F}^{n+1} = -\Dtens^{n}\vd \del \phi^{n+1} \times 
  \max\left( 1, \norm{\Dtens^{n}\vd \frac{\del \phi^{n}}{\phi^{n}}}
  \right)\inv
\end{equation*}


%%%%%%%%%%%%%%%%%%%%%%%%%%%%%%%%%%%%%%%%%%%%%%%%%%%%%%%%%%%%%%%%%%%%%%%%%%%%%%%%
\section{Anisotropic \Pone}
The anisotropic diffusion approximation effectively makes the same quasi-static
approximation as diffusion by making the ansatz
$\frac1c\pder{}{t}=O(\epsilon^2)$. If we instead assume that
$\frac1c\pder{}{t}=O(\epsilon)$, a `

\subsection{Approximating the initial condition term}\label{sec:derPoneIc}
The term accounting for the initial condition in
Eq.~\eqref{eq:approxIntensity1} is (see Eqs.~\eqref{eqs:inverseTransport}):
\begin{align*}
  \mathrm{ \tilde i }(\vec{x}, \vec{\Omega}, t) &=
    \lopinv{i}{I^i - \frac1{4\pi} \phi^i}
  \\
  &= \left[ I^i - \frac1{4\pi} \phi^i \right]_{( \vec{x} - ct
  \vec{\Omega}, \vec{\Omega})}
    \eexp^{ -\tau(\vec{x}, \vec{x} - ct \vec{\Omega})}
    U( \norm{\vec{x} - \vec{x}_b} - ct)\,.
\end{align*}
This describes the contribution at $(\vec{x}, \vec{\Omega}, t)$ from particles
that started at $t=0$ along the direction $\vec{\Omega}$, which have
traveled a distance $ct$ to get to the current point. The Heaviside function
$U$ is so that only particles that started in the interior are considered. The
particles are attenuated by a factor $\eexp^{ -\tau}$ along their travel, so
the contribution from this term in a non-vacuum is exponentially small at large
times.

Now we approximate the terms in the bracket, which is the evaluated nonlocal
initial condition. Let's expand the initial condition in spherical harmonics:
\begin{equation*}
  I^i(\vec{x}, \vec{\Omega}) = \frac{1}{4\pi} \phi^i(\vec{x})
  + \frac{3}{4\pi} \vec{\Omega} \vd \vec{F}^n(\vec{x}) + \varphi(\vec{x}, \vec{\Omega})\,,
\end{equation*}
where $\varphi$ contains all the spherical moments of $I^i$ that are more than
linearly anisotropic. Substituting the spherical harmonic expansion into
$\mathrm{ \tilde i }$ cancels the isotropic component of the initial condition,
leaving
\begin{align} \nonumber
  \mathrm{ \tilde i }(\vec{x}, \vec{\Omega}, t)
  &= \left[ \frac{3}{4\pi} \vec{\Omega} \vd \vec{F}^n(\vec{x})
  + \varphi(\vec{x}, \vec{\Omega}) \right]_{( \vec{x} - ct \vec{\Omega},
  \vec{\Omega})}
    \eexp^{ -\tau(\vec{x}, \vec{x} - ct \vec{\Omega})}
    U( \norm{\vec{x} - \vec{x}_b} - ct)\,.
  \\
  \intertext{Now we make the dubious but (perhaps for now) necessary
  approximation of discarding the higher-order angular moments:} \nonumber
  \mathrm{ \tilde i }(\vec{x}, \vec{\Omega}, t)
  &= \left[ \frac{3}{4\pi} \vec{\Omega} \vd \vec{F}^n(\vec{x}) \right]_{( \vec{x} - ct \vec{\Omega}, \vec{\Omega})}
    \eexp^{ -\tau(\vec{x}, \vec{x} - ct \vec{\Omega})}
    U( \norm{\vec{x} - \vec{x}_b} - ct)\,.
  \\
  \intertext{At this point, we apply the same Taylor series expansion as in
  Eq.~\eqref{eq:taylorPhi}, but this time to $\vec{F}^n$, allowing us to move
  the term outside the bracket:} \nonumber
  \mathrm{ \tilde i }(\vec{x}, \vec{\Omega}, t)
  &= \left[ \frac{3}{4\pi} \right]_{( \vec{x} - ct \vec{\Omega}, \vec{\Omega})}
    \eexp^{ -\tau(\vec{x}, \vec{x} - ct \vec{\Omega})}
    U( \norm{\vec{x} - \vec{x}_b} - ct)  \vec{\Omega} \vd \vec{F}^n(\vec{x})
  \\ \nonumber
  &= \lopinv{i}{\frac{3}{4\pi}} \vec{\Omega} \vd \vec{F}^n(\vec{x})\,.
  \\\label{eq:tildeI}
  &= g(\vec{x}, \vec{\Omega}, t) \vec{\Omega} \vd \vec{F}^n(\vec{x})\,.
\end{align}

Now we have another simple transport problem:
\begin{equation*}
  g(\vec{x}, \vec{\Omega}, t) = 
  \lopinv{b}{0} + \lopinv{i}{\frac{3}{4\pi}} + \lopinv{v}{ 0 },
\end{equation*}
which describes a time-dependent transport problem with an isotropic initial
condition but no source:
\begin{subequations} \label{eqs:gFull}
  \begin{equation} \label{eq:gFullVol}
  \frac{1}{c} \pder{g}{t}(\vec{x}, \vec{\Omega}, t)
    + \vec{\Omega}\vd \grad g(\vec{x}, \vec{\Omega}, t)
    + \sigma^\ast g (\vec{x}, \vec{\Omega}, t)
  =  0 \,, \quad x \in V,\  0 \le t < \Delta_t, \ \vec{\Omega}
  \in 4\pi\,,
  \end{equation}
with the initial condition
\begin{equation} \label{eq:gFullInit}
 g(\vec{x}, \vec{\Omega}, 0) = \frac{3}{4\pi} \,,
 \quad \vec{x} \in V, \ \vec{\Omega} \in 4\pi\,,
\end{equation}
and boundary conditions to be determined.
\end{subequations}

%%%%%%%%%%%%%%%%%%%%%%%%%%%%%%%%%%%%%%%%%%%%%%%%%%%%%%%%%%%%%%%%%%%%%%%%%%%%%%%%
\subsection{Implicit time-dependent formulation}
To take the implicit Euler approximation of Eq.~\eqref{eq:anisotropicP1}, we
average over the time step $0 < t < \Delta_t$, and approximate the
time-average unknowns by their values at $t^{n+1} = t^i + \Delta_t$. This
procedure yields an approximate but numerically convenient equation,
\begin{equation}\label{eq:implicitFlux1}
  \vec{F}^{n+1}(\vec{x}) = \mat{E}^{n+1} (\vec{x}) \vd \vec{F}^n(\vec{x})
  - \Dtens^{n+1}(\vec{x}) \vd \grad \phi^{n+1} (\vec{x}) 
\end{equation}

\begin{subequations} \label{eqs:implicitD}
The implicit approximation to $\Dtens$ is
\begin{equation}\label{eq:implicitDtens}
  \Dtens^{n+1} = \int_{4\pi} \vec{\Omega} \vec{\Omega} \:
  f^{n+1}(\vec{x}, \vec{\Omega}) \ud\Omega\,,
\end{equation}
and the transport equation~\eqref{eqs:fFull} for $f$ discretized implicitly
over time is:
\begin{equation*}
  \frac{f^{n+1} - f^{0}}{c \Delta_t}
  + \vec{\Omega} \vd \grad f^{n+1}
  + \sigma^\ast f^{n+1}
  =  \frac{1}{4\pi} \,,
\end{equation*}
with appropriate boundary conditions and the initial condition $f^{0}=0$. This
can be rearranged to the form of a steady-state transport solve, with modified
opacities:
\begin{equation} \label{eq:implicitF}
  \vec{\Omega} \vd \grad f^{n+1}
  + \left[ \sigma^\ast + \frac{1}{c \Delta_t} \right]f^{n+1}
  =  \frac{1}{4\pi} \,.
\end{equation}
This transport equation takes only one ``sweep'' to solve!
\end{subequations}

\begin{subequations} \label{eqs:implicitE}
Likewise, the implicit approximation to $\mat{E}$ is
\begin{equation}\label{eq:implicitEtens}
  \mat{E}^{n+1} = \int_{4\pi} \vec{\Omega} \vec{\Omega} \:
  g^{n+1}(\vec{x}, \vec{\Omega}) \ud\Omega\,,
\end{equation}
and the transport equation~\eqref{eqs:gFull} for $g$ discretized implicitly
over time is:
\begin{equation*}
  \frac{g^{n+1} - g^{0}}{c \Delta_t}
  + \vec{\Omega} \vd \grad g^{n+1}
  + \sigma^\ast g^{n+1}
=  0 \,,
\end{equation*}
with appropriate boundary conditions and the initial condition
$g^{0}=\frac{3}{4\pi}$. This too can be rearranged! Substituting the
isotropic initial condition $g^i$,
\begin{equation} \label{eq:implicitG}
  \vec{\Omega} \vd \grad g^{n+1}
  + \left[ \sigma^\ast + \frac{1}{c \Delta_t} \right]g^{n+1}
  = \frac{1}{c \Delta_t} \frac{3}{4\pi} \,.
\end{equation}
This transport equation \emph{also} takes only one ``sweep'' to solve, but
there is another glaringly obvious fact that will make our lives even easier:
with the implicit approximation, $g^{n+1} = \frac{3}{c \Delta t} f^{n+1}$!
\end{subequations}

Therefore, $\mat{E}^{n+1} =  \frac{3}{c \Delta t} \mat{D}^{n+1}$, so
Eq.~\eqref{eq:implicitFlux1} becomes
\begin{equation}\label{eq:implicitFlux}
  \vec{F}^{n+1}(\vec{x}) =  \frac{3}{c \Delta t} \Dtens^{n+1} (\vec{x}) \vd
  \vec{F}^n(\vec{x})
  - \Dtens^{n+1}(\vec{x}) \vd \grad \phi^{n+1} (\vec{x}) \,.
\end{equation}
Our implicit approximation comprises this approximation to the first moment,
Eq.~\eqref{eq:implicitFlux}, and the transport equation used to calculate
$\Dtens$, Eqs.~\eqref{eqs:implicitD}.

%%%%%%%%%%%%%%%%%%%%%%%%%%%%%%%%%%%%%%%%%%%%%%%%%%%%%%%%%%%%%%%%%%%%%%%%%%%%%%%%
\section{Discussion}

\subsection{Time dependence}
Starting here, we only consider the implicit Euler discretization of the
\APone\ equations. \emph{However}, the fact that we can calculate
$\mat{E}^{n+1}$ and $\Dtens^{n+1}$ however we want means that we could do crazy
stuff like, for example, using a Monte Carlo transport solve to yield
coefficients that take into account the finite speed of the particles, yielding
true time-averaged values for the two tensors, and possibly obviating the error
inherent in implicit time discretization.

%%%%%%%%%%%%%%%%%%%%%%%%%%%%%%%%%%%%%%%%%%%%%%%%%%%%%%%%%%%%%%%%%%%%%%%%%%%%%%%%
\subsubsection{Limit for large time steps}
As $\Delta_t\to \infty$, the modified opacity in Eq.~\eqref{eq:implicitF}
approaches the true space-dependent opacity $\sigma^\ast$. Thus, $\Dtens$
approaches the same $\Dtens$ as the standard anisotropic diffusion method. In
an infinite homogeneous medium, $f\to 1/4\pi \sigma$ so $\Dtens \to
\mat{I}/3\sigma$, the standard diffusion result.

For these large time steps, the contribution of the initial condition
$\vec{F}^n$ should approach zero, so $\mat{E}^{n+1}$ should approach $
\mat{0}$. We find that it does, because the effect of the initial condition
in Eq.~\eqref{eq:implicitG} diminishes exponentially for large time steps. Thus,
for $c \Delta_t \gg \sigma^\ast$, $g^{n+1}=0$ and therefore
$\mat{E}^{n+1}=\mat{0}$.

Therefore, in the limit of large time steps, the \APone\ method limits to the
standard AD method. This is good!

\paragraph{Note} What makes a time step ``large'' is now a non-local factor,
not a local factor as in \Pone.

%%%%%%%%%%%%%%%%%%%%%%%%%%%%%%%%%%%%%%%%%%%%%%%%%%%%%%%%%%%%%%%%%%%%%%%%%%%%%%%%
\subsubsection{Limit for small time steps}
TODO the same analysis makes it clear that as $\Delta_t \to 0$,
$\Dtens^{n+1}$ approaches $\mat{0}$ and $\mat{E}^{n+1}$ approaches $\mat{I}$,
yielding
\begin{equation*}
  \vec{F}^{n+1} \approx \vec{F}^{n}
\end{equation*}
as $\Delta_t \to 0$.

%%%%%%%%%%%%%%%%%%%%%%%%%%%%%%%%%%%%%%%%%%%%%%%%%%%%%%%%%%%%%%%%%%%%%%%%%%%%%%%%
\subsubsection{Infinite homogeneous medium case}
In an infinite homogeneous medium, Eq.~\eqref{eq:implicitFlux} limits to a
form, obviating the error inherent in implicit discretizations.
of the implicitly time-discretized \Pone\ equation.

The solution to Eq.~\eqref{eq:implicitF} is
\begin{equation*}
  f^{n+1}
  = \frac{1}{4\pi} \frac{1}{\sigma^\ast + 1 /c \Delta_t}
  = \frac{1}{4\pi} \frac{\sigma^\ast}{\sigma^\ast} \frac{c \Delta t}{1 + \sigma^\ast c \Delta_t}
  = \frac{1}{4\pi \sigma} ( 1 - \alpha^n) \,,
\end{equation*}
where $\alpha^n = 1/ \sigma^\ast c \Delta_t$. (This has a physical
interpretation, because $\sigma^\ast c \Delta_t$ is the number of mean free
paths a photon travels inside a time step.) Equation~\eqref{eq:implicitDtens}
then gives
\begin{equation*}
  \Dtens = \frac{1}{3 \sigma} ( 1 - \alpha^n) \mat{I}\,.
\end{equation*}

The solution to Eq.~\eqref{eq:implicitG} is
\begin{equation*}
  g^{n+1}
  = \frac{1}{c \Delta_t} \frac{3}{4\pi} \frac{1}{\sigma^\ast + 1 /c \Delta_t}
  = \frac{3}{4\pi} \frac{1}{1 + \sigma^\ast c \Delta_t}
  = \frac{3}{4\pi} \alpha^n \,.
\end{equation*}
Equation~\eqref{eq:implicitEtens} then gives
\begin{equation*}
  \mat{E} = \alpha^n \mat{I}\,.
\end{equation*}

Substituting $\Dtens$ and $\mat{E}$ into Eq.~\eqref{eq:implicitFlux}, we get the
infinite-medium solution
\begin{align*}
  \vec{F}^{n+1}(\vec{x})
  &= \alpha^n \mat{I} \vd \vec{F}^n(\vec{x})
  - \frac{1}{3 \sigma} ( 1 - \alpha^n) \mat{I} \vd \grad \phi^{n+1} (\vec{x}) 
  \\
  &= \alpha^n \vec{F}^n(\vec{x})
  - \frac{1}{3 \sigma} ( 1 - \alpha^n) \grad \phi^{n+1} (\vec{x})\,.
\end{align*}

We can reshape the \Pone\ equations into this exact form:
\begin{align*}
  \frac{\vec{F}^{n+1} - \vec{F}^n}{c \Delta_t} + \frac{1}{3} \grad \phi^{n+1}
  + \sigma^\ast \vec{F}^{n+1} &= 0
  \\
  \vec{F}^{n+1} - \vec{F}^n 
  + \sigma^\ast c \Delta_t \vec{F}^{n+1}
  &= - c \Delta_t \frac{1}{3} \grad \phi^{n+1}
  \\
  \left[ 1 + \sigma^\ast c \Delta_t \right] \vec{F}^{n+1}
  &= \vec{F}^n - \sigma^\ast c \Delta_t \frac{1}{3\sigma^\ast} \grad \phi^{n+1}
  \\
  \vec{F}^{n+1}
  &= \frac{1}{1 + \sigma^\ast c \Delta_t} \vec{F}^n - \frac{\sigma^\ast c
  \Delta_t}{1 + \sigma^\ast c \Delta_t} \frac{1}{3\sigma^\ast} \grad \phi^{n+1}
  \\
  \vec{F}^{n+1}
  &= \alpha^n \vec{F}^n - (1 - \alpha^n) \frac{1}{3\sigma^\ast} \grad \phi^{n+1}
\end{align*}

Ta-da! Our approximation of the previous time step in \S\ref{sec:derIc} as
being linear in angle is really what gives the undesirable $\frac13\grad\phi$
(which has an incorrect wave speed). It may be possible to either improve that
approximation by supposing $f$ for the previous time step is like $f$ from the
current time step (and same with $g$), which would definitely not be the case
if $\Delta_t$ is changing from one time step to the next.

