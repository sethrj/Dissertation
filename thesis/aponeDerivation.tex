% !TEX root = _individual/aponeDerivation.tex

%%%%%%%%%%%%%%%%%%%%%%%%%%%%%%%%%%%%%%%%%%%%%%%%%%%%%%%%%%%%%%%%%%%%%%%%%%%%%%%%
\chapter{Anisotropic P$_1$}\label{chap:aponeDerivation}

%%%%%%%%%%%%%%%%%%%%%%%%%%%%%%%%%%%%%%%%%%%%%%%%%%%%%%%%%%%%%%%%%%%%%%%%%%%%%%%%
\section{Derivation}

The anisotropic diffusion approximation effectively makes the same quasi-static
approximation as diffusion by making the ansatz
$\frac1c\pder{}{t}=O(\epsilon^2)$. If we follow the derivation of the AD
approximations but instead assume that $\frac1c\pder{}{t}=O(\epsilon)$, we will
arrive at a different time-dependent expression for $\Psi$ that
should better account for problems that vary less slowly in time.
The new approximation that results effectively expands the low order phase
space to include not just $\phi$ (the zeroth angular moment of $I$) but also
$\vec{F}$ (the first angular moment of $I$). We will refer to this new method
as ``anisotropic \Pone'' because of the analogy to the time-dependent \Pone\
equations, which uses the same unknowns $\phi$ and $\vec{F}$.

%%%%%%%%%%%%%%%%%%%%%%%%%%%%%%%%%%%%%%%%%%%%%%%%%%%%%%%%%%%%%%%%%%%%%%%%%%%%%%%%
\subsection{Interior and boundary layer transport equations}

We begin with the transport equation for $\Psi$ derived in \S\ref{sec:capPsi},
which is the exact transport equation with the isotropic component $\phi/4\pi$
subtracted. 
The first step is to separate $\Psi$ into an internal solution $\tilde\Psi$ and
a boundary layer solution
$\Psi_\mathrm{bl}$
\begin{equation} \label{eq:ap1boundaryLayerPsi}
  \Psi(\vec{x}, \vec{\Omega}, t)
  = \tilde\Psi(\vec{x}, \vec{\Omega}, t)
  + \Psi_\mathrm{bl}(\vec{x}, \vec{\Omega}, t)\,.
\end{equation}

\begin{subequations} \label{eqs:ap1tCapPsi}
The internal transport equation accounts for the source term:
\begin{multline} \label{eq:ap1tCapPsiVol}
  \frac{1}{c} \pder{}{t}\tilde\Psi(\vec{x}, \vec{\Omega}, t)
    + \vec{\Omega}\vd \grad \tilde\Psi(\vec{x}, \vec{\Omega}, t)
    + \sigmast(\vec{x}) \tilde\Psi(\vec{x}, \vec{\Omega}, t)
  \\
  = \frac{1}{4\pi} \grad \vd\vec{F}(\vec{x}, t) -
  \frac{1}{4\pi} \vec{\Omega}\vd \grad \phi(\vec{x}, t)\,,
  \qquad
x \in V,\  0 \le t \le \Delta_t, \ \vec{\Omega} \in 4\pi.
\end{multline}
Its boundary condition, however, will take the form
\begin{equation} \label{eq:ap1tCapPsiBndy}
 \tilde\Psi(\vec{x}, \vec{\Omega}, t) 
  = - \zeta(\vec{x}, \vec{\Omega}, t) \vec{\Omega}\vd \grad \phi(\vec{x}, t)
  + \eta(\vec{x}, \vec{\Omega}, t) \vec{\Omega}\vd \vec{F}^i(\vec{x})
  \equiv \tilde\Psi^b(\vec{x}, \vec{\Omega}, t) \,,
\end{equation}
for $\vec{x} \in \partial V_b$, $\vec{\Omega} \vd \vec{n} < 0$,
$0 \le t \le \Delta_t$. Here, the functions $\zeta$ and $\eta$, which live on
the boundary for incident directions, are yet to be determined. The boundary
condition also uses the first moment of the initial condition:
\begin{equation} \label{eq:fluxInit}
  \vec{F}^i(\vec{x}) \equiv \int_{4\pi} \vec{\Omega} I^i(\vec{x}, \vec{\Omega}) \ud \Omega\,.
\end{equation}
On reflecting boundaries, the internal solution is reflecting:
\begin{equation} \label{eq:ap1tCapPsiRefl}
 \tilde\Psi(\vec{x}, \vec{\Omega}, t) 
  = \tilde\Psi(\vec{x}, \vec{\Omega}_r, t)
  \equiv \tilde\Psi^b(\vec{x}, \vec{\Omega}, t) \,,
\end{equation}
for $\vec{x} \in \partial V_r$, $\vec{\Omega} \vd \vec{n} < 0$,
$0 \le t \le \Delta_t$.
Finally, the internal solution contains the same initial condition as
Eq.~\eqref{eq:capPsiInit}:
\begin{equation} \label{eq:ap1tCapPsiInit}
 \tilde\Psi(\vec{x}, \vec{\Omega}, 0)
 = \Psi^i(\vec{x}, \vec{\Omega}, t)\,.
\end{equation}
\end{subequations}

The corresponding transport problem for $\Psi_\mathrm{bl}$ is defined to
satisfy the transport equations for $\Psi$ using the definition in
Eq.~\eqref{eq:ap1boundaryLayerPsi}.
\begin{subequations} \label{eqs:ap1blCapPsi}
It has the same left-hand side as Eq.~\eqref{eq:capPsiBndy} but no internal
source:
\begin{equation} \label{eq:ap1blCapPsiVol}
  \frac{1}{c} \pder{}{t}\Psi_\mathrm{bl}(\vec{x}, \vec{\Omega}, t)
    + \vec{\Omega}\vd \grad \Psi_\mathrm{bl}(\vec{x}, \vec{\Omega}, t)
    + \sigmast(\vec{x}) \Psi_\mathrm{bl}(\vec{x}, \vec{\Omega}, t)
  = 0\,, \quad
x \in V,\  0 \le t \le \Delta_t, \ \vec{\Omega} \in 4\pi.
\end{equation}
The incident boundary condition accounts for the true incident boundary source
as well as the $\zeta$ term we introduced:
\begin{equation} \label{eq:ap1blCapPsiBndy}
 \Psi_\mathrm{bl}(\vec{x}, \vec{\Omega}, t) 
  = I^b(\vec{x}, \vec{\Omega}, t) - \frac{1}{4\pi} \phi(\vec{x}, t)
  + \zeta(\vec{x}, \vec{\Omega}, t) \vec{\Omega}\vd \grad \phi(\vec{x}, t)
  - \eta(\vec{x}, \vec{\Omega}, t) \vec{\Omega}\vd \vec{F}^i(\vec{x})
  \equiv \Psi_\mathrm{bl}^b(\vec{x}, \vec{\Omega}, t) \,.
\end{equation}
For $\vec{x} \in \partial V_r$, the boundary layer solution is reflecting:
\begin{equation} \label{eq:ap1blCapPsiRefl}
 \Psi_\mathrm{bl}(\vec{x}, \vec{\Omega}, t) 
  = \Psi_\mathrm{bl}(\vec{x}, \vec{\Omega}_r, t)\,.
\end{equation}
Finally, because $\tilde\Psi$ accounts for the initial condition, the initial
condition for $\Psi_\mathrm{bl}$ is zero:
Eq.~\eqref{eq:capPsiInit}:
\begin{equation} \label{eq:ap1blCapPsiInit}
 \Psi_\mathrm{bl}(\vec{x}, \vec{\Omega}, 0)
 = 0\,.
\end{equation}
\end{subequations}

If we add Eqs.~\eqref{eqs:ap1blCapPsi} to Eqs.~\eqref{eqs:tCapPsi}, we recover
the original transport equation for $\Psi$. The only difference from the
AD derivation is in the form of the boundary condition, which is modified in
anticipation of a different approximation to $\tilde\Psi$ that depends on
$\vec{F}^i$.

As was done in Eqs.~\eqref{eqs:inverseTransport}, the interior transport
equation for $\tilde\Psi$ is expressed in an integral form, where we use the
same linear operators $\lopinv{}{\cdot}$ as in
Eq.~\eqref{eq:inverseTransportBrief}:
\begin{align}\label{eq:ap1inverseTransportBrief}
  \begin{split}
    \tilde\Psi(\vec{x}, \vec{\Omega}, t)
    &\equiv
    -\lopinv{b}{\zeta \vec{\Omega}\vd \grad \phi}_{\partial V_b}
    +\lopinv{b}{\eta \vec{\Omega}\vd \vec{F}^i}_{\partial V_b}
    + \lopinv{b}{\tilde\Psi(\vec{x}, \vec{\Omega}_r, t)}_{\partial V_r}
  \\&\qquad
    + \lopinv{i}{\Psi^i}
    + \lopinv{v}{\frac{1}{4\pi} \grad \vd\vec{F} }
    - \lopinv{v}{\frac{1}{4\pi} \vec{\Omega}\vd \grad \phi}
    \,.
  \end{split}
\end{align}

%%%%%%%%%%%%%%%%%%%%%%%%%%%%%%%%%%%%%%%%%%%%%%%%%%%%%%%%%%%%%%%%%%%%%%%%%%%%%%%%
\subsection{Asymptotic scaling}

The crucial difference between \APone\ and AD is how the derivatives of $I$ are
scaled.
As with the AD approximation, we make an ansatz that the spatial gradients of
the intensity are weak, the intensity varies slowly in time, and the solution is
mildly (but not necessarily linearly) anisotropic:
\begin{align} \label{eq:ap1ansatz}
  I &= O(1), &
  \grad I &= O(\epsilon), &
  \frac{1}{c}\pder{I}{t} &= O(\epsilon), &
  \int_{4\pi} \vec{\Omega} I\ud\Omega &= O(\epsilon).
\end{align}
Note here that the time derivative is scaled as $O(\epsilon)$, not
$O(\epsilon^2)$ as with anisotropic diffusion. The consequence is that the
contribution from the initial condition is \emph{not} $O(\epsilon^2)$.

Like before, we discard the $O(\epsilon^2)$ term $\grad \vd\vec{F}$ that appears in
Eq.~\eqref{eq:ap1inverseTransportBrief}. Unlike before, the $\Psi^i$ term
remains.
\begin{equation} \label{eq:ap1approxPsi1}
  \tilde\Psi \approx 
  -\lopinv{b}{\zeta \vec{\Omega}\vd \grad \phi}_{\partial V_b}
  + \lopinv{b}{\eta \vec{\Omega}\vd \vec{F}^i}_{\partial V_b}
  + \lopinv{b}{\tilde\Psi(\vec{\Omega}_r)}_{\partial V_r}
  + \lopinv{i}{\Psi^i}
  - \lopinv{v}{\frac{1}{4\pi} \vec{\Omega}\vd \grad \phi}
  + O(\epsilon^2)
\end{equation}

The nonlocal variables buried in Eq.~\eqref{eq:approxPsi1} can still be expanded
about the local spatiotemporal point, because both gradients are $O(\epsilon)$:
\begin{equation} \label{eq:ap1taylorPhi}
  \phi(\vec{x} - s \vec{\Omega}, t-s/c)
  \sim \phi(\vec{x},t) - s \left( \frac{1}{c} \pder{}{t} + \vec{\Omega} \vd
  \grad  \right) \phi (\vec{x}, t) + O(\epsilon^2) \sim \phi(\vec{x},t) +
  O(\epsilon) \,.
\end{equation}

The volumetric source term is identical to that in anisotropic diffusion and was
expanded in \S\ref{sec:adStreaming}:
\begin{equation*}
  - \lopinv{v}{\frac{1}{4\pi} \vec{\Omega}\vd \grad \phi}
  \approx  - \lopinv{v}{\frac{1}{4\pi}} \vec{\Omega}\vd \grad
  \phi(\vec{x},t) \,.
\end{equation*}

As in \S\ref{sec:adIncident}, we expand $\phi$ in the boundary term:
\begin{equation*}
-\lopinv{b}{\zeta \vec{\Omega}\vd \grad \phi}_{\partial V_b}
  \approx  - \lopinv{v}{\zeta} \vec{\Omega}\vd \grad
  \phi(\vec{x},t) \,.
\end{equation*}
For the other component of the boundary term, we expand the $O(\epsilon)$
initial radiation flux $\vec{F}^i$ about $(\vec{x})$:
\begin{align*} \nonumber
\lopinv{b}{\eta \vec{\Omega}\vd \vec{F}^i}_{\partial V_b}
&= \left[\eta(\vec{x} - s_b\vec{\Omega}, \vec{\Omega}, t - s_b/c)
    \vec{\Omega}\vd \vec{F}^i(\vec{x} - s_b\vec{\Omega}) \right]
    \eexp^{ -\tau(\vec{x}, \vec{x} - s_b \vec{\Omega})} U(ct - s_b)
    \\
&\approx \left\{\eta(\vec{x} - s_b\vec{\Omega}, \vec{\Omega}, t - s_b/c)
\vec{\Omega}\vd \left[
\vec{F}^i(\vec{x}) + O(\epsilon^2)\right] \right\}
  \eexp^{ -\tau(\vec{x}, \vec{x} - s_b \vec{\Omega})} U(ct - s_b)\,,
\end{align*}
or more simply,
\begin{equation*}
\lopinv{b}{\eta \vec{\Omega}\vd \vec{F}^i}_{\partial V_b}
  \approx \lopinv{b}{\eta}_{\partial V_b}
  \vec{\Omega}\vd \vec{F}^i(\vec{x}) \,.
\end{equation*}

%%%%%%%%%%%%%%%%%%%%%%%%%%%%%%%%%%%%%%%%%%%%%%%%%%%%%%%%%%%%%%%%%%%%%%%%%%%%%%%%
\subsection{Initial condition treatment}\label{sec:ap1ic}
The integral equation~\eqref{eq:ap1approxPsi1} in the \APone\ derivation
includes the contribution from the initial condition $\Psi^i$, which was not
present in AD because of the assumption that the time dependence was
$O(\epsilon^2)$:
\begin{equation*}
  \lopinv{i}{\Psi^i} = \Psi^i( \vec{x} - ct \vec{\Omega}, \vec{\Omega})
    \eexp^{ -\tau(\vec{x}, \vec{x} - ct \vec{\Omega})} U( s_b - ct)
\end{equation*}
This is the contribution to $\Psi$ of uncollided particles that started
inside the problem at $t=0$. As $t\to\infty$, as in the case of a steady-state
problem, this term vanishes.

The difficulty of treating this term is that the initial condition $\Psi^i
\equiv I^i - \frac{1}{4\pi}\phi^i$ is an arbitrary function of both space and
angle, and in its present form it would have to be fully stored in computer
memory.  The point of developing the anisotropic approximations is to avoid the
expense of calculating and retaining complex functions of space and angle, hence
the restriction of the unknowns in the AD method to the low-order phase space
$\phi(\vec{x})$. With the anisotropic \Pone\ approximation, we expand the
low-order unknowns to include not just the zeroth moment $\phi$ of the intensity
$I$ but also the first moment $\vec{F}$.  Therefore, we desire an approximate
expression for $\Psi^i$ as a function of $\phi^i$ and $\vec{F}^i$.

The most straightforward way to represent $\Psi^i$ is using the \Pone\
approximation that the initial condition is linear in angle:
\begin{equation*}
  \Psi^i(\vec{x},\vec{\Omega})
  \approx \frac{3}{4\pi} \vec{\Omega} \vd \vec{F}^i(\vec{x}) \,.
\end{equation*}
We can now expand the nonlocal initial condition about the local point,
\begin{align*}
  \Psi^i( \vec{x} - ct \vec{\Omega}, \vec{\Omega})
  & \approx  \frac{3}{4\pi} \vec{\Omega} \vd \vec{F}^i( \vec{x} - ct \vec{\Omega})
  \\
  &\sim \frac{3}{4\pi} \vec{\Omega} \vd \left[ 
  \vec{F}^i( \vec{x} ) + O(\epsilon^2) \right]
  \\
  &= \frac{3}{4\pi} \vec{\Omega} \vd \vec{F}^i( \vec{x} ) \,,
\end{align*}
giving the approximation
\begin{equation*}
  \lopinv{i}{\Psi^i} \approx \lopinv{i}{\frac{3}{4\pi}} \vec{\Omega} \vd
  \vec{F}^i( \vec{x} ) \,.
\end{equation*}

We have now reduced Eq.~\eqref{eq:ap1approxPsi1} to
\begin{multline} \label{eq:ap1approxPsi2}
  \tilde\Psi(\vec{x},\vec{\Omega},t)
  \approx 
  - \lopinv{v}{\frac{1}{4\pi}} \vec{\Omega}\vd \grad \phi(\vec{x},t)
  + \lopinv{i}{\frac{3}{4\pi}} \vec{\Omega} \vd \vec{F}^i( \vec{x} )
  \\
  - \lopinv{b}{\zeta}_{\partial V_b} \vec{\Omega}\vd \grad \phi(\vec{x},t)
  + \lopinv{b}{\eta}_{\partial V_b} \vec{\Omega}\vd \vec{F}^i(\vec{x})
  + \lopinv{b}{\tilde\Psi(\vec{\Omega}_r)}_{\partial V_r}
\end{multline}

%%%%%%%%%%%%%%%%%%%%%%%%%%%%%%%%%%%%%%%%%%%%%%%%%%%%%%%%%%%%%%%%%%%%%%%%%%%%%%%%
\subsection{Reflecting boundary condition}
Ignoring the reflecting boundary condition in Eq.~\eqref{eq:ap1approxPsi2}
allows the equation's reformulation as
\begin{equation*}
  \tilde\Psi(\vec{x},\vec{\Omega},t)
  \approx 
 - f(\vec{x},\vec{\Omega},t) \vec{\Omega}\vd \grad \phi(\vec{x},t)
 + g(\vec{x},\vec{\Omega},t) \vec{\Omega}\vd \vec{F}^i(\vec{x})\,.
\end{equation*}
We suppose for a moment that the reflecting boundary term can be approximated
by a similar form, embedding it into the $f$ and $g$ expressions.

As with the derivation for the anisotropic diffusion method, we desire the
approximated reflecting boundary condition to be embedded in $f$ and $g$. To
approximate the reflecting boundary condition term $\lopinv{b}{\cdot}_{\partial
V_r}$, which describes uncollided particles reflected from the boundary
during a time step, we follow a procedure identical to that performed in
\S\ref{sec:derReflBc}.  It begins by substituting the approximate $\tilde\Psi$
into the reflecting boundary term, and writing $\vec{\Omega}_r$ using
Eq.~\eqref{eq:reflection}. 
\begin{align*}
  \lefteqn{\lopinv{b}{\tilde\Psi(\vec{\Omega}_r)}_{\partial V_r}}\qquad&
  \\
  &= \left[\tilde\Psi(\vec{x} - s_b\vec{\Omega}, \vec{\Omega}_r, t - s_b/c)
  \right]
    \eexp^{ -\tau(\vec{x}, \vec{x} - s_b \vec{\Omega})}
    U(ct - s_b)
  \\
  &=\lopinv{b}{ - f(\vec{x} - s_b\vec{\Omega},\vec{\Omega}_r,t - s_b/c)
  \vec{\Omega}_r\vd \grad \phi(\vec{x} - s_b\vec{\Omega}, t - s_b/c)}
  \\
  &\qquad + \lopinv{b}{ g(\vec{x} - s_b\vec{\Omega}, \vec{\Omega}_r, t - s_b/c)
  \vec{\Omega}_r\vd \vec{F}^i(\vec{x} - s_b\vec{\Omega})}
  \\
  &= - \lopinv{b}{ f(\vec{x}_b,\vec{\Omega}_r,t_b)
  \vec{\Omega}\vd \grad \phi(\vec{x}_b, t_b)}
   + \lopinv{b}{ g(\vec{x}_b, \vec{\Omega}_r, t_b)
  \vec{\Omega}\vd \vec{F}^i(\vec{x}_b)}
  \\
  &\qquad + \lopinv{b}{ f(\vec{x}_b,\vec{\Omega}_r,t_b)
  2(\vec{\Omega} \vd \vec{n}) \vec{n} \vd \grad \phi(\vec{x}_b, t_b)}
   - \lopinv{b}{ g(\vec{x}_b, \vec{\Omega}_r, t_b)
  2(\vec{\Omega} \vd \vec{n}) \vec{n} \vd \vec{F}^i(\vec{x}_b)}\,.
\end{align*}
Because $\vec{n} \vd \grad \phi = \vec{n} \vd \vec{F}^i = 0$ on a
reflecting boundary, the second pair of terms disappear, leaving
\begin{equation*}
\lopinv{b}{\tilde\Psi(\vec{\Omega}_r)}_{\partial V_r}
  \approx - \lopinv{b}{ f(\vec{x}_b,\vec{\Omega}_r,t_b)
  \vec{\Omega}\vd \grad \phi(\vec{x}_b, t_b)}
   + \lopinv{b}{ g(\vec{x}_b, \vec{\Omega}_r, t_b)
  \vec{\Omega}\vd \vec{F}^i(\vec{x}_b)}
\end{equation*}
Expanding $\phi(\vec{x}_b, t_b)$ about $\phi(\vec{x}, t)$ and
$\vec{F}^i(\vec{x}_b)$ about $\vec{F}^i(\vec{x})$ allows this to be
approximated to $O(\epsilon^2)$ as
\begin{equation*}
  \lopinv{b}{\tilde\Psi(\vec{\Omega}_r)}_{\partial V_r}
  \approx
  - \lopinv{b}{ f(\vec{x}_b,\vec{\Omega}_r,t_b)} \vec{\Omega}\vd \grad
  \phi(\vec{x}, t)
  + \lopinv{b}{ g(\vec{x}_b, \vec{\Omega}_r, t_b)}
  \vec{\Omega}\vd \vec{F}^i(\vec{x})\,,
\end{equation*}
or, in the more simplified form,
\begin{equation} \label{eq:ap1reflApprox}
\lopinv{b}{\tilde\Psi(\vec{\Omega}_r)}_{\partial V_r}
\approx  
- \lopinv{b}{f(\vec{\Omega}_r)}_{\partial V_r}
\vec{\Omega} \vd \grad \phi(\vec{x}, t)
+ \lopinv{b}{g(\vec{\Omega}_r)}_{\partial V_r}
\vec{\Omega}\vd \vec{F}^i(\vec{x})
\,.
\end{equation}

%%%%%%%%%%%%%%%%%%%%%%%%%%%%%%%%%%%%%%%%
\subsection{Anisotropic \Pone\ approximation to \texorpdfstring{$\Psi$}{Psi}}

Now we have approximations for the boundary terms, the initial condition, and
the volumetric terms that allow $I-\frac{1}{4\pi}\phi$ to be expressed in terms
of local values of $\phi$ and $\vec{F}^i$.
\begin{align} \nonumber
\begin{split}
 \tilde\Psi
 &\approx 
 - \lopinv{v}{\frac{1}{4\pi}} \vec{\Omega}\vd \grad \phi
 + \lopinv{i}{\frac{3}{4\pi}} \vec{\Omega}\vd \vec{F}^i
 - \lopinv{b}{\zeta}_{\partial V_b} \vec{\Omega}\vd \grad \phi
 + \lopinv{b}{\eta}_{\partial V_b} \vec{\Omega}\vd \vec{F}^i
 \\&\qquad
 - \lopinv{b}{f(\vec{\Omega}_r)}_{\partial V_r} \vec{\Omega}\vd \grad \phi
 + \lopinv{b}{g(\vec{\Omega}_r)}_{\partial V_r} \vec{\Omega}\vd \vec{F}^i
\end{split}
\\
\begin{split}
 \tilde\Psi &= 
- \left\{
  \lopinv{b}{\zeta}_{\partial V_b} 
+ \lopinv{b}{f(\vec{\Omega}_r)}_{\partial V_r}
+ \lopinv{v}{\frac{1}{4\pi}}
\right\} \vec{\Omega}\vd \grad \phi
 \\&\qquad
+ \left\{
  \lopinv{b}{\eta}_{\partial V_b} 
+ \lopinv{b}{g(\vec{\Omega}_r)}_{\partial V_r}
+ \lopinv{i}{\frac{3}{4\pi}}
\right\} \vec{\Omega}\vd \vec{F}^i(\vec{x})
\end{split}
\\ \label{eq:ap1approxPsi3}
\tilde\Psi(\vec{x}, \vec{\Omega}, t)
&= - f(\vec{x}, \vec{\Omega}, t) \vec{\Omega}\vd \grad \phi(\vec{x}, t)
+ g(\vec{x}, \vec{\Omega}, t) \vec{\Omega}\vd \vec{F}^i(\vec{x})
\end{align}

Here, we have defined
\begin{equation*}
  f(\vec{x}, \vec{\Omega})
  \equiv
  \lopinv{b}{\zeta}_{\partial V_b} 
+ \lopinv{b}{f(\vec{\Omega}_r)}_{\partial V_r}
+ \lopinv{i}{0}
+ \lopinv{v}{\frac{1}{4\pi}}\,,
\end{equation*}
which is the solution to a time-dependent integral transport equation (see
Eqs.~\eqref{eqs:inverseTransport}). It is very similar to the anisotropic
diffusion formulation for $f$ in Eqs.~\eqref{eqs:fFull}, except that now $f$
varies in time. The differential transport equation for this modified $f$ is
\begin{subequations} \label{eqs:ap1fFull}
  \begin{equation} \label{eq:ap1fFullVol}
    \pder{f}{t}(\vec{x}, \vec{\Omega},t)
    + \vec{\Omega}\vd \grad f(\vec{x}, \vec{\Omega},t)
    + \sigmast f(\vec{x}, \vec{\Omega},t)
  = \frac{1}{4\pi} \,,
  \qquad x \in V,\ \vec{\Omega} \in 4\pi,\  0 \le t \le \Delta_t\,.
\end{equation}
It has boundary and initial conditions:
\begin{alignat}{2} \label{eq:ap1fFullBndy}
  f(\vec{x}, \vec{\Omega}, t) &= \zeta(\vec{x}, \vec{\Omega}, t) \,,
 \qquad && \vec{x} \in \partial V_b, \ \vec{\Omega} \vd \vec{n} < 0
 ,\  0 \le t \le \Delta_t\,,
\\ \label{eq:ap1fFullRefl}
  f(\vec{x}, \vec{\Omega}) &= f(\vec{x}, \vec{\Omega}_r) \,,
 && \vec{x} \in \partial V_r, \ \vec{\Omega} \vd \vec{n} < 0
 ,\  0 \le t \le \Delta_t\,,
\\ \label{eq:ap1fFullInit}
  f(\vec{x}, \vec{\Omega}, 0) &= 0 \,,
 && \vec{x} \in V, \ \vec{\Omega}\in 4\pi\,.
\end{alignat}
\end{subequations}

In Eq.~\eqref{eq:ap1approxPsi3} we also defined
\begin{equation*}
  g(\vec{x}, \vec{\Omega})
  \equiv
  \lopinv{b}{\eta}_{\partial V_b} 
+ \lopinv{b}{g(\vec{\Omega}_r)}_{\partial V_r}
+ \lopinv{i}{\frac{3}{4\pi}}
+ \lopinv{v}{0}\,,
\end{equation*}
which is a slightly different time-dependent integral transport equation.
Here, $g$ is the solution of a purely absorbing
transport equation with a uniform, isotropic initial condition but without any
volumetric source:
\begin{subequations} \label{eqs:ap1gFull}
  \begin{equation} \label{eq:ap1gFullVol}
    \pder{g}{t}(\vec{x}, \vec{\Omega},t)
    + \vec{\Omega}\vd \grad g(\vec{x}, \vec{\Omega},t)
    + \sigmast g(\vec{x}, \vec{\Omega},t)
  = 0 \,,
  \qquad x \in V,\ \vec{\Omega} \in 4\pi,\  0 \le t \le \Delta_t\,,
\end{equation}
with boundary and initial conditions
\begin{alignat}{2} \label{eq:ap1gFullBndy}
  g(\vec{x}, \vec{\Omega}, t) &= \eta(\vec{x}, \vec{\Omega}, t) \,,
 \qquad && \vec{x} \in \partial V_b, \ \vec{\Omega} \vd \vec{n} < 0
 ,\  0 \le t \le \Delta_t\,,
\\ \label{eq:ap1gFullRefl}
  g(\vec{x}, \vec{\Omega}) &= g(\vec{x}, \vec{\Omega}_r) \,,
  && \vec{x} \in \partial V_r, \ \vec{\Omega} \vd \vec{n} < 0
 ,\  0 \le t \le \Delta_t\,,
\\ \label{eq:ap1gFullInit}
g(\vec{x}, \vec{\Omega}, 0) &= \frac{3}{4\pi} \,,
  && \vec{x} \in V, \ \vec{\Omega}\in 4\pi\,.
\end{alignat}
\end{subequations}

%%%%%%%%%%%%%%%%%%%%%%%%%%%%%%%%%%%%%%%%
\subsection{Approximate radiation flux}
Now we have an equation for $\Psi(\vec{x}, \vec{\Omega}, t)$ as a
function of
\begin{itemize}
  \item the simple transport equations $f(\vec{x}, \vec{\Omega}, t)$
    and $g(\vec{x}, \vec{\Omega}, t)$,
  \item the unknown scalar intensity $\phi(\vec{x},t)$, and
  \item the initial radiation flux $\vec{F}^i(\vec{x})$.
\end{itemize}

From Eq.~\eqref{eq:capPsiFirst}, the first moment of $\Psi$ is the flux
$\vec{F}$. Operating on Eq.~\eqref{eq:ap1approxPsi3} by $\int_{4\pi}
\vec{\Omega} (\cdot) \ud\Omega$ gives an expression similar to the anisotropic
diffusion Fick's law:
\begin{align} \nonumber
  \vec{F}(\vec{x}, t)
  &= \int_{4\pi} \vec{\Omega} \tilde \Psi(\vec{x}, \vec{\Omega}, t) \ud\Omega
  \\ \nonumber
  &= 
  - \left[ \int_{4\pi} \vec{\Omega} \vec{\Omega} f(\vec{x}, \vec{\Omega}, t)
  \ud\Omega \right]
  \vd \grad \phi(\vec{x},t)
  + \left[ \int_{4\pi} \vec{\Omega} \vec{\Omega} g(\vec{x}, \vec{\Omega}, t)
  \ud\Omega \right]
  \vd \vec{F}^i(\vec{x})
  \\ \label{eq:ap1Ficks}
  &= - \Dtens(\vec{x},t) \vd \grad \phi(\vec{x},t)
  + \Etens(\vec{x},t) \vd \vec{F}^i(\vec{x})\,.
\end{align}
Here, we have defined the second moment of the time-dependent $f$ as
\begin{equation}\label{eq:ap1dDefinition}
  \Dtens(\vec{x},t) \equiv \int_{4\pi} \vec{\Omega} \vec{\Omega}
  f(\vec{x}, \vec{\Omega},t) \ud\Omega \,,
\end{equation}
and the second moment of the time-dependent $g$ as
\begin{equation}\label{eq:ap1eDefinition}
  \Etens(\vec{x},t) \equiv \int_{4\pi} \vec{\Omega} \vec{\Omega}
  g(\vec{x}, \vec{\Omega}, t) \ud\Omega \,.
\end{equation}

Unlike Fick's law, Eq.~\eqref{eq:ap1Ficks} incorporates the initial condition of
the first moment, $\vec{F}^i$, making it closer to the $P_1$ equations.

%%%%%%%%%%%%%%%%%%%%%%%%%%%%%%%%%%%%%%%%
\subsection{Incident boundary conditions}
Equations~\eqref{eqs:ap1fFull} and~\eqref{eqs:ap1gFull} do not completely
describe the \APone\ method because we have not yet used the boundary layer
equations~\eqref{eqs:ap1blCapPsi} or determined $\zeta$ and $\eta$.

To determine $\zeta$ and $\eta$, we enforce the same relation that determined
$\zeta$ in the AD derivation, Eq.~\eqref{eq:capPsiZeroth}:
\begin{align*}
  0 &= \int_{4\pi} \Psi(\vec{x}, \vec{\Omega}, t) \ud\Omega
  \\
  0 &= \int_{4\pi} \left[  - f(\vec{x}, \vec{\Omega}, t) \vec{\Omega}\vd \grad \phi(\vec{x}, t)
+ g(\vec{x}, \vec{\Omega}, t) \vec{\Omega}\vd \vec{F}^i(\vec{x})
 \right]  \ud\Omega
  \\
  0 &=
- \int_{4\pi} f(\vec{x}, \vec{\Omega}, t) \vec{\Omega} \ud\Omega
  \vd \grad \phi(\vec{x}, t)
+ \int_{4\pi} g(\vec{x}, \vec{\Omega}, t) \vec{\Omega} \ud\Omega
  \vd \vec{F}^i(\vec{x}) \,.
\end{align*}
Because we do not with $f$ and $g$ to depend on each other, on the unknown
$\phi$, or on the known $\vec{F}^i$, we take this relationship to give two
conditions on $\partial V_b$:
\begin{equation*}
  \int_{4\pi} f(\vec{x}, \vec{\Omega}, t) \vec{\Omega} \ud\Omega = 0\,,
  \qquad \text{and} \qquad
  \int_{4\pi} g(\vec{x}, \vec{\Omega}, t) \vec{\Omega} \ud\Omega = 0\,.
\end{equation*}
Substituting $\zeta$ and $\eta$ for incoming directions, these give
\begin{subequations} \label{eqs:ap1zetaCondition}
\begin{align} \label{eq:ap1zetaCondition}
- \int_{\vec{\Omega} \vd \vec{n} < 0}
 \vec{\Omega} \zeta(\vec{x}, \vec{\Omega}, t)\ud\Omega
 &=
 \int_{\vec{\Omega} \vd \vec{n} > 0}
  \vec{\Omega} f(\vec{x}, \vec{\Omega}, t)\ud\Omega \,,
\\ \label{eq:ap1etaCondition}
- \int_{\vec{\Omega} \vd \vec{n} < 0}
 \vec{\Omega} \eta(\vec{x}, \vec{\Omega}, t)\ud\Omega
 &=
 \int_{\vec{\Omega} \vd \vec{n} > 0}
  \vec{\Omega} g(\vec{x}, \vec{\Omega}, t)\ud\Omega \,.
\end{align}
\end{subequations}

Both of these relations are identical in form to the constraint on $f$,
Eq.~\eqref{eq:zetaCondition1}. We shall simply restate the result of the
constraint: given that $f$ (or $g$) is symmetric about $\vec{n}$, a reflecting
boundary will satisfy Eqs.~\eqref{eqs:ap1zetaCondition}, as will any other
angular distribution that preserves the first partial exiting moment.

Boundary conditions for the \APone\ approximation are formulated in the same
manner as standard diffusion and anisotropic diffusion.
Substituting Eq.~\eqref{eq:ap1blCapPsiBndy} into Eq.~\eqref{eq:bcW} gives a
condition to enforce on the boundary:
\begin{align*}
  0 &= \int_{\vec{\Omega} \vd \vec{n} < 0} W(\abs{\vec{\Omega} \vd \vec{n}})
  \Psi_\mathrm{bl} (\vec{x}, \vec{\Omega}, t) \ud\Omega
  \\
  &= \int_{\vec{\Omega} \vd \vec{n} < 0} W(\abs{\vec{\Omega} \vd \vec{n}})
 \left[ I^b(\vec{x}, \vec{\Omega}, t) - \frac{1}{4\pi} \phi(\vec{x}, t)
  + \zeta(\vec{x}, \vec{\Omega}, t) \vec{\Omega}\vd \grad \phi(\vec{x}, t)
  - \eta(\vec{x}, \vec{\Omega}, t) \vec{\Omega}\vd \vec{F}^i(\vec{x})\right]
  \ud\Omega
\end{align*}
Under the assumption that we are using reflecting boundary conditions for $f$
and $g$, we rearrange this equation and express $\zeta$ and $\eta$ in terms of
outgoing values of $f$ and $g$ (see \S\ref{sec:zeta}):
\begin{multline}\label{eq:ap1loBndy}
  2 \int_{\vec{\Omega} \vd \vec{n} < 0} W(\abs{\vec{\Omega} \vd \vec{n}})
  I^b(\vec{x}, \vec{\Omega}, t) \ud\Omega
  \\
  =
 \phi(\vec{x}, t)
  + 2\int_{\vec{\Omega}\vd \vec{n} > 0} W(\vec{\Omega} \vd \vec{n})
  \vec{\Omega} f(\vec{x}, \vec{\Omega},t) \ud\Omega
  \vd \grad \phi(\vec{x}, t) 
  \\
  - 2\int_{\vec{\Omega}\vd \vec{n} > 0} W(\vec{\Omega} \vd \vec{n})
  \vec{\Omega} g(\vec{x}, \vec{\Omega},t) \ud\Omega
  \vd \vec{F}^i(\vec{x}) 
  \,.
\end{multline}

If we use the Marshak-like approximation $W(\mu)\approx 2\mu$,
Eq.~\eqref{eq:ap1loBndy} reduces to a form similar to that constructed for
anisotropic diffusion, Eq.~\eqref{eq:marshakAd}.
\begin{align}\nonumber
  \lefteqn{2\int_{\vec{\Omega}\vd \vec{n} < 0}
  ( 2\abs{\vec{\Omega} \vd \vec{n}} ) I^b(\vec{\Omega}) \ud\Omega}\qquad
\\ \nonumber
  &= \phi
  + 2\int_{\vec{\Omega}\vd \vec{n} > 0} ( 2\vec{\Omega} \vd \vec{n} )
  \vec{\Omega} f(\vec{\Omega}) \ud\Omega \vd \grad \phi
  - 2\int_{\vec{\Omega}\vd \vec{n} > 0} ( 2\vec{\Omega} \vd \vec{n} )
  \vec{\Omega} g(\vec{\Omega}) \ud\Omega \vd \vec{F}^i
\\ \nonumber
  4 F^-
  &= \phi
  + 4 \vec{n} \vd \left[ \int_{\vec{\Omega}\vd \vec{n} > 0} \vec{\Omega}
  \vec{\Omega} f(\vec{\Omega}) \ud\Omega \right] \vd \grad \phi
  - 4 \vec{n} \vd \left[ \int_{\vec{\Omega}\vd \vec{n} > 0} \vec{\Omega}
  \vec{\Omega} f(\vec{\Omega}) \ud\Omega \right] \vd \vec{F}^i
\\ \nonumber
  4 F^-
  &= \phi
  + 2 \vec{n} \vd \int_{4\pi} \vec{\Omega} \vec{\Omega} f \ud\Omega \vd \grad
  \phi
  - 2 \vec{n} \vd \int_{4\pi} \vec{\Omega} \vec{\Omega} g \ud\Omega \vd \grad
  \phi
\\ \label{eq:ap1marshak}
  4 F^-(\vec{x}, t)
  &= \phi(\vec{x}, t)
  + 2 \vec{n} \vd \Dtens(\vec{x},t) \vd \grad \phi(\vec{x}, t)
  - 2 \vec{n} \vd \Etens(\vec{x},t) \vd \vec{F}^i(\vec{x}) \,.
\end{align}
The Marshak form can also be written by substituting $\vec{F}$ from
Eq.~\eqref{eq:ap1Ficks}:
\begin{equation} \label{eq:ap1marshak2}
  4 F^-(\vec{x}, t)
  = \phi(\vec{x}, t)
  - 2 \vec{n} \vd \vec{F}(\vec{x}, t) \,.
\end{equation}

%%%%%%%%%%%%%%%%%%%%%%%%%%%%%%%%%%%%%%%%
\subsection{Reflecting boundary condition}
On a reflecting boundary, $I(\vec{x},\vec{\Omega},t) =
I(\vec{x},\vec{\Omega}_r,t)$, with $\vec{\Omega}_r$ defined in
Eq.~\eqref{eq:reflection} as
\begin{equation*}
  \vec{\Omega}_r = \vec{\Omega} - 2(\vec{\Omega} \vd \vec{n}) \vec{n}\,.
\end{equation*}
As stated before, this identity implies
\begin{equation*}
  \vec{n}\vd \grad I = \vec{n} \vd \grad \phi = 0\,,
\end{equation*}
as well as
\begin{equation*}
  \vec{n}\vd\vec{F}=0\,.
\end{equation*}
We substitute Eq.~\eqref{eq:ap1Ficks} into this last identity to give
\begin{equation}\label{eq:ap1LoRefl1}
  - \vec{n}\vd \Dtens(\vec{x},t) \vd \grad \phi(\vec{x},t)
  + \vec{n}\vd \Etens(\vec{x},t) \vd \vec{F}^i(\vec{x}) = 0\,.
\end{equation}

Reconciling this expression with $\vec{n} \vd \grad \phi=0$ requires the same
jiggery-pokery as the reflecting boundary for anisotropic diffusion. In essence,
the nature of $\Dtens$ generally allows a gradient along a surface to induce a
net flow of radiation across that surface. However, if $\vec{n}$ is an
eigenvector of $\Dtens$ (see \S\ref{sec:eigenvectors}), and of $\Etens$,
Eq.~\eqref{eq:ap1LoRefl1} can be written
\begin{equation*}
  - D(\vec{x},t) \vec{n} \vd \grad \phi(\vec{x},t)
  + E(\vec{x},t) \vec{n} \vd \vec{F}^i(\vec{x}) = 0\,.
\end{equation*}
Because $\vec{n}\vd\vec{F}=0$, $\vec{n}\vd\vec{F}^i$ should also be equal to
zero:
\begin{align}\nonumber
  - D(\vec{x},t) \vec{n} \vd \grad \phi(\vec{x},t) &= 0  
  \\ \label{eq:ap1LoRefl2}
  \vec{n} \vd \grad \phi(\vec{x},t) &= 0\,.
\end{align}

%%%%%%%%%%%%%%%%%%%%%%%%%%%%%%%%%%%%%%%%%%%%%%%%%%%%%%%%%%%%%%%%%%%%%%%%%%%%%%%%
\subsection{Summary}
We briefly recap the equations derived in the previous sections that describe
the anisotropic \Pone\ approximation.

First, we have the low-order conservation equations~\eqref{eqs:loEquations} for
$\phi$,
\begin{equation*}
  \frac{1}{c} \pder{\phi}{t} (\vec{x}, t)
  + \grad \vd \vec{F}(\vec{x},t)
  + \sigmast(\vec{x}) \phi(\vec{x}, t)
  = Q(\vec{x}, t) \,,
  \qquad \vec{x} \in V,\ 0 \le t \le \Delta_t \,,
\end{equation*}
with the initial condition
\begin{equation*}
\phi(\vec{x}, 0) = \phi^i(\vec{x})\,, \vec{x} \in V  \,.
\end{equation*}

The \APone\ method expresses $\vec{F}$ as a time-dependent function that
incorporates the initial condition $\vec{F}^i$. From Eq.~\eqref{eq:ap1Ficks},
\begin{equation*}
  \vec{F}(\vec{x}, t)
= - \Dtens(\vec{x},t) \vd \grad \phi(\vec{x},t)
  + \Etens(\vec{x},t) \vd \vec{F}^i(\vec{x})\,.
\end{equation*}
Here, $\Dtens$ and $\Etens$ are the second angular moments of $f$ and $g$,
respectively, which are solutions to simple transport equations.

The transport equation for $f$ is similar to the formulation of $f$ in the
anisotropic diffusion method, Eqs.~\eqref{eqs:fFull}, but it is time-dependent
with an initial condition equal to zero. We expressed $f$ in
Eqs.~\eqref{eqs:ap1fFull} as
\begin{equation*}
    \pder{f}{t}(\vec{x}, \vec{\Omega},t)
    + \vec{\Omega}\vd \grad f(\vec{x}, \vec{\Omega},t)
    + \sigmast f(\vec{x}, \vec{\Omega},t)
  = \frac{1}{4\pi} \,,
  \qquad x \in V,\ \vec{\Omega} \in 4\pi,\  0 \le t \le \Delta_t\,,
\end{equation*}
with boundary and initial conditions
\begin{alignat*}{2}
  f(\vec{x}, \vec{\Omega}, t) &= \zeta(\vec{x}, \vec{\Omega}, t) \,,
 \qquad && \vec{x} \in \partial V_b, \ \vec{\Omega} \vd \vec{n} < 0
 ,\  0 \le t \le \Delta_t\,,
\\
  f(\vec{x}, \vec{\Omega}, t) &= f(\vec{x}, \vec{\Omega}_r, t) \,,
 && \vec{x} \in \partial V_r, \ \vec{\Omega} \vd \vec{n} < 0
 ,\  0 \le t \le \Delta_t\,,
\\
  f(\vec{x}, \vec{\Omega}, 0) &= 0 \,,
 && \vec{x} \in V, \ \vec{\Omega}\in 4\pi\,.
\end{alignat*}
The function $\zeta$ should satisfy the identity from
Eq.~\eqref{eq:ap1zetaCondition},
\begin{equation*}
- \int_{\vec{\Omega} \vd \vec{n} < 0}
 \vec{\Omega} \zeta(\vec{x}, \vec{\Omega}, t)\ud\Omega
 =
 \int_{\vec{\Omega} \vd \vec{n} > 0}
  \vec{\Omega} f(\vec{x}, \vec{\Omega}, t)\ud\Omega \,.
\end{equation*}
Again, $f$ is related to the low-order equation for $\vec{F}$ through the second
angular moment:
\begin{equation*}
  \Dtens(\vec{x},t) \equiv \int_{4\pi} \vec{\Omega} \vec{\Omega}
  f(\vec{x}, \vec{\Omega},t) \ud\Omega \,,
\end{equation*}

The quantity $g$ is the solution of a similar transport equation, given in
Eqs.~\eqref{eqs:ap1gFull}:
\begin{equation*}
  \pder{g}{t}(\vec{x}, \vec{\Omega},t)
  + \vec{\Omega}\vd \grad g(\vec{x}, \vec{\Omega},t)
  + \sigmast g(\vec{x}, \vec{\Omega},t)
  = 0 \,,
  \qquad x \in V,\ \vec{\Omega} \in 4\pi,\  0 \le t \le \Delta_t\,,
\end{equation*}
with boundary and initial conditions
\begin{alignat*}{2}
  g(\vec{x}, \vec{\Omega}, t) &= \eta(\vec{x}, \vec{\Omega}, t) \,,
 \qquad && \vec{x} \in \partial V_b, \ \vec{\Omega} \vd \vec{n} < 0
 ,\  0 \le t \le \Delta_t\,,
\\
  g(\vec{x}, \vec{\Omega}, t) &= g(\vec{x}, \vec{\Omega}_r, t) \,,
  && \vec{x} \in \partial V_r, \ \vec{\Omega} \vd \vec{n} < 0
 ,\  0 \le t \le \Delta_t\,,
\\
g(\vec{x}, \vec{\Omega}, 0) &= \frac{3}{4\pi} \,,
  && \vec{x} \in V, \ \vec{\Omega}\in 4\pi\,.
\end{alignat*}
The boundary condition $\eta$ should satisfy the identity from
Eq.~\eqref{eq:ap1etaCondition},
\begin{equation*}
- \int_{\vec{\Omega} \vd \vec{n} < 0}
 \vec{\Omega} \eta(\vec{x}, \vec{\Omega}, t)\ud\Omega
 =
 \int_{\vec{\Omega} \vd \vec{n} > 0}
  \vec{\Omega} g(\vec{x}, \vec{\Omega}, t)\ud\Omega \,.
\end{equation*}
The second angular moment of $g$ is the tensor $\Etens$:
\begin{equation*}
  \Etens(\vec{x},t) \equiv \int_{4\pi} \vec{\Omega} \vec{\Omega}
  g(\vec{x}, \vec{\Omega}, t) \ud\Omega \,.
\end{equation*}

We have two low-order boundary conditions. The first is expressible as a
Marshak-like boundary condition relating $\vec{F}$ to $\phi$ on a surface with
incident radiation, given in Eq.~\eqref{eq:ap1marshak2} and valid if $\eta$ and
$\zeta$ are set to be reflecting boundary conditions:
\begin{equation*}
  4 F^-(\vec{x}, t)
  = \phi(\vec{x}, t)
  - 2 \vec{n} \vd \vec{F}(\vec{x}, t) \,,\qquad
   \vec{x} \in \partial V_b,\ 0 \le t \le \Delta_t \,.
\end{equation*}

The second boundary condition, presented in Eq.~\eqref{eq:ap1LoRefl2} under
certain assumptions about the nature of $\Dtens$ and $\Etens$, is for reflecting
boundaries:
\begin{equation*}
  \vec{n} \vd \grad \phi(\vec{x},t) = 0\,,\qquad
   \vec{x} \in \partial V_r,\ 0 \le t \le \Delta_t \,.
\end{equation*}

%%%%%%%%%%%%%%%%%%%%%%%%%%%%%%%%%%%%%%%%%%%%%%%%%%%%%%%%%%%%%%%%%%%%%%%%%%%%%%%%
\section{Discussion}

The biggest approximation in deriving the \APone method was to approximate
$\Psi^i$ as a linear function in angle. Essentially, this implies that at the
end of a time step, all uncollided particles from the initial condition started
with a \Pone-like approximation, and all collided particles are treated with
arbitrary anisotropy.

%%%%%%%%%%%%%%%%%%%%%%%%%%%%%%%%%%%%%%%%%%%%%%%%%%%%%%%%%%%%%%%%%%%%%%%%%%%%%%%%
\subsection{Implicit time-dependent formulation}
To take the implicit Euler approximation of Eq.~\eqref{eq:ap1Ficks}, we
average over the time step $0 \le t \le \Delta_t$, and approximate the
time-average unknowns by their values at the end of the time step, $t=\Delta_t$.
This
procedure yields an approximate but numerically convenient equation,
\begin{equation}\label{eq:ap1implicitFlux1}
  \vec{F}(\vec{x}) = \mat{E} (\vec{x}) \vd \vec{F}^i(\vec{x})
  - \Dtens(\vec{x}) \vd \grad \phi (\vec{x})  \,.
\end{equation}
For simplicity of notation, we assume that variables without a time parameter or
initial condition mark $(\cdot)^i$ are evaluated at the end of the time step:
$\phi(\vec{x})\equiv\phi(\vec{x},\Delta_t)$.

\begin{subequations} \label{eqs:ap1implicitD}
The implicit approximation to the diffusion tensor $\Dtens$ is
\begin{equation}\label{eq:ap1implicitDtens}
  \Dtens(\vec{x}) = \int_{4\pi} \vec{\Omega} \vec{\Omega}
  f(\vec{x}, \vec{\Omega}) \ud\Omega\,,
\end{equation}
and the transport equation~\eqref{eq:ap1fFullVol} for $f$ discretized implicitly
over time is:
\begin{equation*}
  \frac{f - f^i}{c \Delta_t}
  + \vec{\Omega} \vd \grad f
  + \sigmast f
  =  \frac{1}{4\pi} \,,
\end{equation*}
with appropriate boundary conditions and the initial condition $f^i=0$. This
can be rearranged to the form of a steady-state transport solve, with modified
opacities:
\begin{equation} \label{eq:ap1implicitF}
  \vec{\Omega} \vd \grad f
  + \left[ \sigmast + \frac{1}{c \Delta_t} \right]f
  = \frac{1}{4\pi} \,.
\end{equation}
Just like the anisotropic diffusion version of $f$, this transport equation
takes only one ``sweep'' to solve. The value of 
\end{subequations}

\begin{subequations} \label{eqs:ap1implicitE}
Likewise, the implicit approximation to $\mat{E}$ is
\begin{equation}\label{eq:ap1implicitEtens}
  \mat{E} = \int_{4\pi} \vec{\Omega} \vec{\Omega} \:
  g(\vec{x}, \vec{\Omega}) \ud\Omega\,,
\end{equation}
and the transport equation~\eqref{eqs:ap1gFull} for $g$ discretized implicitly
over time is:
\begin{equation*}
  \frac{g - g^i}{c \Delta_t}
  + \vec{\Omega} \vd \grad g
  + \sigmast g
=  0 \,,
\end{equation*}
with appropriate boundary conditions and the initial condition
$g^i=\frac{3}{4\pi}$. This too can be rearranged! Substituting the
isotropic initial condition $g^i$,
\begin{equation} \label{eq:ap1implicitG}
  \vec{\Omega} \vd \grad g
  + \left[ \sigmast + \frac{1}{c \Delta_t} \right]g
  = \frac{1}{c \Delta_t} \frac{3}{4\pi} \,.
\end{equation}
This transport equation \emph{also} takes only one ``sweep'' to solve, but
there is another glaringly obvious fact that will make our lives even easier:
with the implicit approximation, $g = \frac{3}{c \Delta t} f$!
\end{subequations}

Therefore, $\mat{E} =  \frac{3}{c \Delta t} \mat{D}$, so
Eq.~\eqref{eq:ap1implicitFlux1} becomes
\begin{equation}\label{eq:ap1implicitFlux}
  \vec{F}(\vec{x}) =  \frac{3}{c \Delta t} \Dtens (\vec{x}) \vd
  \vec{F}^i(\vec{x})
  - \Dtens(\vec{x}) \vd \grad \phi (\vec{x}) \,.
\end{equation}
Our implicit approximation comprises this approximation to the first moment,
Eq.~\eqref{eq:ap1implicitFlux}, and the transport equation used to calculate
$\Dtens$, Eqs.~\eqref{eqs:ap1implicitD}.

\subsection{Time dependence}
Starting here, we only consider the implicit Euler discretization of the
\APone\ equations. \emph{However}, the fact that we can calculate
$\mat{E}$ and $\Dtens$ however we want means that we could do crazy
stuff like, for example, using a Monte Carlo transport solve to yield
coefficients that take into account the finite speed of the particles, yielding
true time-averaged values for the two tensors, and possibly obviating the error
inherent in implicit time discretization.

%%%%%%%%%%%%%%%%%%%%%%%%%%%%%%%%%%%%%%%%%%%%%%%%%%%%%%%%%%%%%%%%%%%%%%%%%%%%%%%%
\subsubsection{Limit for large time steps}
As $\Delta_t\to \infty$, the modified opacity in Eq.~\eqref{eq:ap1implicitF}
approaches the true space-dependent opacity $\sigmast$. Thus, $\Dtens$
approaches the same $\Dtens$ as the standard anisotropic diffusion method. In
an infinite homogeneous medium, $f\to 1/4\pi \sigma$ so $\Dtens \to
\mat{I}/3\sigma$, the standard diffusion result.

For these large time steps, the contribution of the initial condition
$\vec{F}^i$ should approach zero, so $\mat{E}$ should approach $
\mat{0}$. We find that it does, because the effect of the initial condition
in Eq.~\eqref{eq:ap1implicitG} diminishes exponentially for large time steps. Thus,
for $c \Delta_t \gg \sigmast$, $g=0$ and therefore
$\mat{E}=\mat{0}$.

Therefore, in the limit of large time steps, the \APone\ method limits to the
standard AD method. This is good!

\paragraph{Note} What makes a time step ``large'' is now a non-local factor,
not a local factor as in \Pone.

%%%%%%%%%%%%%%%%%%%%%%%%%%%%%%%%%%%%%%%%%%%%%%%%%%%%%%%%%%%%%%%%%%%%%%%%%%%%%%%%
\subsubsection{Limit for small time steps}
TODO the same analysis makes it clear that as $\Delta_t \to 0$,
$\Dtens$ approaches $\mat{0}$ and $\mat{E}$ approaches $\mat{I}$,
yielding
\begin{equation*}
  \vec{F} \approx \vec{F}^i
\end{equation*}
as $\Delta_t \to 0$.

%%%%%%%%%%%%%%%%%%%%%%%%%%%%%%%%%%%%%%%%%%%%%%%%%%%%%%%%%%%%%%%%%%%%%%%%%%%%%%%%
\subsubsection{Infinite homogeneous medium case}
In an infinite homogeneous medium, Eq.~\eqref{eq:ap1implicitFlux} limits to a
form, obviating the error inherent in implicit discretizations.
of the implicitly time-discretized \Pone\ equation.

The solution to Eq.~\eqref{eq:ap1implicitF} is
\begin{equation*}
  f
  = \frac{1}{4\pi} \frac{1}{\sigmast + 1 /c \Delta_t}
  = \frac{1}{4\pi} \frac{\sigmast}{\sigmast} \frac{c \Delta t}{1 + \sigmast c \Delta_t}
  = \frac{1}{4\pi \sigma} ( 1 - \alpha) \,,
\end{equation*}
where $\alpha = 1/ \sigmast c \Delta_t$. (This has a physical
interpretation, because $\sigmast c \Delta_t$ is the number of mean free
paths a photon travels inside a time step.) Equation~\eqref{eq:ap1implicitDtens}
then gives
\begin{equation*}
  \Dtens = \frac{1}{3 \sigma} ( 1 - \alpha) \mat{I}\,.
\end{equation*}

The solution to Eq.~\eqref{eq:ap1implicitG} is
\begin{equation*}
  g
  = \frac{1}{c \Delta_t} \frac{3}{4\pi} \frac{1}{\sigmast + 1 /c \Delta_t}
  = \frac{3}{4\pi} \frac{1}{1 + \sigmast c \Delta_t}
  = \frac{3}{4\pi} \alpha \,.
\end{equation*}
Equation~\eqref{eq:ap1implicitEtens} then gives
\begin{equation*}
  \mat{E} = \alpha \mat{I}\,.
\end{equation*}

Substituting $\Dtens$ and $\mat{E}$ into Eq.~\eqref{eq:ap1implicitFlux}, we get the
infinite-medium solution
\begin{align*}
  \vec{F}(\vec{x})
  &= \alpha \mat{I} \vd \vec{F}^i(\vec{x})
  - \frac{1}{3 \sigma} ( 1 - \alpha) \mat{I} \vd \grad \phi (\vec{x}) 
  \\
  &= \alpha \vec{F}^i(\vec{x})
  - \frac{1}{3 \sigma} ( 1 - \alpha) \grad \phi (\vec{x})\,.
\end{align*}

We can reshape the \Pone\ equations into this exact form:
\begin{align*}
  \frac{\vec{F} - \vec{F}^i}{c \Delta_t} + \frac{1}{3} \grad \phi
  + \sigmast \vec{F} &= 0
  \\
  \vec{F} - \vec{F}^i 
  + \sigmast c \Delta_t \vec{F}
  &= - c \Delta_t \frac{1}{3} \grad \phi
  \\
  \left[ 1 + \sigmast c \Delta_t \right] \vec{F}
  &= \vec{F}^i - \sigmast c \Delta_t \frac{1}{3\sigmast} \grad \phi
  \\
  \vec{F}
  &= \frac{1}{1 + \sigmast c \Delta_t} \vec{F}^i - \frac{\sigmast c
  \Delta_t}{1 + \sigmast c \Delta_t} \frac{1}{3\sigmast} \grad \phi
  \\
  \vec{F}
  &= \alpha \vec{F}^i - (1 - \alpha) \frac{1}{3\sigmast} \grad \phi
\end{align*}

Ta-da! Our approximation of the previous time step in \S\ref{sec:ap1ic} as
being linear in angle is really what gives the undesirable $\frac13\grad\phi$
(which yields the incorrect energy propagation speed). It may be possible to make some better approximation to how we represent $\Psi^i$.

%%%%%%%%%%%%%%%%%%%%%%%%%%%%%%%%%%%%%%%%%%%%%%%%%%%%%%%%%%%%%%%%%%%%%%%%%%%%%%%%
\section{Summary}
The anisotropic \Pone\ approximation was derived using the same procedure as 
time-dependent anisotropic diffusion. The assumption of a stronger time
dependence of $I$ led to the retention of the initial condition $\Psi^i$ in the
approximate equation for $\Psi$. To avoid the expensive storage of the full form
of $\Psi^i$, we applied the $P_1$ approximation to $\Psi^i$, expanding it as a
linear-in-angle function.

