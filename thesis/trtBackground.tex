% !TEX root = _individual/trtBackground.tex

%%%%%%%%%%%%%%%%%%%%%%%%%%%%%%%%%%%%%%%%%%%%%%%%%%%%%%%%%%%%%%%%%%%%%%%%%%%%%%%%
\chapter{Background: Thermal Radiative Transfer}
%%%%%%%%%%%%%%%%%%%%%%%%%%%%%%%%%%%%%%%%%%%%%%%%%%%%%%%%%%%%%%%%%%%%%%%%%%%%%%%%

Thermal radiative transfer (TRT) is the nonlinear process describing the
dominant form of energy transfer in a very hot material, such as the interior
of a star or the
target of a laser fusion experiment. The equations describing TRT are
time-dependent, contain strong nonlinearities, and reside in a large phase
space. These difficulties make TRT the subject of significant work in methods
development.

A full representation of the physics in
those high-energy-density regimes often includes the consideration of moving
relativistic materials, different electron and ion temperatures, photon
scattering, and thermal conduction in the material \cite{Mih1984}. However,
much theoretical work in the field neglects these complex phenomena by
\prelistpar
\begin{itemize}
  \item working in a fixed medium, disregarding material advection;
  \item assuming local thermodynamic equilibrium (LTE), which uses a single
    material temperature;
  \item neglecting photon scattering, which VERIFY THIS tends to be small for
    very hot materials; and
  \item neglecting thermal conduction, since energy transfer is dominated by
    radiative transfer in the temperature regimes we consider.
\end{itemize}

A further simplification often used for methods development is the ``gray''
approximation to the frequency dependence. Analogous to the one-group
approximation for neutron transport, the full transport equation is integrated
over all frequencies, and the opacities are averaged with some \emph{a priori}
weighting function. The commonly used Rosseland mean satisfies a radiation
diffusion equation found in an asymptotic analysis of the thermal radiative
transfer equations \cite{Lar1983a} and is therefore the best choice for a
method that resembles diffusion.

\section{TRT equations}
After the simplifications, the thermal radiative transfer process can be
described by
\begin{subequations} \label{eqs:fullGrayTRT}
the radiative transfer equation,
\begin{equation} \label{eq:fullGrayTransport}
  \frac{1}{c} \pder{I}{t}
  + \vec{\Omega} \vd \del I +
 \sigma I
  = \frac{\sigma c U_r}{4\pi} 
  + \frac{c Q}{4\pi} \,,
\end{equation}
and the material energy balance equation
\begin{equation} \label{eq:fullGrayMaterial}
  \pder{U_m}{t} = \sigma \int_{4\pi}  I \ud \Omega - \sigma c U_r
  %= \sigma \phi - \sigma c U_r
  \,.
\end{equation}
\end{subequations}

The notation and omitted parameters in Eqs.~\eqref{eqs:fullGrayTRT} are:
\begin{alignat*}{2}
  I &= I(\vec{x}, \vec{\Omega}, t) &&= \text{the angle-dependent
  radiation intensity,}
%  \\
%  \phi &= \phi(\vec{x}, t) &&= \text{the zeroth angular moment of the radiation
%  intensity,}
  \\
  T &= T(\vec{x}, t) &&= \text{the temperature of the material,}
  \\
  \sigma &= \sigma(\vec{x}, T) &&= \text{the absorption opacity,} 
  \\
  Q &= Q(\vec{x}) &&= \text{an extraneous isotropic radiation energy source,}
  \\
  U_m &= U_m(\vec{x}, T) &&= \text{the material energy density,}
  \\
  U_r &= U_r(\vec{x}, T) &&= \text{the equilibrium radiation energy density of
  the material,}
  \\
  c_v &= c_v(\vec{x}, T) &&= \text{the specific heat capacity of a material,}
  \\
  c& &&= \text{the speed of light.}
\end{alignat*}
The ``equilibrium radiation energy density'' of a material is a scaled integral
of the Planckian emission function:
\begin{subequations} \label{eqs:materialU}
\begin{equation} \label{eq:radEnergyDens}
  U_r(T) \equiv aT^4 = \frac{1}{c} \int_{4\pi} \int_{0}^{\infty}B(\nu, T) \ud
  \nu \ud \Omega \,.
\end{equation}
The material energy density is related only to the material's temperature:
\begin{equation} \label{eq:matEnergyDens}
  U_m(T) = \int_{0}^{T} c_v(T') \ud T' \,.
\end{equation}
The specific heat capacity $c_v(T)$ is the amount of energy per unit
volume needed to change the material's temperature.
\end{subequations}
%The zeroth moment of $I$ is equal to the radiation energy density scaled by the
%speed of light:
%\begin{equation*}
%  \phi(\vec{x}, t) = \int_{4\pi} I(\vec{x}, \vec{\Omega}, t) \ud \Omega
%  = \int_{4\pi} c [h \nu] N(\vec{x}, \vec{\Omega}, t) \ud \Omega
%  = c \RadEn(\vec{x}, t)\,.
%\end{equation*}
%Here, $N$ is the photon density, $h\nu$ is the energy of a single photon, and
%$E$ is the radiation energy density.


