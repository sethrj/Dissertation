% !TEX root = _individual/trtBackground.tex

%%%%%%%%%%%%%%%%%%%%%%%%%%%%%%%%%%%%%%%%%%%%%%%%%%%%%%%%%%%%%%%%%%%%%%%%%%%%%%%%
\chapter{Background to Thermal Radiative Transfer}\label{chap:trtBackground}

In order to elucidate the derivation and application of the anisotropic
diffusion approximation, it is necessary to delve deeper into the physical
process that is subject to the approximation. As discussed in the introduction,
thermal radiative transfer is the physical process of energy transfer via
high-energy photons in hot materials. Because the TRT equations and
approximations have been probed and reviewed in countless other works
\cite{Mih1984,Pom1973,Cas2004,Wol2008}, our aim is a concise explanation of the
physics relevant to the anisotropic diffusion approximation and its immediate
application, rather than a thorough overview of the extensive field of radiation
hydrodynamics. We also review the derivation of competing solution methods and
discuss their advantages and shortcomings.

%%%%%%%%%%%%%%%%%%%%%%%%%%%%%%%%%%%%%%%%%%%%%%%%%%%%%%%%%%%%%%%%%%%%%%%%%%%%%%%%
\section{Equations of transfer}\label{sec:bgTrtEquations}
Thermal radiative transfer describes how high-energy photons move energy about
in a very hot material, such as the interior of a star or the target of a laser
fusion experiment. The equations that model TRT are time-dependent, contain
strong nonlinearities, and reside in a large phase space.
A full representation of the physics in
the high-energy-density regime often includes the consideration of moving
relativistic materials, different electron and ion temperatures, photon
scattering, and thermal conduction in the material \cite{Mih1984}. However,
much theoretical work in the field neglects these complex phenomena by
\begin{itemize}
  \item working in a fixed medium, disregarding material advection;
  \item assuming local thermodynamic equilibrium (LTE), which uses a single
    material temperature;
  \item neglecting photon scattering, which tends to be comparatively small for
    very hot materials; and
  \item neglecting thermal conduction, since energy transfer is dominated by
    radiation in the temperature regimes we consider.
\end{itemize}

A further simplification often used for methods development is the ``gray''
approximation to the frequency dependence. Analogous to the one-group
approximation for neutron transport, the full transport equation is integrated
over all frequencies, and the opacities are averaged with some \emph{a priori}
weighting function, typically the Rosseland mean \cite{Lar1983a}.

For the purposes of discussion and the later AD derivation, we consider a
general, 3-D universe, where the spatial coordinates are
\begin{equation*}
  \vec{x}
  = x \vec{i} + y \vec{j} + z \vec{k}\,,
\end{equation*}
and the angular coordinates are the unit vector
\begin{equation*}
  \vec{\Omega}
  = \mu \vec{i}
  + \sqrt{1-\mu^2} \cos \theta \vec{j}
  + \sqrt{1-\mu^2} \sin \theta \vec{k} \,.
\end{equation*}
These angular coordinates reside in the domain $-1 \le \mu \le 1$, $0 \le \theta
< \pi$; we use $\vec{\Omega}\in4\pi$ as a frequent shorthand denoting the entire
unit sphere. See \cite{Lar2007,Pri2010} for a more complete discussion of the
spatial and angular coordinate systems.

After the simplifications described above, the thermal radiative transfer
process in the interior of a problem (away from the initial time and
from boundaries) can be described \cite{Pom1973} by
\begin{subequations} \label{eqs:explanTRT}
the radiative transfer equation,
\begin{equation} \label{eq:explanTransport}
  \frac{1}{c} \pder{I}{t}
  + \vec{\Omega} \vd \del I +
 \sigma I
  = \frac{\sigma a c T^4}{4\pi} 
  + \frac{c q_r}{4\pi} \,,
\end{equation}
and the material energy balance equation,
\begin{equation} \label{eq:explanMaterial}
  c_v \pder{T}{t} = \sigma \int_{4\pi}  I \ud \Omega - \sigma a c T^4 \,.
\end{equation}
\end{subequations}

The notation and omitted parameters in Eqs.~\eqref{eqs:explanTRT} are:
\begin{alignat*}{2}
  I &= I(\vec{x}, \vec{\Omega}, t) &&= \text{the angular
  radiation intensity,}
  \\
  T &= T(\vec{x}, t) &&= \text{the material temperature,}
  \\
  \sigma &= \sigma(\vec{x}, T) &&= \text{the absorption opacity,} 
  \\
  q_r &= q_r(\vec{x}, t) &&= \text{an extraneous isotropic radiation energy source,}
  \\
  c_v &= c_v(\vec{x}, T) &&= \text{the specific heat capacity of a material,}
  \\
  a& &&= \text{the radiation constant, and}
  \\
  c& &&= \text{the speed of light.}
\end{alignat*}
The intensity $I$ and the temperature $T$ are the primary unknowns: they
describe the state of energy in the radiation and in the material.
Each of the terms that depends on $T$ implicitly depends on the time $t$. The
explicit dependence of $c_v$ and $\sigma$ on $\vec{x}$ accounts for different
materials in different parts of the problem.

Equation~\eqref{eq:explanTransport} is a transport equation for $I$. The first
two terms describe how photons ``stream'' in time and space: if the $\sigma$
and source terms were zero, Eq.~\eqref{eq:explanTransport} would reduce to a
wave equation with a wave speed of $c$. However, because the photons are moving
through a material, there is a chance they will collide with the material,
hence the \emph{collision} term $\sigma I$. The first term on the right-hand
side represents particles emitted via the isotropic temperature-dependent
process of black body emission. The additional term $q_r$ is an extraneous
isotropic radiation source that emits with an energy density (energy per
volume per time) of $q_r$.

The material equation~\eqref{eq:explanMaterial} describes an energy balance in
the material. On the left-hand side is the time rate of change in the material
energy density, which is a function of the material's temperature and specific
heat capacity $c_v$:
\begin{equation} \label{eq:matEnergyDens}
  U_m(T) = \int_{0}^{T} c_v(T') \ud T' \,.
\end{equation}
The first term on the right hand side exactly mirrors the collision term in the
radiation equation: it is a ``gain'' term corresponding to photons that
collided with (were absorbed by) the material. The second term describes when
energy is lost from the material and emitted as radiation: black body emission.

In order to conserve energy in a simulation, strict attention must be paid to
$U_m$ and $\phi$, which are the true quantities of energy in
the material and radiation. If the loss, gain, and rate of change terms are not
properly treated, energy will not be conserved.

The physical properties $\sigma$ and $c_v$ are often approximated by simplistic
models in the methods development sphere. The heat capacity $c_v$ of an ideal
gas is a constant, giving the material energy $U_m$ a linear proportionality to
the material temperature $T$. A much-used model \cite{Mou2006,Wol2008} of the
gray opacity is $\sigma \propto \propto T^{-3}$.
Our numerical test problems will use both of these idealized representations of
the physical constants.

The nonlinear coupling between Eqs.~\eqref{eq:explanTransport}
and~\eqref{eq:explanMaterial} via $T^4$ emission and $T^{-3}$ absorption make
the TRT equations extremely ``stiff'' \cite{Kno2003} and therefore even more
difficult to solve than the standard linear transport equation. In this work,
we will not attempt to provide any new solutions to treat the nonlinearities,
but we will formulate our time-dependent anisotropic diffusion approximations to
be compatible with multiple solution techniques.

\subsection{The radiation intensity}

In radiative transfer, the intensity $I$ is the energy-weighted radiation path
length density, similar to the angular flux $\psi$ in reactor physics:
\begin{equation*}
  I(\vec{x},\vec{\Omega},\nu, t) = c h\nu N(\vec{x},\vec{\Omega},\nu, t)\,,
\end{equation*}
where $h\nu$ is the photon energy and $N$ is the photon density,
\begin{align*}
  N(\vec{x},\vec{\Omega}, \nu, t) \ud V \ud \Omega
  &= \topbox{the number of photons inside the differential volume $\ud V$
  about $\vec{x}$, traveling in the directions $\Omega$ about
  $\vec{\Omega}$, inside the frequencies $\ud \nu$ about $\nu$, at time $t$.}
\end{align*}
Integrating over all energy gives the gray intensity,
\begin{equation*}
  I(\vec{x},\vec{\Omega}, t)
  = c \int_0^\infty h\nu N(\vec{x},\vec{\Omega},\nu, t) \ud\nu \,.
\end{equation*}

Additionally, integrating over all angles (i.e.~taking the zeroth angular
moment) yields the ``scalar intensity''
\begin{align} \label{eq:intensityZeroth}
  \phi(\vec{x},t) &\equiv \int_{4\pi} I(\vec{x},\vec{\Omega}, t) \ud \Omega
  \\
  &= c \int_{4\pi} \int_0^\infty h\nu N(\vec{x},\vec{\Omega},\nu, t) \ud\nu
   \ud \Omega
\\ \intertext{which is directly proportional to the radiation energy density:}
\nonumber
\frac{1}{c} \phi(\vec{x},t) \ud V
&= \topbox{the amount of energy in the radiation field inside the differential
  volume $\ud V$ about $\vec{x}$, at time $t$.}
\end{align}
For our work, it is usually more convenient to refer to $\phi$ than to the
radiation energy density (compare, for example, \cite{Den2007} to
\cite{Kno1999a}). In fact, since in our computational experiments we use the
scaled system of variables $c=a=1$, our later results can use ``scalar
intensity'' and ``radiation energy density'' interchangeably.

The first angular moment of the intensity also has physical significance. The
radiation flux is defined as
\begin{equation} \label{eq:intensityFirst}
  \vec{F}(\vec{x},t) \equiv \int_{4\pi} \vec{\Omega}
  I(\vec{x},\vec{\Omega}, t) \ud \Omega \,.
\end{equation}
Analogous to the ``neutron current'' $\vec{J}$ in reactor physics, the
radiation flux is the net rate of energy flowing through a point.

At any particular point in space and time, the radiation intensity $I$ is
generally a complicated function of angle.%
\footnote{See \cite{Ada2001a} for an example of how complicated a function of
angle the transport solution can be in even a simple reactor physics problem.}
For example, at the edge of a
radiation shock wave, the distribution is highly peaked in the directions
pointed away from the hot region, because that is the source of the photons. In
other parts of the problem where the system is closer to an equilibrium state,
the intensity is nearly isotropic: that is, the intensity is almost a uniform
function in angle.

%%%%%%%%%%%%%%%%%%%%%%%%%%%%%%%%%%%%%%%%%%%%%%%%%%%%%%%%%%%%%%%%%%%%%%%%%%%%%%%%
\section{Semi-implicit linearization}\label{sec:bgSemiImplicit}

The anisotropic diffusion approximation is intended for use in a deterministic
manner. It is therefore necessary to discuss \emph{discretization} schemes,
where the continuous unknowns in Eqs.~\eqref{eqs:explanTRT} are approximated
with discrete unknowns. In this section, we tackle the discretization of the
time variable in the context of \emph{linearization}, where the nonlinear
aspects of the TRT equations are approximated to allow a linear algebraic
representation of the system of unknowns, facilitating their solution on a
computer.

To solve the TRT equations with deterministic methods, we use the common
``semi-implicit'' scheme \cite{Kno1999a,Kno2001,Low2004}, where operator splitting
is used to decouple the radiation and material equations inside a time step, and
a backward Euler discretization is used with the unknowns.
Like Fleck and Cummings' IMC method \cite{Fle1971}, this technique
yields a linear transport equation with a pseudo-scattering term. Because of the
extensive use of this
nonlinear treatment in our implementations of anisotropic diffusion and the
other tested methods, and because the semi-implicit scheme for radiation
transport is usually glossed over or presented vaguely in other works, we derive
it here in some level of detail.

For the semi-implicit discretization, it is more convenient to write
Eqs.~\eqref{eqs:explanTRT} in a slightly altered form:
\begin{subequations} \label{eqs:semiTRT}
\begin{equation} \label{eq:semiTransport}
  \frac{1}{c} \pder{I}{t}
  + \vec{\Omega} \vd \del I +
 \sigma I
 = \frac{\sigma c U_r}{4\pi} 
  + \frac{c q_r}{4\pi} \,,
\end{equation}
and the material energy balance equation,
\begin{equation} \label{eq:semiMaterial}
  \pder{U_m}{t} = \sigma \int_{4\pi}  I \ud \Omega - \sigma c U_r \,.
\end{equation}
\end{subequations}
Here, we have defined the ``equilibrium radiation energy density'' of a
material as a scaled integral of the Planckian emission function:
\begin{equation} \label{eq:radEnergyDens}
  U_r(T) \equiv aT^4
  = \frac{1}{c} \int_{4\pi} \int_{0}^{\infty} B(\nu, T) \ud\nu \ud\Omega \,.
\end{equation}
Its physical relevance is that, when the radiation field and material reach
an equilibrium, $I=B$, and the radiation energy density $\phi/c$ is equal to
$U_r$.  The quantity $U_r$ is \emph{not} equal to the energy density stored in
the material, $U_m$.

\subsection{Linearizing the material energy equation}

First, we define a parameter $\beta$ as a function of
Eqs.~\eqref{eq:matEnergyDens} and~\eqref{eq:radEnergyDens}:
\begin{equation} \label{eq:beta}
  \beta(\vec{x}, T) \equiv \pder{U_r}{U_m} 
  = \pder{U_r}{T} \Bigg/ \pder{U_m}{T}
  = \frac{4 a T^3}{c_v(\vec{x}, T)} \,.
\end{equation}
The chain rule allows the left hand side of Eq.~\eqref{eq:semiMaterial} to be
expressed without approximation in terms of the equilibrium radiation energy
density $U_r$:
\begin{equation*}
  \pder{U_m}{t} = \pder{U_m}{U_r} \pder{U_r}{t} = \frac{1}{\beta(T)}
  \pder{U_r}{t} = \sigma \int_{4\pi}  I \ud \Omega - \sigma c U_r \,.
\end{equation*}
The first approximation is to ``freeze'' the parameter $\beta$ at the beginning-of-time-step temperature $T^n$:
\begin{equation}\label{eq:frozenBeta}
  \frac{1}{\beta^n}
  \pder{U_r}{t} \approx \sigma \int_{4\pi}  I \ud \Omega - \sigma c U_r \,.
\end{equation}
The process of freezing $\beta$ is equivalent to approximating the
Planckian emission term with a Taylor series \cite{Kno2007}.

Because the approximation to $\beta$ is an approximation to the rate of change
in material energy, this equation no longer conserves the system's total
energy. To enforce conservation of energy over a time step, we must set the
material energy change over a time step to the time-integrated approximation:
\begin{align}
  \nonumber
  \int_{t^n}^{t^{n+1}}  \pder{U_m}{t}\ud t &= \frac{1}{\beta^n}
  \int_{t^n}^{t^{n+1}} \pder{U_r}{t}\ud t
  \\
  \nonumber
  U_m^{n+1} - U_m^n &= \frac{1}{\beta^n} \left[ U_r^{n+1} - U_r^n \right]
  \\
  \label{eq:matenConservationUpdate}
  U_m^{n+1} &=  U_m^n + \frac{U_r^{n+1} - U_r^n}{\beta^n}\,.
\end{align}
%Thus, the expression of $\tpder{U_m}{t}$ in terms of $\tpder{U_r}{t}$

The next approximation is to explicitly freeze the opacity $\sigma$ in
Eq.~\eqref{eq:frozenBeta}:
\begin{equation*}
  \frac{1}{\beta^n}
  \pder{U_r}{t} \approx \sigma^n \int_{4\pi}  I \ud \Omega - \sigma^n c U_r \,.
\end{equation*}
Now we can time-average the material equation to express it in terms of two
simple time-average unknowns. Operating by
$\frac{1}{\Delta_t^n}\int_{t_n}^{t^{n+1}} (\cdot) \ud t$,
\begin{equation*}
  \frac{1}{\beta^n}
  \frac{U_r^{n+1} - U_r^n}{\Delta_t^n} = \sigma^n \int_{4\pi} \left[
  \frac{1}{\Delta_t^n}\int_{t_n}^{t^{n+1}} I\ud t
  \right] \ud \Omega - \sigma^n c \left[
  \frac{1}{\Delta_t^n}\int_{t_n}^{t^{n+1}} U_r \ud t \right]\,.
\end{equation*}
Next, we apply the implicit Euler approximation%
\footnote{Note that the IMC method only applies the implicit approximation to
$U_r$, allowing the continuous-in-time treatment of $I$.}%
to $U_r(\vec{x}, \vec{\Omega}, t)$ and $I(\vec{x}, \vec{\Omega}, t)$ by setting
their time-averaged values to
the values at $t^{n+1}$:
\begin{equation} \label{eq:semiImplicitMaterial}
  \frac{1}{\beta^n(\vec{x})}
  \frac{U_r^{n+1}(\vec{x}) - U_r^n(\vec{x})}{\Delta_t^n}
  = \sigma^n(\vec{x}) \int_{4\pi} I^{n+1}(\vec{x}, \vec{\Omega})\ud \Omega
  - c \sigma^n(\vec{x}) U_r^{n+1}(\vec{x}) \,.
\end{equation}

We solve Eq.~\eqref{eq:semiImplicitMaterial} for $U_r^{n+1}$ in order to
eliminate the implicit dependence of the transport equation on the material
energy equation.
\begin{align} \nonumber
  U_r^{n+1} [ 1 + c \beta^n \Delta_t^n \sigma^n ]
  &= \beta^n \Delta_t^n \sigma^n\int_{4\pi} I^{n+1}\ud \Omega + U_r^n
   \\ \nonumber
  U_r^{n+1}
  &= \frac1c \frac{ c \beta^n \Delta_t^n \sigma^n }{ 1 + c \beta^n \Delta_t^n \sigma^n}
  \int_{4\pi} I^{n+1}\ud \Omega + \frac1{ 1 + c \beta^n \Delta_t^n \sigma^n}
  U_r^n
  \\ \label{eq:urNPlusOne}
  U_r^{n+1}
  &= \left(1 - f^n\right) \frac1c \int_{4\pi} I^{n+1}\ud \Omega + f^n U_r^n
\end{align}
where we have defined the Fleck factor \cite{Fle1971} as
\begin{equation} \label{eq:fleckFactor}
  f^n = f^n(\vec{x}) \equiv \left[ 1 + \beta^n c \Delta_t^n \sigma^n
  \right]\inv \,.
\end{equation}

\subsection{Linearizing the transport equation}
The next step is apply similar approximations to the nonlinear radiation
transport equation~\eqref{eq:semiTransport}. As with the material equation,
the
opacities are ``frozen'' at their beginning-of-time-step values $\sigma^n$, and
the equation is time-averaged:
\begin{multline*}
  \frac{1}{c} \frac{I^{n+1} - I^n}{\Delta_t^n}
  + \vec{\Omega} \vd \del \left[
  \frac{1}{\Delta_t^n}\int_{t_n}^{t^{n+1}} I\ud t
  \right] +
 \sigma^n \left[
  \frac{1}{\Delta_t^n}\int_{t_n}^{t^{n+1}} I\ud t
  \right]
  \\
  = \frac{\sigma^n c}{4\pi} \left[
  \frac{1}{\Delta_t^n}\int_{t_n}^{t^{n+1}} U_r \ud t \right]
  + \frac{1}{4\pi}\left[
  \frac{1}{\Delta_t^n}\int_{t_n}^{t^{n+1}} q_r \ud t \right] \,.
\end{multline*}
Since the extraneous energy source $q_r$ is assumed to be known \emph{a priori},
we let its time-averaged value be $q_r^n$. As in the material equation, we apply
the implicit Euler approximation to $I$ and $U_r$:
\begin{equation*}
  \frac{1}{c} \frac{I^{n+1} - I^n}{\Delta_t^n}
  + \vec{\Omega} \vd \del I^{n+1}
 + \sigma^n I^{n+1}
 = \frac{\sigma^n c}{4\pi} U_r^{n+1}
  + \frac{c}{4\pi} q_r^n \,.
\end{equation*}
Finally, we substitute $U_r^{n+1}$ from Eq.~\eqref{eq:urNPlusOne},
which was derived from the material equation Eq.~\eqref{eq:semiMaterial}:
\begin{align}\nonumber
  \frac{1}{c} \frac{I^{n+1} - I^n}{\Delta_t^n}
  + \vec{\Omega} \vd \del I^{n+1}
 + \sigma^n I^{n+1}
 &= \frac{\sigma^n c}{4\pi} \left[ \left(1 - f^n\right) \frac1c \int_{4\pi} I^{n+1}\ud \Omega + f^n U_r^n \right]
  + \frac{c}{4\pi} q_r^n
  \\ \label{eq:linearizedGrayTransport}
  \frac{1}{c} \frac{I^{n+1} - I^n}{\Delta_t^n}
  + \vec{\Omega} \vd \del I^{n+1}
 + \sigma^n I^{n+1}
 &=  \left(1 - f^n\right) \sigma^n \frac{1}{4\pi} \int_{4\pi} I^{n+1}\ud \Omega
 + \frac{1}{4\pi} f^n \sigma^n c U_r^n
  + \frac{1}{4\pi} c q_r^n \,.
\end{align}

\subsection{Comments}\label{bgSIComments}
If we compare Eq.~\eqref{eq:linearizedGrayTransport} to a temporally implicit
discretization of a monoenergetic linear transport problem with isotropic
scattering,
\begin{equation*}
  \frac{1}{v} \frac{\psi^{n+1} - \psi^n}{\Delta_t^n} 
  + \vec{\Omega} \vd \del \psi^{n+1}
 + \Sigma_t \psi^{n+1}
 = \frac{1}{4\pi} \int_{4\pi} \psi^{n+1}\ud \Omega
  + \frac{1}{4\pi} q \,,
\end{equation*}
we find equivalences between the two:
\begin{alignat*}{2}
  I &\leftrightarrow \psi &&= \text{the angular flux,}
  \\
  \sigma^n &\leftrightarrow \Sigma_t &&= \text{the total cross section,}
  \\
  \left(1 - f^n\right) \sigma^n &\leftrightarrow \Sigma_s &&= \text{the scattering cross
  section,} 
  \\
  f^n \sigma^n c U_r^n + c q_r^n &\leftrightarrow q &&= \text{the isotropic source for time
  step $n$,}
  \\
  v   &\leftrightarrow c &&= \text{the particle velocity.}
\end{alignat*}
Even though the original radiation transport equation was purely
absorbing, the linearization scheme created a ``pseudoscattering''
term that essentially emulates the absorption and isotropic re\"emission of
radiation during a time step. The IMC literature often refers to the
``effective scattering opacity,''
$\sigma_\text{es}^n \equiv \left(1 - f^n\right) \sigma^n$.

Furthermore, if we take the zeroth angular moment of
Eq.~\eqref{eq:linearizedGrayTransport} and let $\phi^{n+1}(\vec{x}) \equiv
\frac{1}{4\pi} \int_{4\pi} I^{n+1}\ud \Omega$, then we find the radiation
energy conservation equation over the time step to be
\begin{equation}\label{eq:semiImplicitZeroth}
  \frac{1}{c} \frac{\phi^{n+1} - \phi^n}{\Delta_t^n}
  + \del \vd \vec{F}^{n+1} + f^n\sigma^n \phi^{n+1}
 =  f^n \sigma^n c U_r^n + c q_r^n\,.
\end{equation}
The quantity, $\sigma_\text{ea}^n \equiv f^n\sigma^n$ is known as the
``effective absorption opacity.''

\subsection{Solution process summary}
The time-dependent radiation solution is stored in the linear time-dependent
transport solver. The material energy $U_m$ must be stored, but the
material temperature can be either stored (highly recommended because of its
frequent use) or calculated on the fly from $U_m$ by inverting the integral in
Eq.~\eqref{eq:matEnergyDens}.

For the $n$th time step, given the initial radiation field $I^{n}$ and the
initial material energy density $U_m^n$, the solution process follows.
\begin{enumerate}
  \item \emph{Linearize the system.} Using the starting temperature $T^n$,
    calculate
    the frozen $\sigma^n$ and $\beta^n$ in each spatial cell. Use
    Eq.~\eqref{eq:fleckFactor} to calculate $f^n$, which in turn is used to
    calculate the linearized isotropic source $f^n \sigma^n c U_r^n + c q_r^n$
    and the effective scattering cross section $\left(1 - f^n\right) \sigma^n$.
    If using a diffusion method to approximate the transport solution, the
    absorption cross sections and diffusion coefficients must be recalculated.

  \item \emph{Solve the linear transport problem for $\psi^{n+1}=I^{n+1}$.} The new
    radiation
    temperature can optionally be calculated:
    \begin{equation*}
      a (T_\text{rad}^{n+1})^4 = \frac{1}{c} \int_{4\pi} I^{n+1}
      \ud \Omega
      \lra
      T_\text{rad}^{n+1}(\vec{x})
      = \left[ \frac{\phi^{n+1}(\vec{x})}{ac} \right]^{1/4}\,.
    \end{equation*}

  \item \emph{Update the material temperature.} From Eq.~\eqref{eq:urNPlusOne}
    we can calculate the linearized estimate of $U_r^{n+1}$:
    \begin{equation*}
      U_r^{n+1} = \left(1 - f^n\right) \frac1c \phi^{n+1}  + f^n U_r^n\,.
    \end{equation*}
    However, because of the linearization of $\beta$, $U_r^{n+1} \ne a
    (T^{n+1})^4$. Instead, to calculate the material temperature, we must use
    Eq.~\eqref{eq:matenConservationUpdate}. Substituting
    Eq.~\eqref{eq:urNPlusOne} into Eq.~\eqref{eq:matenConservationUpdate}
    and simplifying gives
    \begin{equation}\label{eq:matenConservationUpdate2}
      U_m^{n+1} = U_m^n + f^n \sigma^n \Delta_t^n
      \left[ \phi^{n+1} - c U_r^n \right] \,.
    \end{equation}
    [This form can also be derived by integrating
    Eq.~\eqref{eq:semiMaterial} over a time step after making the
    approximation $\sigma(T) \approx \sigma(T^n)$.]
\end{enumerate}

Figure~\ref{fig:semiImplicitFlowchart} shows how the quantities $\Delta_t$,
$\beta$, $\sigma$, $f$, etc.~relate. This relation is especially important when
implementing the linearization scheme programmatically. 

\begin{sidewaysfigure}[hp]
  \centering
  \includegraphics[width=8in]{semi-implicit}
  \caption{Dependency graph of quantities in the semi-implicit discretization.}
  \label{fig:semiImplicitFlowchart}
\end{sidewaysfigure}

%%%%%%%%%%%%%%%%%%%%%%%%%%%%%%%%%%%%%%%%%%%%%%%%%%%%%%%%%%%%%%%%%%%%%%%%%%%%%%%%
\clearpage
\section{Deterministic radiation transport approximations}
\label{sec:bgApproxMethods}

In this section, we briefly review several existing deterministic radiation transport methods.

The salient difference among these methods is in their treatment of
the angular dependence of the solution. Additionally, each method tends to have
its own set of spatial discretizations that are effective and efficient for the
given angular treatment. The choice of temporal discretization is usually
independent of the method used. Every approximation applied to the transport
equation introduces a source of error.

Approximations to the angular behavior of $I$ introduce definite errors
that are often surprisingly hard to quantify. Asymptotic analysis can be used to
describe quantitative regimes of applicability, but we will confine our
discussion to some of the qualitative unphysical behavior introduced by the
angular approximations.

Innumerable spatial discretizations exist. Typically, a
spatial discretization will introduce a local error into the solution in some
relation to the grid size. In the diffusion (and anisotropic diffusion) methods,
the most convenient methods to use are often finite difference schemes
\cite{Lev2007}. It should also be noted for the sake of completion that for more
complex angular approximations than diffusion such as \SN, the grid error is not
the most serious problem; poor asymptotic behavior for optically
thick cells can lead to highly inaccurate answers \cite{Ada1998a,Ada2001}.

Finally, the treatment of the time dependence of the TRT equations is a lengthy
topic \cite{Low2004}. As in other partial differential equations, the
approximation made to the
$\tpder{I}{t}$ term (or to the time average of the other terms) incurs a
discretization error of $O(\Delta_t)$ in the case of forward and backward Euler,
or $O(\Delta_t^2)$ and higher in other high-order discretization schemes.
Because of the stiff nature of the TRT equations \cite{Kno2003}, unstable explicit
methods such as forward Euler are practically unusable in this application.
Furthermore, the operator split in the semi-implicit treatment produces an
$O(\Delta_t)$ linearization error. Typically, then,
higher-order methods require iteration on the nonlinearities or the
application of Jacobi-free Newton--Krylov techniques. In our work, we will use
only the simple semi-implicit linearization with the backward Euler implicit time
discretization, so our discussion of the methods in this section will represent
the transport equation in the linearized form
\begin{equation}\label{eq:siTransport}
  \frac{1}{c} \frac{I^{n+1} - I^n}{\Delta_t^n}
  + \vec{\Omega} \vd \del I^{n+1}
  + \sigma^n I^{n+1}
  = \frac{1}{4\pi} \sigma_\mathrm{es}^n \phi^{n+1}
  + \frac{1}{4\pi} Q^n\,.
\end{equation}

%%%%%%%%%%%%%%%%%%%%%%%%%%%%%%%%%%%%%%%%
\subsection{Discrete ordinates}

The discrete ordinates (\SN) method approximates the angular dependence of the
intensity with a quadrature set of $M$ angles $\vec{\Omega}_m$ with weights $w_m$:
\begin{equation*}
  I(\vec{x},\vec{\Omega},t) \approx \sum_{m=1}^M I_m(\vec{x},t) \vec{\Omega}_m w_m\,.
\end{equation*}
Applying the angular approximation to Eq.~\eqref{eq:siTransport}, we get $N$
equations 
\begin{equation}\label{eq:siTransport}
  \frac{I_{m}^{n+1} - I_{m}^n}{c \Delta_t^n}
  + \vec{\Omega}_{m} \vd \del I_{m}^{n+1}
  + \sigma^n I^{n+1}
  = \frac{1}{4\pi} \sigma_\mathrm{es}^n \phi^{n+1}
  + \frac{1}{4\pi} Q^n\,.
\end{equation}
The equations for each equation are coupled at every point in space through the
scalar intensity in the scattering term:
\begin{equation*}
  \phi^{n+1} = \sum_{m=1}^M I_m^{n+1} w_m \,.
\end{equation*}
The equations can also be coupled at the boundaries of a problem, e.g.~via
specular reflection.

To interpret Eq.~\eqref{eq:siTransport} as a linear algebraic expression, we
rearrange it slightly:
\begin{equation*}
  \left( \vec{\Omega}_{m} \vd \del + \frac{1}{c\Delta_t^n}
  + \sigma^n \right) I_{m}^{n+1}
  = \frac{1}{c\Delta_t^n} I_{m}^n
  + \frac{1}{4\pi} \sigma_\mathrm{es}^n \phi^{n+1}
  + \frac{1}{4\pi} Q^n
\end{equation*}
or, combining all $M$ equations into abstract linear operators (similar to
\cite{War2004}),
\begin{equation*}
  L \psi = S \psi + q\,,
\end{equation*}
where $L$ is the operator representing ``streaming plus collision'' on the left
hand side, $S$ represents the isotropic redistribution of photons via
scattering, $\psi$ is the vector of unknown angular intensity at the new time
step, and $q$ is the isotropic source plus the initial condition. At every time
step, the \SN\ equations are typically solved iteratively with Richardson
iteration (or ``source iteration'') \cite{Lew1984},
\begin{equation*}
  \psi^{(k+1)} = L\inv ( S \psi^{(k)} + q )\,.
\end{equation*}
Every application of $L\inv$ to the unknowns is known as a ``transport sweep,''
as it normally implemented not as an explicit matrix of unknowns but rather in
an algorithm that sweeps across a mesh, progressively solving the transport
equation in each cell.
In the absence of scattering ($\sigma_\mathrm{es}^n=0$) and without boundaries,
the \SN\ equations can be solved in only one sweep, which is particularly
advantageous for the anisotropic diffusion solution process.

In highly scattering systems, \SN\ can take an arbitrarily long time to converge
using pure source iteration. Means of overcoming this, including diffusion
synthetic acceleration, multigrid treatments, and Krylov methods, are well
outside the scope of this background chapter. Our primary use of the \SN\ method
as presented here is to visualize the angular dependence of the intensity at
different points in space-time.

%%%%%%%%%%%%%%%%%%%%%%%%%%%%%%%%%%%%%%%%
\subsection{Spherical harmonics}\label{sec:bgPn}
The spherical harmonics (\PN) method takes a very different approach to
approximating the angular dependence of the intensity. It expresses the
intensity as a truncated series of angular moments,
\begin{align*}
  I(\vec{x},\vec{\Omega},t)
  &\approx \sum_{l=0}^{L} \sum_{m=-l}^{l} Y_{l,m}(\vec{\Omega}) \left[
  \int_{4\pi} Y_{l,m}^\ast(\vec{\Omega}') I(\vec{x},\vec{\Omega}',t) \ud \Omega'
  \right]
  \\
  &= \frac{1}{4\pi} \phi(\vec{x}, t) + \frac{3}{4\pi} \vec{\Omega}\vd
  \vec{F}(\vec{x},t) + \cdots\,,
\end{align*}
where $Y_{l,m}(\vec{\Omega})$ and $Y_{l,m}^*(\vec{\Omega})$ are the spherical
harmonic functions and their complex conjugates \cite{McC2008a,Lar2007}.

The full spherical harmonic equations \cite{McC2007} are lengthy, complicated, and
largely irrelevant to our work. However, the \Pone\ equations, which result from
taking $L=1$, are relevant. The first step in their derivation is to take the
zeroth and first angular moments of Eq.~\eqref{eq:siTransport}. The zeroth
angular moment of the linearized transport equation is
\begin{equation}\label{eq:p1ZerothMoment}
  \frac{\phi^{n+1} - \phi^n }{c \Delta_t}
  + \grad \vd \vec{F}
  + \sigma_\mathrm{ea}^n \phi^{n+1}
  = Q^n\,,
\end{equation}
and the first moment is the vector equation
\begin{equation}\label{eq:p1FirstMoment}
  \frac{\vec{F}^{n+1} - \vec{F}^n }{c \Delta_t} + \grad \vd \int_{4\pi}
  \vec{\Omega}\vec{\Omega} I^{n+1} \ud\Omega
  + \sigma^n \vec{F}^{n+1}
  = 0\,.
\end{equation}

The \Pone\ approximation provides a closure for Eq.~\eqref{eq:p1FirstMoment} by
approximating the intensity as
\begin{equation}\label{eq:p1Approx}
  I^{n+1}(\vec{x},\vec{\Omega})
  \approx \frac{1}{4\pi} \phi^{n+1}(\vec{x})
  + \frac{3}{4\pi} \vec{\Omega}\vd \vec{F}^{n+1}(\vec{x}) \,,
\end{equation}
so the second angular moment is
\begin{equation*}
  \int_{4\pi} \vec{\Omega}\vec{\Omega} I^{n+1} \ud\Omega
  \approx \frac{1}{4\pi} \phi^{n+1}
  \int_{4\pi} \vec{\Omega}\vec{\Omega}\ud\Omega + 0
  = \frac{4\pi}{3} \Identitytens \phi^{n+1}\,,
\end{equation*}
where $\Identitytens$ is the identity matrix (the unit dyad). Thus
Eq.~\eqref{eq:p1FirstMoment} becomes
\begin{equation} \label{eq:p1FirstMoment2}
  \frac{\vec{F}^{n+1} - \vec{F}^n }{c \Delta_t} + \frac{1}{3} \grad \phi^{n+1}
  + \sigma^n \vec{F}^{n+1}
  = 0\,.
\end{equation}

Rather than the $M$ unknowns per spatial coordinate of the \SN\ method, the
\Pone has four (or, in 2-D space, three). It is therefore easier to solve and
store in computer memory. The disadvantages of \Pone\ are pronounced, however.
First, most obviously, the reduction in unknowns means the solution will almost
always lag in accuracy compared to a transport method. Second, it happens that
the approximation in Eq.~\eqref{eq:p1FirstMoment2} yields a plane wave
propagation speed of $c/\sqrt{3}$ rather than the physical $c$
\cite{Mih1984,War2002}. Even worse, the \Pone\ equation and in fact the entire
\PN\ family can produce negative solutions for $\phi^{n+1}$ in the presence of
steep gradients \cite{Bru2002,McC2007}. This is a particular problem in TRT
applications because a negative $\phi$ can lead to a negative temperature and
the catastrophic failure of a simulation.

An alternative to using the linear-in-angle approximation in the closure for
$\int_{4\pi} \vec{\Omega}\vec{\Omega} I^{n+1} \ud\Omega$ is to define an
Eddington tensor \cite{Pom1982,Ols2000} as
\begin{equation*}
  \Etens \equiv \frac{\int_{4\pi} \vec{\Omega}\vec{\Omega} I^{n+1}
  \ud\Omega}{\int_{4\pi} I^{n+1} \ud\Omega}\,,
\end{equation*}
so that Eq.~\eqref{eq:p1FirstMoment} becomes
\begin{equation*}
  \frac{\vec{F}^{n+1} - \vec{F}^n }{c \Delta_t}
  + \grad \vd \left( \Etens\phi^{n+1} \right)
  + \sigma^n \vec{F}^{n+1}
  = 0\,.
\end{equation*}
Typically, $\Etens$ is calculated via a transport problem or some \emph{a
priori} relation based on $\phi$ and its derivatives.
The variable Eddington factor, or quasidiffusion, family of methods can be much
more accurate than the \Pone\ equations, but the nonlinear closure for $\Etens$
can lead to nonphysical shocks and instabilities.

%%%%%%%%%%%%%%%%%%%%%%%%%%%%%%%%%%%%%%%%
\subsection{Diffusion}\label{sec:bgDiffusion}

After the \Pone\ approximation is applied to the first angular moment of the
transport equation, Eq.~\eqref{eq:p1Approx}, one further simplification leads to
the very common diffusion approximation. The ``quasi-static'' \cite{Dud1976}
approximation is to discard the time derivative term in
Eq.~\eqref{eq:p1FirstMoment}, simplifying the expression and allowing for an
explicit solution of $\vec{F}^{n+1}$ to yield Fick's law:
\begin{equation*}
  \vec{F}^{n+1} = - \frac{1}{3\sigma^n} \grad \phi^{n+1}\,.
\end{equation*}
This simple expression approximates the entire angular distribution of the
intensity $I$ with a single unknown, yielding a very small memory footprint and
typically very fast solution.

However, Fick's law is only a very coarse approximation, and it is not
necessarily accurate. An asymptotic analysis \cite{Lar1975,Lar1983a} shows that
the diffusion approximation actually is a leading
order solution of the transport equation given certain conditions---%
primarily, that the material properties vary slowly in space, and that the
system is highly scattering. For problems that have a fast time scale or have
strong absorbers, the diffusion approximation will not yield transport-quality
answers.

One particular property of the diffusion approximation in time-dependent
applications is that it yields a parabolic equation, allowing energy to be
transferred faster than the speed of light.

%%%%%%%%%%%%%%%%%%%%%%%%%%%%%%%%%%%%%%%%
\subsection{Flux-limited diffusion}\label{sec:bgFld}

The exact radiation intensity satisfies the mathematical identity $\norm{F} \le
\phi$,
essentially limiting the leakage of radiation at a point to the radiation
energy at that point. The identity can be proven using the triangle inequality:
\begin{align}
  \norm{\vec{F}} &= \norm{ \int_{4\pi} \vec{\Omega} I \ud \Omega}
  \\ \nonumber
  &\le \int_{4\pi} \norm{\vec{\Omega}} \abs{I} \ud \Omega 
  \\ \nonumber
  &\le \int_{4\pi} [1] I \ud\Omega
  \\ \nonumber
  \norm{\vec{F}} &\le \phi\,.
  \\ 
  \intertext{Substituting Fick's law for $\vec{F}$ gives a condition that the
  diffusion coefficient should ideally satisfy in the presence of large
  gradients:} \nonumber
  \norm{- D \grad \phi} &\le \phi
  \\ \label{eq:fluxLimit}
  D & \le \frac{\phi}{\norm{\grad \phi}} \,.
\end{align}
In a void, where $\sigma\approx 0$, this condition tends to be violated in
time-dependent problems.

A ``flux limiter'' is designed to combat this problem in an \emph{ad hoc} but
effective manner. Although certain flux limiters \cite{Lev1984} are based on
idealized forms that the radiation intensity might take where
Eq.~\eqref{eq:fluxLimit} tends to be violated, the usual approach is more
straightforward. The flux-limited diffusion (FLD) coefficient should assume the
standard diffusion coefficient $1/3\sigma$ where the spatial gradients are weak,
but it should satisfy Eq.~\eqref{eq:fluxLimit} when the gradients are strong.
A standard formulation \cite{Ols2000} is
\begin{equation} \label{eq:fluxLimiter}
  D = \left[ (3\sigma)^{m} + \left( \frac{\norm{\grad
  \phi}}{\phi} \right)^{m}\right]^{-1/m} \,.
\end{equation}
The ``sum'' limiter at $m=1$ due to Wilson \cite{Mor2000} leads to inaccuracies
at the diffusion limit, but using the ``square-root'' limiter at $m=2$ (as
suggested by Larsen \cite{Ols2000}) is accurate to leading order. Taking
$n\to\infty$ leads to the ``max'' limiter, which is also accurate in the
diffusion limit but has discontinuous derivatives.

Typically, Eq.~\eqref{eq:fluxLimiter} is evaluated explicitly (``lagged'')
because of its inherent nonlinearity, giving the FLD version of Fick's law:
\begin{equation*}
  \vec{F}^{n+1} \approx - \left[ (3\sigma)^{m} + \left( \frac{\norm{\grad
  \phi^{n}}}{\phi^{n}} \right)^{m}\right]^{-1/m} \grad \phi^{n+1} \,.
\end{equation*}

The discretization of the gradient in Eq.~\eqref{eq:fluxLimiter} incidentally
has a strong effect on the solution \cite{Ols2007}. The implementation of FLD
used in this thesis takes a geometric average of the normalized gradient on the
face of a computational cell:
\begin{multline*}
 \frac{1}{\phi} \frac{\partial \phi}{\partial x}
  \approx
\sqrt{\abs{ \left(  \frac{1}{(\phi_{i+1} + \phi_{i})/2} \frac{\phi_{i+1} - \phi_{i}}
 { (\Delta_{x,i+1}/2 + \Delta_{x,i}/2)} \right)
\left(  \frac{1}{(\phi_{i} + \phi_{i-1})/2} \frac{\phi_{i} - \phi_{i-1}}
{ (\Delta_{x,i}/2 + \Delta_{x,i-1}/2)} \right) }}
\\
\times
 \mathrm{sgn}(\phi_{i+1} - \phi_{i-1})\,,
\end{multline*}
rather than the arithmetic average
\begin{equation*}
 \frac{1}{\phi} \frac{\partial \phi}{\partial x}
  \approx
 \frac{1}{\phi_{i}} \frac{\phi_{i+1} - \phi_{i-1}}
 { (\Delta_{x,i+1}/2 + \Delta_{x,i} + \Delta_{x,i-1}/2)}\,.
\end{equation*}

%%%%%%%%%%%%%%%%%%%%%%%%%%%%%%%%%%%%%%%%%%%%%%%%%%%%%%%%%%%%%%%%%%%%%%%%%%%%%%%%
\section{Conclusions}

In this chapter, we gave an overview of the equations underlying thermal
radiative transfer, as well as some existing techniques for solving the
equations. We presented some of the strengths and weaknesses of several
deterministic methods that will compete with our anisotropic diffusion methods,
for better methods are developed by better understanding of deficiencies.

