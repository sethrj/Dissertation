\documentclass[11pt,letter,twoside]{mc2011}
%%%%%%%%%%%%%%%%%%%%%%%%%%%%%%%%%%%%%%%%%%%%%%%%%%%%%%%%%%%%%%%%%%%%%%%%%%%%
\usepackage{bm}
\usepackage{amsmath}
\usepackage{amssymb}
\usepackage{microtype}
\usepackage{booktabs} % \toprule, \midrule, \bottomrule
/Users/seth/Documents/Compositions/SRJinclude.tex
\newcommand{\epsiloncolor}[1]{#1}
%%%%%%%%%%%%%%%%%%%%%%%%%%%%%%%%%%%%%%%%%%%%%%%%%%%%%%%%%%%%%%%%%%%%%%%%%%%%

\usepackage[numbers,sort&compress]{natbib}

\usepackage{fancyhdr}
\usepackage{lastpage}
\usepackage{graphicx}

\pagestyle{fancy}

\globalmc2011

%%%%%%%%%%%%%%%%%%%%%%%%%%%%%%%%%%%%%%%%%%%%%%%%%%%%%%%%%%%%%%%%%%%%%%%%%%%%
\begin{document}

\title{A Physics-based Anisotropic Diffusion Method for Thermal Radiative
Transfer}

\author{
\textbf{Seth R.~Johnson and Edward W.~Larsen}\\
Department of Nuclear Engineering \& Radiological Sciences\\
University of Michigan \\
2355 Bonisteel Boulevard, Ann Arbor, MI, 48109\\
sethrj@umich.edu; edlarsen@umich.edu
}

\maketitle

\thispagestyle{empty}

\begin{abstract}
This paper describes an anisotropic diffusion (AD) method that uses
transport-calculated AD coefficients to efficiently and accurately solve the
thermal radiative transfer (TRT) equations. By assuming weak gradients and angular moments in the radiation
intensity, we derive an expression for the radiation energy density that
depends on a non-local function of the opacity that yields an approximate
expression for the intensity that does not assume linearity in angle. The AD
coefficients can be solved with a single steady-state transport sweep once per
time step.
To demonstrate the AD method's efficacy, we model radiation flow down a tube
in ``flatland'' geometry. 

\keywords{Thermal radiative transfer, Anisotropic diffusion, Flux-limited
diffusion, Hybrid methods, Flatland geometry}

\end{abstract}

\newcommand\authorname{Seth~R.~Johnson and Edward~W.~Larsen}
\newcommand\shorttitlename{A physics-based anisotropic diffusion method for TRT}

\fancymc2011

%%%%%%%%%%%%%%%%%%%%%%%%%%%%%%%%%%%%%%%%%%%%%%%%%%%%%%%%%%%%%%%%%%%%%%%%%%%%
\section{Introduction}
Thermal radiative transfer (TRT) is the nonlinear process describing the
dominant form of energy transfer in a very hot material, such as the interior
of a star or the
target of a laser-driven shock experiment. The equations describing TRT are
time-dependent, contain strong nonlinearities, and reside in a large phase
space $(\vec{x}, \vec{\Omega}, h\nu, t)$.
These difficulties make TRT the subject of significant work in methods
development, which typically balances fidelity against computational cost. 
Among high-fidelity methods are Fleck and Cummings' Implicit Monte Carlo
(IMC) method \cite{Fle1971} and the discrete ordinates (\SN) method
\cite{Mor1996}. Both of these methods require large amounts of computer time:
IMC requires large numbers of particles to accurately sample the large phase
space, and \SN\ has to iteratively ``sweep'' through all spatial cells and all
angles in the quadrature set. Furthermore, the requirement of storing the full
time-dependent angular intensity imposes a heavy burden on computer memory: IMC
must store millions of particles in a ``census'' at the end of the time step,
and \SN\ must store the calculated angular flux in all angles at all spatial
points.

This paper describes a new hybrid method that uses transport-calculated
anisotropic diffusion coefficients to efficiently and accurately solve the TRT
equations. By making assumptions about the strength of the gradients and
angular moment of the radiative intensity $I$, we systematically derive an
approximate expression for the radiation energy density $E$. The result is an
expression for the radiation flux, a function of the local radiation energy
gradient and a nonlocal anisotropic diffusion tensor. The AD tensor is the
second angular moment of a steady-state, purely absorbing transport problem
with a uniform source.

%%%%%%%%%%%%%%%%%%%%%%%%%%%%%%%%%%%%%%%%%%%%%%%%%%%%%%%%%%%%%%%%%%%%%%%%%%%%
\section{Theory}
We consider the gray TRT equations, which eliminate the energy unknown from the
full TRT description by approximating the energy dependence of the opacity
$\sigma$ and integrating over all energies. 
\begin{subequations} \label{eqs:fullGrayTRT}
The radiation field is described by a monoenergetic Boltzmann transport
equation:
\begin{equation} \label{eq:fullGrayTransport}
  \frac{1}{c} \pder{I}{t}(\vec{x}, \vec{\Omega}, t)
  + \vec{\Omega} \vd \del I(\vec{x}, \vec{\Omega}, t) +
 \sigma(\vec{x}, T) I(\vec{x}, \vec{\Omega}, t)
  = \frac{\sigma(\vec{x}, T) a c [T(\vec{x},t)]^4}{4\pi} 
  + \frac{c Q(\vec{x},t)}{4\pi}
\end{equation}
for $\vec{x} \in V$, $\vec{\Omega} \in 4\pi$, and $t >= 0$. The material energy
equation describes the time rate of change in the material energy:
\begin{equation} \label{eq:fullGrayMaterial}
  \pder{U_m}{t}(\vec{x}, t)
  = \sigma(\vec{x}, T) \int_{4\pi}  I \ud \Omega
    - \sigma(\vec{x}, T) ac [T(\vec{x},t)]^4 
  = c \sigma(\vec{x}, T) \left[
  E(\vec{x},  t) - B(\vec{x}, t) \right]\,.
\end{equation}
\end{subequations}
Here, $E=\frac1c \int_{4\pi} I \ud \Omega$ is the radiation energy density and
$B=aT^4$ is the Planck function integrated over all energy. The material energy
density $U_m$ is related to the temperature by 
\begin{equation} \label{eq:matEnergyDens}
  U_m(\vec{x},T) = \int_{0}^{T} c_v(\vec{x},T') \ud T' \,.
\end{equation}
%The nonlinear Planck function $B$ and the generally nonlinear opacity $\sigma$ 

Integrating Eq.~\eqref{eq:fullGrayTransport} over all angles gives an
expression for the conservation of energy in the radiation field:
\begin{equation} \label{eq:grayTransportZeroth}
  \pder{E}{t}(\vec{x}, t)
  + c\del \vd \vec{F}(\vec{x}, t) +
 c \sigma(\vec{x}, T) E(\vec{x}, t)
  = c \sigma(\vec{x}, T) B(\vec{x},t) + cQ(\vec{x},t) \,.
\end{equation}
$\vec{F}(\vec{x}, t)$ is the radiation flux, analogous to the ``current'' of
the neutron transport world.

To begin, we informally make some assumptions about the strength of the
derivatives and the angular moments of $I$:
\begin{align*}
  I &= O(\epsiloncolor{1}), &
  \sigma &= O(\epsiloncolor{1}), &
  \del I &= O(\epsiloncolor{\epsilon}), &
  \frac1c\pder{I}{t} &= O(\epsiloncolor{\epsilon}), &
  \frac1c\pder{\sigma}{t} &= O(\epsiloncolor{\epsilon}), &
  \int_{4\pi} \vec{\Omega} I\ud \Omega &= O(\epsiloncolor{\epsilon}).
\end{align*}
These assumptions are different (and less rigorous) than the set traditionally
used to derive the radiation diffusion equations \cite{Lar1983a}. They will
lead to an approximate expression for the angular intensity that is
\emph{not} linear in angle.

%%%%%%%%%%%%%%%%%%%%%%%%%%%%%%%%%%%%%%%%%%%%%%%%%%%%%%%%%%%%%%%%%%%%%%%%%%%%
\section{Numerical Results}

%%%%%%%%%%%%%%%%%%%%%%%%%%%%%%%%%%%%%%%%%%%%%%%%%%%%%%%%%%%%%%%%%%%%%%%%%%%%
\subsection{Methods compared}
%%%%%%%%%%%%%%%%%%%%%%%%%%%%%%%%%%%%%%%%%%%%%%%%%%%%%%%%%%%%%%%%%%%%%%%%%%%%
\subsection{Problem description}
%%%%%%%%%%%%%%%%%%%%%%%%%%%%%%%%%%%%%%%%%%%%%%%%%%%%%%%%%%%%%%%%%%%%%%%%%%%%
\subsection{Results}

%%%%%%%%%%%%%%%%%%%%%%%%%%%%%%%%%%%%%%%%%%%%%%%%%%%%%%%%%%%%%%%%%%%%%%%%%%%%
\section{Conclusions}


%%%%%%%%%%%%%%%%%%%%%%%%%%%%%%%%%%%%%%%%%%%%%%%%%%%%%%%%%%%%%%%%%%%%%%%%%%%%
\section*{Acknowledgements}
This material is based upon work supported under a National Science Foundation
Graduate Research Fellowship and a Department of Energy Nuclear
Energy University Programs Graduate Fellowship.

%%%%%%%%%%%%%%%%%%%%%%%%%%%%%%%%%%%%%%%%%%%%%%%%%%%%%%%%%%%%%%%%%%%%%%%%%%%%
\nocite{Mih1984}
\bibliographystyle{ans}
\bibliography{../SRJall}

%%%%%%%%%%%%%%%%%%%%%%%%%%%%%%%%%%%%%%%%%%%%%%%%%%%%%%%%%%%%%%%%%%%%%%%%%%%%
\end{document}
