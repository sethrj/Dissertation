\documentclass[11pt,letter,twoside]{mc2011}

\usepackage[numbers,sort&compress]{natbib}

\usepackage{fancyhdr}
\usepackage{lastpage}
\usepackage{graphicx}

\pagestyle{fancy}

\globalmc2011

\begin{document}

\title{A Physics-based Anisotropic Diffusion Method for Thermal Radiative
Transfer}

\author{
\textbf{Author A and Author B\footnote{Footnote, if necessary, in Times New Roman and font size 9}}\\
Name of Institute\\
Corresponding Address\\
A@institute.gov; B@institute.gov \vspace{1.0em} \\ 
\textbf{Double space and list Author C}\\
Department of Nuclear Engineering\\
Name of University\\
Address\\
c@name.univ.edu
}

\maketitle

\thispagestyle{empty}

\begin{abstract}
Use 8.5 x 11 paper size, with 1" margins on all sides. A required 200-250 word abstract starts on this line.  
Leave two blank lines before ABSTRACT and one after.  Use 10 point Times New Roman here and single 
(10 point) spacing.  The abstract is a very brief summary highlighting main accomplishments, what is new, 
and how it relates to the state-of-the-art.

\keywords{List of minimum 3 and maximum 6 key words separated by commas.}

\end{abstract}

\newcommand\authorname{Seth~R.~Johnson and Edward~W.~Larsen}
\newcommand\shorttitlename{A physics-based anisotropic diffusion method for TRT}

\fancymc2011

%%%%%%%%%%%%%%%%%%%%%%%%%%%%%%%%%%%%%%%%%%%%%%%%%%%%%%%%%%%%%%%%%%%%%%%%%%%%
\section{Introduction}

%%%%%%%%%%%%%%%%%%%%%%%%%%%%%%%%%%%%%%%%%%%%%%%%%%%%%%%%%%%%%%%%%%%%%%%%%%%%
\section{Theory}

%%%%%%%%%%%%%%%%%%%%%%%%%%%%%%%%%%%%%%%%%%%%%%%%%%%%%%%%%%%%%%%%%%%%%%%%%%%%
\section{Results}

%%%%%%%%%%%%%%%%%%%%%%%%%%%%%%%%%%%%%%%%%%%%%%%%%%%%%%%%%%%%%%%%%%%%%%%%%%%%
\section{Conclusions}


%%%%%%%%%%%%%%%%%%%%%%%%%%%%%%%%%%%%%%%%%%%%%%%%%%%%%%%%%%%%%%%%%%%%%%%%%%%%
\section*{Acknowledgements}

%%%%%%%%%%%%%%%%%%%%%%%%%%%%%%%%%%%%%%%%%%%%%%%%%%%%%%%%%%%%%%%%%%%%%%%%%%%%
\bibliographystyle{ans}
\bibliography{../SRJall}

%%%%%%%%%%%%%%%%%%%%%%%%%%%%%%%%%%%%%%%%%%%%%%%%%%%%%%%%%%%%%%%%%%%%%%%%%%%%
\end{document}
