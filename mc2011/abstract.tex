\documentclass[11pt,letter,twoside]{mc2011}

\globalabst

\begin{document}

\title{An Anisotropic Diffusion Approximation to Thermal Radiative
Transfer}

\author{
\textbf{Seth R.~Johnson and Edward W.~Larsen}\\
Department of Nuclear Engineering \& Radiological Sciences\\
University of Michigan \\
2355 Bonisteel Boulevard, Ann Arbor, MI, 48109\\
sethrj@umich.edu; edlarsen@umich.edu
}

\maketitle

\thispagestyle{empty}

\section*{ABSTRACT}

\small

This paper describes an anisotropic diffusion (AD) method that uses
transport-calculated AD coefficients to efficiently and accurately solve the
thermal radiative transfer equations. By assuming weak gradients and angular moments in the radiation
intensity, we derive an expression for the radiation energy density that
depends on a non-local function of the opacity. This nonlocal function is the
solution of a transport equation that can be solved with a single steady-state
transport sweep once per time step, and the function's second angular moment is
the anisotropic diffusion tensor.
To demonstrate the AD method's efficacy, we model radiation flow down a channel 
in ``flatland'' geometry. 

\keywords{Anisotropic diffusion, Thermal radiative transfer, Flux-limited
diffusion, Hybrid methods, Flatland geometry}


\end{document}
