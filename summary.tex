\documentclass[11pt]{SRJresearch}
\author{Seth R.~Johnson}
\date{\today}
\title{Research summary}

\newcommand{\Dtens}{\mat{D}}

\begin{document}
%%%%%%%%%%%%%%%%%%%%%%%%%%%%%%%%%%%%%%%%%%%%%%%%%%%%%%%%%%%%%%%%%%%%%%%%%%%%%%%%
This is a list of what I've done, what I'm currently working on, and what I've
left to do.

%%%%%%%%%%%%%%%%%%%%%%%%%%%%%%%%%%%%%%%%%%%%%%%%%%%%%%%%%%%%%%%%%%%%%%%%%%%%%%%
\section{Done}
\subsection{Theory}
\begin{itemize}
  \item Discretization schemes: Gol'din, nine-point stencil, alternate
    nine-point stencil
  \item Realization that AD coefficients are continuous, so first derivative of
    scalar intensity is continuous (no kinks)
  \item Semi-implicit anisotropic diffusion approximation
  \item Method of moments (ultimately fruitless)
\end{itemize}

\subsection{Implementation}
\begin{itemize}
  \item Lots of already existing methods (IMC, MC, SN, D, P1, FLD, etc.)
  \item Lots of analysis tools (lineout, angleout, wavefront detection, Silo)
  \item Transport-calculated AD coefficients (MOC, DD)
  \item Nine-point stencils for AD
  \item Diffusion-matched boundary conditions
\end{itemize}

\subsection{Analysis}
\begin{itemize}
  \item Looked at off-diagonal terms of $\Dtens$ which were neglected in
    previous AD work (dogleg problem)
  \item Compared angular intensity representation for different methods (AD,
    SN, D, FLD)
  \item Test problem for non-orthogonal channels
\end{itemize}

%%%%%%%%%%%%%%%%%%%%%%%%%%%%%%%%%%%%%%%%%%%%%%%%%%%%%%%%%%%%%%%%%%%%%%%%%%%%%%%
\section{In progress}

%%%%%%%%%%%%%%%%%%%%%%%%%%%%%%%%%%%%%%%%%%%%%%%%%%%%%%%%%%%%%%%%%%%%%%%%%%%%%%%
\section{Needs to be done}

\subsection{Theory}
\begin{itemize}
  \item Time-dependent P1, including elimination of face-centered terms?
  \item Boundary conditions
  \item Higher-order approximations with powers of $s$
\end{itemize}

\subsection{Implementation}
\begin{itemize}
  \item Coarse-grid AD coefficient calculation using multigrid routines
  \item Flux-limited AD
\end{itemize}

\subsection{Analysis}
\begin{itemize}
  \item CRASH-like test problem
  \item Boundary conditions in, e.g., the pipe problem
  \item Comparison using wavefront detection
\end{itemize}


%%%%%%%%%%%%%%%%%%%%%%%%%%%%%%%%%%%%%%%%%%%%%%%%%%%%%%%%%%%%%%%%%%%%%%%%%%%%%%%
\end{document}

