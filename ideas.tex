\documentclass[11pt]{SRJresearch}
\author{Seth R.~Johnson}
\date{\today}
\title{Summary of work and ideas}

\begin{document}
%%%%%%%%%%%%%%%%%%%%%%%%%%%%%%%%%%%%%%%%%%%%%%%%%%%%%%%%%%%%%%%%%%%%%%%%%%%%%%%%
\section{2010/01/28}
\subsection{Summary of week's work}
I began with understanding Larsen's anisotropic diffusion method
\cite{Lar2009c}: what approximations he made, what approximations he didn't
make, etc. My conclusion is that he is a magician.

The first thing I went to was the fully time-dependent integral transport
equation \cite{Pri2010}. I played around with it and tried to see what would
happen if we try to look at an infinite medium in a small temporal range about
the current time. 

I implemented a 2-D version of my mesh code, and it works great. I also
implemented output for VisIt with Silo, and managed to get a contour plot with
Matplotlib.

\subsection{Ideas this week}
If we look at the exact time-dependent problem first, what are the
qualifications for being ``diffusive''?  What are the approximations that lead
to diffusion from the regular transport equation? Densmore's asymptotic paper
\cite{Den2004} gives a good summary: the speed of light is fast ($c \to
c/\eps$), opacities are large ($\sigma \to \sigma/\eps$), and small amounts of
absorption result in large changes in temperature ($c_v \to \eps c_v$). Do we
have to make all these approximations when deriving the TRT AD method?

We need to look at the scattering ratio $f$, opacity $\sigma$, current
$\vec{J}$, and scalar flux $\phi$ as a function of time near a Marshak wave.
What assumptions can we make about these functions and their derivatives with
respect to space and time? They usually have strong spatial gradients, and they
basically always change quickly with respect to time.

It is very unlikely that extra terms from the first moment can show in the
zeroth moment without the potential for negative answers.

If I come up with a time-dependent current term, or whatever, we need to see if
$\tpder{\vec{J}}t$ is zero.


\subsection{To do}

%%%%%%%%%%%%%%%%%%%%%%%%%%%%%%%%%%%%%%%%%%%%%%%%%%%%%%%%%%%%%%%%%%%%%%%%%%%%%%%
\bibliographystyle{amsalpha}
\bibliography{SRJall}
\end{document}
