\documentclass[11pt]{SRJresearch}
\author{Seth R.~Johnson}
\date{\today}
\title{Summary of work and ideas}

\begin{document}
%%%%%%%%%%%%%%%%%%%%%%%%%%%%%%%%%%%%%%%%%%%%%%%%%%%%%%%%%%%%%%%%%%%%%%%%%%%%%%%%
This document is supposed to give some organization to my research. Too often,
ideas and thoughts will get lost amidst a cascade of scratch paper.
Additionally, it is too easy to forget what one has accomplished on a
month-to-month basis, and it is too easy to accomplish little without proper
motivation.


%%%%%%%%%%%%%%%%%%%%%%%%%%%%%%%%%%%%%%%%%%%%%%%%%%%%%%%%%%%%%%%%%%%%%%%%%%%%%%%
\section{2010/01/28}
\subsection{Summary of week's work}
I began with understanding Larsen's anisotropic diffusion method
\cite{Lar2009c}: what approximations he made, what approximations he didn't
make, etc. My conclusion is that he is a magician.

The first thing I went to was the fully time-dependent integral transport
equation \cite{Pri2010}. I played around with it and tried to see what would
happen if we try to look at an infinite medium in a small temporal range about
the current time. 

I implemented a 2-D version of my mesh code, and it works great. I also
implemented output for VisIt with Silo, and managed to get a contour plot with
Matplotlib.

\subsection{Ideas this week}
\begin{enumerate}
  \item 
If we look at the exact time-dependent problem first, what are the
qualifications for being ``diffusive''?  What are the approximations that lead
to diffusion from the regular transport equation?
Do we have to make all the same approximations when deriving the TRT AD method?

  \item 
We need to look at the scattering ratio $f$, opacity $\sigma$, current
$\vec{J}$, and scalar flux $\phi$ as a function of time near a Marshak wave.
What assumptions can we make about these functions and their derivatives with
respect to space and time? They usually have strong spatial gradients, and they
basically always change quickly with respect to time.

  \item 
It is very unlikely that extra terms from the first moment can show in the
zeroth moment without the potential for negative answers.

  \item 
If I come up with a time-dependent current term, or whatever, we need to see if
$\tpder{\vec{J}}t$ is zero.
\end{enumerate}

\subsection{To do}
\begin{itemize}
  \item Taylor series-ing stuff with the time-dependent method

  \item Time-dependent linear infinite medium setup
\end{itemize}
%%%%%%%%%%%%%%%%%%%%%%%%%%%%%%%%%%%%%%%%%%%%%%%%%%%%%%%%%%%%%%%%%%%%%%%%%%%%%%%%
\section{2010/02/04}

\subsection{Old business}

\subsection{Summary of week's work}
Starting work on 2D diffusion solver. I'll need to use CG or something rather
than the direct inversion that was done in the tridiagonal 1D case. I also
spent a lot of time with the class I'm taking

\subsection{Ideas this week}
\begin{enumerate}
\item Can we somehow make diffusion ``flux-limiting'' by modifying the transport
problem that gives us $D$?

\item What are the proper transport boundary conditions with the modified $D$?
  Are they calculated by considering the terms dropped in the integral
  transport equation?

\end{enumerate}

\subsection{To do}

%%%%%%%%%%%%%%%%%%%%%%%%%%%%%%%%%%%%%%%%%%%%%%%%%%%%%%%%%%%%%%%%%%%%%%%%%%%%%%%%
\section{2010/02/11}

\subsection{Summary of week's work}
I basically spent the whole time deriving the boundary conditions.
I learned a lot in the process, but it's not clear that what I got is right.

\subsection{Ideas this week}
\begin{enumerate}
\item 
  Since Larsen derived the method in a different way (using integration by
  parts rather than the ``probability density function'' interpretation),
  shouldn't it be possible to stick the time derivatives in there? (They'd pop
  out along with the boundary conditions, I think). We might be able to treat
  the problem with an initial layer analysis to match the problem at the end
  of long time steps.

\item
  As part of the boundary layer analysis, it was clear that the right thing to
  do in the derivation is to leave in \emph{all} the terms in $\psi$,
  discarding them only when calculating the diffusion coefficient in the
  interior. The other terms are important near the boundary but unimportant
  ($O(\epsilon)$) away from it.

\end{enumerate}

\subsection{To do}
%%%%%%%%%%%%%%%%%%%%%%%%%%%%%%%%%%%%%%%%%%%%%%%%%%%%%%%%%%%%%%%%%%%%%%%%%%%%%%%
%%%%%%%%%%%%%%%%%%%%%%%%%%%%%%%%%%%%%%%%%%%%%%%%%%%%%%%%%%%%%%%%%%%%%%%%%%%%%%%
\bibliographystyle{amsalpha}
\bibliography{SRJall}
\end{document}
