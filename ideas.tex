\documentclass[11pt]{SRJresearch}
\author{Seth R.~Johnson}
\date{\today}
\title{Summary of work and ideas}

\newcommand{\pyrev}[1]{\textsf{PyTRT} revision \texttt{#1}}
\begin{document}
%%%%%%%%%%%%%%%%%%%%%%%%%%%%%%%%%%%%%%%%%%%%%%%%%%%%%%%%%%%%%%%%%%%%%%%%%%%%%%%%
This document is supposed to give some organization to my research. Too often,
ideas and thoughts will get lost amidst a cascade of scratch paper.
Additionally, it is too easy to forget what one has accomplished on a
month-to-month basis, and it is too easy to accomplish little without proper
motivation.


%%%%%%%%%%%%%%%%%%%%%%%%%%%%%%%%%%%%%%%%%%%%%%%%%%%%%%%%%%%%%%%%%%%%%%%%%%%%%%%
\section{2010/01/28}
\subsection{Summary of week's work}
I began with understanding Larsen's anisotropic diffusion method
\cite{Lar2009c}: what approximations he made, what approximations he didn't
make, etc. My conclusion is that he is a magician.

The first thing I went to was the fully time-dependent integral transport
equation \cite{Pri2010}. I played around with it and tried to see what would
happen if we try to look at an infinite medium in a small temporal range about
the current time. 

I implemented a 2-D version of my mesh code, and it works great. I also
implemented output for VisIt with Silo, and managed to get a contour plot with
Matplotlib.

\subsection{Ideas this week}
\begin{enumerate}
  \item 
If we look at the exact time-dependent problem first, what are the
qualifications for being ``diffusive''?  What are the approximations that lead
to diffusion from the regular transport equation?
Do we have to make all the same approximations when deriving the TRT AD method?

  \item 
We need to look at the scattering ratio $f$, opacity $\sigma$, current
$\vec{J}$, and scalar flux $\phi$ as a function of time near a Marshak wave.
What assumptions can we make about these functions and their derivatives with
respect to space and time? They usually have strong spatial gradients, and they
basically always change quickly with respect to time.

  \item 
It is very unlikely that extra terms from the first moment can show in the
zeroth moment without the potential for negative answers.

  \item 
If I come up with a time-dependent current term, or whatever, we need to see if
$\tpder{\vec{J}}t$ is zero.
\end{enumerate}

\subsection{To do}
\begin{itemize}
  \item Taylor series-ing stuff with the time-dependent method

  \item Time-dependent linear infinite medium setup
\end{itemize}
%%%%%%%%%%%%%%%%%%%%%%%%%%%%%%%%%%%%%%%%%%%%%%%%%%%%%%%%%%%%%%%%%%%%%%%%%%%%%%%%
\section{2010/02/04}

\subsection{Summary of week's work}
In preparation for 2D diffusion solver, completed Silo output routines for
visualization. I'll need to use
CG or something rather
than the direct inversion that was done in the tridiagonal 1D case. I also
spent a lot of time with the class I'm taking.

\subsection{Ideas this week}
\begin{enumerate}
\item Can we somehow make diffusion ``flux-limiting'' by modifying the transport
problem that gives us $D$?

\item What are the proper transport boundary conditions with the modified $D$?
  Are they calculated by considering the terms dropped in the integral
  transport equation?

\end{enumerate}

%%%%%%%%%%%%%%%%%%%%%%%%%%%%%%%%%%%%%%%%%%%%%%%%%%%%%%%%%%%%%%%%%%%%%%%%%%%%%%%%
\section{2010/02/11}

\subsection{Summary of week's work}
I basically spent the whole time deriving the boundary conditions.
I learned a lot in the process, but it's not clear that what I got is right.
The closest thing I got to a true boundary condition is:
\begin{equation*}
  J^-(\vec{x}_b) = \tfrac14 \phi(\vec{x}_b)
\end{equation*}

Using some ridiculous asymptotic analysis stuff which I'm not sure is right, I
also calculated a requirement for the \emph{derivative} of the half-angle flux
near the boundary.

\subsection{Ideas this week}
\begin{enumerate}
\item 
  Since Larsen derived the method in a different way (using integration by
  parts rather than the ``probability density function'' interpretation),
  shouldn't it be possible to stick the time derivatives in there? (They'd pop
  out along with the boundary conditions, I think). We might be able to treat
  the problem with an initial layer analysis to match the problem at the end
  of long time steps.

\item
  As part of the boundary layer analysis, it was clear that the right thing to
  do in the derivation is to leave in \emph{all} the terms in $\psi$,
  discarding them only when calculating the diffusion coefficient in the
  interior. The other terms are important near the boundary but unimportant
  ($O(\epsilon)$) away from it.

\end{enumerate}

%%%%%%%%%%%%%%%%%%%%%%%%%%%%%%%%%%%%%%%%%%%%%%%%%%%%%%%%%%%%%%%%%%%%%%%%%%%%%%%%
\section{2010/02/18}

\subsection{Summary of week's work}
Implemented 2D cell-centered diffusion. Rewrote 1D cell-centered diffusion,
re-benchmarked, and fixed boundary conditions. Added zero-flux boundary
condition in case our derivation is right.

\subsection{Ideas this week}
\begin{enumerate}
  \item Should we compare against SP$_3$, which is actually sort of
    transport-like?
\end{enumerate}

\subsection{To do}
Boundary conditions, clarification on $D$-calculating transport problem, 1D
adjoint transport if necessary, 2D 
%%%%%%%%%%%%%%%%%%%%%%%%%%%%%%%%%%%%%%%%%%%%%%%%%%%%%%%%%%%%%%%%%%%%%%%%%%%%%%%%
\section{2010/02/25}

\subsection{Summary of week's work}
Transitioning toward full 2D transport with SN; reorganized library and added
boundary faces, and BC manager (as well as that clever functor code) etc.

\subsection{Ideas this week}
None to speak of.
%%%%%%%%%%%%%%%%%%%%%%%%%%%%%%%%%%%%%%%%%%%%%%%%%%%%%%%%%%%%%%%%%%%%%%%%%%%%%%%%
\section{2010/03/04}

\subsection{Summary of week's work}
Working on transport solvers still.

\subsection{Ideas this week}
None to speak of.

%%%%%%%%%%%%%%%%%%%%%%%%%%%%%%%%%%%%%%%%%%%%%%%%%%%%%%%%%%%%%%%%%%%%%%%%%%%%%%%%
\section{2010/03/11}

\subsection{Summary of week's work}
Wrote tensor/vector output to VisIT. Have transport solver kind of working, in
certain situations.

\subsection{Ideas this week}
In deriving discretized equations for the anisotropic diffusion tensor, I
realized that the net current across a face needs special attention. In regular
diffusion, considering leakage to the right of a cell, it's
\begin{equation*}
  \vec{n} \vd \vec{J} = - D(\vec{x}) \vec{n} \vd \del \phi(\vec{x}) 
  = -D \pder{\phi}{x}
\end{equation*}
which is calculated from the left as
\begin{equation*}
  (\vec{n} \vd \vec{J})^- = -D_{ij} \frac{\phi_{i+1/2,j} -
  \phi_{ij}}{\Delta_{x,i} /2} \,.
\end{equation*}

However, in anisotropic diffusion, it's 
\begin{equation*}
  \vec{n} \vd \vec{J} = - \vec{n} \mat{D} \del \phi(\vec{x}) 
  = -
  \begin{bmatrix} 1 & 0 \end{bmatrix}
  \begin{bmatrix}
    D_{xx} & D_{xy} \\ 
    D_{yx} & D_{yy}
  \end{bmatrix}
  \begin{bmatrix}
    \pder{\phi}{x} \\
    \pder{\phi}{y}
  \end{bmatrix}
  = - D_{xx} \pder{\phi}{x} - D_{xy} \pder{\phi}{y}
\end{equation*}
However, with cell-centered diffusion, you don't consider vertical variations
in $\phi$ when looking at the right face. So we're going to have to use some
sort of stencil to approximate $\tpder{\phi}{y}$ along that vertical face.

%%%%%%%%%%%%%%%%%%%%%%%%%%%%%%%%%%%%%%%%%%%%%%%%%%%%%%%%%%%%%%%%%%%%%%%%%%%%%%%%
\section{2010/03/18}

\subsection{Summary of week's work}
Talked to Will (Dima Anistratov's student) about quasidiffusion. His thesis
\cite{Wie2009} stores fluxes and currents on cell faces in addition to cell
centers, which we'd like to avoid.

\subsection{Ideas this week}
I just realized that the previously published work in anisotropic diffusion
\emph{only} considered analytic solutions and only for the $D_{xx}$ and
$D_{xy}$ components.

For the dog leg problem, I have plotted (Fig.~\ref{fig:offdiagFrac}) the
relative importance of the off-diagonal terms:
\begin{equation*}
  \frac{\abs{D_{xy}}}{\text{max}(D_{xx}, D_{yy})} \,.
\end{equation*}

\begin{figure}[htb]
  \centering
  \includegraphics[width=4in]{ideas_include/dogleg_offdiag_frac}
  \caption{Fractional importance of off-diagonal terms}
  \label{fig:offdiagFrac}
\end{figure}
%%%%%%%%%%%%%%%%%%%%%%%%%%%%%%%%%%%%%%%%%%%%%%%%%%%%%%%%%%%%%%%%%%%%%%%%%%%%%%%%
\section{2010/03/25}

\subsection{Summary of week's work}
Implemented a bunch of 2D/flatland Monte Carlo code. Did other fooling around
stuff with the Git repository and such. Got a copy of orthogonal-mesh
quasidiffusion derivation \cite{Val2002}.

%%%%%%%%%%%%%%%%%%%%%%%%%%%%%%%%%%%%%%%%%%%%%%%%%%%%%%%%%%%%%%%%%%%%%%%%%%%%%%%%
\section{2010/04/01}

\subsection{Summary of week's work}
Added Russian roulette combing to IMC code.

%%%%%%%%%%%%%%%%%%%%%%%%%%%%%%%%%%%%%%%%%%%%%%%%%%%%%%%%%%%%%%%%%%%%%%%%%%%%%%%%
\section{2010/04/22}

\subsection{Summary of week's work}
Fellowship application, UQ project, etc. They're all finished now. I also
worked on IMC stuff: it seems that my code has some problems. Su Olson doesn't
converge as I crank down the time step; full clip Russian roulette was the
source of some of the error. Generally, although I can't be sure, energy isn't
traveling as far as it should as quickly as it should. This looks like it was a
regression since \pyrev{698d1a0b}.

%%%%%%%%%%%%%%%%%%%%%%%%%%%%%%%%%%%%%%%%%%%%%%%%%%%%%%%%%%%%%%%%%%%%%%%%%%%%%%%%
\section{2010/04/29}

\subsection{Summary of week's work}

\subsubsection{Benchmark problem}
Looked up benchmark for radiation streaming problem \cite{Ack1989}. It probably
won't be useful because it's purely absorbing and streaming: that's a problem
for diffusion.

\subsubsection{Discretization}
Worked on anisotropic diffusion discretization. Gol'din's method
\cite{Val2002} is actually the same as regular cell-centered diffusion in 1D.
Rather than approximating the derivative $\tpder\phi x$, it approximates the
half-cell-average current.

We had earlier talked about how to calculate the transverse derivative term;
Larsen thought it would make sense, in the conservation of current term on the
boundary, to require that the ``top'' flux on the left is equal to the ``top''
flux on the right, in order that on that whole face, there could be a kink in
the flux along the $x$ axis but no discontinuity along the $y$. (If both
$\tpder \phi y$ and $\tpder \phi x$ are discontinuous along the face, then
$\phi$ will be discontinuous somewhere on the face.) After playing around with
it, I'm not convinced we should try to require that $\phi_{T-} = \phi_{T+}$. It
looks like there aren't enough degrees of freedom to do it; plus, there are
other ways that the ``continuity'' of the flux is violated even in the regular
cell-centered discretization (think of how $\tpder \phi x|_{i,j} \ne \tpder \phi
x|_{i,j+1}$).

\subsubsection{Anisotropic diffusion coefficients and code}
Worked on Larsen--Trahan channel problem. Previously, we had found that a
line-out of $D_{yy}$ wasn't showing the right coefficients in the middle of the
channel (they were too low). I surmised that the angular quadrature set wasn't
fine enough, because only the very steepest angles would be able to see the
infinite height of the channel. Today, I wrote a Python script to use a
standard numerical integration method to calculate the diffusion coefficients.
I also added the option to change the high-quality integration to a coarse
midpoint Riemann sum, which would be equivalent to a coarse-angle transport
sweep. Fig.~\ref{fig:anisotropicCalculated} shows the actual
transport-calculated coefficients (\pyrev{2e8d536c}) for
a cell width of 0.25, with an 8-angle-per-quadrant flatland quadrature set.
Fig.~\ref{fig:anisotropicAnalyticCoarse} shows the values calculated from the
analytic expressions given in \cite{Lar2009c} using the coarse integration.
(The angular integration and grid spacing are both chosen to match the
transport grid.) Clearly, they are pretty damn close. Refining the number of
points used in the Riemann sum causes the values to grow closer to the exact
value. Fig.~\ref{fig:anisotropicChannelConvergence} shows how increasing the
number of angles brings the answer closer to the true value. Using 128
azimuthal angles per quadrant still leaves a 10\% error in the diffusion
coefficient.

\begin{figure}[htb]
  \centering
  \includegraphics[width=3in]{ideas_include/anisotropic_calculated}
  \caption{Line-out of anisotropic diffusion coefficients in the channel
  problem, at $y=50$.}
  \label{fig:anisotropicCalculated}
\end{figure}

\begin{figure}[htb]
  \centering
  \includegraphics[width=3in]{ideas_include/anisotropic_analytic_coarse}
  \caption{Analytic diffusion coefficients, calculated using a midpoint Riemann
  sum with 16 points in $(0,\pi)$.}
  \label{fig:anisotropicAnalyticCoarse}
\end{figure}

\begin{figure}[htb]
  \centering
  \includegraphics[width=3in]{ideas_include/anisotropic_channel_convergence}
  \caption{Convergence of $D_{yy}(0.125)$ as the number of ``angles per
  quadrant'' in the Riemann sum increases.}
  \label{fig:anisotropicChannelConvergence}
\end{figure}

\subsection{Ideas this week}
\begin{enumerate}
  \item Will this discretization be diagonally dominant, symmetric, positive,
    \ldots? What about after applying boundary conditions? CG only works on SPD
    matrices, if I remember correctly\ldots

  \item How well will the method work if our transport problem (that calculates
    the anisotropic diffusion coefficients) is coarser than our diffusion grid?
    Are there some problems where it might be advantageous to have the grid on
    the diffusion problem be finer than the anisotropic diffusion coefficients?
\end{enumerate}

\subsection{To do}
\begin{itemize}
  \item Finish implementing anisotropic diffusion
  \item Compare solutions of transport-calculated coefficients to
    analytic-calculated coefficients
\end{itemize}
%%%%%%%%%%%%%%%%%%%%%%%%%%%%%%%%%%%%%%%%%%%%%%%%%%%%%%%%%%%%%%%%%%%%%%%%%%%%%%%
%%%%%%%%%%%%%%%%%%%%%%%%%%%%%%%%%%%%%%%%%%%%%%%%%%%%%%%%%%%%%%%%%%%%%%%%%%%%%%%
\bibliographystyle{amsalpha}
\bibliography{SRJall}
\end{document}
