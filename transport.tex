\documentclass[11pt]{SRJresearch}
\author{Seth R.~Johnson}
\date{\today}
\title{General transport}

\begin{document}
%%%%%%%%%%%%%%%%%%%%%%%%%%%%%%%%%%%%%%%%%%%%%%%%%%%%%%%%%%%%%%%%%%%%%%%%%%%%%%%%
\section{Transport equations and moments}
The well-known time-dependent transport equation is
\begin{subequations} \label{eqs:generalTransport}
\begin{multline} \label{eq:generalPsiTransport}
  \frac{1}{v} \pder{\psi}{t}(\vec{x}, \vec{\Omega}, E, t)
  + \vec{\Omega}\vd \del \psi(\vec{x}, \vec{\Omega}, E, t)
  + \Sigma_t(\vec{x}, E, t) \psi (\vec{x}, \vec{\Omega}, E, t)
\\
= \int_{0}^{\infty} \int_{4\pi} \Sigma_s(  \vec{\Omega}' \vd
\vec{\Omega}, E' \to E, t) \psi(\vec{x}, \vec{\Omega}', E', t) \ud \Omega' \ud E'
+ \frac{1}{4\pi} Q(\vec{x}, E, t)\,,
\end{multline}
for $x \in V$, $0 < E < \infty$, $0 \le t < \infty$, and $\vec{\Omega} \in
4\pi$;
with the boundary condition
\begin{equation} \label{eq:generalPsiBc}
 \psi(\vec{x}, \vec{\Omega}, E, t) =\psi^b(\vec{x}, \vec{\Omega}, E, t) \,,
 \quad \vec{x} \in \partial V, \ \vec{\Omega} \vd \vec{n} < 0,\ 0 < E < \infty,
 0 \le t < \infty
\end{equation}
and the initial condition
\begin{equation} \label{eq:generalPsiIc}
 \psi(\vec{x}, \vec{\Omega}, E, 0) =\psi^i(\vec{x}, \vec{\Omega}, E, t) \,,
 \quad \vec{x} \in V, \ \vec{\Omega} \in 4\pi,\ 0 < E < \infty\,.
\end{equation}
\end{subequations}

\subsection{Moments}
The angular moments of the transport equation are very useful. The zeroth
angular moment is the result of integrating
Eq.~\eqref{eq:generalPsiTransport} over all angles:
\begin{multline*}
  \frac{1}{v} \pder{}{t}\int_{4\pi} \psi(\vec{x}, \vec{\Omega}, E, t) \ud \Omega
  + \del \vd \int_{4\pi} \vec{\Omega}\psi(\vec{x}, \vec{\Omega}, E, t) \ud \Omega
  + \Sigma_t(\vec{x}, E, t) \int_{4\pi} \psi (\vec{x}, \vec{\Omega}, E, t)   \ud \Omega
\\
= \int_{0}^{\infty} \int_{4\pi}
\int_{4\pi} \Sigma_s(  \vec{\Omega}' \vd \vec{\Omega}, E' \to E, t) \ud \Omega
\psi(\vec{x}, \vec{\Omega}', E', t) \ud \Omega' \ud E'
+ \frac{1}{4\pi} Q(\vec{x}, E, t)\int_{4\pi}  \ud \Omega\,.
\end{multline*}
Ubiquitously in nuclear engineering, the zeroth angular moment of $\psi$ is
known as the ``scalar flux'':
\begin{subequations} \label{eqs:generalMomentDefinitions}
  \begin{equation} \label{eq:generalMomentPhi}
  \phi(\vec{x}, E, t) 
  \equiv \int_{4\pi} \psi(\vec{x}, \vec{\Omega}, E, t) \ud \Omega\,.
  \end{equation}
The ``current'' is a vector quantity, the first angular moment of $\psi$:
\begin{equation} \label{eq:generalMomentJ}
  \vec{J}(\vec{x}, E, t)
  \equiv \int_{4\pi} \vec{\Omega}\psi(\vec{x}, \vec{\Omega}, E, t) \ud \Omega\,.
\end{equation}
\end{subequations}
With these definitions, and applying some tricks to the scattering integral, the
zeroth moment of Eq.~\eqref{eq:generalPsiTransport} is
\begin{equation} \label{eq:generalZerothMoment}
  \frac{1}{v} \pder{\phi}{t} (\vec{x}, E, t)
  + \del \vd \vec{J}(\vec{x}, E, t)
  + \Sigma_t(\vec{x}, E, t) \phi(\vec{x}, E, t)
= \int_{0}^{\infty} \Sigma_{s0}(\vec{x}, E' \to E, t) 
\phi(\vec{x}, E', t) \ud E'
+ Q(\vec{x}, E, t)\,.
\end{equation}


%%%%%%%%%%%%%%%%%%%%%%%%%%%%%%%%%%%%%%%%%%%%%%%%%%%%%%%%%%%%%%%%%%%%%%%%%%%%%%%%
\section{Integral transport}
In the context of integral transport, the right hand side of
Eq.~\eqref{eq:generalPsiTransport} is rewritten to be a space-, angle-, energy-,
and time-dependent source $\hat Q$, which itself
depends on the solution $\psi$:
\begin{equation*}
  \frac{1}{v} \pder{\psi}{t}(\vec{x}, \vec{\Omega}, E, t)
  + \vec{\Omega}\vd \psi(\vec{x}, \vec{\Omega}, E, t)
  + \Sigma_t(\vec{x}, E, t) \psi (\vec{x}, \vec{\Omega}, E, t)
=  \hat Q(\vec{x}, \vec{\Omega}, E, t) \,.
\end{equation*}

It is possible to invert this equation exactly, assuming a known $Q$
\cite{Pri2010}. The exact result for the angular flux is
\begin{subequations} \label{eqs:integralAngularFlux}
  \begin{equation} \label{eq:integralAngularFluxFull}
  \begin{split}
    \psi(\vec{x}, \vec{\Omega}, E, t)
    &=
    \psi^b(\vec{x}_b, \vec{\Omega}, E, t - \norm{\vec{x} - \vec{x}_b}/v)
    \eexp^{ -\tau(\vec{x}, \vec{x}_b, \vec{\Omega}, E, t)}
    U(vt - \norm{\vec{x} - \vec{x}_b})
    \\
    &\qquad + \psi^i( \vec{x} - vt \vec{\Omega}, \vec{\Omega}, E)
    \eexp^{ -\tau(\vec{x}, \vec{x}- vt \vec{\Omega}, \vec{\Omega}, E, t)}
    U( \norm{\vec{x} - \vec{x}_b} - vt)
    \\
    &\qquad +  \int_{0}^{\norm{\vec{x} - \vec{x}_b}}
    \eexp^{ -\tau(\vec{x}, \vec{x} - s, \vec{\Omega}, E, t)}
    \hat Q(\vec{x} - s \vec{\Omega}, \vec{\Omega}, E, t-s/v) \ud s\,.
  \end{split}
  \end{equation}
  Here, $U(\zeta)$ is the heaviside function, unity for $\zeta \ge 0$ and zero
  otherwise. The optical thickness of the medium between points $\vec{x}$ and
  $\vec{x}'$ along direction $\vec{\Omega}$ is 
  \begin{equation} \label{eq:fullTauDefinition}
    \tau(\vec{x}, \vec{x}', \vec{\Omega}, E, t) = \int_{0}^{\norm{\vec{x} -
    \vec{x}'}} \Sigma_t(\vec{x}-s\vec{\Omega}, E, t-s/v) \ud s \,.
  \end{equation}
  The point $\vec{x}_b$ is defined as the point on the boundary that,
  following direction $\vec{\Omega}$, intersects point $\vec{x}$. In other
  words, $\vec{x}_b = \vec{x} - d \vec{\Omega}$ where $d$ is positive.
\end{subequations}
It is possible to reformulate the last integral in
Eq.~\eqref{eq:integralAngularFluxFull} to be a volume integral:
\begin{equation*}
  \int_{0}^{\norm{\vec{x} - \vec{x}_b}}
    \eexp^{ -\tau(\vec{x}, \vec{x} - s, \vec{\Omega}, E, t)}
    \hat Q(\vec{x} - s \vec{\Omega}, \vec{\Omega}, E, t-s/v) \ud s
    = \int_{V} \hat Q(\vec{x}', \vec{\Omega}', E, t-\norm{\vec{x} - \vec{x}'}/v)
    \frac{\eexp^{-\tau(\vec{x}, \vec{x}', \vec{\Omega}, E, t) }}
    {\norm{\vec{x} - \vec{x}'}^2}
    \delta(\vec{\Omega} - \vec{\Omega}') \ud V'
\end{equation*}
%%%%%%%%%%%%%%%%%%%%%%%%%%%%%%%%%%%%%%%%%%%%%%%%%%%%%%%%%%%%%%%%%%%%%%%%%%%%%%%
\bibliographystyle{ans}
\bibliography{SRJall}
\end{document}
