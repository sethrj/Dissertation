\documentclass[11pt]{SRJresearch}
\author{Seth R.~Johnson}
\date{\today}
\title{Facts}

\begin{document}

Fact 1: Larsen is a magician. Do not meddle in the affairs of transport
wizards, for they are subtle and quick to look at you like you're an idiot.
%%%%%%%%%%%%%%%%%%%%%%%%%%%%%%%%%%%%%%%%%%%%%%%%%%%%%%%%%%%%%%%%%%%%%%%%%%%%%%%%
\section{General transport}

\subsection{Flux limiters}

For the angular flux $\psi(\vec{\Omega}) \ge 0$, and $\norm{\vec{\Omega}} = 1$,
with the current defined as
\begin{equation*}
  \vec{J} \equiv \int_{4\pi} \vec{\Omega} \psi\ud \Omega \,,
\end{equation*}
then
\begin{equation*}
  \norm{\vec{J}} = \norm{\int_{4\pi} \vec{\Omega} \psi\ud \Omega }
  \le \int_{4\pi} \norm{\vec{\Omega} } \abs{\psi(\vec{\Omega})} \ud \Omega 
  = \int_{4\pi} [1] \psi(\vec{\Omega}) \ud \Omega 
  = \phi
\end{equation*}
so
\begin{equation*}
  \frac{ \norm{\vec{J}} }{\phi} \le 1\,.
\end{equation*}
This limit is pretty easy to satisfy in a steady-state problem; however,
time-dependent diffusion problems can exceed the limit.

%%%%%%%%%%%%%%%%%%%%%%%%%%%%%%%%%%%%%%%%%%%%%%%%%%%%%%%%%%%%%%%%%%%%%%%%%%%%%%%
\section{Time-dependent transport}
%%%%%%%%%%%%%%%%%%%%%%%%%%%%%%%%%%%%%%%%%%%%%%%%%%%%%%%%%%%%%%%%%%%%%%%%%%%%%%%
\section{TRT}

\subsection{Asymptotic diffusion limit}
The approximations made for diffusion are, from \cite{Den2004}:
the speed of light is fast ($c \to c/\eps$),
opacities are large ($\sigma \to \sigma/\eps$),
and small amounts of absorption result in large changes in temperature ($c_v
  \to \eps c_v$).
%%%%%%%%%%%%%%%%%%%%%%%%%%%%%%%%%%%%%%%%%%%%%%%%%%%%%%%%%%%%%%%%%%%%%%%%%%%%%%%
\bibliographystyle{amsalpha}
\bibliography{SRJall}
\end{document}
