\section{Diffusion}

An accurate flatland diffusion formulation
is needed for benchmarking the anisotropic diffusion
approximation against diffusion solutions.
In the following section we derive ``Marshak'' and ``variational'' boundary
conditions for the flatland diffusion equation. (A summary of this original
work is published in \cite{Joh2011a}.)
%There is no loss of generality in
%ignoring the time dependence because of the quasi-static approximation made in
%the derivation of the diffusion coefficient (see \S\ref{sec:bgDiffusion}).

The difference between diffusion in flatland and 2-D results from the
angular moments in the two geometries, which
are defined (and evaluated in Table~\ref{tab:angularMoments}) as:
\begin{equation*}
  \gamma_n \equiv \int_S \abs{\vec{\Omega} \vd \vec{i}}^n \ud\Omega\,.
\end{equation*}
These give rise not only to a different diffusion coefficient in the interior
but also different boundary conditions.

%%%%%%%%%%%%%%%%%%%%%%%%%%%%%%%%%%%%%%%%
\subsection{Interior diffusion approximation}

The diffusion approximation begins by assuming that $\psi$ is linear in angle:
\begin{equation*}
  \psi(\vec{x}, \vec{\Omega}) \approx f(\vec{x}) + \vec{\Omega} \vd
  \vec{g}(\vec{x})\,.
\end{equation*}
The zeroth angular moment of $\psi$ determines $f$:
\begin{equation*}
  \phi = \int_S \psi \ud\Omega
= \int_S \left( f + \vec{\Omega}\vd \vec{g} \right) \ud\Omega
= \int_S\ud\Omega f + 0
= \gamma_0 f \,,
\end{equation*}
and the first moment of $\psi$ determines $g$:
\begin{equation*}
  \vec{J} = \int_S \vec{\Omega} \psi \ud\Omega
= f \int_S \vec{\Omega} \ud\Omega
  + \vec{g} \vd \int_S \vec{\Omega}\vec{\Omega} \ud\Omega
= \gamma_2 \vec{g} \,.
\end{equation*}
This is the \Pone\ approximation to the radiation angular flux:
\begin{equation}\label{eq:ssPone}
  \psi(\vec{x}, \vec{\Omega})
  \approx \frac{1}{\gamma_0} \phi(\vec{x})
  + \frac{1}{\gamma_2} \vec{\Omega} \vd \vec{J}(\vec{x})\,.
\end{equation}

The diffusion approximation is a closure for the first angular moment of
the transport equation. Operating on Eq.~\eqref{eq:ssTransportVol} by
$\int_S \vec{\Omega} (\cdot) \ud\Omega$ and substituting
the approximation in Eq.~\eqref{eq:ssPone} reduces the first angular moment of
the transport equation to the following:
\begin{align*}
  \grad \vd \int_S \vec{\Omega} \vec{\Omega} \psi
  \ud\Omega
  + \sigma \int_S \vec{\Omega} \psi \ud\Omega
  &=
  \frac{c\sigma}{\gamma_0} \phi \int_S \vec{\Omega} \ud\Omega
  + \frac{1}{\gamma_0} q \int_S \vec{\Omega} \ud\Omega
  \\
  \grad \vd \int_S \vec{\Omega} \vec{\Omega} \left(
  \frac{1}{\gamma_0}\phi + \frac{1}{\gamma_2} \vec{\Omega} \vd \vec{J}
  \right)
  \ud\Omega
  + \sigma \vec{J}
  &= 0
  \\
  \frac{1}{\gamma_0} \grad \vd \int_S \vec{\Omega} \vec{\Omega}
  \ud\Omega\, \phi 
  + \sigma \vec{J} &= 0
  \\
  \frac{\gamma_2}{\gamma_0} \grad \phi + \sigma \vec{J} &= 0 \,.
\end{align*}
Solving for $\vec{J}$ gives Fick's law, expressed in the general form:
\begin{equation} \label{eq:fickGeneral}
  \vec{J}(\vec{x})
  = - \frac{\gamma_2}{\gamma_0} \frac{1}{\sigma(\vec{x})} \grad \phi(\vec{x})
  \equiv -D(\vec{x}) \grad \phi(\vec{x})\,.
\end{equation}
In 2-D and 3-D, $\gamma_2/\gamma_0 = (4\pi / 3) / (4\pi) = 1/3$; in
flatland, $\gamma_2/\gamma_0 = \pi / (2\pi) = 1/2$. Thus, $D=\frac{1}{3\sigma}$ in
2-D but $D=\frac{1}{2\sigma}$ in flatland.

Substituting Fick's law into the linear-in-angle approximation,
Eq.~\eqref{eq:ssPone}, we obtain the diffusion approximation to the angular
angular flux:
\begin{align} \nonumber
  \psi(\vec{x}, \vec{\Omega})
  &\approx \frac{1}{\gamma_0} \phi(\vec{x})
  + \frac{1}{\gamma_2} \vec{\Omega} \vd \left[ - \frac{\gamma_2}{\gamma_0}
  \frac{1}{\sigma(\vec{x})} \grad \phi(\vec{x}) \right]
  \\ \label{eq:diffusionIntensity}
  \psi(\vec{x}, \vec{\Omega})
  &= \frac{1}{\gamma_0} \left[ \phi(\vec{x})
  - \frac{1}{\sigma(\vec{x})}
  \vec{\Omega} \vd \grad \phi(\vec{x}) \right] \,.
\end{align}
In physical geometry this is the standard diffusion approximation
\begin{equation*}
 \psi(\vec{x}, \vec{\Omega})
= \frac{1}{4\pi} \left[ \phi(\vec{x}) - \frac{1}{\sigma(\vec{x})} \vec{\Omega}
\vd \grad \phi(\vec{x}) \right] \,,
\end{equation*}
and in flatland, the diffusion approximation is
\begin{equation}\label{eq:flatlandDiffusion}
 \psi(\vec{x}, \vec{\Omega})
= \frac{1}{2\pi} \left[ \phi(\vec{x}) - \frac{1}{\sigma(\vec{x})} \vec{\Omega}
\vd \grad \phi(\vec{x}) \right]\,.
\end{equation}

%%%%%%%%%%%%%%%%%%%%%%%%%%%%%%%%%%%%%%%%
\subsection{Marshak boundary condition}
The Marshak boundary condition \cite{Mar1947} preserves the incident radiation
current (the partial first moment for incoming directions) on the boundary. It is
derived by substituting the approximate diffusion
angular flux from Eq.~\eqref{eq:diffusionIntensity} into the boundary condition,
Eq.~\eqref{eq:ssBndy}, multiplying by $\abs{\vec{\Omega}\vd \vec{n}}$, and integrating over
incident directions:
\begin{align*}
\int_{\vec{\Omega}\vd \vec{n} < 0 } \abs{\vec{\Omega}\vd \vec{n}}
\psi^b \ud\Omega
 &= 
\int_{\vec{\Omega}\vd \vec{n} < 0 } \abs{\vec{\Omega}\vd \vec{n}} 
 \frac{1}{\gamma_0} \left[ \phi - \frac{1}{\sigma}
  \vec{\Omega} \vd \grad \phi \right]
  \ud\Omega
\\
J^{-}
&= 
\frac{1}{\gamma_0} \phi \left( \int_{\vec{\Omega}\vd \vec{n} < 0 }
\abs{\vec{\Omega}\vd \vec{n}} \ud\Omega \right) 
  - \frac{1}{\gamma_0}\frac{1}{\sigma}
  \left( \int_{\vec{\Omega}\vd \vec{n} < 0 } [-\vec{\Omega}\vd \vec{n}]
  \vec{\Omega} \ud\Omega  \right) \vd \grad \phi
\\
J^{-}
&=
\frac{1}{\gamma_0} \phi \left( \frac{\gamma_1}{2} \right) 
  + \frac{1}{\gamma_0}\frac{1}{\sigma} \left( \vec{n} \vd
  \frac{\gamma_2}{2} \Identitytens \right)
  \vd \grad \phi
\\
J^{-}
&=
\frac{\gamma_1}{2\gamma_0} \phi
+ \frac{\gamma_2}{2\gamma_0}\frac{1}{\sigma} \vec{n} \vd \grad \phi\,.
\end{align*}
This is the Marshak diffusion boundary condition:
\begin{equation} \label{eq:marshak}
\frac{2\gamma_0}{\gamma_1} J^{-}
=
\phi + \frac{\gamma_2}{\gamma_1}\frac{1}{\sigma} \vec{n} \vd \grad \phi\,.
\end{equation}

The value
\begin{equation*}
  \frac{\gamma_2}{\gamma_1}
  =
  \begin{cases}
    \frac{2}{3} \approx 0.6667 & \text{in 1-D, 2-D, 3-D; and} \\
    \frac{\pi}{4} \approx 0.7854 & \text{in flatland,}
  \end{cases}
\end{equation*}
is the Marshak extrapolation distance.
The underlying physical reason for the longer extrapolation distance in flatland
is that in 2-D, a greater fraction of particles travel at a steep angle to the
$x,y$-plane, yielding a steeper slope for $\phi$ on the boundary.

We can also rewrite the Marshak boundary condition in terms of the diffusion
coefficient by substituting $D$ from Eq.~\eqref{eq:fickGeneral}:
\begin{equation*}
\frac{2\gamma_0}{\gamma_1} J^{-}
= \phi + \frac{\gamma_0}{\gamma_1} D \vec{n} \vd \grad \phi\,.
\end{equation*}
In 1-D, 2-D, and 3-D geometries, this is the standard Marshak boundary condition
\begin{equation*}
4 J^{-}
= \phi + 2 D \vec{n} \vd \grad \phi\,.
\end{equation*}
In flatland, it is the following:
\begin{equation}\label{eq:flatMarshak}
\pi J^{-}
= \phi + \frac{\pi}{2} D \vec{n} \vd \grad \phi\,.
\end{equation}

%%%%%%%%%%%%%%%%%%%%%%%%%%%%%%%%%%%%%%%%
\subsection{Variational boundary condition} \label{sec:varBndy}
It is known that the Marshak boundary condition is heuristic and that
a more accurate boundary condition for diffusion can be derived from an
asymptotic matched boundary layer analysis. However, a simpler
method of deriving a more accurate (than Marshak) boundary condition is to use
a variational analysis \cite{Mal1991}.
A shorter but equivalent analysis, adapted to flatland geometry, follows.

We consider a homogeneous, source-free ($q=0$), purely scattering ($c=1$)
transport problem in a
semi-infinite flatland plane.\footnote{%
The justification for setting $c=1$ and $q=0$ relates to the asymptotic
scaling used to derive diffusion from the transport equation:
both $q$ and $1-c$ are $O(\epsilon^2)$ quantities \cite{Mal1991}.}
The transport equation~\eqref{eq:ssTransportVol} becomes
\begin{subequations} \label{eqs:flatTransport}
\begin{equation}\label{eq:flatTransportVol}
  \cos \omega \pder{\psi}{x} + \sin \omega \pder{\psi}{y} + \sigma \psi
  = \frac{\sigma}{2\pi} \int_{0}^{2\pi} \psi \ud \omega'\,,\quad
 -\infty < x < \infty,\ 0 \le y < \infty,\ 0 \le \omega < 2\pi\,.
\end{equation}
It has a uniform incident boundary condition,
\begin{equation}\label{eq:flatTransportBndy}
  \psi(x, 0, \omega) = \psi^b(\omega) \,,\quad -\infty < x < \infty,\ 
  0 \le \omega < \pi \,.
\end{equation}
\end{subequations}

Because neither the boundary condition nor $\sigma$ varies
in $x$, $\tpder{\psi}{x}=0$, and Eq.~\eqref{eq:flatTransportVol} reduces to the
one-dimensional flatland transport equation 
\begin{equation*}
  \sin \omega \pder{}{y}\psi(y,\omega) + \sigma \psi(y,\omega)
  = \frac{\sigma}{2\pi} \int_{0}^{2\pi} \psi(y,\omega') \ud \omega'\,.
\end{equation*}
which is \emph{not} the 1-D planar geometry transport equation.

We define the $y$ components of the angular moments of $\psi$ as
\begin{equation} \label{eq:flatPhi}
  \phi_m(y) = \int_{0}^{2\pi} (\vec{\Omega}\vd\vec{j})^m \psi(y,\omega) \ud\omega
  = \int_{0}^{2\pi} (\sin\omega)^m \psi(y,\omega) \ud\omega \,.
\end{equation}
As $y\to\infty$, the angular flux $\psi$ will approach a constant $\varphi/2\pi$,
which gives $\phi_0(\infty)=\phi(\infty)\equiv\varphi$.
Concordantly, $\phi_1(\infty)=0$.

Operating on the transport equation by $\int_{0}^{2\pi} (\sin\omega)^m (\cdot)
\ud\omega$ gives the $m$th angular moment in the $y$ direction:
\begin{align} \nonumber
  \pder{}{y} \int_{0}^{2\pi} (\sin\omega)^{m+1} \psi \ud\omega
  + \sigma \int_{0}^{2\pi} (\sin\omega)^{m} \psi \ud\omega
  &= \frac{\sigma}{2\pi} \int_{0}^{2\pi} \psi \ud \omega'
  \int_{0}^{2\pi} (\sin\omega)^{m} \ud\omega
  \\ \label{eq:flatMoments}
  \pder{\phi_{m+1}}{y}
  + \sigma \phi_{m}
  &= \frac{\sigma}{2\pi} \phi_{0}
  \int_{0}^{2\pi} (\sin\omega)^{m} \ud\omega\,.
\end{align}
For $m=0$, the conservation equation, Eq.~\eqref{eq:flatMoments} evaluates to
\begin{equation*}
  \pder{\phi_{1}}{y}
  + \sigma \phi_{0}
  = \frac{\sigma}{2\pi} \phi_{0} (2\pi)
  \lra
  \pder{\phi_{1}}{y} = 0\,.
\end{equation*}
In other words, the current is a constant, and because $\phi_1(\infty)=0$,
that constant is zero. Physically, a constant $\phi_1$ means that at every
point, the rate of energy being transferred away from the boundary is balanced
by energy moving toward the boundary. This logically follows from the lack of
absorption in the problem: at steady-state, the only means of energy loss is
through exiting the boundary.

Evaluating Eq.~\eqref{eq:flatMoments} for $m=1$ and using the result that
$\phi_{1}=0$, we obtain
\begin{equation*}
  \pder{\phi_{2}}{y}
  + \sigma \phi_{1}
  = \frac{\sigma}{2\pi} \phi_{0} (0)
  \lra
  \pder{\phi_{2}}{y} = 0\,.
\end{equation*}
Thus $\phi_{2}$ is also a constant. At large distances from the boundary,
$y\to\infty$, the radiation assumes an isotropic distribution,
$\psi\to\varphi/2\pi$. From these two facts we relate the second angular moment
throughout the problem to the equilibrium scalar flux $\varphi$:
\begin{equation*}
  \phi_{2} = \int_{0}^{2\pi} (\sin\omega)^2 \frac{\varphi}{2\pi} \ud\omega
  = \frac{1}{2} \varphi\,.
\end{equation*}
% in 1-D, this would be $\int_{-1}^{1}\mu^2\frac{\varphi}{2} \ud \mu =
% \frac{1}{3} \varphi$.

Since $\phi_1=0$, we can add $\alpha \phi_1$ to the previous equation for any
$\alpha$:
\begin{align*}
 \alpha\phi_1 + \phi_{2} &= \frac{\varphi}{2} \\
 \int_{0}^{2\pi} (\alpha \sin\omega + \sin^2\omega)
 \psi(y,\omega) \ud\omega
 &= \frac{\varphi}{2}\,.
\end{align*}
At the boundary $y=0$, $\psi=\psi^b$ for incident angles $0 \le \omega < \pi$.
The variational analysis in \cite{Mal1991} reveals that certain trial functions
allow an exiting angular distribution that is isotropic to second order accuracy,
so we make the ``variational'' approximation that $\psi(0,\omega)=\psi^\text{out}$.
The previous equation then yields
\begin{equation*}
 \int_{0}^{\pi} (\alpha \sin\omega + \sin^2\omega)
 \psi^b(\omega) \ud\omega
 + \int_{\pi}^{2\pi} (\alpha \sin\omega + \sin^2\omega)\ud\omega \psi^\text{out}
 = \frac{\varphi}{2}\,.
\end{equation*}
The value $\alpha=\pi/4$ eliminates the integral over outgoing directions and
gives the following relation between moments of the incident angular current and the
magnitude of the angular current as $y\to\infty$:
\begin{equation}\label{eq:varBoundary}
 \varphi = 2\int_{0}^{\pi} \left( \frac{\pi}{4} \sin\omega + \sin^2\omega \right)
 \psi^b(\omega) \ud\omega
 \,.
\end{equation}

We wish our boundary condition to preserve the value of
$\varphi$ when the diffusion method is used, so we substitute the diffusion
approximation, Eq.~\eqref{eq:flatlandDiffusion}:
\begin{align*}
 \varphi &= 2\int_{0}^{\pi} \left( \frac{\pi}{4} \sin\omega + \sin^2\omega \right)
 \psi^b(\omega) \ud\omega
 \\
 &= 
  2\int_{0}^{\pi} \left( \frac{\pi}{4} \sin\omega + \sin^2\omega \right)
 \left( \frac{1}{2\pi} \phi -
  \frac{1}{\sigma} \sin\omega \pder{\phi}{y}\right)\ud\omega
\\
 &= 
\frac{1}{2\pi} \int_{0}^{\pi} \left( \frac{\pi}{2} \sin\omega + 2 \sin^2\omega
\right)\ud\omega
 \,\phi -
 \frac{1}{2\pi} \int_{0}^{\pi} \left( \frac{\pi}{2} \sin^2\omega + 2 \sin^3\omega \right)\ud\omega
  \,\frac{1}{\sigma} \pder{\phi}{y}
  \\
 &= 
 \frac{1}{2\pi} \left( \frac{\pi}{2} [2] + 2 \frac{\pi}{2}
\right) \phi
-
\frac{1}{2\pi} \left( \frac{\pi}{2} \left[ \frac{\pi}{2} \right] + 2 \left[
\frac{4}{3} \right] \right) \frac{1}{\sigma} \pder{\phi}{y}
\\
 &= 
  \phi
- \left( \frac{\pi}{8} + \frac{4}{3\pi} \right) \frac{1}{\sigma} \pder{\phi}{y}
\,.
\end{align*}
In this problem, the boundary surface outer normal is $\vec{n}=-\vec{j}$. 
Replacing $\sin \omega$ with $-\vec{\Omega}\vd\vec{n}$, we obtain the following
flatland variational boundary condition:
\begin{equation} \label{eq:flatVarBc}
\int_{\vec{\Omega}\vd\vec{n} < 0} \left[ \frac{\pi}{2}
\abs{\vec{\Omega}\vd\vec{n}} + 2 (\vec{\Omega}\vd\vec{n})^2 \right]
\psi^b(\vec{x}, \vec{\Omega}) \ud\Omega
= 
  \phi(\vec{x})
  - \left( \frac{\pi}{8} + \frac{4}{3\pi} \right) \frac{1}{\sigma}
  \vec{n}\vd\grad \phi(\vec{x})\,.
\end{equation}
Compared to the flatland Marshak boundary condition,
Eq.~\eqref{eq:flatMarshak}, the variational boundary condition not only yields a
different extrapolation distance $\frac{\pi}{8} + \frac{4}{3\pi} \approx
0.8171$ but also uses a different moment of the incident boundary current.

%%%%%%%%%%%%%%%%%%%%%%%%%%%%%%%%%%%%%%%%
\subsection{Generalization}\label{sec:flatlandV}

Recall from the discussion of boundary conditions in
Chapter~\ref{chap:adDerivation} that to eliminate the boundary layer solution
in the interior of a 3-D problem, the necessary condition is
\begin{equation}\tagref{eq:tdKillBndy}
  \int_{\vec{\Omega}\vd\vec{n} < 0}
  W(\abs{\vec{\Omega}\vd\vec{n}}) \Ibl(\vec{x},\vec{\Omega}) \ud\Omega
  = 0\,,
  \quad \vec{x}\in \partial V ,\ \vec{\Omega}\vd \vec{n} < 0\,,
\end{equation}
where $W$ is related to Chandrasekhar's $H$-function \cite{Cha1960} and can be
approximated by a variationally-derived polynomial $W_2$:
\begin{equation} \tagref{eq:chandraW}
  W(\mu) = \frac{\sqrt{3}}{2} \mu H(\mu)
  \approx W_2(\mu) \equiv \mu + \tfrac{3}{2} \mu^2 \,, \quad 0 < \mu \le 1 \,.
\end{equation}
The exact extrapolation distance is the first moment of $W$:
\begin{equation*}
  z_0 = \int_{0}^{1} \mu W(\mu) \ud\mu
  \approx 0.7104\,,
\end{equation*}
and the variational approximation gives the following extrapolation distance:
\begin{equation*}
  z_0 \approx \int_{0}^{1} \mu W_2(\mu) \ud\mu = \tfrac{17}{24}
  \approx 0.7083\,.
\end{equation*}
Similarly, the 3-D Marshak boundary condition uses
\begin{equation*}
  W(\mu) \approx W_1(\mu) \equiv 2 \mu \,,
\end{equation*}
which gives the Marshak extrapolation distance
\begin{equation*}
  z_0 \approx \int_{0}^{1} \mu W_1(\mu) \ud\mu = \tfrac{2}{3}
  \approx 0.6667\,.
\end{equation*}

In our analysis of flatland boundary conditions, we have essentially been
investigating flatland equivalent of the $W$ function, which we term ``$V$'',
that preserves the interior solution of a flatland problem:
\begin{equation}\label{eq:flatKillBndy}
\int_{\vec{\Omega}\vd\vec{n} < 0} V( \abs{\vec{\Omega}\vd\vec{n}})
\psi^b(\vec{x}, \vec{\Omega}) \ud\Omega
=
\int_{\vec{\Omega}\vd\vec{n} < 0} V( \abs{\vec{\Omega}\vd\vec{n}})
I_\text{approx}(\vec{x}, \vec{\Omega}) \ud\Omega \,.
\end{equation}
The ``true'' function $V$ might be calculable, for example, by using a
flatland formulation of the $F_N$ method \cite{Sie1979}, but we instead used a
``variational'' analysis and the standard Marshak treatment to approximate $V$.

We define $V(\mu)$ on the domain $0 < \mu \le 1$, where for our purposes $\mu =
\abs{\vec{\Omega} \vd \vec{n}}$, and normalize the function and its
approximations so that in flatland geometry:
\begin{equation*}
  \int_{\vec{\Omega}\vd\vec{n} < 0} V( \abs{\vec{\Omega}\vd\vec{n}}) \ud\Omega
  = \int_{0}^{\pi} V(\sin\omega) \ud\omega
  = 1 \,.
\end{equation*}
The approximations to $V$ should also have this normalization. Analogous to the
$W$ function, the flatland extrapolation distance is the first moment of $V$:
\begin{equation*}
  z_0 = \int_{\vec{\Omega}\vd\vec{n} < 0} \abs{\vec{\Omega}\vd\vec{n}}
    V( \abs{\vec{\Omega}\vd\vec{n}}) \ud\Omega
  = \int_{0}^{\pi} V(\sin\omega) \sin\omega \ud\omega \,.
\end{equation*}

The flatland Marshak approximation uses only the first angular moment, which,
after normalization, is:
\begin{equation}\label{eq:flatV1}
  V_1(\mu) \equiv \frac{1}{2} \mu \,,
  \quad 0 < \mu \le 1 \,,
\end{equation}
giving the flatland extrapolation distance
\begin{equation*}
 z_0 = \int_{0}^{\pi} V_1(\sin\omega) \sin\omega \ud\omega
  = \frac{\pi}{4} \approx 0.7854 \,.
\end{equation*}

The variational analysis yielded the following two-term approximation to $V$:
\begin{equation}\label{eq:flatV2}
  V_2(\mu) \equiv \frac{1}{2} \mu + \frac{1}{\pi}\mu^2 \,,
  \quad 0 < \mu \le 1 \,.
\end{equation}
The resulting variational flatland extrapolation distance is:
\begin{equation*}
 z_0 = \int_{0}^{\pi} V_2(\sin\omega) \sin\omega \ud\omega
  = \frac{\pi}{8} + \frac{4}{3\pi} \approx 0.8171\,.
\end{equation*}

Using the variational $V(\mu)\approx V_2(\mu)$ with the diffusion
approximation, Eq.~\eqref{eq:flatlandDiffusion}, we obtain
Eq.~\eqref{eq:flatVarBc}. With the Marshak $V(\mu)\approx V_1(\mu)$, the result
is the less accurate Eq.~\eqref{eq:flatMarshak}.

