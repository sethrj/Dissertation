\documentclass[draft]{beamer}
%\documentclass{beamer}
%%%%%%%%%%%%%%%%%%%%%%%%%%%%%%%%%%%%%%%%%%%%%%%%%%%%%%%%%%%%%%%%%%%%%%%%%%%%
\let\Tiny=\tiny
\usefonttheme[onlymath]{serif}
\usepackage{bm}
\usepackage{amsmath}
\usepackage{amssymb}
\usepackage{microtype}
\usepackage{booktabs}
/Users/seth/Documents/Compositions/SRJinclude.tex
\setSRJthesisfigurepaths

\definecolor{lightgray}{gray}{0.85}
%%%%%%%%%%%%%%%%%%%%%%%%%%%%%%%%%%%%%%%%%%%%%%%%%%%%%%%%%%%%%%%%%%%%%%%%%%%%

%30 minute presentation
%15 minutes scheduled for questions
%1 hour allotted total

\usetheme{AnnArbor}
\usecolortheme{seahorse}
\usecolortheme{orchid}
\setbeamercolor*{frametitle}{use=structure,bg=structure.fg!20!white}
\setbeamercolor*{frametitle right}{use=structure,bg=structure.fg!20!white}
\setbeamertemplate{navigation symbols}{\insertframenavigationsymbol}
\setbeamertemplate{caption}[numbered]

\title[Flatland]%
{Diffusion Boundary Conditions in Flatland Geometry}

\author[SRJ, EWL]{Seth~R.~Johnson \and Edward~W.~Larsen}

\institute[UMich]{
University of Michigan, Ann Arbor
}
\date[10/31/2011]{October 31, 2011}

%\AtBeginSection[]
%{
%\begin{frame}
%  \frametitle{Outline}
%  \tableofcontents[currentsection]
%\end{frame}
%}

\hypersetup{colorlinks=true,linkcolor=black}

% only show section headings in table of contents
\setcounter{tocdepth}{1}

%use symbols for footnote
\renewcommand{\thefootnote}{\fnsymbol{footnote}}

\begin{document}
%%%%%%%%%%%%%%%%%%%%%%%%%%%%%%%%%%%%%%%%%%%%%%%%%%%%%%%%%%%%%%%%%%%%%%%%%%%%

\begin{frame}
\titlepage
\begin{center}
  \includegraphics[width=0.2\textwidth]{umlogo}
\end{center}
\smash{\includegraphics[width=0.2\textwidth]{neup-small}}
\end{frame}

%%%%%%%%%%%%%%%%%%%%%%%%%%%%%%%%%%%%%%%%%%%%%%%%%%%%%%%%%%%%%%%%%%%%%%%%%%%%
%\begin{frame}
%  \frametitle{Outline}
%  \tableofcontents
%\end{frame}
%%%%%%%%%%%%%%%%%%%%%%%%%%%%%%%%%%%%%%%%%%%%%%%%%%%%%%%%%%%%%%%%%%%%%%%%%%%%
\section{Introduction}
%%%%%%%%%%%%%%%%%%%%%%%%%%%%%%%%%%%%%%%%
\begin{frame}{What is flatland?}
  \begin{itemize}
    \item Fictional two-dimensional universe \cite{Abb1884} in which particles
      are constrained to exist and travel in a 2-D plane
    \item Multi-dimensional geometry with smaller phase space in angle
      \begin{itemize}
        \item Flatland: $(x,y,\omega)$; 2-D: $(x,y,\mu,\omega)$
        \item Polar cosine is set to zero
      \end{itemize}
    \item Useful testbed for methods development
      \begin{itemize}
        \item Smaller phase space means computationally cheaper
        \item Angle-dependent quantities are easier to visualize
      \end{itemize}
  \end{itemize}
\end{frame}

%%%%%%%%%%%%%%%%%%%%%%%%%%%%%%%%%%%%%%%%
\begin{frame}{Channel surfing}
\begin{center}
  How do you design a 2-D test problem to model radiation in a channel?
\end{center}
\vspace{-.75in}

\begin{minipage}[c]{2.25in}%
  \vspace{.75in}%
  \includegraphics<1-2>[width=2in]{chord-flatland}%
  \includegraphics<3>[width=2in]{channel-xy}%
\end{minipage}
\begin{minipage}[c]{2.25in}%
  \rule{0pt}{3in}%
  \includegraphics<2>[width=2in]{chord-xyz}%
  \includegraphics<3>[width=2in]{channel-xyz}%
\end{minipage}
\end{frame}

%%%%%%%%%%%%%%%%%%%%%%%%%%%%%%%%%%%%%%%%
\begin{frame}{Channels in 2-D}
\begin{minipage}[c]{2.25in}%
  \vspace{.75in}%
  \includegraphics<1-2>[width=2in]{chord-flatland}%
  \includegraphics<3>[width=2in]{channel-xy}%
\end{minipage}
\begin{minipage}[c]{2.25in}%
  \rule{0pt}{3in}%
  \includegraphics<2>[width=2in]{chord-xyz}%
  \includegraphics<3>[width=2in]{channel-xyz}%
\end{minipage}
\end{frame}

%%%%%%%%%%%%%%%%%%%%%%%%%%%%%%%%%%%%%%%%
\begin{frame}{Motivation}
  \begin{itemize}
    \item Testing new methods in flatland requires existing methods to be fully
      formulated in flatland
    \item Flatland diffusion has been derived \cite{Asa2008} but without
      boundary conditions
  \end{itemize}
\end{frame}

%%%%%%%%%%%%%%%%%%%%%%%%%%%%%%%%%%%%%%%%%%%%%%%%%%%%%%%%%%%%%%%%%%%%%%%%%%%%
\section{Theory}
\subsection{Transport}
\begin{frame}{Transport equation}

\begin{equation}\label{eq:generalTransport}
  \vec{\Omega}\vd \grad \psi + \sigma \psi
  = \frac{c \sigma}{\gamma_0} \int_{S} \psi \ud\Omega + \frac{q}{\gamma_0}
  \quad \vec{x}\in V,\ \vec{\Omega}\in S \,,
\end{equation}

  \vspace{2ex}

  \small
  \centering
  Angular variables:
  \begin{tabular}{rccc}
\toprule
   Geometry & $\vec{\Omega}=(\Omega_x, \Omega_y)$ & Domain $S$ & $\ud\Omega$
\\ \midrule
2-D & $( \sqrt{1-\mu^2} \cos \omega,
   \sqrt{1-\mu^2} \sin \omega)$
   & $-1 \le \mu \le 1$, $0 \le \omega < 2\pi$ & $\ud\mu \ud \omega$
   \\
   Flatland & $ ( \cos \omega, \sin \omega )$
   & $0 \le \omega < 2\pi$ & $\ud \omega$
\\ \bottomrule
  \end{tabular}

  \vspace{2ex}

Angular moments:
  \centering
  \begin{tabular}{rccc}
\toprule
   Geometry
   & $\gamma_0 \equiv \int_S \ud\Omega$
   & $\gamma_1 \equiv \int_S \abs{\vec{\Omega}\vd\vec{i}} \ud\Omega$
   & $\gamma_2 \equiv \int_S (\vec{\Omega}\vd\vec{i})^2 \ud\Omega$
\\ \midrule
   2-D & $4\pi$ & $2\pi$ & $\frac{4\pi}{3}$
   \\
   Flatland & $2\pi$ & $4$ & $\pi$
\\ \bottomrule
  \end{tabular}

\end{frame}

%%%%%%%%%%%%%%%%%%%%%%%%%%%%%%%%%%%%%%%%
\begin{frame}{Monte Carlo sampling}
Example of lower computational cost of flatland transport: isotropic volume
source
\begin{equation}\label{eq:volumeSourcePdf}
  f(\vec{\Omega}) \ud\Omega = \frac{\ud\Omega}{\gamma_0} \,,
  \quad \vec{\Omega} \in S\,.
\end{equation}

\begin{columns}
  \column{.5\textwidth}
  \begin{block}{2D}
\begin{multline*}
  f(\mu,\omega) \ud\mu\ud\omega = \frac{\ud\mu}{2}\frac{\ud\omega}{2\pi} \,,
  \\
  -1\le \mu \le 1,\ 0 \le \omega < 2\pi\,,
\end{multline*}
\begin{equation*}
  \omega = 2\pi \xi_1\,, 
  \mu = \sqrt{ \xi_2} \,, 
\end{equation*}
  \end{block}

  \column{.5\textwidth}
  \begin{block}{Flatland}
\begin{equation*}
  f(\omega) \ud\omega = \frac{\ud\omega}{2\pi} \,,
  \quad 0 \le \omega < 2\pi\,,
\end{equation*}
\begin{equation*}
  \omega = 2\pi \xi_1\,.
\end{equation*}
  \end{block}
\end{columns}
\end{frame}

%%%%%%%%%%%%%%%%%%%%%%%%%%%%%%%%%%%%%%%%
\subsection{Diffusion}
\begin{frame}
Assume $\psi$ is linear in angle:
\begin{equation} \label{eq:diffusionIntensity}
  \psi(\vec{x}, \vec{\Omega})
  = \frac{1}{\gamma_0} \left( \phi(\vec{x})
  - \frac{1}{\sigma(\vec{x})}
  \vec{\Omega} \vd \grad \phi(\vec{x}) \right) \,.
\end{equation}

\begin{equation} \label{eq:fickGeneral}
  \vec{J}(\vec{x})
  = - \frac{\gamma_2}{\gamma_0} \frac{1}{\sigma(\vec{x})} \grad \phi(\vec{x})
  \equiv -D(\vec{x}) \grad \phi(\vec{x})\,.
\end{equation}
\end{frame}<++>

%%%%%%%%%%%%%%%%%%%%%%%%%%%%%%%%%%%%%%%%
\subsection{Marshak boundaries}

%%%%%%%%%%%%%%%%%%%%%%%%%%%%%%%%%%%%%%%%
\subsection{``Variational'' boundaries}

%%%%%%%%%%%%%%%%%%%%%%%%%%%%%%%%%%%%%%%%
\section{Results}

%%%%%%%%%%%%%%%%%%%%%%%%%%%%%%%%%%%%%%%%%%%%%%%%%%%%%%%%%%%%%%%%%%%%%%%%%%%%
\section{Conclusions}
%%%%%%%%%%%%%%%%%%%%%%%%%%%%%%%%%%%%%%%%%%%%%%%%%%%%%%%%%%%%%%%%%%%%%%%%%%%%
\appendix
%%%%%%%%%%%%%%%%%%%%%%%%%%%%%%%%%%%%%%%%
\begin{frame}
  %\frametitle{Questions?}

{\par\centering%
  \includegraphics[height=2.75in]{bunny-transport}

}
{\setlength{\baselineskip}{-\baselineskip} \tiny 
This material is based upon work supported under a National Science Foundation
Graduate Research Fellowship and a Department of Energy Nuclear Energy
University Programs Graduate Fellowship. Any opinions, findings, conclusions or
recommendations expressed in this publication are those of the author and do
not necessarily reflect the views of the National Science Foundation or the
Department of Energy Office of Nuclear Energy.\par}
\end{frame}

%%%%%%%%%%%%%%%%%%%%%%%%%%%%%%%%%%%%%%%%
\begin{frame}
  \frametitle{References}
\bibliographystyle{ans}
\bibliography{../../SRJall}
\end{frame}
%%%%%%%%%%%%%%%%%%%%%%%%%%%%%%%%%%%%%%%%%%%%%%%%%%%%%%%%%%%%%%%%%%%%%%%%%%%
\end{document}

