\documentclass[11pt,letter,twoside]{mc2011}

\usepackage[numbers,sort&compress]{natbib}

\usepackage{fancyhdr,lastpage}

\usepackage[dvips]{graphicx,color}

\pagestyle{fancy}

\globalmc2011

\begin{document}

\title{TITLE OF THE PAPER: CENTERED, UPPERCASE, 14 POINT TIMES NEW ROMAN, 
       ON SECOND LINE FROM THE TOP MARGIN, NOT MORE THAN 3 LINES LONG}

\author{
\textbf{Author A and Author B\footnote{Footnote, if necessary, in Times New Roman and font size 9}}\\
Name of Institute\\
Corresponding Address\\
A@institute.gov; B@institute.gov \vspace{1.0em} \\ 
\textbf{Double space and list Author C}\\
Department of Nuclear Engineering\\
Name of University\\
Address\\
c@name.univ.edu
}

\maketitle

\thispagestyle{empty}

\begin{abstract}
Use 8.5 x 11 paper size, with 1" margins on all sides. A required 200-250 word abstract starts on this line.  
Leave two blank lines before ABSTRACT and one after.  Use 10 point Times New Roman here and single 
(10 point) spacing.  The abstract is a very brief summary highlighting main accomplishments, what is new, 
and how it relates to the state-of-the-art.

\keywords{List of minimum 3 and maximum 6 key words separated by commas.}

\end{abstract}

\newcommand\authorname{Authors'names, use et al. if more than 3}
\newcommand\shorttitlename{Short version of title}

\fancymc2011

% ------------------------------------------------------------------------

\section{INTRODUCTION}

Paper starts here with two blank lines before first section title.  Use 8.5 x 11 paper size, with 1" margins on all sides. Double-space before and after each subsequent section�s title.  Section titles have style Heading 1, are 12 point font, must be all uppercase and centered, and must be numbered in Arabic numerals as shown above.  Introduce the topic of your work in this section.

Do not indent the first line of a paragraph; rather double-space between paragraphs.  There are four types of reference styles: journal paper~\cite{AB19xx}, proceeding paper~\cite{CDA19xx}, book~\cite{EFA19xx}, and website~\cite{http02}.  References to websites are discouraged, but acceptable if absolutely necessary.  It is the author�s responsibility to check links in the pdf file.

Since the paper will be included as a .pdf file for the proceedings, the author will get the best results from Word or WordPerfect by using the Acrobat Distiller or Acrobat PDFWriter as the default printer.  When creating the PDF version, check the Embed All Fonts option.  Note that it is the author�s responsibility to review the final PDF version of the paper to ensure proper translation into PDF.  Final PDF file size should be no more than 4 MB.  Recommended paper length is 5-15 pages, suggested 8-12.

% ------------------------------------------------------------------------

\section{SECOND OR SUBSEQUENT MAJOR HEADING}

A logical division of your paper into sections, etc., makes it so much easier to understand.  The style for subsection titles and all text in this template is Heading 2, Heading 3, etc. All text in this template is Body Text 3.  Make sure to avoid widow/orphan lines.

\subsection{Subsection Title: First Character of Each Non-trivial Word is Uppercase}

Double-space before and after secondary titles.  Secondary titles should start flush left, and are numbered as illustrated above.

Equations should be centered and sequentially numbered to the flush right of the formula.

\begin{equation}
  \frac{\partial U_{k}}{\partial x} + \frac{\partial }{\partial x} F_{k}(U) =
  \frac{\partial K}{\partial x}U_{k}^{\frac{3}{2}}\frac{\partial U_{k}}{\partial x}
\end{equation}

The continuation of a paragraph after an equation is not indented. All paragraphs, as well as section or subsection headings, are separated by just one single empty line.

\subsubsection{Sub-subsection level and lower: only first character uppercase}

Figures and tables should appear as closely as possible to where they are first cited, e.g. Fig.~\ref{Fig1}, in the text.  Figures are numbered in Arabic numerals, with the caption centered below the figure, in boldface.  Triple-space before the figure, and after the figure caption.

\begin{figure}[h]
  \begin{center}
  \includegraphics[scale=0.8]{fig1.eps}
  \caption{Sample figure.\label{Fig1}}
  \end{center}
\end{figure}

When importing figures or any graphical image please verify two things:

\begin{itemize}

\item Any number, text or symbol is in Times font and is not smaller than 10-point after reduction to the actual window in your paper;

\item That it can be translated into PDF.

\end{itemize}

Tables, like Table~\ref{Tab1}, are numbered in Roman numerals, with the caption centered above the table, in boldface.  Triple-space before and after the table.

\begin{center}
\begin{table}[h]
\caption{Sample table: accuracy of nodal and characteristic methods}
\vspace{12pt}
\begin{center}
\begin{tabular}{||c||c|c|c|c||} \hline \hline
\rule[-5mm]{0mm}{10mm}
Mesh            & 8 X 8            & 16 X 16         & 32 X 32         & 64 X 64        \\ \hline
\rule[-5mm]{0mm}{10mm}
Nodal	        & 1.000 10$^{-1}$  & 2.500 10$^{-2}$ & 6.250 10$^{-3}$ & 1.563 10$^{-3}$ \\ \hline
\rule[-5mm]{0mm}{10mm}
Characteristic	& 1.000 10$^{-1}$  & 2.500 10$^{-2}$ & 6.250 10$^{-3}$ & 1.563 10$^{-3}$ \\ \hline
\hline
\end{tabular}
\end{center}
\label{Tab1}
\end{table}
\end{center}

% ------------------------------------------------------------------------

\section{CONCLUSIONS}

Present your summary and conclusions here.

% ------------------------------------------------------------------------

\section*{ACKNOWLEDGEMENTS}

This template was adapted from the template for PHYSOR 2002 posted on the
Internet.  This Latex version was made by our colleague Helio P. Amaral Souto,
Professor of the University of the State of Rio de Janeiro, Brazil. Acknowledge
the help of colleagues, and sources of funding, if you wish. 


% ------------------------------------------------------------------------

\begin{thebibliography}{99}

\bibitem[Author B., 19xx]{AB19xx}
B.~Author(s).
\newblock Title.
\newblock {\it Journal Name in Italic.}, {\bf Vol. in Bold}, pp. 34-89 (19xx).

\bibitem[Author C. D., 19xx]{CDA19xx}
C. D.~Author(s).
\newblock Article Title.
\newblock {\it Proceeding of Meeting in Italic.}, Location, Dates of Meeting, Vol. n, pp. 134-156 (19xx).

\bibitem[Author E. F., 19xx]{EFA19xx}
E. F. Author
\newblock {\it Book Title in Italic}.
\newblock Publisher, City \& Country (19xx).

\bibitem[Author(s) G. H., 2002]{http02}
G. H. Author(s).
\newblock Spallation Neutron Source: The next-generation neutron-scattering facility
for the United States, \underline{http://www.sns.gov/documentation/sns\_brochure.pdf} (2002).

\end{thebibliography}

% ------------------------------------------------------------------------

\appendix
% please do not modify 
\section{ }

If necessary, include Appendices numbered in upper case alphabetical order.

In order to ensure a uniform, professional look to the proceedings, please {\bf do not modify} the format of this template without checking with the organizers first.


% ------------------------------------------------------------------------

\end{document}
