%\documentclass[draft]{beamer}
\documentclass{beamer}
%%%%%%%%%%%%%%%%%%%%%%%%%%%%%%%%%%%%%%%%%%%%%%%%%%%%%%%%%%%%%%%%%%%%%%%%%%%%
\let\Tiny=\tiny
\usefonttheme[onlymath]{serif}
\usepackage{bm}
\usepackage{amsmath}
\usepackage{amssymb}
\usepackage{microtype}
\usepackage{booktabs}
/Users/seth/Documents/Compositions/SRJinclude.tex
\setSRJthesisfigurepaths

\definecolor{lightgray}{gray}{0.85}
%%%%%%%%%%%%%%%%%%%%%%%%%%%%%%%%%%%%%%%%%%%%%%%%%%%%%%%%%%%%%%%%%%%%%%%%%%%%

%30 minute presentation
%15 minutes scheduled for questions
%1 hour allotted total

\usetheme{AnnArbor}
\usecolortheme{seahorse}
\usecolortheme{orchid}
\setbeamercolor*{frametitle}{use=structure,bg=structure.fg!20!white}
\setbeamercolor*{frametitle right}{use=structure,bg=structure.fg!20!white}
\setbeamertemplate{navigation symbols}{\insertframenavigationsymbol}
\setbeamertemplate{caption}[numbered]

\title[AD BCs]%
{Boundary Conditions for the Anisotropic Diffusion Approximation}

\author[SRJ, EWL]{Seth~R.~Johnson \and Edward~W.~Larsen}

\institute[UMich]{
University of Michigan, Ann Arbor
}
\date[11/1/2011]{November 1, 2011}

%\AtBeginSection[]
%{
%\begin{frame}
%  \frametitle{Outline}
%  \tableofcontents[currentsection]
%\end{frame}
%}

\hypersetup{colorlinks=true,linkcolor=black}

% only show section headings in table of contents
\setcounter{tocdepth}{1}

%use symbols for footnote
\renewcommand{\thefootnote}{\fnsymbol{footnote}}

\begin{document}
%%%%%%%%%%%%%%%%%%%%%%%%%%%%%%%%%%%%%%%%%%%%%%%%%%%%%%%%%%%%%%%%%%%%%%%%%%%%

\begin{frame}
\titlepage
\begin{center}
  \includegraphics[width=0.2\textwidth]{umlogo}
\end{center}
\smash{\includegraphics[width=0.2\textwidth]{neup-small}}
\end{frame}

%%%%%%%%%%%%%%%%%%%%%%%%%%%%%%%%%%%%%%%%%%%%%%%%%%%%%%%%%%%%%%%%%%%%%%%%%%%%
%\begin{frame}
%  \frametitle{Outline}
%  \tableofcontents
%\end{frame}
%%%%%%%%%%%%%%%%%%%%%%%%%%%%%%%%%%%%%%%%%%%%%%%%%%%%%%%%%%%%%%%%%%%%%%%%%%%%
\section{Introduction}
%%%%%%%%%%%%%%%%%%%%%%%%%%%%%%%%%%%%%%%%
\begin{frame}
\end{frame}

%%%%%%%%%%%%%%%%%%%%%%%%%%%%%%%%%%%%%%%%%%%%%%%%%%%%%%%%%%%%%%%%%%%%%%%%%%%%
\section{Theory}
%%%%%%%%%%%%%%%%%%%%%%%%%%%%%%%%%%%%%%%%
\begin{frame}
\end{frame}

%%%%%%%%%%%%%%%%%%%%%%%%%%%%%%%%%%%%%%%%%%%%%%%%%%%%%%%%%%%%%%%%%%%%%%%%%%%%
\section{Results}

\subsection{Problem description}
\begin{frame}
\begin{figure}[tb]
  \includegraphics[height=2.75in]{adbc-reactor/xsn}

  \caption{Test problem (flatland geometry). Gaussian source in lower-left;
  black and white give cross sections; reflecting boundaries on left, bottom,
  right.}
  \label{fig:problem}
\end{figure}
\end{frame}
  
\begin{frame}{Compared methods}
  \begin{itemize}
    \item Monte Carlo (and \SN\ for visualization of $\psi$)
    \item AD with specularly reflecting boundary for $f$ on top
    \item AD with white boundary for $f$ on top
    \item AD with ``na\"ive'' boundary: vacuum $f$ on top, use Marshak diffusion
      boundary with $\Dtens \vd \vec{n}$
    \item Diffusion
  \end{itemize}
\end{frame}

%%%%%%%%%%%%%%%%%%%%%%%%%%%%%%%%%%%%%%%%
\subsection{Lineouts}
\begin{frame}
\begin{figure}[tb]
  \centering
  % GNUPLOT: LaTeX picture with Postscript
\begingroup
  \makeatletter
  \providecommand\color[2][]{%
    \GenericError{(gnuplot) \space\space\space\@spaces}{%
      Package color not loaded in conjunction with
      terminal option `colourtext'%
    }{See the gnuplot documentation for explanation.%
    }{Either use 'blacktext' in gnuplot or load the package
      color.sty in LaTeX.}%
    \renewcommand\color[2][]{}%
  }%
  \providecommand\includegraphics[2][]{%
    \GenericError{(gnuplot) \space\space\space\@spaces}{%
      Package graphicx or graphics not loaded%
    }{See the gnuplot documentation for explanation.%
    }{The gnuplot epslatex terminal needs graphicx.sty or graphics.sty.}%
    \renewcommand\includegraphics[2][]{}%
  }%
  \providecommand\rotatebox[2]{#2}%
  \@ifundefined{ifGPcolor}{%
    \newif\ifGPcolor
    \GPcolortrue
  }{}%
  \@ifundefined{ifGPblacktext}{%
    \newif\ifGPblacktext
    \GPblacktexttrue
  }{}%
  % define a \g@addto@macro without @ in the name:
  \let\gplgaddtomacro\g@addto@macro
  % define empty templates for all commands taking text:
  \gdef\gplbacktext{}%
  \gdef\gplfronttext{}%
  \makeatother
  \ifGPblacktext
    % no textcolor at all
    \def\colorrgb#1{}%
    \def\colorgray#1{}%
  \else
    % gray or color?
    \ifGPcolor
      \def\colorrgb#1{\color[rgb]{#1}}%
      \def\colorgray#1{\color[gray]{#1}}%
      \expandafter\def\csname LTw\endcsname{\color{white}}%
      \expandafter\def\csname LTb\endcsname{\color{black}}%
      \expandafter\def\csname LTa\endcsname{\color{black}}%
      \expandafter\def\csname LT0\endcsname{\color[rgb]{1,0,0}}%
      \expandafter\def\csname LT1\endcsname{\color[rgb]{0,1,0}}%
      \expandafter\def\csname LT2\endcsname{\color[rgb]{0,0,1}}%
      \expandafter\def\csname LT3\endcsname{\color[rgb]{1,0,1}}%
      \expandafter\def\csname LT4\endcsname{\color[rgb]{0,1,1}}%
      \expandafter\def\csname LT5\endcsname{\color[rgb]{1,1,0}}%
      \expandafter\def\csname LT6\endcsname{\color[rgb]{0,0,0}}%
      \expandafter\def\csname LT7\endcsname{\color[rgb]{1,0.3,0}}%
      \expandafter\def\csname LT8\endcsname{\color[rgb]{0.5,0.5,0.5}}%
    \else
      % gray
      \def\colorrgb#1{\color{black}}%
      \def\colorgray#1{\color[gray]{#1}}%
      \expandafter\def\csname LTw\endcsname{\color{white}}%
      \expandafter\def\csname LTb\endcsname{\color{black}}%
      \expandafter\def\csname LTa\endcsname{\color{black}}%
      \expandafter\def\csname LT0\endcsname{\color{black}}%
      \expandafter\def\csname LT1\endcsname{\color{black}}%
      \expandafter\def\csname LT2\endcsname{\color{black}}%
      \expandafter\def\csname LT3\endcsname{\color{black}}%
      \expandafter\def\csname LT4\endcsname{\color{black}}%
      \expandafter\def\csname LT5\endcsname{\color{black}}%
      \expandafter\def\csname LT6\endcsname{\color{black}}%
      \expandafter\def\csname LT7\endcsname{\color{black}}%
      \expandafter\def\csname LT8\endcsname{\color{black}}%
    \fi
  \fi
  \setlength{\unitlength}{0.0500bp}%
  \begin{picture}(5400.00,4320.00)%
    \gplgaddtomacro\gplbacktext{%
      \csname LTb\endcsname%
      \put(1020,640){\makebox(0,0)[r]{\strut{} $10^{-7}$}}%
      \put(1020,1131){\makebox(0,0)[r]{\strut{} $10^{-6}$}}%
      \put(1020,1623){\makebox(0,0)[r]{\strut{} $10^{-5}$}}%
      \put(1020,2114){\makebox(0,0)[r]{\strut{} 0.0001}}%
      \put(1020,2605){\makebox(0,0)[r]{\strut{} 0.001}}%
      \put(1020,3096){\makebox(0,0)[r]{\strut{} 0.01}}%
      \put(1020,3588){\makebox(0,0)[r]{\strut{} 0.1}}%
      \put(1020,4079){\makebox(0,0)[r]{\strut{} 1}}%
      \put(1140,440){\makebox(0,0){\strut{} 0.01}}%
      \put(2585,440){\makebox(0,0){\strut{} 0.1}}%
      \put(4029,440){\makebox(0,0){\strut{} 1}}%
      \put(200,2359){\rotatebox{-270}{\makebox(0,0){\strut{}Absolute error}}}%
      \put(3089,140){\makebox(0,0){\strut{}$\Delta_x$}}%
    }%
    \gplgaddtomacro\gplfronttext{%
      \csname LTb\endcsname%
      \put(4136,1403){\makebox(0,0)[r]{\strut{}Gol'din}}%
      \csname LTb\endcsname%
      \put(4136,1203){\makebox(0,0)[r]{\strut{}9-point}}%
      \csname LTb\endcsname%
      \put(4136,1003){\makebox(0,0)[r]{\strut{}9-point$*$}}%
      \csname LTb\endcsname%
      \put(4136,803){\makebox(0,0)[r]{\strut{}Diagonal}}%
    }%
    \gplbacktext
    \put(0,0){\includegraphics{/Users/seth/_thesis/figures/manufactured/convergence-multisolve-diag/convergence-multisolve-diag.pdf}}%
    \gplfronttext
  \end{picture}%
\endgroup

  \caption{Scalar flux along the centerline of the channel.}
  \label{fig:phiChannel}
\end{figure}
\end{frame}

\begin{frame}
\begin{figure}[tb]
  \centering
  % GNUPLOT: LaTeX picture with Postscript
\begingroup
  \makeatletter
  \providecommand\color[2][]{%
    \GenericError{(gnuplot) \space\space\space\@spaces}{%
      Package color not loaded in conjunction with
      terminal option `colourtext'%
    }{See the gnuplot documentation for explanation.%
    }{Either use 'blacktext' in gnuplot or load the package
      color.sty in LaTeX.}%
    \renewcommand\color[2][]{}%
  }%
  \providecommand\includegraphics[2][]{%
    \GenericError{(gnuplot) \space\space\space\@spaces}{%
      Package graphicx or graphics not loaded%
    }{See the gnuplot documentation for explanation.%
    }{The gnuplot epslatex terminal needs graphicx.sty or graphics.sty.}%
    \renewcommand\includegraphics[2][]{}%
  }%
  \providecommand\rotatebox[2]{#2}%
  \@ifundefined{ifGPcolor}{%
    \newif\ifGPcolor
    \GPcolortrue
  }{}%
  \@ifundefined{ifGPblacktext}{%
    \newif\ifGPblacktext
    \GPblacktexttrue
  }{}%
  % define a \g@addto@macro without @ in the name:
  \let\gplgaddtomacro\g@addto@macro
  % define empty templates for all commands taking text:
  \gdef\gplbacktext{}%
  \gdef\gplfronttext{}%
  \makeatother
  \ifGPblacktext
    % no textcolor at all
    \def\colorrgb#1{}%
    \def\colorgray#1{}%
  \else
    % gray or color?
    \ifGPcolor
      \def\colorrgb#1{\color[rgb]{#1}}%
      \def\colorgray#1{\color[gray]{#1}}%
      \expandafter\def\csname LTw\endcsname{\color{white}}%
      \expandafter\def\csname LTb\endcsname{\color{black}}%
      \expandafter\def\csname LTa\endcsname{\color{black}}%
      \expandafter\def\csname LT0\endcsname{\color[rgb]{1,0,0}}%
      \expandafter\def\csname LT1\endcsname{\color[rgb]{0,1,0}}%
      \expandafter\def\csname LT2\endcsname{\color[rgb]{0,0,1}}%
      \expandafter\def\csname LT3\endcsname{\color[rgb]{1,0,1}}%
      \expandafter\def\csname LT4\endcsname{\color[rgb]{0,1,1}}%
      \expandafter\def\csname LT5\endcsname{\color[rgb]{1,1,0}}%
      \expandafter\def\csname LT6\endcsname{\color[rgb]{0,0,0}}%
      \expandafter\def\csname LT7\endcsname{\color[rgb]{1,0.3,0}}%
      \expandafter\def\csname LT8\endcsname{\color[rgb]{0.5,0.5,0.5}}%
    \else
      % gray
      \def\colorrgb#1{\color{black}}%
      \def\colorgray#1{\color[gray]{#1}}%
      \expandafter\def\csname LTw\endcsname{\color{white}}%
      \expandafter\def\csname LTb\endcsname{\color{black}}%
      \expandafter\def\csname LTa\endcsname{\color{black}}%
      \expandafter\def\csname LT0\endcsname{\color{black}}%
      \expandafter\def\csname LT1\endcsname{\color{black}}%
      \expandafter\def\csname LT2\endcsname{\color{black}}%
      \expandafter\def\csname LT3\endcsname{\color{black}}%
      \expandafter\def\csname LT4\endcsname{\color{black}}%
      \expandafter\def\csname LT5\endcsname{\color{black}}%
      \expandafter\def\csname LT6\endcsname{\color{black}}%
      \expandafter\def\csname LT7\endcsname{\color{black}}%
      \expandafter\def\csname LT8\endcsname{\color{black}}%
    \fi
  \fi
  \setlength{\unitlength}{0.0500bp}%
  \begin{picture}(5400.00,4320.00)%
    \gplgaddtomacro\gplbacktext{%
      \csname LTb\endcsname%
      \put(1020,640){\makebox(0,0)[r]{\strut{} $10^{-7}$}}%
      \put(1020,1131){\makebox(0,0)[r]{\strut{} $10^{-6}$}}%
      \put(1020,1623){\makebox(0,0)[r]{\strut{} $10^{-5}$}}%
      \put(1020,2114){\makebox(0,0)[r]{\strut{} 0.0001}}%
      \put(1020,2605){\makebox(0,0)[r]{\strut{} 0.001}}%
      \put(1020,3096){\makebox(0,0)[r]{\strut{} 0.01}}%
      \put(1020,3588){\makebox(0,0)[r]{\strut{} 0.1}}%
      \put(1020,4079){\makebox(0,0)[r]{\strut{} 1}}%
      \put(1140,440){\makebox(0,0){\strut{} 0.01}}%
      \put(2585,440){\makebox(0,0){\strut{} 0.1}}%
      \put(4029,440){\makebox(0,0){\strut{} 1}}%
      \put(200,2359){\rotatebox{-270}{\makebox(0,0){\strut{}Absolute error}}}%
      \put(3089,140){\makebox(0,0){\strut{}$\Delta_x$}}%
    }%
    \gplgaddtomacro\gplfronttext{%
      \csname LTb\endcsname%
      \put(4136,1403){\makebox(0,0)[r]{\strut{}Gol'din}}%
      \csname LTb\endcsname%
      \put(4136,1203){\makebox(0,0)[r]{\strut{}9-point}}%
      \csname LTb\endcsname%
      \put(4136,1003){\makebox(0,0)[r]{\strut{}9-point$*$}}%
      \csname LTb\endcsname%
      \put(4136,803){\makebox(0,0)[r]{\strut{}Diagonal}}%
    }%
    \gplbacktext
    \put(0,0){\includegraphics{/Users/seth/_thesis/figures/manufactured/convergence-multisolve-diag/convergence-multisolve-diag.pdf}}%
    \gplfronttext
  \end{picture}%
\endgroup

  \caption{Scalar flux along the left edge of the problem.}
  \label{fig:phiLeft}
\end{figure}
\end{frame}

%%%%%%%%%%%%%%%%%%%%%%%%%%%%%%%%%%%%%%%%
\subsection{Purely absorbing transport solution}
\begin{frame}
\begin{figure}[tb]
  \centering
  % GNUPLOT: LaTeX picture with Postscript
\begingroup
  \makeatletter
  \providecommand\color[2][]{%
    \GenericError{(gnuplot) \space\space\space\@spaces}{%
      Package color not loaded in conjunction with
      terminal option `colourtext'%
    }{See the gnuplot documentation for explanation.%
    }{Either use 'blacktext' in gnuplot or load the package
      color.sty in LaTeX.}%
    \renewcommand\color[2][]{}%
  }%
  \providecommand\includegraphics[2][]{%
    \GenericError{(gnuplot) \space\space\space\@spaces}{%
      Package graphicx or graphics not loaded%
    }{See the gnuplot documentation for explanation.%
    }{The gnuplot epslatex terminal needs graphicx.sty or graphics.sty.}%
    \renewcommand\includegraphics[2][]{}%
  }%
  \providecommand\rotatebox[2]{#2}%
  \@ifundefined{ifGPcolor}{%
    \newif\ifGPcolor
    \GPcolortrue
  }{}%
  \@ifundefined{ifGPblacktext}{%
    \newif\ifGPblacktext
    \GPblacktexttrue
  }{}%
  % define a \g@addto@macro without @ in the name:
  \let\gplgaddtomacro\g@addto@macro
  % define empty templates for all commands taking text:
  \gdef\gplbacktext{}%
  \gdef\gplfronttext{}%
  \makeatother
  \ifGPblacktext
    % no textcolor at all
    \def\colorrgb#1{}%
    \def\colorgray#1{}%
  \else
    % gray or color?
    \ifGPcolor
      \def\colorrgb#1{\color[rgb]{#1}}%
      \def\colorgray#1{\color[gray]{#1}}%
      \expandafter\def\csname LTw\endcsname{\color{white}}%
      \expandafter\def\csname LTb\endcsname{\color{black}}%
      \expandafter\def\csname LTa\endcsname{\color{black}}%
      \expandafter\def\csname LT0\endcsname{\color[rgb]{1,0,0}}%
      \expandafter\def\csname LT1\endcsname{\color[rgb]{0,1,0}}%
      \expandafter\def\csname LT2\endcsname{\color[rgb]{0,0,1}}%
      \expandafter\def\csname LT3\endcsname{\color[rgb]{1,0,1}}%
      \expandafter\def\csname LT4\endcsname{\color[rgb]{0,1,1}}%
      \expandafter\def\csname LT5\endcsname{\color[rgb]{1,1,0}}%
      \expandafter\def\csname LT6\endcsname{\color[rgb]{0,0,0}}%
      \expandafter\def\csname LT7\endcsname{\color[rgb]{1,0.3,0}}%
      \expandafter\def\csname LT8\endcsname{\color[rgb]{0.5,0.5,0.5}}%
    \else
      % gray
      \def\colorrgb#1{\color{black}}%
      \def\colorgray#1{\color[gray]{#1}}%
      \expandafter\def\csname LTw\endcsname{\color{white}}%
      \expandafter\def\csname LTb\endcsname{\color{black}}%
      \expandafter\def\csname LTa\endcsname{\color{black}}%
      \expandafter\def\csname LT0\endcsname{\color{black}}%
      \expandafter\def\csname LT1\endcsname{\color{black}}%
      \expandafter\def\csname LT2\endcsname{\color{black}}%
      \expandafter\def\csname LT3\endcsname{\color{black}}%
      \expandafter\def\csname LT4\endcsname{\color{black}}%
      \expandafter\def\csname LT5\endcsname{\color{black}}%
      \expandafter\def\csname LT6\endcsname{\color{black}}%
      \expandafter\def\csname LT7\endcsname{\color{black}}%
      \expandafter\def\csname LT8\endcsname{\color{black}}%
    \fi
  \fi
  \setlength{\unitlength}{0.0500bp}%
  \begin{picture}(5400.00,4320.00)%
    \gplgaddtomacro\gplbacktext{%
      \csname LTb\endcsname%
      \put(1020,640){\makebox(0,0)[r]{\strut{} $10^{-7}$}}%
      \put(1020,1131){\makebox(0,0)[r]{\strut{} $10^{-6}$}}%
      \put(1020,1623){\makebox(0,0)[r]{\strut{} $10^{-5}$}}%
      \put(1020,2114){\makebox(0,0)[r]{\strut{} 0.0001}}%
      \put(1020,2605){\makebox(0,0)[r]{\strut{} 0.001}}%
      \put(1020,3096){\makebox(0,0)[r]{\strut{} 0.01}}%
      \put(1020,3588){\makebox(0,0)[r]{\strut{} 0.1}}%
      \put(1020,4079){\makebox(0,0)[r]{\strut{} 1}}%
      \put(1140,440){\makebox(0,0){\strut{} 0.01}}%
      \put(2585,440){\makebox(0,0){\strut{} 0.1}}%
      \put(4029,440){\makebox(0,0){\strut{} 1}}%
      \put(200,2359){\rotatebox{-270}{\makebox(0,0){\strut{}Absolute error}}}%
      \put(3089,140){\makebox(0,0){\strut{}$\Delta_x$}}%
    }%
    \gplgaddtomacro\gplfronttext{%
      \csname LTb\endcsname%
      \put(4136,1403){\makebox(0,0)[r]{\strut{}Gol'din}}%
      \csname LTb\endcsname%
      \put(4136,1203){\makebox(0,0)[r]{\strut{}9-point}}%
      \csname LTb\endcsname%
      \put(4136,1003){\makebox(0,0)[r]{\strut{}9-point$*$}}%
      \csname LTb\endcsname%
      \put(4136,803){\makebox(0,0)[r]{\strut{}Diagonal}}%
    }%
    \gplbacktext
    \put(0,0){\includegraphics{/Users/seth/_thesis/figures/manufactured/convergence-multisolve-diag/convergence-multisolve-diag.pdf}}%
    \gplfronttext
  \end{picture}%
\endgroup

  \caption{Transport solution $f(2.5,10,\omega)$, at end of channel.}
  \label{fig:fMid}
\end{figure}
\end{frame}

\begin{frame}
\begin{figure}[tb]
  \centering
  % GNUPLOT: LaTeX picture with Postscript
\begingroup
  \makeatletter
  \providecommand\color[2][]{%
    \GenericError{(gnuplot) \space\space\space\@spaces}{%
      Package color not loaded in conjunction with
      terminal option `colourtext'%
    }{See the gnuplot documentation for explanation.%
    }{Either use 'blacktext' in gnuplot or load the package
      color.sty in LaTeX.}%
    \renewcommand\color[2][]{}%
  }%
  \providecommand\includegraphics[2][]{%
    \GenericError{(gnuplot) \space\space\space\@spaces}{%
      Package graphicx or graphics not loaded%
    }{See the gnuplot documentation for explanation.%
    }{The gnuplot epslatex terminal needs graphicx.sty or graphics.sty.}%
    \renewcommand\includegraphics[2][]{}%
  }%
  \providecommand\rotatebox[2]{#2}%
  \@ifundefined{ifGPcolor}{%
    \newif\ifGPcolor
    \GPcolortrue
  }{}%
  \@ifundefined{ifGPblacktext}{%
    \newif\ifGPblacktext
    \GPblacktexttrue
  }{}%
  % define a \g@addto@macro without @ in the name:
  \let\gplgaddtomacro\g@addto@macro
  % define empty templates for all commands taking text:
  \gdef\gplbacktext{}%
  \gdef\gplfronttext{}%
  \makeatother
  \ifGPblacktext
    % no textcolor at all
    \def\colorrgb#1{}%
    \def\colorgray#1{}%
  \else
    % gray or color?
    \ifGPcolor
      \def\colorrgb#1{\color[rgb]{#1}}%
      \def\colorgray#1{\color[gray]{#1}}%
      \expandafter\def\csname LTw\endcsname{\color{white}}%
      \expandafter\def\csname LTb\endcsname{\color{black}}%
      \expandafter\def\csname LTa\endcsname{\color{black}}%
      \expandafter\def\csname LT0\endcsname{\color[rgb]{1,0,0}}%
      \expandafter\def\csname LT1\endcsname{\color[rgb]{0,1,0}}%
      \expandafter\def\csname LT2\endcsname{\color[rgb]{0,0,1}}%
      \expandafter\def\csname LT3\endcsname{\color[rgb]{1,0,1}}%
      \expandafter\def\csname LT4\endcsname{\color[rgb]{0,1,1}}%
      \expandafter\def\csname LT5\endcsname{\color[rgb]{1,1,0}}%
      \expandafter\def\csname LT6\endcsname{\color[rgb]{0,0,0}}%
      \expandafter\def\csname LT7\endcsname{\color[rgb]{1,0.3,0}}%
      \expandafter\def\csname LT8\endcsname{\color[rgb]{0.5,0.5,0.5}}%
    \else
      % gray
      \def\colorrgb#1{\color{black}}%
      \def\colorgray#1{\color[gray]{#1}}%
      \expandafter\def\csname LTw\endcsname{\color{white}}%
      \expandafter\def\csname LTb\endcsname{\color{black}}%
      \expandafter\def\csname LTa\endcsname{\color{black}}%
      \expandafter\def\csname LT0\endcsname{\color{black}}%
      \expandafter\def\csname LT1\endcsname{\color{black}}%
      \expandafter\def\csname LT2\endcsname{\color{black}}%
      \expandafter\def\csname LT3\endcsname{\color{black}}%
      \expandafter\def\csname LT4\endcsname{\color{black}}%
      \expandafter\def\csname LT5\endcsname{\color{black}}%
      \expandafter\def\csname LT6\endcsname{\color{black}}%
      \expandafter\def\csname LT7\endcsname{\color{black}}%
      \expandafter\def\csname LT8\endcsname{\color{black}}%
    \fi
  \fi
  \setlength{\unitlength}{0.0500bp}%
  \begin{picture}(5400.00,4320.00)%
    \gplgaddtomacro\gplbacktext{%
      \csname LTb\endcsname%
      \put(1020,640){\makebox(0,0)[r]{\strut{} $10^{-7}$}}%
      \put(1020,1131){\makebox(0,0)[r]{\strut{} $10^{-6}$}}%
      \put(1020,1623){\makebox(0,0)[r]{\strut{} $10^{-5}$}}%
      \put(1020,2114){\makebox(0,0)[r]{\strut{} 0.0001}}%
      \put(1020,2605){\makebox(0,0)[r]{\strut{} 0.001}}%
      \put(1020,3096){\makebox(0,0)[r]{\strut{} 0.01}}%
      \put(1020,3588){\makebox(0,0)[r]{\strut{} 0.1}}%
      \put(1020,4079){\makebox(0,0)[r]{\strut{} 1}}%
      \put(1140,440){\makebox(0,0){\strut{} 0.01}}%
      \put(2585,440){\makebox(0,0){\strut{} 0.1}}%
      \put(4029,440){\makebox(0,0){\strut{} 1}}%
      \put(200,2359){\rotatebox{-270}{\makebox(0,0){\strut{}Absolute error}}}%
      \put(3089,140){\makebox(0,0){\strut{}$\Delta_x$}}%
    }%
    \gplgaddtomacro\gplfronttext{%
      \csname LTb\endcsname%
      \put(4136,1403){\makebox(0,0)[r]{\strut{}Gol'din}}%
      \csname LTb\endcsname%
      \put(4136,1203){\makebox(0,0)[r]{\strut{}9-point}}%
      \csname LTb\endcsname%
      \put(4136,1003){\makebox(0,0)[r]{\strut{}9-point$*$}}%
      \csname LTb\endcsname%
      \put(4136,803){\makebox(0,0)[r]{\strut{}Diagonal}}%
    }%
    \gplbacktext
    \put(0,0){\includegraphics{/Users/seth/_thesis/figures/manufactured/convergence-multisolve-diag/convergence-multisolve-diag.pdf}}%
    \gplfronttext
  \end{picture}%
\endgroup

  \caption{Transport solution $f(1.5,10,\omega)$, at end of channel.}
  \label{fig:fLeft}
\end{figure}
\end{frame}

%%%%%%%%%%%%%%%%%%%%%%%%%%%%%%%%%%%%%%%%
\subsection{Angular flux}
%\begin{frame}
%\begin{figure}[tb]
%  \centering
%  % GNUPLOT: LaTeX picture with Postscript
\begingroup
  \makeatletter
  \providecommand\color[2][]{%
    \GenericError{(gnuplot) \space\space\space\@spaces}{%
      Package color not loaded in conjunction with
      terminal option `colourtext'%
    }{See the gnuplot documentation for explanation.%
    }{Either use 'blacktext' in gnuplot or load the package
      color.sty in LaTeX.}%
    \renewcommand\color[2][]{}%
  }%
  \providecommand\includegraphics[2][]{%
    \GenericError{(gnuplot) \space\space\space\@spaces}{%
      Package graphicx or graphics not loaded%
    }{See the gnuplot documentation for explanation.%
    }{The gnuplot epslatex terminal needs graphicx.sty or graphics.sty.}%
    \renewcommand\includegraphics[2][]{}%
  }%
  \providecommand\rotatebox[2]{#2}%
  \@ifundefined{ifGPcolor}{%
    \newif\ifGPcolor
    \GPcolortrue
  }{}%
  \@ifundefined{ifGPblacktext}{%
    \newif\ifGPblacktext
    \GPblacktexttrue
  }{}%
  % define a \g@addto@macro without @ in the name:
  \let\gplgaddtomacro\g@addto@macro
  % define empty templates for all commands taking text:
  \gdef\gplbacktext{}%
  \gdef\gplfronttext{}%
  \makeatother
  \ifGPblacktext
    % no textcolor at all
    \def\colorrgb#1{}%
    \def\colorgray#1{}%
  \else
    % gray or color?
    \ifGPcolor
      \def\colorrgb#1{\color[rgb]{#1}}%
      \def\colorgray#1{\color[gray]{#1}}%
      \expandafter\def\csname LTw\endcsname{\color{white}}%
      \expandafter\def\csname LTb\endcsname{\color{black}}%
      \expandafter\def\csname LTa\endcsname{\color{black}}%
      \expandafter\def\csname LT0\endcsname{\color[rgb]{1,0,0}}%
      \expandafter\def\csname LT1\endcsname{\color[rgb]{0,1,0}}%
      \expandafter\def\csname LT2\endcsname{\color[rgb]{0,0,1}}%
      \expandafter\def\csname LT3\endcsname{\color[rgb]{1,0,1}}%
      \expandafter\def\csname LT4\endcsname{\color[rgb]{0,1,1}}%
      \expandafter\def\csname LT5\endcsname{\color[rgb]{1,1,0}}%
      \expandafter\def\csname LT6\endcsname{\color[rgb]{0,0,0}}%
      \expandafter\def\csname LT7\endcsname{\color[rgb]{1,0.3,0}}%
      \expandafter\def\csname LT8\endcsname{\color[rgb]{0.5,0.5,0.5}}%
    \else
      % gray
      \def\colorrgb#1{\color{black}}%
      \def\colorgray#1{\color[gray]{#1}}%
      \expandafter\def\csname LTw\endcsname{\color{white}}%
      \expandafter\def\csname LTb\endcsname{\color{black}}%
      \expandafter\def\csname LTa\endcsname{\color{black}}%
      \expandafter\def\csname LT0\endcsname{\color{black}}%
      \expandafter\def\csname LT1\endcsname{\color{black}}%
      \expandafter\def\csname LT2\endcsname{\color{black}}%
      \expandafter\def\csname LT3\endcsname{\color{black}}%
      \expandafter\def\csname LT4\endcsname{\color{black}}%
      \expandafter\def\csname LT5\endcsname{\color{black}}%
      \expandafter\def\csname LT6\endcsname{\color{black}}%
      \expandafter\def\csname LT7\endcsname{\color{black}}%
      \expandafter\def\csname LT8\endcsname{\color{black}}%
    \fi
  \fi
  \setlength{\unitlength}{0.0500bp}%
  \begin{picture}(5400.00,4320.00)%
    \gplgaddtomacro\gplbacktext{%
      \csname LTb\endcsname%
      \put(1020,640){\makebox(0,0)[r]{\strut{} $10^{-7}$}}%
      \put(1020,1131){\makebox(0,0)[r]{\strut{} $10^{-6}$}}%
      \put(1020,1623){\makebox(0,0)[r]{\strut{} $10^{-5}$}}%
      \put(1020,2114){\makebox(0,0)[r]{\strut{} 0.0001}}%
      \put(1020,2605){\makebox(0,0)[r]{\strut{} 0.001}}%
      \put(1020,3096){\makebox(0,0)[r]{\strut{} 0.01}}%
      \put(1020,3588){\makebox(0,0)[r]{\strut{} 0.1}}%
      \put(1020,4079){\makebox(0,0)[r]{\strut{} 1}}%
      \put(1140,440){\makebox(0,0){\strut{} 0.01}}%
      \put(2585,440){\makebox(0,0){\strut{} 0.1}}%
      \put(4029,440){\makebox(0,0){\strut{} 1}}%
      \put(200,2359){\rotatebox{-270}{\makebox(0,0){\strut{}Absolute error}}}%
      \put(3089,140){\makebox(0,0){\strut{}$\Delta_x$}}%
    }%
    \gplgaddtomacro\gplfronttext{%
      \csname LTb\endcsname%
      \put(4136,1403){\makebox(0,0)[r]{\strut{}Gol'din}}%
      \csname LTb\endcsname%
      \put(4136,1203){\makebox(0,0)[r]{\strut{}9-point}}%
      \csname LTb\endcsname%
      \put(4136,1003){\makebox(0,0)[r]{\strut{}9-point$*$}}%
      \csname LTb\endcsname%
      \put(4136,803){\makebox(0,0)[r]{\strut{}Diagonal}}%
    }%
    \gplbacktext
    \put(0,0){\includegraphics{/Users/seth/_thesis/figures/manufactured/convergence-multisolve-diag/convergence-multisolve-diag.pdf}}%
    \gplfronttext
  \end{picture}%
\endgroup

%  \caption{Angular flux $\psi(2.5,0,\omega)$, at beginning of channel.}
%  \label{fig:psi0}
%\end{figure}
%\end{frame}

\begin{frame}
\begin{figure}[tb]
  \centering
  % GNUPLOT: LaTeX picture with Postscript
\begingroup
  \makeatletter
  \providecommand\color[2][]{%
    \GenericError{(gnuplot) \space\space\space\@spaces}{%
      Package color not loaded in conjunction with
      terminal option `colourtext'%
    }{See the gnuplot documentation for explanation.%
    }{Either use 'blacktext' in gnuplot or load the package
      color.sty in LaTeX.}%
    \renewcommand\color[2][]{}%
  }%
  \providecommand\includegraphics[2][]{%
    \GenericError{(gnuplot) \space\space\space\@spaces}{%
      Package graphicx or graphics not loaded%
    }{See the gnuplot documentation for explanation.%
    }{The gnuplot epslatex terminal needs graphicx.sty or graphics.sty.}%
    \renewcommand\includegraphics[2][]{}%
  }%
  \providecommand\rotatebox[2]{#2}%
  \@ifundefined{ifGPcolor}{%
    \newif\ifGPcolor
    \GPcolortrue
  }{}%
  \@ifundefined{ifGPblacktext}{%
    \newif\ifGPblacktext
    \GPblacktexttrue
  }{}%
  % define a \g@addto@macro without @ in the name:
  \let\gplgaddtomacro\g@addto@macro
  % define empty templates for all commands taking text:
  \gdef\gplbacktext{}%
  \gdef\gplfronttext{}%
  \makeatother
  \ifGPblacktext
    % no textcolor at all
    \def\colorrgb#1{}%
    \def\colorgray#1{}%
  \else
    % gray or color?
    \ifGPcolor
      \def\colorrgb#1{\color[rgb]{#1}}%
      \def\colorgray#1{\color[gray]{#1}}%
      \expandafter\def\csname LTw\endcsname{\color{white}}%
      \expandafter\def\csname LTb\endcsname{\color{black}}%
      \expandafter\def\csname LTa\endcsname{\color{black}}%
      \expandafter\def\csname LT0\endcsname{\color[rgb]{1,0,0}}%
      \expandafter\def\csname LT1\endcsname{\color[rgb]{0,1,0}}%
      \expandafter\def\csname LT2\endcsname{\color[rgb]{0,0,1}}%
      \expandafter\def\csname LT3\endcsname{\color[rgb]{1,0,1}}%
      \expandafter\def\csname LT4\endcsname{\color[rgb]{0,1,1}}%
      \expandafter\def\csname LT5\endcsname{\color[rgb]{1,1,0}}%
      \expandafter\def\csname LT6\endcsname{\color[rgb]{0,0,0}}%
      \expandafter\def\csname LT7\endcsname{\color[rgb]{1,0.3,0}}%
      \expandafter\def\csname LT8\endcsname{\color[rgb]{0.5,0.5,0.5}}%
    \else
      % gray
      \def\colorrgb#1{\color{black}}%
      \def\colorgray#1{\color[gray]{#1}}%
      \expandafter\def\csname LTw\endcsname{\color{white}}%
      \expandafter\def\csname LTb\endcsname{\color{black}}%
      \expandafter\def\csname LTa\endcsname{\color{black}}%
      \expandafter\def\csname LT0\endcsname{\color{black}}%
      \expandafter\def\csname LT1\endcsname{\color{black}}%
      \expandafter\def\csname LT2\endcsname{\color{black}}%
      \expandafter\def\csname LT3\endcsname{\color{black}}%
      \expandafter\def\csname LT4\endcsname{\color{black}}%
      \expandafter\def\csname LT5\endcsname{\color{black}}%
      \expandafter\def\csname LT6\endcsname{\color{black}}%
      \expandafter\def\csname LT7\endcsname{\color{black}}%
      \expandafter\def\csname LT8\endcsname{\color{black}}%
    \fi
  \fi
  \setlength{\unitlength}{0.0500bp}%
  \begin{picture}(5400.00,4320.00)%
    \gplgaddtomacro\gplbacktext{%
      \csname LTb\endcsname%
      \put(1020,640){\makebox(0,0)[r]{\strut{} $10^{-7}$}}%
      \put(1020,1131){\makebox(0,0)[r]{\strut{} $10^{-6}$}}%
      \put(1020,1623){\makebox(0,0)[r]{\strut{} $10^{-5}$}}%
      \put(1020,2114){\makebox(0,0)[r]{\strut{} 0.0001}}%
      \put(1020,2605){\makebox(0,0)[r]{\strut{} 0.001}}%
      \put(1020,3096){\makebox(0,0)[r]{\strut{} 0.01}}%
      \put(1020,3588){\makebox(0,0)[r]{\strut{} 0.1}}%
      \put(1020,4079){\makebox(0,0)[r]{\strut{} 1}}%
      \put(1140,440){\makebox(0,0){\strut{} 0.01}}%
      \put(2585,440){\makebox(0,0){\strut{} 0.1}}%
      \put(4029,440){\makebox(0,0){\strut{} 1}}%
      \put(200,2359){\rotatebox{-270}{\makebox(0,0){\strut{}Absolute error}}}%
      \put(3089,140){\makebox(0,0){\strut{}$\Delta_x$}}%
    }%
    \gplgaddtomacro\gplfronttext{%
      \csname LTb\endcsname%
      \put(4136,1403){\makebox(0,0)[r]{\strut{}Gol'din}}%
      \csname LTb\endcsname%
      \put(4136,1203){\makebox(0,0)[r]{\strut{}9-point}}%
      \csname LTb\endcsname%
      \put(4136,1003){\makebox(0,0)[r]{\strut{}9-point$*$}}%
      \csname LTb\endcsname%
      \put(4136,803){\makebox(0,0)[r]{\strut{}Diagonal}}%
    }%
    \gplbacktext
    \put(0,0){\includegraphics{/Users/seth/_thesis/figures/manufactured/convergence-multisolve-diag/convergence-multisolve-diag.pdf}}%
    \gplfronttext
  \end{picture}%
\endgroup

  \caption{Angular flux $\psi(2.5,5,\omega)$, in middle of channel.}
  \label{fig:psi5}
\end{figure}
\end{frame}

\begin{frame}
\begin{figure}[tb]
  \centering
  % GNUPLOT: LaTeX picture with Postscript
\begingroup
  \makeatletter
  \providecommand\color[2][]{%
    \GenericError{(gnuplot) \space\space\space\@spaces}{%
      Package color not loaded in conjunction with
      terminal option `colourtext'%
    }{See the gnuplot documentation for explanation.%
    }{Either use 'blacktext' in gnuplot or load the package
      color.sty in LaTeX.}%
    \renewcommand\color[2][]{}%
  }%
  \providecommand\includegraphics[2][]{%
    \GenericError{(gnuplot) \space\space\space\@spaces}{%
      Package graphicx or graphics not loaded%
    }{See the gnuplot documentation for explanation.%
    }{The gnuplot epslatex terminal needs graphicx.sty or graphics.sty.}%
    \renewcommand\includegraphics[2][]{}%
  }%
  \providecommand\rotatebox[2]{#2}%
  \@ifundefined{ifGPcolor}{%
    \newif\ifGPcolor
    \GPcolortrue
  }{}%
  \@ifundefined{ifGPblacktext}{%
    \newif\ifGPblacktext
    \GPblacktexttrue
  }{}%
  % define a \g@addto@macro without @ in the name:
  \let\gplgaddtomacro\g@addto@macro
  % define empty templates for all commands taking text:
  \gdef\gplbacktext{}%
  \gdef\gplfronttext{}%
  \makeatother
  \ifGPblacktext
    % no textcolor at all
    \def\colorrgb#1{}%
    \def\colorgray#1{}%
  \else
    % gray or color?
    \ifGPcolor
      \def\colorrgb#1{\color[rgb]{#1}}%
      \def\colorgray#1{\color[gray]{#1}}%
      \expandafter\def\csname LTw\endcsname{\color{white}}%
      \expandafter\def\csname LTb\endcsname{\color{black}}%
      \expandafter\def\csname LTa\endcsname{\color{black}}%
      \expandafter\def\csname LT0\endcsname{\color[rgb]{1,0,0}}%
      \expandafter\def\csname LT1\endcsname{\color[rgb]{0,1,0}}%
      \expandafter\def\csname LT2\endcsname{\color[rgb]{0,0,1}}%
      \expandafter\def\csname LT3\endcsname{\color[rgb]{1,0,1}}%
      \expandafter\def\csname LT4\endcsname{\color[rgb]{0,1,1}}%
      \expandafter\def\csname LT5\endcsname{\color[rgb]{1,1,0}}%
      \expandafter\def\csname LT6\endcsname{\color[rgb]{0,0,0}}%
      \expandafter\def\csname LT7\endcsname{\color[rgb]{1,0.3,0}}%
      \expandafter\def\csname LT8\endcsname{\color[rgb]{0.5,0.5,0.5}}%
    \else
      % gray
      \def\colorrgb#1{\color{black}}%
      \def\colorgray#1{\color[gray]{#1}}%
      \expandafter\def\csname LTw\endcsname{\color{white}}%
      \expandafter\def\csname LTb\endcsname{\color{black}}%
      \expandafter\def\csname LTa\endcsname{\color{black}}%
      \expandafter\def\csname LT0\endcsname{\color{black}}%
      \expandafter\def\csname LT1\endcsname{\color{black}}%
      \expandafter\def\csname LT2\endcsname{\color{black}}%
      \expandafter\def\csname LT3\endcsname{\color{black}}%
      \expandafter\def\csname LT4\endcsname{\color{black}}%
      \expandafter\def\csname LT5\endcsname{\color{black}}%
      \expandafter\def\csname LT6\endcsname{\color{black}}%
      \expandafter\def\csname LT7\endcsname{\color{black}}%
      \expandafter\def\csname LT8\endcsname{\color{black}}%
    \fi
  \fi
  \setlength{\unitlength}{0.0500bp}%
  \begin{picture}(5400.00,4320.00)%
    \gplgaddtomacro\gplbacktext{%
      \csname LTb\endcsname%
      \put(1020,640){\makebox(0,0)[r]{\strut{} $10^{-7}$}}%
      \put(1020,1131){\makebox(0,0)[r]{\strut{} $10^{-6}$}}%
      \put(1020,1623){\makebox(0,0)[r]{\strut{} $10^{-5}$}}%
      \put(1020,2114){\makebox(0,0)[r]{\strut{} 0.0001}}%
      \put(1020,2605){\makebox(0,0)[r]{\strut{} 0.001}}%
      \put(1020,3096){\makebox(0,0)[r]{\strut{} 0.01}}%
      \put(1020,3588){\makebox(0,0)[r]{\strut{} 0.1}}%
      \put(1020,4079){\makebox(0,0)[r]{\strut{} 1}}%
      \put(1140,440){\makebox(0,0){\strut{} 0.01}}%
      \put(2585,440){\makebox(0,0){\strut{} 0.1}}%
      \put(4029,440){\makebox(0,0){\strut{} 1}}%
      \put(200,2359){\rotatebox{-270}{\makebox(0,0){\strut{}Absolute error}}}%
      \put(3089,140){\makebox(0,0){\strut{}$\Delta_x$}}%
    }%
    \gplgaddtomacro\gplfronttext{%
      \csname LTb\endcsname%
      \put(4136,1403){\makebox(0,0)[r]{\strut{}Gol'din}}%
      \csname LTb\endcsname%
      \put(4136,1203){\makebox(0,0)[r]{\strut{}9-point}}%
      \csname LTb\endcsname%
      \put(4136,1003){\makebox(0,0)[r]{\strut{}9-point$*$}}%
      \csname LTb\endcsname%
      \put(4136,803){\makebox(0,0)[r]{\strut{}Diagonal}}%
    }%
    \gplbacktext
    \put(0,0){\includegraphics{/Users/seth/_thesis/figures/manufactured/convergence-multisolve-diag/convergence-multisolve-diag.pdf}}%
    \gplfronttext
  \end{picture}%
\endgroup

  \caption{Angular flux $\psi(2.5,10,\omega)$, at end of channel.}
  \label{fig:psi10}
\end{figure}
\end{frame}

\begin{frame}
\begin{figure}[tb]
  \centering
  % GNUPLOT: LaTeX picture with Postscript
\begingroup
  \makeatletter
  \providecommand\color[2][]{%
    \GenericError{(gnuplot) \space\space\space\@spaces}{%
      Package color not loaded in conjunction with
      terminal option `colourtext'%
    }{See the gnuplot documentation for explanation.%
    }{Either use 'blacktext' in gnuplot or load the package
      color.sty in LaTeX.}%
    \renewcommand\color[2][]{}%
  }%
  \providecommand\includegraphics[2][]{%
    \GenericError{(gnuplot) \space\space\space\@spaces}{%
      Package graphicx or graphics not loaded%
    }{See the gnuplot documentation for explanation.%
    }{The gnuplot epslatex terminal needs graphicx.sty or graphics.sty.}%
    \renewcommand\includegraphics[2][]{}%
  }%
  \providecommand\rotatebox[2]{#2}%
  \@ifundefined{ifGPcolor}{%
    \newif\ifGPcolor
    \GPcolortrue
  }{}%
  \@ifundefined{ifGPblacktext}{%
    \newif\ifGPblacktext
    \GPblacktexttrue
  }{}%
  % define a \g@addto@macro without @ in the name:
  \let\gplgaddtomacro\g@addto@macro
  % define empty templates for all commands taking text:
  \gdef\gplbacktext{}%
  \gdef\gplfronttext{}%
  \makeatother
  \ifGPblacktext
    % no textcolor at all
    \def\colorrgb#1{}%
    \def\colorgray#1{}%
  \else
    % gray or color?
    \ifGPcolor
      \def\colorrgb#1{\color[rgb]{#1}}%
      \def\colorgray#1{\color[gray]{#1}}%
      \expandafter\def\csname LTw\endcsname{\color{white}}%
      \expandafter\def\csname LTb\endcsname{\color{black}}%
      \expandafter\def\csname LTa\endcsname{\color{black}}%
      \expandafter\def\csname LT0\endcsname{\color[rgb]{1,0,0}}%
      \expandafter\def\csname LT1\endcsname{\color[rgb]{0,1,0}}%
      \expandafter\def\csname LT2\endcsname{\color[rgb]{0,0,1}}%
      \expandafter\def\csname LT3\endcsname{\color[rgb]{1,0,1}}%
      \expandafter\def\csname LT4\endcsname{\color[rgb]{0,1,1}}%
      \expandafter\def\csname LT5\endcsname{\color[rgb]{1,1,0}}%
      \expandafter\def\csname LT6\endcsname{\color[rgb]{0,0,0}}%
      \expandafter\def\csname LT7\endcsname{\color[rgb]{1,0.3,0}}%
      \expandafter\def\csname LT8\endcsname{\color[rgb]{0.5,0.5,0.5}}%
    \else
      % gray
      \def\colorrgb#1{\color{black}}%
      \def\colorgray#1{\color[gray]{#1}}%
      \expandafter\def\csname LTw\endcsname{\color{white}}%
      \expandafter\def\csname LTb\endcsname{\color{black}}%
      \expandafter\def\csname LTa\endcsname{\color{black}}%
      \expandafter\def\csname LT0\endcsname{\color{black}}%
      \expandafter\def\csname LT1\endcsname{\color{black}}%
      \expandafter\def\csname LT2\endcsname{\color{black}}%
      \expandafter\def\csname LT3\endcsname{\color{black}}%
      \expandafter\def\csname LT4\endcsname{\color{black}}%
      \expandafter\def\csname LT5\endcsname{\color{black}}%
      \expandafter\def\csname LT6\endcsname{\color{black}}%
      \expandafter\def\csname LT7\endcsname{\color{black}}%
      \expandafter\def\csname LT8\endcsname{\color{black}}%
    \fi
  \fi
  \setlength{\unitlength}{0.0500bp}%
  \begin{picture}(5400.00,4320.00)%
    \gplgaddtomacro\gplbacktext{%
      \csname LTb\endcsname%
      \put(1020,640){\makebox(0,0)[r]{\strut{} $10^{-7}$}}%
      \put(1020,1131){\makebox(0,0)[r]{\strut{} $10^{-6}$}}%
      \put(1020,1623){\makebox(0,0)[r]{\strut{} $10^{-5}$}}%
      \put(1020,2114){\makebox(0,0)[r]{\strut{} 0.0001}}%
      \put(1020,2605){\makebox(0,0)[r]{\strut{} 0.001}}%
      \put(1020,3096){\makebox(0,0)[r]{\strut{} 0.01}}%
      \put(1020,3588){\makebox(0,0)[r]{\strut{} 0.1}}%
      \put(1020,4079){\makebox(0,0)[r]{\strut{} 1}}%
      \put(1140,440){\makebox(0,0){\strut{} 0.01}}%
      \put(2585,440){\makebox(0,0){\strut{} 0.1}}%
      \put(4029,440){\makebox(0,0){\strut{} 1}}%
      \put(200,2359){\rotatebox{-270}{\makebox(0,0){\strut{}Absolute error}}}%
      \put(3089,140){\makebox(0,0){\strut{}$\Delta_x$}}%
    }%
    \gplgaddtomacro\gplfronttext{%
      \csname LTb\endcsname%
      \put(4136,1403){\makebox(0,0)[r]{\strut{}Gol'din}}%
      \csname LTb\endcsname%
      \put(4136,1203){\makebox(0,0)[r]{\strut{}9-point}}%
      \csname LTb\endcsname%
      \put(4136,1003){\makebox(0,0)[r]{\strut{}9-point$*$}}%
      \csname LTb\endcsname%
      \put(4136,803){\makebox(0,0)[r]{\strut{}Diagonal}}%
    }%
    \gplbacktext
    \put(0,0){\includegraphics{/Users/seth/_thesis/figures/manufactured/convergence-multisolve-diag/convergence-multisolve-diag.pdf}}%
    \gplfronttext
  \end{picture}%
\endgroup

  \caption{Angular flux $\psi(1.5,5)$, at left edge of middle of channel.}
  \label{fig:psiEdge}
\end{figure}
\end{frame}

%%%%%%%%%%%%%%%%%%%%%%%%%%%%%%%%%%%%%%%%%%%%%%%%%%%%%%%%%%%%%%%%%%%%%%%%%%%%
\section{Conclusions}
\begin{frame}
\end{frame}

%%%%%%%%%%%%%%%%%%%%%%%%%%%%%%%%%%%%%%%%%%%%%%%%%%%%%%%%%%%%%%%%%%%%%%%%%%%%
\appendix
%%%%%%%%%%%%%%%%%%%%%%%%%%%%%%%%%%%%%%%%
\begin{frame}

{\par\centering\hspace{-.35in}
  \includegraphics[height=2.75in]{larsen-tux}
  \par}

\vspace{-3in}
{\par\Huge Questions?
\par}%
\vspace{3in}

{\vspace{-.25in}%
\setlength{\baselineskip}{-\baselineskip} \tiny %\raggedright
This material is based upon work supported under a National Science Foundation
Graduate Research Fellowship and a Department of Energy Nuclear Energy
University Programs Graduate Fellowship. Any opinions, findings, conclusions or
recommendations expressed in this publication are those of the author and do
not necessarily reflect the views of the National Science Foundation or the
Department of Energy Office of Nuclear Energy.\par}

\end{frame}

%%%%%%%%%%%%%%%%%%%%%%%%%%%%%%%%%%%%%%%%
\begin{frame}
  \frametitle{References}
\bibliographystyle{ans}
\bibliography{../../SRJall}
\end{frame}
%%%%%%%%%%%%%%%%%%%%%%%%%%%%%%%%%%%%%%%%%%%%%%%%%%%%%%%%%%%%%%%%%%%%%%%%%%%
\end{document}

