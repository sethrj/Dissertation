\documentclass{anstrans}
%%%%%%%%%%%%%%%%%%%%%%%%%%%%%%%%%%%
\title{Boundary conditions for diffusion in flatland geometry}
\author{Seth R.~Johnson}

%% uncomment these next five only if using anstrans
\institute{Department of Nuclear Engineering \& Radiological Sciences, University of Michigan, Ann Arbor, MI, 48109}
\email{sethrj@umich.edu}
\usepackage{bm}
\usepackage{amsmath}
\usepackage{amssymb}
\usepackage{microtype}
\usepackage{booktabs} % \toprule, \midrule, \bottomrule
/Users/seth/Documents/Compositions/SRJinclude.tex

\date{2011/05/30}
%%%%%%%%%%%%%%%%%%%%%%%%%%%%%%%%%%%
\begin{document}
%%%%%%%%%%%%%%%%%%%%%%%%%%%%%%%%%%%%%%%%%%%%%%%%%%%%%%%%%%%%%%%%%%%%%%%%%%%%%%%%
\section{Introduction}
Flatland geometry is a fictional two-dimensional space where particles are
constrained to the page \cite{Asa2008}. This differs from standard 2-D
geometry, which is the projection of a 3-D problem onto the page. The
constraint of living in the page reduces the phase space of the transport
equation, as the flatland solution is only a function of the azimuthal angle
rather than both azimuthal and polar angles. This reduction in phase space
makes flatland geometry a computationally less burdensome testing ground for new
methods \cite{Lar2009a}.

Previous work in flatland has shown that the diffusion
coefficient for flatland geometry $\frac{1}{2\sigma}$ is different from
the physical diffusion coefficients $\frac{1}{3\sigma}$, but the correct
formulation for flatland diffusion boundary conditions have remained an
unanswered and indeed unasked question. This summary derives both a Marshak
boundary condition and a variational boundary condition.

%%%%%%%%%%%%%%%%%%%%%%%%%%%%%%%%%%%%%%%%%%%%%%%%%%%%%%%%%%%%%%%%%%%%%%%%%%%%%%%%
\section{Theory}
%%%%%%%%%%%%%%%%%%%%%%%%%%%%%%%%%%%%%%%%%%%%%%%%%%%%%%%%%%%%%%%%%%%%%%%%%%%%%%%%
\section{Results and Analysis}
%%%%%%%%%%%%%%%%%%%%%%%%%%%%%%%%%%%%%%%%%%%%%%%%%%%%%%%%%%%%%%%%%%%%%%%%%%%%%%%%
\section{Conclusions}
%%%%%%%%%%%%%%%%%%%%%%%%%%%%%%%%%%%%%%%%%%%%%%%%%%%%%%%%%%%%%%%%%%%%%%%%%%%%%%%%
\bibliographystyle{ans}
\bibliography{../SRJall}
\end{document}
