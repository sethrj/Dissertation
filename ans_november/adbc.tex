\documentclass{anstrans}
%%%%%%%%%%%%%%%%%%%%%%%%%%%%%%%%%%%
\title{Boundary Conditions for the Anisotropic Diffusion Method}
\author{Seth R.~Johnson \and Edward W.~Larsen}

\institute{Department of Nuclear Engineering \& Radiological Sciences, University of Michigan, Ann Arbor, MI, 48109}
\email{sethrj@umich.edu \and edlarsen@umich.edu}
\usepackage{amssymb}
\usepackage{microtype}
\usepackage{graphicx}
\usepackage{booktabs} % \toprule, \midrule, \bottomrule
%%% INCLUDE FILE FOR DEFINITIONS
%%% These may require various packages.

% Shortcuts in regular text
\newcommand{\degs}{\ensuremath{^\circ}}
\newcommand{\EE}[1]{\ensuremath{\times 10^{#1}}}
\newcommand{\ttimes}{\ensuremath{{}\times{}}}
\newcommand{\cclicense}{%
  \smash{\raisebox{-0.45ex}{%
  \setlength{\unitlength}{1em}%
  \begin{picture}(1,1)%
    \put(0.5,0.5){\circle{1}}
    \put(0.5,0.5){\hbox to 0pt{\hss\raisebox{-.45ex}{\tiny\textsf{CC}}\hss}}
  \end{picture}%
  }}%
  \hskip -1em%
  \href{http://creativecommons.org/licenses/by-nc-sa/3.0/}%
  {\ \hskip 1em \textsf{BY-NC-SA}}%
}

%\newcommand{\horizsep}{{\par\noindent\centering\rule[.25ex]{.75\columnwidth}{2pt}\par}}
\newcommand{\horizsep}{\vspace{\baselineskip}\noindent\hspace{\stretch{1}}$
\ast\qquad \ast\qquad \ast\qquad
$ \hspace{\stretch{1}} \vspace{\baselineskip}}
\newcommand{\pytrt}{\textsf{PyTRT}}

% Research
\newcommand{\lop}[1]{\mathcal{L}\!\left[#1\right]}
\newcommand{\lopinv}[2]{\mathcal{I}_{#1}\!\left[#2\right]}
\newcommand{\Dtens}{\mat{D}}
\newcommand{\Etens}{\mat{E}}
\newcommand{\Identitytens}{\mat{I}}
\newcommand{\APone}{AP$_1$}
\newcommand{\Pone}{P$_1$}
\newcommand{\SN}{S$_N$}%{S$_\text{N}$}%{$S_N$}%
\newcommand{\PN}{P$_N$}%{P$_\text{N}$}%{$P_N$}%
\newcommand{\CN}{Crank--Nicolson} %Yes, it's Nic not Nich
\newcommand{\Eddington}{\mathcal{E}} %whatever symbol I decided for Eddington
\newcommand{\RadEn}{E} %whatever symbol I decide for radiation energy
\newcommand{\Sigmatr}{\Sigma_{\mathit{tr}}}

% Program names
\newcommand{\cpp}{\textsf{C\raisebox{0.2ex}{++}}}

% General math shortcuts
\newcommand{\ud}{\mathop{}\!\mathrm{d}}
\newcommand{\pder}[2]{\frac{\partial #1}{\partial #2}}
\newcommand{\oder}[2]{\frac{\mathrm{d} #1}{\mathrm{d} #2}}
\newcommand{\tpder}[2]{{\partial #1}/{\partial #2}} %inlined
\newcommand{\toder}[2]{{\mathrm{d} #1}/{\mathrm{d} #2}} %inlined
\newcommand{\lra}{ \quad \Longrightarrow \quad }
\newcommand{\eexp}{\mathop{}\!\mathrm{e}} % upright ``e'' for exponent
\newcommand{\expp}[1]{\exp\!\left( {#1} \right)} % exp with parentheses
\newcommand{\qeq}{\stackrel{\mathrm{?}}{=}}

% Probability
\newcommand{\expectation}[1]{\mathop{}\!\mathrm{E}\!\left[ #1 \right]}
\DeclareMathOperator{\Var}{Var} % variance

% Asymptotic analysis
\DeclareMathOperator{\Ei}{Ei} % Exponential function
\newcommand{\lapl}[1]{\mathcal{L}[{#1}]} %laplace

%change the Re and Im operators from fancy curly letters
\DeclareMathOperator{\MathOpRe}{Re}
\renewcommand{\Re}{\MathOpRe}
\DeclareMathOperator{\MathOpIm}{Im}
\renewcommand{\Im}{\MathOpIm}

%imaginary ``i'' , upright 'i' or \imath
\newcommand{\iimag}{\mathrm{i}}

% Finite differences
\newcommand{\hot}{\text{h.o.t.}}
\newcommand{\inv}{^{-1}}

% Numerical Linear Algebra
\newcommand{\conj}{^{\ast}} % complex conjugate (transpose)
\newcommand{\norm}[1]{\left\| #1 \right\|} % double pipe
\newcommand{\abs}[1]{\left| #1 \right|} % single pipe
\newcommand{\eps}{\varepsilon}
\DeclareMathOperator{\fl}{fl}

\DeclareMathOperator{\acosh}{arccosh} 

% Define a command to write a nice-looking element, e.g. 4,2 He
\newcommand{\elem}[3]{\ensuremath{{}^{{#1}}_{{#2}}\mathrm{{#3}}}}

% Vector definitions
\newcommand{\mat}[1]{\mathbf{#1}} %matrix is bold upright
\renewcommand{\vec}[1]{\bm{#1}} %vector is bold italic
\newcommand{\op}[1]{\mathsf{#1}} % ``operator'' is sans serif

\newcommand{\vd}{\bm{\cdot}} % slightly bold vector dot
\newcommand{\del}{\vec{\nabla}} % gradient (Del) is bold
\newcommand{\grad}{\vec{\nabla}} % gradient

%\newcommand{\abr}[1]{\langle {#1} \rangle}
\newcommand{\abr}[1]{\left\langle {#1} \right\rangle} % angle brackets for avg.

%% topbox is useful in extended definitions of math terms inside an align
\newcommand{\topbox}[2][0.6]{\parbox[t]{#1\columnwidth}{\raggedright{}#2}}

% commands to make text in math mode appear as zero-width (better-looking
% integrals/sums, e.g.)
% from mathmode.pdf page 74, or Alexander R. Perlis ``A complement to \smash,
% \llap, and \rlap''

\def\mathllap{\mathpalette\mathllapinternal}
	\def\mathllapinternal#1#2{%
	\llap{$\mathsurround=0pt#1{#2}$}%
}
\def\clap#1{\hbox to 0pt{\hss#1\hss}}%
\def\mathclap{\mathpalette\mathclapinternal}%
\def\mathclapinternal#1#2{%
	\clap{$\mathsurround=0pt#1{#2}$}%
}
\def\mathrlap{\mathpalette\mathrlapinternal}%
\def\mathrlapinternal#1#2{%
	\rlap{$\mathsurround=0pt#1{#2}$}%
}

\newcommand{\url}[1]{\texttt{#1}}

% graphics paths
\graphicspath{{/Users/seth/_research/figures/}}
\makeatletter
\def\input@path{{/Users/seth/_research/figures/}}
\makeatother

\renewcommand{\bottomfraction}{0.99}
\renewcommand{\topfraction}{0.99}

%\hyphenpenalty=800
%\tolerance=400

%\usepackage{setspace}
%\doublespacing

\begin{document}
%%%%%%%%%%%%%%%%%%%%%%%%%%%%%%%%%%%%%%%%%%%%%%%%%%%%%%%%%%%%%%%%%%%%%%%%%%%%%%%%
\section{Introduction}
A new anisotropic diffusion (AD) approximation has recently been developed
that uses transport-calculated AD coefficients to cheaply but accurately treat
particle transport, particularly in voided regions where standard diffusion
theory breaks down \cite{Lar2009c}. Prior derivations of the AD method have
assumed an infinite medium: boundary conditions have been unaddressed.

This summary presents a different derivation that formulates boundary
conditions for the AD method. The proposed boundary conditions are
implemented in and tested with the \pytrt\ research code \cite{Pytrt}. 

%%%%%%%%%%%%%%%%%%%%%%%%%%%%%%%%%%%%%%%%%%%%%%%%%%%%%%%%%%%%%%%%%%%%%%%%%%%%%%%%
\section{Analysis}
We consider a 3-D, monoenergetic, steady-state transport equation
with isotropic scattering and an isotropic extraneous source:
\begin{subequations} \label{eqs:fullTransport}
\begin{multline} \label{eq:fullTransportVol}
  \vec{\Omega}\vd \grad \psi(\vec{x}, \vec{\Omega})
  + \sigma(\vec{x}) \psi (\vec{x}, \vec{\Omega})
\\ =
  \frac{\sigma_s(\vec{x})}{4\pi} \int_{4\pi} \psi(\vec{x},\vec{\Omega}')
  \ud\Omega' + \frac{q(\vec{x})}{4\pi} 
  \equiv \frac{1}{4\pi} Q(\vec{x}) \,,
\end{multline}
inside $x \in V$, $\vec{\Omega} \in 4\pi$, with an incident boundary
condition
\begin{equation} \label{eq:fullTransportBndy}
  \psi(\vec{x}, \vec{\Omega}) = \psi^b(\vec{x}, \vec{\Omega}) \,,
 \quad \vec{x} \in \partial V, \ \vec{\Omega} \vd \vec{n} < 0\,.
\end{equation}
\end{subequations}
Operating on Eq.~\eqref{eq:fullTransportVol} by $\int_{4\pi} (\cdot) \ud
\Omega$ gives the particle conservation equation
\begin{equation} \label{eq:loVol}
  \grad \vd\vec{J}(\vec{x})
  + \sigma(\vec{x}) \phi(\vec{x})
  = Q(\vec{x})\,,
  \quad \vec{x} \in V\,.
\end{equation}
If we add $-\vec{\Omega}\vd \grad \phi$ to both sides of Eq.~\eqref{eq:loVol},
divide the result by $4\pi$, and subtract from
Eq.~\eqref{eq:fullTransportVol}, the isotropic scattering and extraneous
sources cancel, and we get a transport equation:
\begin{equation*}
  \vec{\Omega}\vd \grad \left[ \psi
  - \frac{1}{4\pi} \phi \right]
  + \sigma \left[ \psi
  - \frac{1}{4\pi} \phi \right]
  = 0 + \frac{1}{4\pi} \grad \vd\vec{J} -
  \frac{1}{4\pi} \vec{\Omega}\vd \grad \phi\,.
\end{equation*}
Defining the ``anisotropic angular flux''
\begin{equation}\label{eq:capPsi}
  \Psi(\vec{x},\vec{\Omega}) \equiv \psi(\vec{x},\vec{\Omega})
  - \frac{1}{4\pi} \phi(\vec{x})\,,
\end{equation}
which is the angular flux with the isotropic component subtracted off, we have
an exact transport equation
\begin{subequations} \label{eqs:capPsiTransport}
\begin{equation} \label{eq:capPsiVol}
  \vec{\Omega}\vd \grad \Psi(\vec{x}, \vec{\Omega})
  + \sigma(\vec{x}) \Psi(\vec{x}, \vec{\Omega})
  = \frac{1}{4\pi} \grad \vd\vec{J}(\vec{x}) -
  \frac{1}{4\pi} \vec{\Omega}\vd \grad \phi(\vec{x})\,.
\end{equation}
Subtracting $\phi/4\pi$ from Eq.~\eqref{eq:fullTransportBndy} gives the
boundary condition
\begin{equation}\label{eq:capPsiBndy}
 \Psi(\vec{x}, \vec{\Omega}) 
  =\psi^b(\vec{x}, \vec{\Omega}) - \frac{1}{4\pi} \phi(\vec{x})\,,
  \quad \vec{x} \in \partial V,\ \vec{\Omega} \vd \vec{n} < 0\,.
\end{equation}
\end{subequations}

The true transport solution will satisfy these equations exactly. The
true anisotropic angular flux also satisfies the identities
\begin{equation} \label{eq:capPsiIdentities}
  \int_{4\pi} \Psi(\vec{x}, \vec{\Omega}) \ud\Omega = 0
  \quad\text{and}\quad
  \int_{4\pi} \vec{\Omega}\Psi(\vec{x}, \vec{\Omega}) \ud\Omega =
  \vec{J}(\vec{x})\,.
\end{equation}

Introducing the approximation $\Psi\approx \frac{3}{4\pi}
\vec{\Omega}\vd\vec{J}$ into Eq.~\eqref{eq:capPsiVol} and taking the first
angular moment would yield Fick's law, but the AD method does something far
stranger.

%%%%%%%%%%%%%%%%%%%%%%%%%%%%%%%%%%%%%%%%%%%%%%%%%%%%%%%%%%%%%%%%%%%%%%%%%%%%%%%%
\subsection{Anisotropic Diffusion}
In order to formulate transport-matched boundary conditions, we separate $\Psi$
into an internal solution $\tilde\Psi$ and a boundary layer solution
$\Psi_\mathrm{bl}$:
\begin{equation} \label{eq:boundaryLayerPsi}
  \Psi(\vec{x}, \vec{\Omega})
  = \tilde\Psi(\vec{x}, \vec{\Omega})
  + \Psi_\mathrm{bl}(\vec{x}, \vec{\Omega})\,.
\end{equation}
The internal transport equation is just like Eq.~\eqref{eq:capPsiVol}:
\begin{subequations} \label{eqs:tCapPsi}
\begin{equation} \label{eq:tCapPsiVol}
  \vec{\Omega}\vd \grad \tilde\Psi(\vec{x}, \vec{\Omega})
  + \sigma(\vec{x}) \tilde\Psi(\vec{x}, \vec{\Omega})
  = \frac{1}{4\pi} \grad \vd\vec{J}(\vec{x}) -
  \frac{1}{4\pi} \vec{\Omega}\vd \grad \phi(\vec{x})\,.
\end{equation}
However, we will define incident boundary conditions for this internal solution
to be
\begin{equation}\label{eq:tCapPsiBndy}
 \tilde\Psi(\vec{x}, \vec{\Omega}) 
  = - \zeta(\vec{x}, \vec{\Omega}) \vec{\Omega}\vd \grad \phi(\vec{x})
  \equiv \tilde\Psi^b(\vec{x}, \vec{\Omega}) \,.
\end{equation}
\end{subequations}
The boundary layer solution complements the internal solution. Its transport
equation has no internal source, and its boundary condition is just
Eq.~\eqref{eq:tCapPsiBndy} subtracted from Eq.~\eqref{eq:capPsiBndy}:
\begin{equation} \label{eq:blCapPsiBndy}
 \Psi_\mathrm{bl}(\vec{x}, \vec{\Omega}) 
  = \psi^b(\vec{x}, \vec{\Omega}) - \frac{1}{4\pi} \phi(\vec{x})
  + \zeta(\vec{x}, \vec{\Omega}) \vec{\Omega}\vd \grad \phi(\vec{x})\,.
\end{equation}

Integrating along a characteristic ray transforms the differential
equation~\eqref{eq:tCapPsiVol} into an integral transport equation
\cite{Pri2010}:
\begin{subequations} \label{eqs:inverseTransport}
\begin{align} \label{eq:inverseTransportFull}
\begin{split}
  \tilde\Psi(\vec{x}, \vec{\Omega})
  &=
  \tilde\Psi^b(\vec{x} - s_b\vec{\Omega}, \vec{\Omega})
  \eexp^{ -\tau(\vec{x}, \vec{x} - s_b \vec{\Omega})}
  \\
  &\quad +  \int_{0}^{s_b}
  \Big[\tfrac{1}{4\pi} \grad \vd\vec{J}(\vec{x} - s \vec{\Omega}, \vec{\Omega})
  \\
  &\quad\qquad - \tfrac{1}{4\pi} \vec{\Omega}\vd \grad \phi(\vec{x} - s \vec{\Omega}, \vec{\Omega}) \Big]
  \eexp^{ -\tau(\vec{x}, \vec{x} - s \vec{\Omega})} \ud s
\end{split}
\\
  \label{eq:inverseTransportBrief}
\begin{split}
  \tilde\Psi(\vec{x}, \vec{\Omega})
    &\equiv
    -\lopinv{b}{\zeta \vec{\Omega}\vd \grad \phi}
    + \lopinv{v}{\tfrac{1}{4\pi} \grad \vd\vec{J} }
    - \lopinv{v}{\tfrac{1}{4\pi} \vec{\Omega}\vd \grad \phi}\,.
\end{split}
\end{align}
The optical thickness of the medium between points $\vec{x}$ and $\vec{x}'$
along direction $\vec{\Omega} = (\vec{x}'- \vec{x})/\norm{\vec{x}'-\vec{x}}$ is 
\begin{equation} \label{eq:fullTauDefinition}
  \tau(\vec{x}, \vec{x}') = \int_{0}^{\norm{\vec{x} -
  \vec{x}'}} \sigma(\vec{x}-s\vec{\Omega}) \ud s \,.
\end{equation}
The term $s_b$ is the distance to the boundary along $-\vec{\Omega}$ from
$\vec{x}$.
\end{subequations}

Now we make an ansatz that the spatial gradients of the angular flux
are weak, and the solution is mildly (but not necessarily linearly) anisotropic:
\begin{align} \label{eq:ansatz}
  \psi &= O(1), &
  \grad \psi &= O(\epsilon), &
  \int_{4\pi} \vec{\Omega} \psi\ud\Omega &= O(\epsilon).
\end{align}
With this ansatz, $\grad \vd\vec{J}=O(\epsilon^2)$.
We can also expand the nonlocal variables in Eq.~\eqref{eq:inverseTransportFull} about the local spatial point:
\begin{equation} \label{eq:taylorPhi}
  \phi(\vec{x} - s \vec{\Omega})
  \sim \phi(\vec{x}) - s \vec{\Omega} \vd \grad \phi (\vec{x}) + O(\epsilon^2)
  \sim \phi(\vec{x}) + O(\epsilon) \,.
\end{equation}
Thus, the third term in Eq.~\eqref{eq:inverseTransportBrief} simplifies to
\begin{align}\nonumber
- \lopinv{v}{\frac{1}{4\pi} \vec{\Omega}\vd \grad \phi(\vec{x})}
  &\approx \int_{0}^{\norm{\vec{x} - \vec{x}_b}}
    \left[ -\frac1{4\pi}\vec{\Omega}\vd \grad \tilde\phi(\vec{x}) \right]
    \eexp^{ -\tau(\vec{x}, \vec{x} - s \vec{\Omega})}
    \ud s
  \\\nonumber
  &= - \int_{0}^{\norm{\vec{x} - \vec{x}_b}}
    \left[ \frac1{4\pi}\right]
    \eexp^{ -\tau(\vec{x}, \vec{x} - s \vec{\Omega})} \ud s \,
    \vec{\Omega}\vd \grad \phi(\vec{x})
  \\\label{eq:streamingApprox}
  &= - \lopinv{v}{ \frac1{4\pi} } \vec{\Omega}\vd \grad \phi(\vec{x}) +
  O(\epsilon^2) \,.
\end{align}
Similarly,
\begin{equation} \label{eq:bndyApprox}
-\lopinv{b}{\zeta \vec{\Omega}\vd \grad \phi}_{\partial V_b}
\approx -\lopinv{b}{\zeta}_{\partial V_b} \vec{\Omega}\vd \grad \phi(\vec{x})
+ O(\epsilon^2) \,.
\end{equation}
Substituting Eqs.~\eqref{eq:streamingApprox} and~\eqref{eq:bndyApprox} into
Eq.~\eqref{eq:inverseTransportBrief}, and discarding $O(\epsilon^2)$ terms, we
have an approximate representation for the anisotropic angular flux:
\begin{align} \label{eq:approxPsi2}
  \tilde\Psi
  &\approx 
- \lopinv{b}{\zeta}_{\partial V_b} \vec{\Omega}\vd \grad \phi
- \lopinv{v}{\tfrac{1}{4\pi}}  \vec{\Omega}\vd \grad \phi
%\\ \label{eq:approxPsi2}
%  \tilde\Psi &= 
%- \left\{ \lopinv{b}{\zeta}_{\partial V_b} 
%+ \lopinv{b}{f(\vec{\Omega}_r)}_{\partial V_r}
%+ \lopinv{v}{\frac{1}{4\pi}} \right\} \vec{\Omega}\vd \grad \phi
\\ \label{eq:approxPsi3}
\tilde\Psi(\vec{x}, \vec{\Omega}) &\equiv - f(\vec{x}, \vec{\Omega})
\vec{\Omega}\vd \grad \phi(\vec{x})\,.
\end{align}
Converting the $\lopinv{}{\cdot}$ terms back into the differential form, we
see that $f$ is the solution of a purely absorbing transport equation with a
uniform, isotropic source:
\begin{subequations} \label{eqs:fFull}
  \begin{equation} \label{eq:fFullVol}
    \vec{\Omega}\vd \grad f(\vec{x}, \vec{\Omega})
    + \sigma(\vec{x}) f (\vec{x}, \vec{\Omega})
  = \tfrac{1}{4\pi} \,, \quad x \in V,\ \vec{\Omega} \in 4\pi\,,
  \end{equation}
and to-be-determined boundary conditions,
\begin{equation} \label{eq:fFullBndy}
  f(\vec{x}, \vec{\Omega}) = \zeta(\vec{x}, \vec{\Omega}) \,,
 \quad \vec{x} \in \partial V, \ \vec{\Omega} \vd \vec{n} < 0\,.
\end{equation}
\end{subequations}

We use the identity from Eq.~\eqref{eq:capPsiIdentities} to get an expression
for the current by taking the first moment of Eq.~\eqref{eq:approxPsi3}:
\begin{align} \nonumber
  \vec{J}(\vec{x})
  &= 
  - \left[ \int_{4\pi} \vec{\Omega} \vec{\Omega} f(\vec{x}, \vec{\Omega})
  \ud\Omega \right]
  \vd \grad \phi(\vec{x})
  \\ \label{eq:anisotropicFicks}
  &= - \Dtens(\vec{x}) \vd \grad \phi(\vec{x}) \,.
\end{align}
This resembles Fick's law, but instead of a scalar diffusion coefficient,
the anisotropic diffusion method has a diffusion \emph{tensor}, $\Dtens$, the
second angular moment of $f$.

The unknown function $\zeta(\vec{x}, \vec{\Omega})$ that lives on the boundary
is a degree of freedom introduced at the beginning of the anisotropic
diffusion derivation. It allowed us to formulate a specified boundary condition
such that the effect of $\zeta$ could be embedded in the anisotropic diffusion
tensor $\Dtens$. To make use of this degree of freedom, we choose to enforce on
the boundary the truth from Eq.~\eqref{eq:capPsiIdentities}, that the zeroth
moment of $\Psi(\vec{x}, \vec{\Omega})$ is zero. Applying the identity to
Eq.~\eqref{eq:approxPsi3} shows that for that truth to hold, $f$ must be an
even function of $\vec{\Omega}$, so $\zeta(\vec{\Omega})=f(-\vec{\Omega})$. If
$f$ is azimuthally symmetric about $\vec{n}$ on the boundary, then
\begin{equation*}
f(\vec{x}, \vec{\Omega}) = \zeta(\vec{x},\vec{\Omega})=f(\vec{x},\vec{\Omega}-2(\vec{n}\vd\vec{\Omega})\vec{n})\,,
 \quad \vec{x} \in \partial V, \ \vec{\Omega} \vd \vec{n} < 0\,,
\end{equation*}
a reflecting boundary condition for $f$.

Now we return to the boundary layer transport equation~\eqref{eq:blCapPsiBndy}.
A lengthy analysis shows that the transport boundary layer decays most rapidly
if the solution of the approximate method satisfies the boundary condition
\begin{equation} \label{eq:bcW}
  0 = \int_{\vec{\Omega} \vd \vec{n} < 0} W(\abs{\vec{\Omega} \vd \vec{n}})
  \Psi_\mathrm{bl} (\vec{x}, \vec{\Omega}) \ud \Omega\,,\qquad \vec{x} \in
  \partial V\,,
\end{equation}
where $W(\mu)$ is well-approximated by the simple polynomial $\mu +
\tfrac{3}{2} \mu^2$ \cite{Mal1991}. Using $W(\mu)=2\mu$ with
Eq.~\eqref{eq:bcW} is the Marshak approximation.

Applying Eq.~\eqref{eq:bcW} to Eq.~\eqref{eq:blCapPsiBndy}, using
$\zeta(\vec{\Omega})=f(-\vec{\Omega})$, we find
\begin{equation*}
  2\int_{\vec{\Omega}\vd \vec{n} < 0}
  W(\abs{\vec{\Omega} \vd \vec{n}}) \psi^b(\vec{x}, \vec{\Omega}) \ud\Omega
  = \phi(\vec{x})
  - 2\int_{\vec{\Omega}\vd \vec{n} < 0} W(\abs{\vec{\Omega} \vd \vec{n}})
  f(\vec{x}, -\vec{\Omega}) \vec{\Omega} \ud\Omega
  \vd \grad \phi(\vec{x})
\end{equation*}

%%%%%%%%%%%%%%%%%%%%%%%%%%%%%%%%%%%%%%%%%%%%%%%%%%%%%%%%%%%%%%%%%%%%%%%%%%%%%%%%
\section{Numerical Results}

%%%%%%%%%%%%%%%%%%%%%%%%%%%%%%%%%%%%%%%%%%%%%%%%%%%%%%%%%%%%%%%%%%%%%%%%%%%%%%%%
\section{Conclusions}

%%%%%%%%%%%%%%%%%%%%%%%%%%%%%%%%%%%%%%%%%%%%%%%%%%%%%%%%%%%%%%%%%%%%%%%%%%%%%%%%
\section{Acknowledgments}
This material is based upon work supported a Department of Energy Nuclear
Energy University Programs Graduate Fellowship.

%%%%%%%%%%%%%%%%%%%%%%%%%%%%%%%%%%%%%%%%%%%%%%%%%%%%%%%%%%%%%%%%%%%%%%%%%%%%%%%%
\bibliographystyle{ans}
\bibliography{../SRJall}
\end{document}

