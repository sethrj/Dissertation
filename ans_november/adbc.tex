\documentclass{anstrans}
%%%%%%%%%%%%%%%%%%%%%%%%%%%%%%%%%%%
\title{Anisotropic Diffusion Boundary Conditions}
\author{Seth R.~Johnson \and Edward W.~Larsen}

%% uncomment these next five only if using anstrans
\institute{Department of Nuclear Engineering \& Radiological Sciences, University of Michigan, Ann Arbor, MI, 48109}
\email{sethrj@umich.edu \and edlarsen@umich.edu}
\usepackage{amssymb}
\usepackage{microtype}
\usepackage{graphicx}
\usepackage{booktabs} % \toprule, \midrule, \bottomrule
/Users/seth/Documents/Compositions/SRJinclude.tex

% graphics paths
\graphicspath{{/Users/seth/_research/figures/}}
\makeatletter
\def\input@path{{/Users/seth/_research/figures/}}
\makeatother

\renewcommand{\bottomfraction}{0.99}
\renewcommand{\topfraction}{0.99}

%\usepackage{setspace}
%\doublespacing

\begin{document}
%%%%%%%%%%%%%%%%%%%%%%%%%%%%%%%%%%%%%%%%%%%%%%%%%%%%%%%%%%%%%%%%%%%%%%%%%%%%%%%%
\section{Introduction}
A new anisotropic diffusion (AD) approximation has recently been developed
that uses transport-calculated AD coefficients to cheaply but accurately treat
particle transport, particularly in voided regions where standard diffusion
theory breaks down \cite{Lar2009a}.

%%%%%%%%%%%%%%%%%%%%%%%%%%%%%%%%%%%%%%%%%%%%%%%%%%%%%%%%%%%%%%%%%%%%%%%%%%%%%%%%
\section{Analysis}

%%%%%%%%%%%%%%%%%%%%%%%%%%%%%%%%%%%%%%%%%%%%%%%%%%%%%%%%%%%%%%%%%%%%%%%%%%%%%%%%
\subsection{Boundary Condition}

%%%%%%%%%%%%%%%%%%%%%%%%%%%%%%%%%%%%%%%%%%%%%%%%%%%%%%%%%%%%%%%%%%%%%%%%%%%%%%%%
\section{Numerical Results}

%%%%%%%%%%%%%%%%%%%%%%%%%%%%%%%%%%%%%%%%%%%%%%%%%%%%%%%%%%%%%%%%%%%%%%%%%%%%%%%%
\section{Conclusions}

%%%%%%%%%%%%%%%%%%%%%%%%%%%%%%%%%%%%%%%%%%%%%%%%%%%%%%%%%%%%%%%%%%%%%%%%%%%%%%%%
\bibliographystyle{ans}
\bibliography{../SRJall}
\end{document}

