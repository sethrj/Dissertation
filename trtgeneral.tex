\documentclass[11pt]{SRJresearch}
\author{Seth R.~Johnson}
\date{\today}
\title{Thermal radiative transport: basic stuff}

\begin{document}
%%%%%%%%%%%%%%%%%%%%%%%%%%%%%%%%%%%%%%%%%%%%%%%%%%%%%%%%%%%%%%%%%%%%%%%%%%%%%%%%
\section{Thermal radiative transport equations}

\subsection{Full, multi-D}
The thermal radiative transport (TRT) equations consist of two coupled partial
differential equations.
\begin{subequations} \label{eqs:fullTRT}
The first describes the behavior of the radiation field, where photons stream,
are absorbed, and are emitted by the material or an extraneous source:
\begin{equation} \label{eq:fullTransport}
  \frac{1}{c} \pder{}{t} I + \vec{\Omega} \vd \del I +\sigma I
  = \sigma B + \frac{Q}{4\pi} \,.
\end{equation}
The second equation describes an energy balance in the material, where energy
is gained by absorption of radiation and is lost by thermal emission into the
radiation field:
\begin{equation} \label{eq:fullMaterial}
  \pder{U_m}{t} = \int_{0}^{\infty}  \int_{4\pi} \sigma\left[ I -
  B \right] \! \ud \Omega\ud \nu \,.
\end{equation}
\end{subequations}
The unknowns $I$ and $T$ require appropriate initial conditions, and $I$ needs
boundary conditions for all $t$.

The material energy $U_m$ is related to the unknown temperature $T$ by the
specific heat capacity:
\begin{equation} \label{eq:heatCapacity}
  U_m(T) = \int_{0}^{T} c_v(T') \ud T'\,.
\end{equation}
(If $c_v$ is a constant, then $U_m(T) = c_v T$; otherwise,
Eq.~\eqref{eq:heatCapacity} adds another source of nonlinearity to an already
strongly nonlinear problem.)

The notation and omitted parameters in Eqs.~\eqref{eqs:fullTRT} are:
\begin{alignat*}{2}
  I &= I(\vec{x}, \vec{\Omega}, \nu, t) &&= \text{the angle-dependent
  radiation intensity,}
  \\
  \sigma &= \sigma(\vec{x}, \nu, T) &&= \text{the absorption opacity,} 
  \\
  Q &= Q(\vec{x}, \nu) &&= \text{an extraneous isotropic source with units
  [energy]/([length]$^3$-[time]),}
  \\
  B &= B(\nu, T) &&= \text{the Planck function,}
  \\
  U_m &= U_m(\vec{x}, T) &&= \text{the material energy density,}
  \\
  c_v &= c_v(\vec{x}, T) &&= \text{the specific heat capacity of a material,}
  \\
  T &= T(\vec{x}, t) &&= \text{the temperature of the material, and}
  \\
  c& &&= \text{the speed of light.}
\end{alignat*}
Often in the Monte Carlo community, the Planck function $B$ is decomposed into
an emissive intensity and a normalized, frequency-dependent shape function:
\begin{equation*}
  B(\nu, T) \equiv a c T^4 \frac{b}{4\pi}
\end{equation*}
where $\int_{0}^{\infty} b(\nu,T) \ud \nu=1$ for any $T$, and $a$ is the
Stefan--Boltzmann radiation constant.

If we define the Planck-weighted averaged opacity
\begin{equation} \label{eq:planckCrossSection}
  \sigma_P(T) = \frac{\int_{0}^{\infty} \sigma(\nu, T) B(\nu,T) \ud
  \nu}{\int_{0}^{\infty}B(\nu, T) \ud T}
  = \int_{0}^{\infty} \sigma(\nu, T) b(\nu, T) \ud \nu
\end{equation}
then Eq.~\eqref{eq:fullMaterial} can be rewritten
\begin{equation} \label{eq:fullMaterialSimplified}
  \pder{U_m}{t} = \int_{0}^{\infty} \sigma  \int_{4\pi} I \ud \Omega\ud \nu -
  \sigma_P ac T^4 \,.
\end{equation}
%%%%%%%%%%%%%%%%%%%%%%%%%%%%%%%%%%%%%%%%%%%%%%%%%%%%%%%%%%%%%%%%%%%%%%%%%%%%%%%
\subsection{Intensity}
The intensity is a scalar quantity in a seven-dimensional phase space:
\begin{equation*}
  I(\vec{x}, \vec{\Omega}, \nu, t)= [h \nu] \psi(\vec{x}, \vec{\Omega}, \nu, t)
  = [h\nu] [c\, n(\vec{x}, \vec{\Omega}, \nu, t)]
\end{equation*}
where $h \nu$ is the quantum of energy in a photon with frequency
$\nu$, $\psi(\vec{x}, \vec{\Omega}, \nu, t)$ is (in neutron transport
terminology) the angular flux, and
\begin{equation*}
  n(\vec{x}, \vec{\Omega}, \nu, t) \ud V \ud \Omega \ud \nu
  = \topbox{the number of photons 
  in the differential volume $\ud V$ about $\vec{x}$,
  traveling in directions $\ud \Omega$ about $\vec{\Omega}$,
  with frequencies inside $\ud \nu$ about $\nu$,
 % in $(\ud V, \ud \Omega, \ud \nu)$
 % about $(\vec{x}, \vec{\Omega}, \nu)$
 at time $t$.}
\end{equation*}
The units of the photon density $n$ and the intensity $I$ are
\begin{align*}
  n &:
  \frac{\text{[\#]}}{[\text{length}]^3 [\text{solid angle}] [\text{freqency}]}
  &
  I &:
  \frac{\text{[energy][length]/[time]}}{[\text{length}]^3 [\text{solid
  angle}] [\text{freqency}]} \,.
\end{align*}

The time-dependent radiation energy---the amount of energy
in the radiation per unit volume---is the scalar quantity (with units
[energy]/[length]$^3$)
\begin{subequations} \label{eqs:fancyMoments}
\begin{equation} \label{eq:energyDensity}
  \RadEn(\vec{x}, t) = \frac{1}{c} \int_{0}^{\infty} \int_{4\pi} I(\vec{x},
  \vec{\Omega}, \nu, t) \ud \Omega \ud \nu \,.
\end{equation}
The energy flux (which, in neutron transport lingo, is known as the
``current'', and is not the scalar flux) is the vector quantity
\begin{equation} \label{eq:energyFlux}
  \vec{F}(\vec{x}, t) = \int_{0}^{\infty}  \int_{4\pi}  \vec{\Omega}\, I(\vec{x},
  \vec{\Omega}, \nu, t) \ud \Omega \ud \nu \,.
\end{equation}
\end{subequations}

%%%%%%%%%%%%%%%%%%%%%%%%%%%%%%%%%%%%%%%%%%%%%%%%%%%%%%%%%%%%%%%%%%%%%%%%%%%%%%%
\subsection{Conservation of energy}
The zeroth moment of the transport equation describes the conservation of
energy in the radiation field. Operating on Eq.~\eqref{eq:fullTransport} by
$\int_{0}^{\infty} \int_{4\pi} (\cdot) \ud \Omega \ud \nu$,
\begin{equation*}
  \frac{1}{c}\pder{}{t} \int_{0}^{\infty} \int_{4\pi}  I \ud \nu
  + \del \vd \int_{0}^{\infty} \int_{4\pi} \vec{\Omega}  I \ud \Omega \ud \nu
  + \int_{0}^{\infty} \sigma \int_{4\pi} I \ud \Omega \ud \nu
  = a c T^4 \int_{0}^{\infty} \sigma b \ud \nu + \int_{0}^{\infty} Q \ud \nu
\end{equation*}
Substituting Eqs.~\eqref{eqs:fancyMoments}, and the Planck-averaged opacity,
this is:
\begin{equation}\label{eq:fullConservation}
  \pder{}{t}\RadEn(\vec{x}, t)
  + \del \vd \vec{F}(\vec{x}, t)
  + \int_{0}^{\infty} \sigma \int_{4\pi} I \ud \Omega \ud \nu
  = \sigma_P a c T^4  + \int_{0}^{\infty} Q \ud \nu \,.
\end{equation}
Rearranging this is merely a balance equation:
\begin{align*}
  \left[ \pder{}{t}\RadEn(\vec{x}, t) \right]
  &= 
  - \left[ \del \vd \vec{F}(\vec{x}, t) \right]
  - \left[ \int_{0}^{\infty} \sigma \int_{4\pi} I \ud \Omega \ud \nu \right]
  + \left[ \sigma_P a c T^4 \right]
  + \left[ \int_{0}^{\infty} Q \ud \nu \right]
  \\
  \left[\parbox[c]{12em}{\raggedleft Time rate of change in
  radiation energy} \right]
  &= {}-{}[\text{Net outleakage rate}]- [\text{Absorption rate}]
  \\
  &\hphantom{ {}={}} + [\text{Emission rate from material}]
  \\
  &\hphantom{ {}={}} + [\text{Emission rate from extraneous source}] \,.
\end{align*}

If we add this balance equation to the material energy balance equation
Eq.~\eqref{eq:fullMaterialSimplified}, the material absorption and emission
terms cancel,
\begin{equation*}
  \pder{}{t} \left[ U_m(\vec{x}, t) + \RadEn(\vec{x}, t)  \right]
  = - \left[ \del \vd \vec{F}(\vec{x}, t) \right]
+ \left[ \int_{0}^{\infty} Q (\vec{x}, t) \ud \nu \right] 
\end{equation*} 
The only way for energy to enter or leave a differential volume is to cross the
boundary or to be born from an extraneous source.
%%%%%%%%%%%%%%%%%%%%%%%%%%%%%%%%%%%%%%%%%%%%%%%%%%%%%%%%%%%%%%%%%%%%%%%%%%%%%%%
\subsection{Material energy}
The material energy density represents how much heat a material has stored.
Absorption increases the value, and emission decreases it: see
Eq.~\eqref{eq:fullMaterialSimplified}. For a given material, is strictly a
function of temperature (which, of course, is a function of time and space)
determined by an intrinsic quantity known as the specific heat capacity:
\begin{subequations} \label{eqs:materialU}
\begin{equation} \label{eq:matEnergyDens}
  U_m(T) = \int_{0}^{T} c_v(T') \ud T' \,.
\end{equation}
The specific heat capacity $c_v(T)$ is the amount of energy per unit
volume needed to change the material's temperature.

The ``equilibrium radiation energy density'' of any material at a certain
temperature is 
\begin{equation} \label{eq:radEnergyDens}
  U_r(T) = aT^4 = \frac{1}{c} \int_{4\pi} \int_{0}^{\infty}B(\nu, T) \ud
  \nu \ud \Omega \,,
\end{equation}
and $U_r = \RadEn$ in a steady-state infinite homogeneous medium (where $I =
B$).
\end{subequations}
It is important to note that, although $U_r$'s name may seem to imply that it
is related to the radiation field in any given problem, $U_r$ is strictly a
property of the material's temperature. In fact, it is essentially a shorthand
for the $T^4$ source term.

The quantity $U_r$ and its name are best explained by considering
an infinite, homogeneous, steady-state problem without external sources.
In this situation, the transport equation Eq.~\eqref{eq:fullTransport}
simplifies to
\begin{align*}
  0 + 0 +\sigma I( \vec{\Omega}, \nu, t) &= \sigma a c T^4 \frac{b(\nu)}{4\pi} + 0 \,.
  \\ 
  \intertext{Dividing by $\sigma c$ and integrating over all frequencies and
  angles, we can substitute Eq.~\eqref{eq:energyDensity}:}
  \frac{1}{c} \int_{0}^{\infty} \int_{4\pi} I( \vec{\Omega}, \nu, t) \ud \Omega \ud \nu
  &=  a T^4 \int_{4\pi} \frac{1}{4\pi} \ud \Omega\int_{0}^{\infty} b(\nu) \ud \nu
  \\
\RadEn &=  U_r \,.
\end{align*}
Thus, at a steady-state equilibrium, the energy density of the radiation
$\RadEn$ is identically equal to the material's equilibrium radiation
energy density $U_r$.

Often, because the material energy equation
Eq.~\eqref{eq:fullMaterialSimplified} contains the derivative of material
energy rather than temperature, a parameter $\beta$ is defined as a function
of Eqs.~\eqref{eqs:materialU}:
\begin{equation} \label{eq:beta}
  \beta(T) = \pder{U_r}{U_m} 
  = \pder{U_r}{T} \Bigg/ \pder{U_m}{T}
  = \frac{4 a T^3}{c_v(T)} \,.
\end{equation}
The time rate of change in the material energy is, by the chain rule,
\begin{equation*}
  \pder{U_m}{t} = \pder{U_m}{U_r} \pder{U_r}{t}
  = \frac{1}{ \beta(T)}\left[ a \pder{}{t} T^4 \right] \,.
\end{equation*}
Then, the material energy equation, Eq.~\eqref{eq:fullMaterialSimplified}, can
be written just as a function of temperature:
\begin{equation} \label{eq:betaMaterial}
  \frac{1}{\beta(T)}  \pder{}{t}\left[ a T^4 \right]
  = \int_{0}^{\infty} \sigma  \int_{4\pi} I \ud \Omega\ud \nu -
  \sigma_P ac T^4 \,.
\end{equation}
The functional dependence of $c_v(T)$ for a material is always known, so
$\beta(T)$ is always known. However, because $T = T(\vec{x},t)$ is a
time-dependent variable, this material energy equation is
nonlinear%
\footnote{Eq.~\eqref{eq:betaMaterial} can actually be linear for a
contrived scenario \cite{Pom1979,Su1997}. When $c_v(T)$ is proportional to
$T^3$, then $\beta$ becomes a constant, and (if $\sigma$ is independent of
temperature) the TRT equations become linear in $T^4$.
}.

The material energy's change as a function can also be explicitly written as a
function of temperature by using the chain rule:
\begin{equation*}
  \pder{U_m}{t} = \pder{U_m}{T} \pder{T}{t}
  = c_v(T) \pder{T}{t}
\end{equation*}
so Eq.~\eqref{eq:fullMaterialSimplified} can be written in yet another way, 
\begin{equation} \label{eq:cvdTdtMaterial}
  c_v(T) \pder{T}{t}
  = \int_{0}^{\infty} \sigma  \int_{4\pi} I \ud \Omega\ud \nu -
  \sigma_P ac T^4 \,.
\end{equation}

%\subsection{Heat capacity models}<++>
%Saha heat capacity model:
%\begin{gather*}
%    c_v(T) = 1 + \alpha (T + 0.3) \frac{\partial \alpha}{\partial T}
%    \\
%    \alpha = \tfrac12 e^{-0.3/T} \left( \sqrt{1+4 e^{0.3/T}} - 1 \right) 
%    \f]    
%\\
%\frac{\partial \alpha}{\partial T} = \frac{0.3}{T^2} \left( \alpha -
%\frac{1}{\sqrt{1+4e^{0.3/T}}} \right)
%\end{gather*}
%%%%%%%%%%%%%%%%%%%%%%%%%%%%%%%%%%%%%%%%%%%%%%%%%%%%%%%%%%%%%%%%%%%%%%%%%%%%%%%
\section{Simplifications}
%%%%%%%%%%%%%%%%%%%%%%%%%%%%%%%%%%%%%%%%%%%%%%%%%%%%%%%%%%%%%%%%%%%%%%%%%%%%%%%
\subsection{Planar geometry}
If all the material properties are dependent only on a single Cartesian
dimension $x$, then the intensity
and temperature distributions will also depend only on $x$. The intensity will
also be independent of the azimuthal direction $\omega$ and will thus be a
function only of the cosine to the $x$ axis, $\mu \in [-1,1]$. Then, we define
the ``slab geometry'' intensity to be the full, three-dimensional intensity
integrated over the unimportant azimuthal variable:
\begin{equation*}
  I(x, \mu, \nu, t) = \int_{0}^{2\pi} I(\vec{x}, \vec{\Omega}, \nu, t) \ud
  \omega = 2\pi I(\vec{x}, \vec{\Omega}, \nu, t) \,.
\end{equation*}
Then, also operating on Eq.~\eqref{eq:fullTransport} by $\int_{0}^{2\pi}
(\cdot)  \ud \omega$, the planar-geometry, frequency-dependent transport
equation becomes
\begin{subequations} \label{eqs:freqonedTRT}
\begin{equation} \label{eq:freqonedTransport}
  \frac{1}{c} \pder{}{t} I + \mu\pder{}{x} I +\sigma I
  = \sigma a c T^4 \frac{b}{2} + \frac{Q}{2}
\end{equation}
where $I=I(x, \mu, \nu, t)$ is the new, one-dimensional intensity. The material
energy equation, Eq.~\eqref{eq:fullMaterialSimplified}, is then (with the new,
one-dimensional intensity) just
\begin{equation} \label{eq:freqonedMaterial}
  \pder{U_m}{t} = \int_{0}^{\infty} \sigma  \int_{-1}^{1} I \ud \mu \ud \nu -
  \sigma_P ac T^4 
\end{equation}
\end{subequations}
%%%%%%%%%%%%%%%%%%%%%%%%%%%%%%%%%%%%%%%%%%%%%%%%%%%%%%%%%%%%%%%%%%%%%%%%%%%%%%%
\subsection{Gray}
The gray, frequency-independent TRT equations are derived by integrating the
transport equation Eq.~\eqref{eq:fullTransport} over all frequency:
\begin{equation*}
  \frac{1}{c} \pder{}{t} \int_{0}^{\infty} I \ud \nu
  + \vec{\Omega} \vd \del \int_{0}^{\infty}  I \ud \nu +
  \int_{0}^{\infty} \sigma I \ud \nu
  = \int_{0}^{\infty}  \sigma B  \ud \nu
  + \frac{1}{4\pi} \int_{0}^{\infty} Q \ud \nu
\end{equation*}
Using the definition of the Planck-averaged opacity, letting the
frequency-integrated intensity be
\begin{equation*}
  \hat I(\vec{x}, \vec{\Omega}, t) = \int_{0}^{\infty}  I(\vec{x},
  \vec{\Omega}, \nu, t) \ud \nu \,,
\end{equation*}
%defining a frequency-averaged opacity \marginpar{Is this right? Do we have to
%integrate over angles?}
%\begin{equation*}
%  \hat \sigma = \frac{\int_{0}^{\infty} \sigma I \ud
%  \nu}{\int_{0}^{\infty} I \ud \nu} =  \frac{\int_{0}^{\infty} \sigma I \ud
%  \nu}{\hat I} \,,
%\end{equation*}
and using the frequency-integrated source $\hat Q$, the transport
equation becomes
\begin{equation} \label{eq:fullGrayTransport}
  \frac{1}{c} \pder{}{t} \hat I
  + \vec{\Omega} \vd \del \hat I +
 \int_{0}^{\infty} \sigma I \ud \nu
  = \sigma_P \frac{a c T^4}{4\pi} 
  + \frac{\hat Q}{4\pi} \,.
\end{equation}

The angle-dependent quantity $\int_{0}^{\infty} \sigma I \ud \nu$ cannot be
known without an exact solution to the full transport equation (i.e.,
knowledge of the full, frequency-dependent intensity). The ubiquitous closure
is to
assume a frequency-independent $\hat \sigma$, which is usually based on some
\emph{a priori} approximation to the intensity used to calculate a weighted
integral of $\sigma(\nu)$. (In an emission-dominated problem, it is appropriate
to assume that the intensity will be Planckian, which gives $\hat \sigma
\approx \sigma_P$; in a diffusive problem, the Rosseland-averaged opacity $\hat
\sigma \approx \sigma_R$ is better).

The gray transport equation is then
\begin{equation} \label{eq:fullGrayTransport}
  \frac{1}{c} \pder{}{t} \hat I
  + \vec{\Omega} \vd \del \hat I +
 \hat \sigma \hat I
  = \frac{a c T^4}{4\pi} 
  + \frac{\hat Q}{4\pi} \,.
\end{equation}

TODO: material energy equation with $\hat \sigma$. Also reformulate in terms of
$I - B$, describe/look up Rosseland and Planck.
%%%%%%%%%%%%%%%%%%%%%%%%%%%%%%%%%%%%%%%%%%%%%%%%%%%%%%%%%%%%%%%%%%%%%%%%%%%%%%%
\subsection{Gray planar geometry}
Integrating over both frequency and azimuthal angle, as in the previous two
subsections, the gray, planar geometry TRT equations are the radiation
transport equation:
\begin{subequations} \label{eqs:onedTRTgray}
\begin{equation} \label{eq:onedTransportgray}
  \frac{1}{c} \pder{}{t} I + \mu\pder{}{x} I +\sigma I
  = \frac{\sigma a c T^4}{2} + \frac{Q}{2}
\end{equation}
where $I=I(x, \mu, t)$ is the one-dimensional, gray intensity; and the material
energy equation:
\begin{equation} \label{eq:onedMaterialgray}
  \pder{U_m}{t} = \sigma \left[ \int_{-1}^{1} I \ud \mu - ac T^4 \right] \,.
\end{equation}
\end{subequations}
If the cross-sections of a problem are truly frequency-independent
($\sigma(\nu) = \sigma$), then $\sigma_P = \sigma$, and for a planar-geometry
problem the above equations are exact.
%%%%%%%%%%%%%%%%%%%%%%%%%%%%%%%%%%%%%%%%%%%%%%%%%%%%%%%%%%%%%%%%%%%%%%%%%%%%%%%%
\section{Low-order equations}

\subsection{Energy conservation}
The angular moments of the gray, slab-geometry intensity are defined as
\begin{equation*}
  I_m(x) = \int_{-1}^{1} \mu^m I \ud \mu \,.
\end{equation*}
Expressed as moments, the current [Eq.~\eqref{eq:energyFlux}] is $F(x) =
I_1(x)$, and the energy
density [Eq.~\eqref{eq:energyDensity}] is $\RadEn(x) = I_0(x) / c$.

As we did for the full transport equation in Eq.~\eqref{eq:fullConservation},
we operate on the one-D gray equation [Eq.~\eqref{eq:onedTransportgray}] by
$\int_{-1}^{1} (\cdot) \ud \mu$:
\begin{align*}
  \frac{1}{c} \pder{}{t} \int_{-1}^{1} I \ud \mu
  + \pder{}{x} \int_{-1}^{1}\mu I  \ud \mu 
  +\sigma \int_{-1}^{1} I \ud \mu 
  &= \sigma a c T^4 + Q
  \\
  \frac{1}{c} \pder{}{t} I_0
  + \pder{}{x} I_1
  +\sigma(T) I_0
  &= \sigma a c T^4 + Q \,.
\end{align*}
Adding to the material energy Eq.~\eqref{eq:onedMaterialgray}
\begin{equation*}
  \frac{1}{c} \pder{}{t} I_0
  + \pder{U_m}{t}
  = \left[ \sigma a c T^4 + Q
  - \pder{}{x} I_1
  - \sigma I_0\right]
  + \left[ \sigma I_0 - \sigma ac T^4 \right]
\end{equation*}
which gives the conservation equation
\begin{equation} \label{eq:loConservation}
   \pder{}{t} \left[ \frac{I_0(x,t)}{c} + U_m(x,t) \right]
   = Q(x,t) - \pder{}{x} I_1(x,t) \,.
\end{equation}

Integrating over a single time step by operating on
Eq.~\eqref{eq:loConservation} by $\int_{t_n}^{t_{n+1}} (\cdot) \ud t$, if we
don't care about spatial discretization:
\begin{equation*}
  \left[ \frac{I_0(x,t_{n+1})}{c} + U_m(x,t_{n+1}) \right]
 - \left[ \frac{I_0(x,t_{n})}{c} + U_m(x,t_{n}) \right]
   = \int_{t_n}^{t_{n+1}} Q(x,t) \ud t - \pder{}{x}\int_{t_n}^{t_{n+1}}  I_1(x,t) \ud t\,.
\end{equation*}
This balance equation states that the energy in some differential volume at the
end of the time step minus the energy at the beginning is equal to the energy
added by the source over the time step, minus 

If we consider both a time step $[t_n, t_{n+1}]$ and a spatial
cell $[x_k, x_{k+1}]$, we operate on Eq.~\eqref{eq:loConservation} with
$\int_{x_k}^{x_{k+1}} \int_{t_n}^{t_{n+1}} (\cdot) \ud t \ud x$:
\begin{multline*}
  \frac{1}{c} \int_{x_k}^{x_{k+1}} I_0(x, t_{n+1}) \ud x
  - \frac{1}{c} \int_{x_k}^{x_{k+1}} I_0(x, t_{n}) \ud x
  + \int_{x_k}^{x_{k+1}} U_m(x, t_{n+1}) \ud x
  - \int_{x_k}^{x_{k+1}} U_m(x, t_{n}) \ud x
\\
= \int_{x_k}^{x_{k+1}} \int_{t_n}^{t_{n+1}} Q(x,t) \ud t \ud x
- \int_{t_n}^{t_{n+1}} I_1(x_{k+1}, t) \ud t + \int_{t_n}^{t_{n+1}} I_1(x_{k},
t) \ud t \,,
\end{multline*}
or
\begin{multline*}
  [\text{Radiation energy in cell at end}] 
  - [\text{Radiation energy in cell at beginning}] 
  \\
  + [\text{Material energy in cell at end}] 
  - [\text{Material energy in cell at beginning}]
  \\
  = 
  [\text{Energy added to the cell from the source over the time step}]
  \\
 - [\text{Net leakage to the right over the time step}] 
 + [\text{Net leakage from the left over the time step}] \,.
\end{multline*}
%%%%%%%%%%%%%%%%%%%%%%%%%%%%%%%%%%%%%%%%%%%%%%%%%%%%%%%%%%%%%%%%%%%%%%%%%%%%%%%
\subsection{Formulation}
The material energy equation can also be rewritten (see
Eq.~\eqref{eq:cvdTdtMaterial}) as
\begin{equation} \label{eq:loCvMaterial}
  c_v(T)  \pder{T}{t}
  = \sigma I_0 - \sigma ac T^4 \,.
\end{equation}

The zeroth moment of Eq.~\eqref{eq:onedTransportgray} is
\begin{subequations} \label{eqs:loMoments}
\begin{equation} \label{eq:loZeroth}
  \frac{1}{c} \pder{}{t} I_0(x,t)
  + \pder{}{x} I_1(x,t)
  +\sigma(T(x,t)) I_0(x,t)
  = \sigma(T(x,t)) a c [T(x,t)]^4 + Q(x,t) \,,
\end{equation}
and the first moment of Eq.~\eqref{eq:onedTransportgray} is
\begin{equation} \label{eq:loFirst}
  \frac{1}{c} \pder{}{t} I_1(x,t)
  + \pder{}{x} I_2(x,t)
  +\sigma(T(x,t)) I_1(x,t)
  = 0 \,.
\end{equation}
\end{subequations}

The second moment of the angular intensity is unknown. The variable Eddington
factor \cite{Ols2000} is defined by this second moment as
\begin{equation} \label{eq:eddington}
  \Eddington(x,t) \equiv \frac{I_2(x,t)}{I_0(x,t)} \,.
\end{equation}
It is bounded in $[0,1]$.

%%%%%%%%%%%%%%%%%%%%%%%%%%%%%%%%%%%%%%%%%%%%%%%%%%%%%%%%%%%%%%%%%%%%%%%%%%%%%%%
\bibliographystyle{ans}
\bibliography{SRJall}
\end{document}
